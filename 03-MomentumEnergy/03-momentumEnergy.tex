\documentclass[12pt,compress,aspectratio=169]{beamer}

\mode<presentation>
{
  \usetheme{Singapore}
  \setbeamersize{text margin left=.5cm,text margin right=.5cm}
%  \setbeamertemplate{navigation symbols}{} % suppress nav bar
%  \setbeamercovered{transparent}
}
\usefonttheme{professionalfonts}
\usepackage{amsmath,bm}
\usepackage{siunitx}
%\usepackage{graphicx}
\usepackage{tikz}
\usepackage{mathpazo}
\usepackage[scaled]{helvet}
\usepackage{xcolor,colortbl}
%\usepackage{hyperref}

\sisetup{number-math-rm=\mathnormal}

\title{3.\ Energy, Momentum and Collisions}
\subtitle{AP Physics}
\author[TML]{Dr.\ Timothy Leung}
\institute{Olympiads School}
\date{Fall 2017}

\newcommand{\pic}[2]{\includegraphics[width=#1\textwidth]{#2}}
\newcommand{\mb}[1]{\ensuremath\mathbf{#1}}

\begin{document}

\begin{frame}
  \maketitle
\end{frame}

\begin{frame}
  \frametitle{Files for You to Download}
  \begin{itemize}
  \item\texttt{00-outline.pdf}--The course outline
  \item\texttt{03-momentumEnergy.pdf}--This week's slides
  \item\texttt{03-Homework.pdf} This week's homework
  \end{itemize}
  Please download/print the PDF file.
\end{frame}

\section{Energy}
\begin{frame}
  \frametitle{Work}
  \begin{itemize}
  \item\textbf{Work} is defined as
    {\Large
      \begin{displaymath}
        \boxed{W=\int\mb{F}\cdot d\mb{r}}
      \end{displaymath}
    }
  \item Dot product of force $\mb{F}$ and infinitesimal displacement $d\mb{r}$

  \item No work done if $\mb{F}\cdot\mb{r}=0$ i.e.\ force is perpendicular to
    displacement
  \item No work done if no displacement
  \item Negative work done is possible
  \end{itemize}
\end{frame}

\begin{frame}
  \frametitle{Kinetic Energy}
  \begin{itemize}
  \item When we apply a force on an object to accelerate it, and the resulting
    amount of work done is given by

    \vspace{-0.3in}{\Large
      \begin{align*}
        W&=\int\mb{F}\cdot d\mb{x}=\int m\mb{a}\cdot d\mb{x}\\
        &=\int m\frac{d\mb{v}}{dt}\cdot d\mb{x}
        =\int md\mb{v}\cdot\frac{d\mb{x}}{dt}=\int m\mb{v}\cdot d\mb{v}\\
        &=\int_{v_1}^{v_2} mvdv=\Delta\left(\frac{1}{2}mv^2\right)=\Delta K
      \end{align*}
    }
  \item where $K=\frac{1}{2}mv^2$ is the (translational) kinetic energy
  \end{itemize}
\end{frame}


\begin{frame}
  \frametitle{Example}
  \textbf{Example 1:}A force $F=(\SI{4}{kg/s^2})x\bm{\hat{\imath}}$ acts on an
  object of mass
  \SI{2}{kg} as it moves from $x=0$ to $x=\SI{5}{m}$. Given that the object is
  at rest at $x=0$,
  \begin{enumerate}[(a)]
  \item Calculate the net work
  \item What is the final speed of the object?
  \end{enumerate}
\end{frame}


\begin{frame}
  \frametitle{Gravitational Force and Potential Energy}
  \begin{itemize}
  \item For objects near Earth, the force of gravity is
   
    \vspace{-.2in}{\Large
      \begin{displaymath}
        F_g=mg
      \end{displaymath}
    }
  \item The work done to raise an object from $h_1$ to $h_2$ is therefore:

    \vspace{-.3in}{\Large
      \begin{align*}
        W&=\int \mb{F}_g\cdot d\mb{h}
        =\int_{h_1}^{h_2} -mg\bm{\hat{\jmath}}\cdot dh\hat{\bm{\jmath}}\\
        &=-mgh\Big|^{h_2}_{h_1}=-\Delta(mgh)=-\Delta U_g
    \end{align*}
    }
  \item $U_g=mgh$ is the gravitational potential energy
  \end{itemize}
\end{frame}
 
\begin{frame}
  \frametitle{Spring Force \& Elastic Potential Energy}
  \begin{itemize}
  \item The spring force $F_x$ is the force a compressed or stretched spring
    exerts onto objects connected to it. It obeys Hooke's Law:
    
    \vspace{-.2in}{\Large
      \begin{displaymath}
        F_x=-kx
      \end{displaymath}
    }
  \item If we apply Hooke's law into the work equation, we can find the
    work done when compressing/stretching a spring:

    \vspace{-.35in}{\Large
      \begin{align*}
        W&=\int F_xdx=\int -kxdx=-\frac{1}{2}kx^2\Big|^{x_2}_{x_1}
        =-\Delta\left(\frac{1}{2}kx^2\right)\\
        &=-\Delta U_e
      \end{align*}
    }
  \item $U_e=\frac{1}{2}kx^2$ is the elastic potential energy
  \end{itemize}
\end{frame}

\begin{frame}
  \frametitle{Conservative Forces}
  \begin{itemize}
  \item Gravitational force and spring force (and also electrostatic force)
    are called \textbf{conservative forces}
  \item A conservative force has the property that the work done in moving a
    particle
    between two points is independent of the path taken. This force is related
    to a potential energy by:
    
    \vspace{-.2in}{\Large
      \begin{displaymath}
        \boxed{F_x=-\frac{dU}{dx}}
      \end{displaymath}
    }
  \item The direction of the force decreases the potential energy
  \end{itemize}
\end{frame}

\begin{frame}
  \frametitle{Conservation of Energy}
  %  \framesubtitle{What if there are non-conservative forces?}
  \begin{itemize}
  \item If there are only conservative forces, the change
    {\Large
    \begin{displaymath}
      \Delta K+\Delta U=0\;\;\rightarrow\;\; \boxed{K_1+U_1=K_2+U_2}
    \end{displaymath}
  }
  \item When there are non-conservative forces doing work, 
  {\Large
    \begin{displaymath}
      \boxed{K_1+U_1+W_\mathrm{non-conservative}=K_2+U_2}
    \end{displaymath}
  }
  \item Work done by non-conservative forces $W_\mathrm{non-conservative}$
    are usually friction forces, convert mechanical energy in the system into
    sound and heat
  \end{itemize}
\end{frame}

\begin{frame}
  \frametitle{Example}
  \textbf{Example 2:} A mass $m$ is dropped from a height of $h$ above the
  equilibrium position of a spring. Set up the equation that determines the
  spring's compression $d$ when the object is instantaneously at rest.
  \begin{center}
    \pic{.35}{spring-example1.png}
  \end{center}
\end{frame}


\begin{frame}
  \frametitle{Example}
  \textbf{Example 3:} A mass $m$ is pulled a distance $d$ up an incline (angle
  of elevation $\theta$) at constant speed using a rope that is parallel to
  the incline. The coefficient of friction is $\mu_k$.
  \begin{enumerate}[(a)]
  \item What is the magnitude of the tension force in the rope?
  \item What is the magnitude of the normal force?
  \item What is the work done by the normal force?
  \item What is the work done by friction?
  \item What is the work done by the tension force?
  \item What is the net work?
  \item What is the change in total mechanical energy?
  \item Show that $\Delta E_\mathrm{mech}=W_\mathrm{non-conservative}$.
  \end{enumerate}
\end{frame}


\begin{frame}
  \frametitle{Energy Diagrams}
  \begin{itemize}
  \item Plots of potential energy ($U$) vs.\ position for a conservative force
    \begin{center}
      \pic{0.5}{energy-diagram.png}
    \end{center}
  \item If more than one conservative force, they can be combined into one graph
  \item Where slope is zero means no force acting on it--\textbf{equilibrium}
  \item An object placed at an equilibrium point with $K=0$ it will remain there
  \end{itemize}
\end{frame}

\section{Momentum}


\begin{frame}
  \frametitle{Linear Momentum}
  \textbf{Linear momentum} is directly proportional to the object's
  \textbf{mass} and its \textbf{velocity}.

  \vspace{-0.35in}{\Huge
    \begin{displaymath}
      \boxed{\mb{p}=m\mb{v}}
    \end{displaymath}
  }
  \begin{center}
    \begin{tabular}{l|c|l}
      \rowcolor{pink}
      \textbf{Quantity} & \textbf{Symbol} & \textbf{SI Unit} \\ \hline
      Momentum & $\mb{p}$ & \si{kg.m/s} (kilogram meters per second) \\
      Mass      & $m$    & \si{kg} (kilograms) \\
      Velocity  & $\mb{v}$ & \si{m/s} (meters per second) \\
    \end{tabular}
  \end{center}
  \begin{itemize}
  \item Momentum $\mb{p}$ is a vector in the same direction as velocity
  \item Like all vectors, $\mb{p}$ obeys \emph{superposition}
  \end{itemize}
\end{frame}

\begin{frame}
  \frametitle{Newton's Second Law of Motion}

  Start with our ``standard form'' of Newton's second law of motion with
  constant $m$, we can find out how $\Delta\mb{p}$ relates to $\mb{F}$:
 
  \vspace{-0.25in}{\Large
    \begin{displaymath}
      \sum\mb{F}=m\mb{a}=m\frac{d\mb{v}}{dt}=\frac{d(m\mb{v})}{dt}
      =\frac{d\mb{p}}{dt}
    \end{displaymath}
  }
  \begin{itemize}
  \item $\mb{F}=\mb{p}'(t)$ is the general form, $\mb{F}=m\mb{a}$ is a special
    case
  \item Momentum is conserved (i.e.\ $\sum\mb{p}$ constant) when the net force
    on an object or a system of objects is zero.
  \item Internal forces do not contribute to net force, in that case:
    \begin{displaymath}
      \sum\mb{p}(t_1)=\sum\mb{p}(t_2)
    \end{displaymath}
  \end{itemize}
\end{frame}

\begin{frame}
  \frametitle{Impulse}
  Let's get this by looking at Newton's 2nd law again. If we rearrange the
  variables:
  
  \vspace{-.2in}{\LARGE
    \begin{displaymath}
      \mb{F}=\frac{d\mb{p}}{dt}\;\rightarrow\;
      \mb{F}dt=d\mb{p}
    \end{displaymath}
  }
  We can integrate both sides to get the \textbf{impulse-momentum theorem}.
 
  \vspace{-.2in}{\LARGE
    \begin{displaymath}
      \boxed{\mb{J}=\int_{t_1}^{t_2}\mb{F}dt=\int d\mb{p}=\Delta\mb{p}}
    \end{displaymath}
  }
  The quantity $\mb{J}$ is called the impulse.
\end{frame}


\begin{frame}
  \frametitle{Impulse}
  \begin{itemize}
  \item $\mb{F}$, $\mb{p}$ and $\mb{J}$ are all vectors, so the integral can
    be evaluated in each of the $x$, $y$ and $z$ axis, e.g.

    \vspace{-.3in}{\Large
      \begin{displaymath}
        J_x=\int_{t_1}^{t_2}F_xdt=\int dp_x=\Delta p_x
      \end{displaymath}
    }
    
    \vspace{-.2in}for the $x$ direction.
  \item In Physics 12, we used the ``average force'' to compute
    impulse. In reality, the average force really is a ``force'' that gets the
    same impulse as the integral on the last slide, i.e.

    \vspace{-.35in}{\Large
      \begin{displaymath}
        \mb{F}_\mathrm{ave}=\frac{\int_{t_1}^{t_2}\mb{F}dt}{t_2-t_1}
        =\frac{\mb{J}}{\Delta t}
      \end{displaymath}
     }
  \item Note that impulse does not depend on whether the object moves
  \end{itemize}
\end{frame}

\begin{frame}
  \frametitle{Impulse: An Example}
  \textbf{Example 4:} Jim pushes a box with mass \SI{1}{kg} with a \SI{5}{N}
  force for \SI{10}{\s} while the box stays on the same place. Find the impulse
  of the pushing force, friction force, the gravitational force, and the net
  force.
\end{frame}

\begin{frame}
  \frametitle{Impulse: Another Example}
  \textbf{Example 5:} Two balls of the same mass are dropped from the same
  height onto the floor. The first ball bounces upwards from the floor
  elastically. The second ball sticks to the floor. The first applies an
  impulse to the floor of $I_1$ and the second applies an impulse $I_2$. The
  two impulses obey:
  \begin{enumerate}[(a)]
  \item $I_2=2I_1$
  \item $I_2=I_1/2$
  \item $I_2=4I_1$
  \item $I_2=I_1/4$
  \end{enumerate}
\end{frame}

\begin{frame}
  \frametitle{Conservation of Momentum}
  \begin{itemize}
  \item From Newton's third law, we know that the action and reaction forces are
    always equal in magnitude and in opposite direction. Thus, their total
    impulse would be zero. 
    
  \item Suppose there is no external force, the momentum of the total system
    would always be constant. We saw that a few slides ago:

    \vspace{-.2in}{\Large
      \begin{displaymath}
        \sum\mb{p}(t_1)=\sum\mb{p}(t_2)
      \end{displaymath}
    }
  \end{itemize}
\end{frame}

\begin{frame}
  \frametitle{How to Solve Conservation of Momentum Problem}
  \begin{enumerate}
  \item Check whether the condition for the conservation of momentum is
    satisfied.
  \item If so, write out expressions for initial momentum and final momentum,
    and equate the two. You will get $1$ to $3$ equations (one for each
    direction).
  \item Solve these equations, find the quantity you need to find.
  \end{enumerate}
\end{frame}

\begin{frame}
  \frametitle{Two Remarks}
  \begin{itemize}
  \item Sometimes, the external force \emph{does} exist, but are too small, or
    the time interval of the external force is very short. In these cases, we
    can still regard the total momentum as conserved.
  \item Remember that momentum is a vector. If there is no external force
    component in some direction, then the momentum component in this
    direction is still conserved.
  \end{itemize}
\end{frame}

\begin{frame}
  \frametitle{Example}
  \textbf{Example 6:} Two blocks A and B, both have mass \SI{1}{\kg}. Block A
  has velocity \SI{3}{m/\s} and block B is at rest. Their distance is
  \SI{1}{m}. The surface is has dynamic friction coefficient $0.02$. After they
  collide, they move together, what would be the final velocity of these two
  blocks? How far can they go after the collision?
\end{frame}

%\begin{frame}
%  \frametitle{More Example}
%  Max throws a ball into the air with an initial speed $\SI{10}{m/s} at an
%  angle of $60$ degree with the horizontal direction. By accident, the ball
%  splits into two parts (horizontally) in the air. Suppose both parts land at
%  the same time, neglecting the air resistance,
%  \begin{enumerate}
%  \item If one part is \SI{5}{\m} away from its original position (same
%    direction as the initial speed), where is the second part?
%  \item How about one parties \SI{5}{m} away from the original position in the
%    direction that has an angle of $30$ degree with its initial speed?
%  \end{enumerate}
%\end{frame}
%
\begin{frame}
  \frametitle{Before We Dive Into Some Exercises}
  \begin{itemize}
  \item The most typical applications of momentum conservation are collision
    and explosions
  \item\textbf{Collision: object A hits object B}. Regardless of whether they
    move together or not afterwards, momentum is conserved.
    \begin{itemize}
    \item Head-on collisions are usually 1D
    \item Glancing collisions are usually 2D or 3D.
    \end{itemize}
  \item\textbf{Explosion: A explodes and becomes B and C (and D and E\ldots)}.
    Total momentum of B and C (and D and E\ldots) is the same as A in the
    beginning. 
  \end{itemize}
\end{frame}

\begin{frame}
  \frametitle{Collision Problem}
  \textbf{Example 7:} Two objects with equal mass are heading towards each
  other with equal speeds, undergo a head-on collision. Which one of the
  following statement is correct?
  \begin{enumerate}[(a)]
  \item Their final velocities are zero
  \item Their final velocities may be zero
  \item Each must have a final velocity equal to the other's initial velocity
  \item Their velocities must be reduced in magnitude
  \end{enumerate}
\end{frame}

\begin{frame}
  \frametitle{Conservation of Momentum Example}
  \textbf{Example 8:} Two astronauts, each of mass \SI{75}{kg}, are floating
  next to each other in space, outside the space shuttle. One of them pushes
  the other through a distance of \SI{1}{m} (an arm's length) with a force of
  \SI{300}{\newton}. What is the final relative velocity of the two?
  \begin{enumerate}[(a)]
  \item \SI{2.0}{m/s}
  \item \SI{2.83}{m/s}
  \item \SI{4.0}{m/s}
  \item \SI{16.0}{m/s}
  \end{enumerate}
\end{frame}

\begin{frame}
  \frametitle{Continuous Problems in the Application of Momentum}

%  \textbf{Example 9:} A water fountain sprays water with a flow of
%  \SI{30}{L/min}. Suppose the water has no initial velocity, find the impulse
%  of the pushing force in $1$ hour and estimate the pushing force, assuming the
%  force is constant.
%\end{frame}
%
%\begin{frame}
%  \frametitle{Example: Rocket Thrust}
  \textbf{Example 10:} A rocket generates a thrust force by ejecting hot gases
  from an engine. If it takes \SI{1}{\milli\second} to combust \SI{1}{kg} of
  fuel, ejecting it at a speed of \SI{1000}{m/s}, what thrust is generated?
  \begin{enumerate}[(a)]
  \item \SI{1000}{\newton}
  \item \SI{10000}{\newton}
  \item \SI{100000}{\newton}
  \item \SI{1000000}{\newton}
  \end{enumerate}
\end{frame}

\begin{frame}
  \frametitle{Another Space Example}
  \textbf{Example 11:} A rocket for mining the asteroid belt is designed like a
  large scoop. It is approaching asteroids at a velocity of \SI{e4}{m/s}. The
  asteroids are much smaller than the rocket. If the rocket scoops asteroids at
  at rate of \SI{100}{kg/s}, what thrust (force) must the rocket's engine
  provide in order for the rocket to maintain constant velocity? Ignore any
  variation in the rocket's mass due to the burning fuel.
  \begin{enumerate}[(a)]
  \item \SI{e3}{\newton}
  \item \SI{e6}{\newton}
  \item \SI{e9}{\newton}
  \item \SI{e12}{\newton}
  \end{enumerate}
\end{frame}


\begin{frame}
  \frametitle{Example}
  \textbf{Example 12:} A ball is dropped from a height $h$. It hits the ground
  and bounces up with a momentum loss of $10\%$ due to the impact. The maximum
  height it will reach is:
  \begin{enumerate}[(a)]
  \item $0.90h$
  \item $0.81h$
  \item $0.949h$
  \item $0.3h$
  \end{enumerate}
\end{frame}

\begin{frame}
  \frametitle{Conservation of Energy Example}
  \textbf{Example 13:} A simple pendulum has a bob of mass \SI{2}{kg} hanging
  on a cord of length \SI{1}{m}. Suppose the pendulum is raised until it is
  horizontal (and angular displacement of \ang{90}) and then released. What is
  the speed of the bob at the bottom of its swing?
  \begin{enumerate}[(a)]
  \item\SI{9.91}{m/s}
  \item\SI{19.6}{m/s}
  \item\SI{3.13}{m/s}
  \item\SI{4.43}{m/s}
  \end{enumerate}
\end{frame}
 
\begin{frame}
  \frametitle{Conservation of Energy Example}
  \textbf{Example 14:} A toy firing a ball vertically consists of a vertical
  spring which is compressed by \SI{0.10}{m}. A force of \SI{10.0}{\newton}
  is needed to hold the spring at that compression. If a ball of mass
  \SI{0.050}{kg} is placed on the compressed spring and the spring is released,
  the ball will reach a height (above its initial position) of:
  \begin{enumerate}[(a)]
  \item \SI{1.0}{m}
  \item \SI{1.2}{m}
  \item \SI{1.4}{m}
  \item \SI{1.6}{m}
  \end{enumerate}
\end{frame}


\section{Elastic Collisions}

\begin{frame}
  \frametitle{Classifications of Collisions}
  \begin{itemize}
  \item Elastic Collision:
    \begin{itemize}
    \item Total kinetic energy is conserved
    \item Momentum is conserved
    \end{itemize}
  \item Inelastic collision:
    \begin{itemize}
    \item Kinetic energy is \textbf{not} conserved
    \item Momentum is conserved
    \end{itemize}
  \item Completely inelastic collision:
    \begin{itemize}
    \item ``Perfectly inelastic collision''
    \item The objects move together after the collision
    \item Kinetic energy is \textbf{not} conserved
    \item Momentum is conserved
    \end{itemize}
  \end{itemize}
\end{frame}

\begin{frame}
  \frametitle{Elastic Collision}
  If two objects 1 and 2 of mass $m_1$ and $m_2$ and initial velocities
  $v_{1,i}$ and $v_{2,i}$ collide elastically, their final velocities will be:
  
  {\Large
    \begin{displaymath}
      v_{1,f}=\frac{v_{1,i}(m_1-m_2)+2m_2v_{2,i}}{m_1+m_2}
    \end{displaymath}
    
    \begin{displaymath}
      v_{2,f}=\frac{v_{2,i}(m_2-m_1)+2m_1v_{1,i}}{m_1+m_2}
    \end{displaymath}
  }
\end{frame}

\begin{frame}
  \frametitle{Elastic Collision Example}

  \textbf{Example 15:} Blocks A and B have the same mass; A hits B with a speed
  of \SI{5}{m/s} while B is initially at rest. If the collision is elastic,
  what would be the final speed of these two objects?
\end{frame}


\begin{frame}
  \frametitle{Elastic Collision Example}
  \textbf{Example 16:} Blocks A and B with the same mass; A has a velocity
  \SI{3}{m/s} to the east while B has \SI{2}{m/s} to the west. If the collision
  is elastic, after the collision, what would the velocity of the two blocks be?
\end{frame}


\begin{frame}
  \frametitle{Elastic Collision Example}
  
  \textbf{Example 17:} Throw a ball to a really big wall, when the ball reaches
  the wall, it has a velocity \SI{10}{m/s} towards the wall. If the collision
  is elastic, what would the final velocity of the ball be?
\end{frame}


\begin{frame}
  \frametitle{Elastic Collision Example}
  \textbf{Example 18:} Throw a ball with a velocity \SI{4}{m/s} towards a train
  with a velocity \SI{40}{m/s} towards the ball. If the collision is elastic,
  what would the final velocity of the ball be?
\end{frame}


\begin{frame}
  \frametitle{Inelastic Collision: Calculating Energy Loss}
  Two blocks A and B with mass \SI{2}{kg}, block A hits B with velocity
  \SI{4}{m/s} while B is at rest.
  \begin{enumerate}[(a)]
  \item Suppose the collision is completely inelastic, what would the final
    velocity of A and B be?
  \item What is the loss of energy?
  \end{enumerate}
\end{frame}

\end{document}
