\documentclass[12pt]{article}

\usepackage[margin=0.8in,letterpaper]{geometry}
\usepackage{enumitem}
\usepackage{graphicx}
%\usepackage{tikz,graphicx,wrapfig}
\usepackage{mathpazo}
\usepackage[scaled]{helvet}
\usepackage{siunitx}

%\sisetup{detect-all}

\renewcommand{\familydefault}{\sfdefault}

\newcommand{\pic}[2]{\includegraphics[width=#1\textwidth]{#2}}
\newcommand{\magdir}[2]{$#1\;[\mathrm{#2}]$}
\newcommand{\mb}[1]{\mathbf{#1}}

\begin{document}
\pagestyle{empty}
\begin{center}
  Student \#: \underline{\hspace{1in}}\hspace{1.9in}
  Student Name: \underline{\hspace{2in}}\\
  \vspace{0.3in}
  {\LARGE
    AP Physics \hspace{0.75in}
    Class 3: Momentum \& Energy
  }
\end{center}

\textbf{Multiple-Choice Questions}

\begin{enumerate}[leftmargin=15pt]
  
\item If a projectile thrown directly upward reaches a maximum height $h$ and
  spends a total time in the air of $T$, the average power of the gravitational
  force during the trajectory is
  \begin{enumerate}[noitemsep,topsep=0pt]
  \item $P=2mgh/T$
  \item $P=-2mgh/T$
  \item 0
  \item $P=mgh/T$
  \item $P=-mgh/T$
  \end{enumerate}

\item Given that the constant net force on an object and the object's 
  displacement, which of the following quantities can be calculated?
  \begin{enumerate}[noitemsep,topsep=0pt]
  \item the net change in the object's velocity
  \item the net change in the object's mechanical energy
  \item the average acceleration
  \item the net change in the object's kinetic energy
  \item the net change in the object's potential energy
  \end{enumerate}

\item Consider the potential energy function shown below. Assuming that no
  non-conservative forces are present, if a particle of mass $m$ is released
  from position $x_0$, what is the maximum speed it will achieve?\\
  \begin{minipage}{0.7\textwidth}
    \begin{enumerate}[noitemsep,topsep=0pt]
    \item $\sqrt{4U_0/m}$
    \item $\sqrt{2U_0/m}$
    \item $\sqrt{U_0/m}$
    \item $\sqrt{U_0/2m}$
    \item The particle will achieve no maximum speed but instead will continue
      to accelerate indefinitely.
    \end{enumerate}
  \end{minipage}
  \begin{minipage}{0.3\textwidth}
    \pic{.7}{potential-well.png}
  \end{minipage}

\item Which of the following is the most accurate description of the system
  introduced in the previous question?
  \begin{enumerate}[noitemsep,topsep=0pt]
  \item stable equilibrium
  \item unstable equilibrium
  \item neutral equilibrium
  \item a bound system
  \item There is a linear restoring force
  \end{enumerate}

\item If the only force acting on an object is given by the equation
  $F(x)=2-4x$ (where the force is measured in newtons and position in meters),
  what is the change in the object's kinetic energy as it moves from $x=2$ to
  $x=1$?
  \begin{enumerate}[noitemsep,topsep=0pt]
  \item $+4$ J
  \item $-4$ J
  \item $+2$ J
  \item $-2$ J
  \item $+8$ J
  \end{enumerate}
  \newpage

\item A pendulum bob of mass $m$ is released from rest as shown in the figure
  below. What is the tension in the string as the pendulum swings through the
  lowest point of its motion?\\
  \begin{minipage}{.5\textwidth}
    \begin{enumerate}[noitemsep,topsep=0pt]
    \item $T=\frac{1}{2}mg$
    \item $T=mg$
    \item $T=\frac{3}{2}mg$
    \item $T=2mg$
    \item None of the above
    \end{enumerate}
  \end{minipage}
  \begin{minipage}{.3\textwidth}
    \pic{.6}{pendulum1.png}
  \end{minipage}

\item Two masses moving along the coordinates axes as shown collide at the
  origin and stick to each other. What is the angle $\theta$ that the final
  velocity that makes with the $x$-axis?\\
  \begin{minipage}{.5\textwidth}
    \begin{enumerate}[noitemsep,topsep=0pt]
    \item $\tan^{-1}(v_2/v_1)$
    \item $\tan^{-1}[m_1v_1/(m_1+m_2)]$
    \item $\tan^{-1}(m_1v_2/m_2v_1)$
    \item $\tan^{-1}(m_2v_2^2/m_1v_1^1)$
    \item $\tan^{-1}(m_2v_2/m_1v_1)$
    \end{enumerate}
  \end{minipage}
  \begin{minipage}{.35\textwidth}
    \pic{1}{collision1.png}
  \end{minipage}

\item A mass traveling in the $+x$ direction collides with a mass at rest. Which
  of the following statements is true?
  \begin{enumerate}[noitemsep,topsep=0pt]
  \item After the collision, the two masses will move with parallel velocities
  \item After the collision, the masses will move with antiparallel velocities
  \item After the collision, the masses will both move along the x-axis
  \item After the collision, the $y$-components of the velocities of the two
    particles will sum to zero.
  \item None of the above
  \end{enumerate}

\item A mass $m_1$ initially moving at speed $v_0$ collides with and sticks to a
  spring attached to a second, initially stationary mass $m_2$. The two masses
  continue to move to the right on a frictionless surface as the length of the
  spring oscillates. At the instant that the spring is maximally extended, the
  velocity of the first mass is
  \begin{center}
    \pic{0.7}{mass-spring-1.png}
  \end{center}
  \begin{enumerate}[noitemsep,topsep=0pt]
  \item $v_0$
  \item $m_1^2v_0/(m_1+m_2)^2$
  \item $m_2v_0/m_1$
  \item $m_1v_0/m_2$
  \item $m_1v_0/(m_1+m_2)$
  \end{enumerate}
\end{enumerate}
\newpage

\textbf{Free-Response Questions}

\begin{enumerate}[leftmargin=15pt]
  \setcounter{enumi}{9}
\item A mass $m$ is placed on an incline of angle $\theta$ at a distance $d$
  from the end of a spring as shown below. The coefficient of kinetic friction
  between the mass and the plane is $\mu$.
  \begin{center}
    \pic{.37}{ramp1.png}
  \end{center}
  \begin{enumerate}[noitemsep]
  \item The mass is released from rest at the position shown. Using Newton's
    laws, calculate the block's speed when it reaches the spring.
  \item Using energy conservation, calculate the block's speed when it reaches
    the spring.
    
  \item The spring has spring constant $k$. At what value $x$ of the compression
    of the spring does the object reach its maximum speed?
  \end{enumerate}

\item A mass $m$ attached to a string of length $2r$ swings, starting at rest
  when the string is horizontal, until the string is vertical. At the instant
  the string is vertical, the mass makes contact with the horizontal surface,
  the string is cut, and the mass continues along a frictionless track as shown
  below.
  \begin{center}
    \vspace{-0.2in}\pic{0.4}{string2.png}
  \end{center}
  \begin{enumerate}[noitemsep]
  \item What is the speed of the mass attached to the string the instant the
    string is cut?
  \item Sketch the forces acting on the mass when it is in the position shown
    below.
  \end{enumerate}
  When the mass is in the position shown below,
  \begin{center}
    \pic{0.25}{circle1.png}
  \end{center}
  \begin{enumerate}[noitemsep]
    \setcounter{enumii}{2}
  \item find the object's speed as a function of $\theta$
  \item find the object's centripetal acceleration as a function of $\theta$
  \item determine at what angle $\theta$ the mass will fall of the track
  \end{enumerate}

\item A projectile is fired from the edge of a cliff \SI{100}{m} high with an
  initial speed of \SI{60}{m/s} at an angle of elevation of \ang{45}.
  \begin{enumerate}[noitemsep]
  \item Write equation for $x(t)$, $y(t)$, $v_x$ and $v_y$. Choose the origin of
    your coordinate system at the particle's original location.
  \item Calculate the location and velocity of the particle at time
    $t=\SI{5}{s}$.
  \end{enumerate}
  Suppose the projectile experiences an internal explosion at time $t=\SI{4}{s}$
  with an internal force purely in the $y$-direction, causing it to break into
  \SI{2}{\kg} and a \SI{1}{\kg} fragment.
  \begin{enumerate}[noitemsep]
    \setcounter{enumii}{2}

  \item If the \SI{2}{\kg} fragment is \SI{77}{m} above the height of the
    cliff at $t=\SI{5}{s}$, what is the $y$-coordinate of the position of the
    \SI{1}{\kg} piece?
  \item If the speed of the \SI{2}{kg} fragment is \SI{46}{m/s} and the
    fragment is falling at $t=\SI{5}{s}$, what is the $y$-component of the
    velocity of the \SI{1}{kg} fragment?
  \end{enumerate}

\item The Ballastic Pendulum. To determine the muzzle speed of a gun, a bullet
  is shot into a mass $M$ from a string as shown below, causing $M$ to swing
  upward through a maximum angle of $\theta$.
  \begin{center}
    \pic{.4}{ballastic.png}
  \end{center}
  \begin{enumerate}[noitemsep]
  \item What is the speed of $M$ the instant after the bullet lodges in it?
  \item What is the speed of the bullet before it hits $M$?
  \item What is the tension in the string at the highest point of the pendulum's
    swing (when the string makes an angle of $\theta$ with the vertical as
    shown)?
  \end{enumerate}
  \newpage
\item Two masses are connected by a spring (spring constant $k$) resting on a
  frictionless horizontal surface as shown. The right mass is initially in
  contact with a wall. A brief blow to the left block leaves it with an initial
  velocity $v_0$ to the right.
  \begin{enumerate}
  \item What is the maximum compression of the spring as the left block moves
    to the right?
  \end{enumerate}
  After the spring is maximally compressed, it eventually moves to the left,
  away from wall. As it moves away from the wall, it continues oscillating.
  \begin{center}
    \pic{.5}{mass-spring-2.png}
  \end{center}
  \begin{enumerate}
    \setcounter{enumii}{1}
  \item What is the net momentum of the two masses after they leave the wall?
  \item What is the total mechanical energy of the oscillating spring system?
  \item What is the relative velocity of the two masses when the spring is
    maximally compressed?
  \item What is the maximum compression of the spring after the two masses have
    left the wall? Compare the compression to the maximum compression calculated
    in part (a) and explain any similarities and differences.
  \end{enumerate}




\end{enumerate}
\end{document}
