\documentclass{../../../oss-handout}
\usepackage{amsmath,bm}
\usepackage{enumitem}
\usepackage{tikz}
\usepackage{siunitx}
\usepackage{titlesec}

\titleformat{\section}{\fontsize{13}{15}\bfseries}{\thesection}{1em}{}
\titlespacing{\section}{0pt}{14pt}{2pt}

\titleformat{\subsection}{\bfseries}{\thesubsection}{1em}{}
\titlespacing{\subsection}{0pt}{6pt}{-3pt}

\newcommand{\iii}{\bm{\hat{\imath}}}
\newcommand{\jjj}{\bm{\hat{\jmath}}}

\tikzset{>=latex}
\setlength{\parindent}{0pt}
\setlength{\parskip}{6pt}
\setlength{\headheight}{26pt}


% Set the page style for the document
\pagestyle{plain}

% Course & handout information
\renewcommand{\institution}{Olympiads School, Toronto, ON, Canada}
\renewcommand{\coursetitle}{Advanced Placement Physics C}
\renewcommand{\term}{Summer 2020}

\title{Projectile Motion}
\author{Dr.\ Timothy Leung}
\date{\today}

\begin{document}
\thispagestyle{title}
\gentitle

\section{General Projectile Motion}
A \textbf{projectile} is an object that is launched with an initial velocity
of $\bm{v}_0$ at an angle $\theta$ above the horizontal. It travels along a
parabolic trajectory and accelerates only due to gravity, as shown in
Figure~\ref{fig:projectile}. 
\begin{figure}[ht]
  \centering
  \begin{tikzpicture}[scale=1.1]
    \draw[->](0,0)--(2,0) node[right]{$x$};
    \draw[->](0,0)--(0,2) node[above]{$y$};
    \draw[dotted,domain=0:6.2,thick] plot (\x, {1.2*\x-.2*\x*\x});
    \draw[very thick,->](0,0)--(.75,.9)node[above]{$\bm{v}_0$};
    \draw[very thick,red!80!black,->](0,0)--(0,.9)node[left]{$v_{y0}$};
    \draw[very thick,blue!80!black,->](0,0)--(.75,0)node[below]{$v_x$};
    \draw[->](.5,0)arc(0:52:.5) node[pos=.65,right]{$\theta$};
  \end{tikzpicture}
  \caption{The parameters defining the motion of a projectile.}
  \label{fig:projectile}
\end{figure}

In general, when solving a projectile motion project, the $x$-axis is defined
as the \emph{horizontal} direction, with the ($+$) direction pointing forward,
while the $y$-axis is the \emph{vertical} direction, with the ($+$) direction
pointing upwards. For simplicity, the origin of the coordinate system is usually
the position where the projectile is launched. This method is consistent with
the standard right-handed Cartesian coordinate system.

The initial velocity $\bm{v}_0$ can be resolved into its $\iii$ and $\jjj$
components, along the $x$ and $y$ axes:%, also shown in Fig.~\ref{sym}:
\begin{equation}
  \bm{v}_0
  =v_x\iii+v_{y0}\jjj
  =v_0\cos\theta\iii + v_0\sin\theta\jjj
\end{equation}
There is no acceleration (i.e.\ $a_x=0$) along $\iii$ direction, therefore
horizontal velocity $v_x$ is constant, and the kinematic equations are reduce
to a single equation:
\begin{equation}
  x=v_xt=\left[v_0\cos\theta\right] t
  \label{eq:x}
\end{equation}
where $x$ is the horizontal position at time $t$, $v_0=|\bm{v}_0|$ is the
magnitude of the initial velocity, $v_x=v_0\cos\theta$ is its horizontal
component.

There is a constant acceleration due to gravity alone along the $\jjj$
direction, i.e.\ $a_y=-g$. (Acceleration is \emph{negative} due to the fact that
the positive $y$-axis points upwards.) The kinematic equations along the
vertical direction are therefore:
\begin{align}
  y &= \left[v_0\sin\theta\right]t-\frac12gt^2\label{eq:y}\\
  v_y &= \left[v_0\sin\theta\right] -gt\\
  v_y^2&=\left[v_0\sin\theta\right]^2-2gy
\end{align}
For most projectile motion problems, Equations~\ref{eq:x} and \ref{eq:y} are
the most important ones, and also most likely to be used for problem solving.

Because $\iii$ and $\jjj$ are orthogonal\footnote{i.e.\ they are linearly
  independent}, horizontal and vertical motions are completely independent of
each other. However, there are variables that are shared in both directions,
namely:
\begin{itemize}[nosep]
\item Time $t$
\item Launch angle $\theta$, measured above the horizontal\footnote{It should
  be obvious that If the projectile is launched below horizontal, then
  $\theta<0$}
\item Initial speed $v_0$
\end{itemize}
When solving any projectile motion problems, there will be two unknowns that
need to be solved (although you may not be explicitly told what one of them
is), requiring two equations (the $x$ and $y$ kinematic equations). In some
rare cases, if an object lands on an incline, there will be a third equation
describing the relationship between $x$ and $y$ of the incline.

\section{Symmetric Trajectory}
A \textbf{symmetric trajectory} is a special case of projectile motion where an
object is launched at an angle of $\theta$ (between \ang{0} and \ang{90}) above
the horizontal\footnote{This may be obvious, but angles \emph{below} the 
  horizontal will never have a symmetric trajectory.} and then lands at the
same height. Examples may include hitting a golf ball toward the hole, or
shooting a bullet toward a horizontal target\footnote{Shooting a bullet toward
  a horizontal target always require an upward angle because of gravity.}. The
equations for symmetric trajectory are \emph{not} included in the AP Exam
equation sheet; if you need these equations during the exams, you will need to
derive them yourself. Thankfully, the derivation is quite striaghtforward.

\textbf{Total time of flight} $t_\text{max}$: We apply the kinematic equation
first in the $y$ direction. When the object lands at the same height, the final
velocity is the same in magnitude and opposite in direction as the initial
velocity, i.e.\  $v_y=-v_{y0}=-v_0\sin\theta$:
\begin{align*}
  v_y &=v_{y0}+a_yt\\
  -v_0\sin\theta &=v_0\sin\theta -g t_\text{max}
\end{align*}
Solving for $t_\text{max}$ we have:
\begin{equation}
  \boxed{
    t_\text{max}=\frac{2v_0\sin\theta}g
  }
  \label{tmax}
\end{equation}
Not surprisingly, a projectile will stay in the air the longest when it is
launched at $\displaystyle\theta=\frac{\pi}2$, or \ang{90}. (As a fun exercise,
for a known initial speed $v_0$, you can plot $t$ vs.\ $\sin\theta$ to find the
acceleration due to gravity $g$!)

\textbf{Maximum height} $H$: Apply the kinematic equation in the $y$-direction.
Recognizing that at maximum height $H=y-y_0$, the vertical component of
velocity is zero $v_y=0$:
\begin{align*}
  v_y^2 &= v_{y0}^2 + 2a_y(y-y_0)\\
  0  &= (v_0\sin\theta)^2-2gH
\end{align*}
Solving for $H$, we get the maximum height equation:
\begin{equation}
  \boxed{H=\frac{v_0^2\sin^2\theta}{2g}}
\end{equation}
The maximum height also (not surprisingly) has a maximum value at
$\displaystyle\theta=\frac{\pi}2$.

\textbf{Range} $R$: We substitute the expression for $t_\text{max}$ from
Equation~\ref{tmax} into the $t$ term, then apply the kinematic equation in
the $x$-direction to compute $R=x-x_0$ for any given launch angle and initial
speed:
\begin{equation}
  R=(x-x_0)=v_xt=v_0\cos\theta\left(\frac{2v_0\sin\theta}{g}\right)
  \label{step1}
\end{equation}
Using the trigonometric identity $\sin(2\theta)=2\sin\theta\cos\theta$, we
simplify Equation~\ref{step1} to:
\begin{equation}
  \boxed{R=\frac{v_0^2\sin(2\theta)}g}
\end{equation}
It is obvious that for any given initial speed $v_0$, the maximum range
$R_\text{max}$ occurs at an angle where $\sin(2\theta)=1$
(i.e.\ $\theta=\pi/4$, or \ang{45}), with a value of
\begin{equation}
  \boxed{R_\text{max}=\frac{v_0^2}g}
\end{equation}
Also, for a known initial speed $v_0$ and range $R$ we can compute the launch
angle $\theta$:
\begin{displaymath}
  \theta_1=\frac{1}{2}\sin^{-1}\left(\frac{gR}{v_0^2}\right)
\end{displaymath}
This angle is labelled $\theta_1$ because it is \emph{not} the only angle that
can reach this range. Recall that for any angle $0<\phi<\pi/2$, there
is also another angle $\phi_2$ where $\sin(\phi)$ is the same:
\begin{displaymath}
  \sin\phi=\sin(\pi-\phi)
\end{displaymath}
Which means that for any $\theta_1$, there is also another angle
$\theta_2$ where $2\theta_2=\pi-2\theta_1$, or quite simply:
\begin{displaymath}
  \theta_2=\frac{\pi}{2}-\theta_1
\end{displaymath}

%These equations are \emph{not} provided in the AP Exam
%  equation sheet, but it can save you a lot of time if you can use them, instead
%  of deriving them during the exam.
%  \begin{itemize}
%  \item Time of flight
%    \eq{-.1in}{t_\text{max}=\frac{2v_0\sin\theta}{g}}
%  \item Range
%    \eq{-.1in}{R=\frac{v_0^2\sin(2\theta)}{g}}
%  \item Maximum height
%    \eq{-.1in}{y_\text{max}=\frac{v_0^2\sin^2\theta}{2g}}
%  \end{itemize}
%  The angle $\theta$ is measured above the the horizontal.
%\end{frame}
%
%
%
%\begin{frame}{Maximum Range}
%  \eq{-.1in}{
%    R=\frac{v_0^2\sin(2\theta)}g
%  }
%  
%  \begin{itemize}
%  \item Maximum range occurs at $\displaystyle\theta=\ang{45}$
%  \item For a given initial speed $v_0$ and range $R$, launch angle $\theta$ is
%    given by:
%    
%    \eq{-.2in}{
%      \theta_1=\frac{1}{2}\sin^{-1}\left(\frac{Rg}{v_0^2}\right)
%    }
%
%    But there is another angle that \emph{gives the same range}!
%
%    \eq{-.2in}{
%      \theta_2=\ang{90}-\theta_1
%    }
%  \end{itemize}
%\end{frame}











%%  \begin{itemize}
%%  \item For 2D problems, resolve the problem into its
%%    horizontal ($x$) and vertical ($y$) directions, and apply kinematic
%%    equations independently
%For projectile motion, there is no acceleration in the horizontal
%($\bm{\hat{\imath}}$) direction, i.e.\ $a_x=0$. Therefore the kinematic
%equations is reduced to a single equation
%\begin{equation}
%  x-x_0=v_xt
%\end{equation}
%It is customary to define the $\bm{\hat{\imath}}$ direction to be the forward
%horizontal direction.
%
%The only acceleration is due to gravity in the vertical ($\bm{\hat{\jmath}}$)
%direction. In the standard Cartesian coordinate system, $\bm{\hat{\jmath}}$ is
%direction is therefore usually \emph{up}, and the acceleration due to gravity is
%\begin{equation}
%  a_y=-g
%\end{equation}
%%\begin{multicols}{2}
%The kinematic equation are reduced to:
%\begin{align}
%  v_y&=v_{y0}-gt\\
%  y&=y_0+v_{y0}t-\frac12 gt^2\label{eq:needthis}\\
%  v_y^2&=v_{y0}^2-2g(y-y_o)
%\end{align}
%In general, only Eq.~\ref{eq:needthis} is needed to solve projectile motion
%problems.
%%\end{multicols}
%%The variable that connects the two directions is time $t$.
%
%\section{Symmetric Trajectory}
%

%equations, we use the $x$-axis for the horizontal direction and $y$-axis for
%the vertical.
%\begin{figure}[ht]
%  \begin{center}
%    \begin{tikzpicture}
%      \draw[->](0,0)--(7,0) node[pos=1,right]{$x$};
%      \draw[->](0,0)--(0,2) node[pos=1,above]{$y$};
%      \draw[dotted,domain=0:6,thick] plot (\x, {1.2*\x-.2*\x*\x});
%      \draw[ultra thick,->](0,0)--(.75,.9)node[pos=1,above]{$\bm{v}_0$};
%      \draw[very thick,red!80!black,->]
%      (0,0)--(0,.9)node[midway,left]{$\bm{v}_{y0}$};
%      \draw[very thick,blue!80!black,->]
%      (0,0)--(.75,0)node[midway,below]{$\bm{v}_x$};
%      \draw[->](.5,0)arc(0:52:.5) node[pos=.6,right]{$\theta$};
%    \end{tikzpicture}
%  \end{center}
%  \vspace{-.2in}
%  \caption{Symmetric project trajectory}
%  \label{sym}
%\end{figure}
%
%The $x$ component of velocity $\bm{v}_x$ remains constant during the motion, as
%there are no forces acting in the $x$ direction (as long as air resistance can
%be ignored), and therefore no acceleration. In the $y$ direction, acceleration
%is due to gravity alone, $a_y=-g$.
%

\end{document}
