\documentclass[11pt]{article}
\usepackage{bm}
\usepackage[margin=1.7cm,top=2.5cm,bottom=2.5cm,letterpaper]{geometry}
\usepackage{enumitem}
\usepackage{tikz,graphicx,wrapfig}
\usepackage{amsmath}
\usepackage{times}
\usepackage[document]{ragged2e}
\usepackage[none]{hyphenat}
\usepackage{siunitx}
\usepackage{multicol}
\usepackage{fancyhdr}

\newcommand{\mb}[1]{\mathbf{#1}}
\newcommand{\iii}{\bm{\hat{\imath}}}
\newcommand{\jjj}{\bm{\hat{\jmath}}}
\newcommand{\pic}[2]{\includegraphics[width=#1\textwidth]{#2}}

\setlength{\parskip}{15pt}

\sisetup{
  %  mode=text,
  detect-all,
  %number-math-rm=\mathnormal,
  %  text-sf=\sffamily,
  %  text-rm=\sffamily,
  inter-unit-product =\ensuremath{\cdot{}},
  per-mode=symbol,
  group-separator={,},
}

\pagestyle{fancy}
\chead{}
\lhead{Olympiads School}
\rhead{Advanced Placement Physics}
\lfoot{Symmetric Trajectory}
\cfoot{-\thepage-}
\rfoot{}


\setlength{\parskip}{15pt}

\begin{document}

\begin{center}
  \vspace{.3in}{\Large\textbf{Symmetric Projectile Trajectory}}
\end{center}

A \textbf{symmetric trajectory} is a special case of projectile motion where an
object is launched at an angle of $\theta$ (between \ang{0} and \ang{90}) above
the horizontal\footnote{This may be obvious, but any angles \emph{below} the 
  horizontal will never have a symmetric trajectory.} with an initial speed
$\mb{v}_0$, and then lands at the same height, as shown below in Fig.~\ref{sym}.
Examples may include hitting a golf ball towards the hole, or shooting a bullet
towards a horizontal target\footnote{Shooting a bullet towards a horizontal
  target always require an upward angle because of gravity.}. The equations for
symmetric trajectory is \emph{not} included in the AP Exams equation sheet; if
you need these equations during the exams, you will need to derive them during
the exam. Thankfully, the derivation is not difficult. To derive the equations,
we use the $x$-axis for the horizontal direction and $y$-axis for the vertical.
\begin{figure}[ht]
  \begin{center}
    \begin{tikzpicture}
      \draw[->](0,0)--(7,0) node[pos=1,right]{$x$};
      \draw[->](0,0)--(0,2) node[pos=1,above]{$y$};
      \draw[dotted,domain=0:6,thick] plot (\x, {1.2*\x-.2*\x*\x});
      \draw[ultra thick,->](0,0)--(.75,.9)node[pos=1,above]{$\mb{v}_0$};
      \draw[very thick,red!80!black,->]
      (0,0)--(0,.9)node[midway,left]{$\mb{v}_{y0}$};
      \draw[very thick,blue!80!black,->]
      (0,0)--(.75,0)node[midway,below]{$\mb{v}_x$};
      \draw[->](.5,0)arc(0:52:.5) node[pos=.6,right]{$\theta$};
    \end{tikzpicture}
  \end{center}
  \vspace{-.2in}
  \caption{Symmetric project trajectory}
  \label{sym}
\end{figure}

The initial velocity $\mb{v}_0$ can be resolved into its $\iii$ and $\jjj$
components, also shown in Figure~\ref{sym}:
\begin{equation}
  \mb{v}_0
  =v_x\iii+v_{y0}\jjj
  =v_0\cos\theta\iii + v_0\sin\theta\jjj
\end{equation}
$\mb{v}_x$ remains constant during the motion, as there are no forces acting in
the $x$ direction (if we can ignore air resistance), and therefore no
acceleration. In the $y$ direction, there is an acceleration due to gravity
$a_y=-g$.

\textbf{Maximum height} $H$: Apply the kinematic equation in the $y$-direction.
Recognizing that at maximum height $H=y-y_0$, the vertical component of
velocity is zero $v_y=0$:
\begin{align*}
  v_y^2 &= v_{y0}^2 + 2a_y(y-y_0)\\
  0  &= (v_0\sin\theta)^2-2gH
\end{align*}
Solving for $H$, we get the maximum height equation:
\begin{equation}
  \boxed{H=\frac{v_0^2\sin^2\theta}{2g}}
\end{equation}
\newpage

\textbf{Total time of flight} $t_\mathrm{max}$: We apply the kinematic equation
in the $y$ direction. When the object lands at the same height, the final
velocity is the same in magnitude and opposite in direction as the initial
velocity, i.e.\  $v_{y2}=-v_{y1}=-v_0\sin\theta$:
\begin{align*}
  v_y &=v_{y0}+a_yt\\
  -v_0\sin\theta &=v_0\sin\theta -g t_\mathrm{max}
\end{align*}
Solving for $t_{\text{max}}$ we have:
\begin{equation}
  \boxed{t_{\text{max}}=\frac{2v_0\sin\theta}{g}}
  \label{tmax}
\end{equation}

\textbf{Range} $R$: We substitute the expression for $t_{\text{max}}$ from
Eq.~\ref{tmax} into the $t$ term, then apply the kinematic equation in
the $x$-direction to compute $R=x-x_0$ for any given launch angle and initial
speed:
\begin{align*}
  x&=x_0+v_xt\\
  R &=v_0\cos\theta\left(\frac{2v_0\sin\theta}{g}\right)
\end{align*}
Using the trigonometric identity $\sin(2\theta)=2\sin\theta\cos\theta$, we
simplify the equation to:
\begin{equation}
  \boxed{R=\frac{v_0^2\sin(2\theta)}{g}}
\end{equation}
It is obvious that for any given initial speed $v_0$, the maximum range
$R_{\text{max}}$ occurs at an angle where $\sin(2\theta)=1$
(i.e.\ $\theta=\pi/4$), with a value of
\begin{equation}
  \boxed{R_\mathrm{max}=\frac{v_0^2}{g}}
\end{equation}
Also, for a known initial speed $v_0$ and range $R$ we can compute the launch
angle $\theta$:
\begin{displaymath}
  \theta_1=\frac{1}{2}\sin^{-1}\left(\frac{gR}{v_0^2}\right)
\end{displaymath}
This angle is labelled $\theta_1$ because it is \emph{not} the only angle that
can reach this range. Recall that for any angle $0<\phi<\pi/2$, there
is also another angle where the $\sin$ are equal:
\begin{displaymath}
  \sin\phi=\sin(\pi-\phi)
\end{displaymath}
Which means that for any $\theta_1$, there is also another angle
$\theta_2$ where $2\theta_2=\pi-2\theta_1$, or quite simply:
\begin{displaymath}
  \theta_2=\frac{\pi}{2}-\theta_1
\end{displaymath}
\end{document}
