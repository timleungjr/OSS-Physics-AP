\documentclass{../../oss-handout}
\usepackage{txfonts}  % must be loaded after amsmath?
\usepackage{enumitem}
\usepackage{titlesec}
\usepackage{siunitx}

\sisetup{
  detect-all,
  per-mode=symbol
}

\setlength{\parindent}{0pt}
\setlength{\parskip}{6pt}
\setlength{\headheight}{26pt}

\newcommand{\mb}[1]{\ensuremath\mathbf{#1}}
\newcommand{\pic}[2]{\includegraphics[width=#1\textwidth]{#2}}

\titleformat*{\section}{\bfseries}

% Set the page style for the document
\pagestyle{plain}

% Course & handout information
\renewcommand{\institution}{Olympiads School}
\renewcommand{\coursetitle}{AP Physics C}
\renewcommand{\term}{Summer 2020}

\title{Advanced Placement Physics C: Course Outline}
\author{Dr.\ Timothy M.\ Leung
  \footnote{Ph.D., M.A.Sc. (Toronto), B.A.Sc. (British Columbia), E-mail:
    \texttt{tleung@olympiadsmail.ca}}
}
\date{\today}

\begin{document}
\thispagestyle{title}
\gentitle

\section*{Class Time}
The Advanced Placement Physics C course at Olympiads School runs for a total of
$40$ hours, on Saturdays and Sundays from 7:00 pm to  9:30 pm
%Saturdays 4:10pm--6:40pm (Starts November 2)
%Class time in September TBA


\section*{Course Pre-requisites}
When taking the course, you should be comfortable and competent in the material
covered in:
\begin{itemize}[noitemsep,topsep=0pt,leftmargin=18pt]
\item\textbf{Physics 11 and 12:} Student will need to be competent in all the
  topics covered in the high-school level physics courses. Many topics from
  Phyiscs 11 and 12 are covered more in-depth in this course.
\item\textbf{Calculus:} Both C exams (Mechanics, and E\&M) are calculus based,
  and students will be required to perform basic differentiation and
  integration.
\item\textbf{Vectors:} Students need to have basic understanding of vector
  operations, including addition and subtraction, as well as dot products and
  cross products.
\end{itemize}


\section*{Course Outline}
The course is divided into two parts, and covers all the topics in the two
calculus-based exams:
\begin{itemize}[noitemsep,topsep=0pt]
\item Mechanics
  \begin{enumerate}[noitemsep,topsep=0pt]
  \item Kinematics
  \item Dynamics
  \item Work and energy
  \item Momentum and collisions
  \item Center of mass
  \item General circular motion
  \item Rotational motion
  \item Harmonic motion (oscillations)
  \item Universal gravitation and planetary motion
  \item\textbf{Practice AP Physics C: Mechanics exam}
  \end{enumerate}
\item Electricity and Magnetism (``E\&M'')
  \begin{enumerate}[noitemsep,topsep=0pt,resume]
  \item Electrostatics
  %\item Gauss's law
  \item Capacitance
  \item Magnetism
  %\item Inductance
  \item Circuit analysis (RC, RL, LC and RLC circuits)
  \item Maxwell's equations and electromagnetic wave
  \item\textbf{Practice AP Physics C: E\&M exam}
  \end{enumerate}
%\item Additional topics in \emph{AP Physics 1} and \emph{AP Physics 2}
%  \begin{enumerate}[itemsep=0pt,leftmargin=18pt]
%  \item Fluid dynamics
%  \item Thermal physics
%  %\item Mechanical waves
%  %\item Light and optics
%  \item Special relativity
%  \item Quantum mechanics
%  \item\textbf{Practice AP Physics 2 Exam}
%  \end{enumerate}
\end{itemize}



\section*{Course Material}
\begin{itemize}[topsep=0pt,noitemsep,leftmargin=18pt]
\item Textbook is not required. However, lecture slides, handouts and any
  additional resources are posted on the school website each week (please check
  and download)
\item Homework questions are generally based on university-level textbook
  problems, and past AP C exams
\item Students are expected to bring the following to each class:
  \begin{itemize}[noitemsep,topsep=0pt]
  \item A pen/pencil for note-taking
  \item Paper/notebook/binder
  \item A scientific calculator for working in-class example problems
  \end{itemize}
\end{itemize}


\section*{Classroom Expectations}
Students attending this course will be expected to:
\begin{itemize}[noitemsep,topsep=0pt,leftmargin=18pt]
  %\item Be in your seat and ready to learn and participate during class
\item Be ready to learn and participate during class
\item Stay on task without disturbing or distracting others
\item Raise your hand if you have any questions or comments and wait to be
  called. Don't wait too long before you ask a question
\item If you need to leave the class early, your parent needs to pick you up at
  the classrom door
\item Be respectful for yourself, others, and the facilities; act in a
  responsible manner in everything you do
\end{itemize}


\subsection{Homework Expectation}
\begin{itemize}[noitemsep,topsep=0pt,leftmargin=18pt]
\item Homework is assigned after every topic is finished (approximately every
  week during the fall/winter session, and twice a week during the summer),
  depending on the length and difficulty of the material for that topic
\item Homework questoins, like the AP exams themselves, will include both
  multiple-choice questions and free-response questions
\item Late homework is always accepted. However, the longer you wait, the less
  meaningful they will be n your learning
  \newpage
\item For free-response questions:
  \begin{itemize}[noitemsep,topsep=0pt,leftmargin=15pt]
  \item Show \emph{all} work by providing complete and organized steps. In all
    AP exams, only part marks are awarded for the correct answer; most of the
    marks are for applying the correct concepts and diligent work.
  %\item Hint: Answer all questions as if the reader is learning the concept
  %  from you, not as if s/he already understands it.
  \item If a question requires you to \emph{explain}, please do so using
    short complete sentences with sufficient supporting detail.
  \item Proper math format must be used, e.g.\ proper use of ``='' sign, units,
    etc.
  \item Circle or box all your final answers.
  \end{itemize}
\item Some of the more difficult questions will be taken up during class.
  However, this does \emph{not} mean you don't need to do your homework at
  home. Always do your best.
\end{itemize}


\section*{Notes}
To help students with getting comfortable with the format of the AP exams, all
effort have been made to ensure that
\begin{itemize}[noitemsep,topsep=0pt,leftmargin=15pt]
\item Whenever possible, the course material uses American spelling (e.g.\
  meter instead of \emph{metre}, color instead of \emph{colour})
\item The homework questions are formatted to look like the actual AP exams
\end{itemize}
\end{document}


