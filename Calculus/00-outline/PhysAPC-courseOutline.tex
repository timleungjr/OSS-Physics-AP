\documentclass{../../oss-handout}
\usepackage[sfdefault,lf]{carlito}
\usepackage{enumitem}
\usepackage{titlesec}

\setlength{\parindent}{0pt}
\setlength{\parskip}{5pt}
\setlength{\headheight}{26pt}

\titleformat*{\section}{\centering\large\bfseries}
\titlespacing*{\section}{0pt}{4pt}{1.5pt}

% Set the page style for the document
\pagestyle{plain}

% Course & handout information
\renewcommand{\institution}{Olympiads School}
\renewcommand{\coursetitle}{AP Physics C}
\renewcommand{\term}{Summer 2021}

\title{Advanced Placement Physics C: Course Outline}
\author{Dr.\ Timothy Leung}%\footnote{
%  Ph.D., M.A.Sc.\ (Toronto), B.A.Sc.\ (British Columbia), E-mail:
%    \texttt{tleung@olympiadsmail.ca}}
%}
\date{\today}

\begin{document}
\thispagestyle{title}
\gentitle

\section*{Class Time}
The Advanced Placement Physics C course at Olympiads School runs for a total of
approximately 50 hours. For the Summer 2021 session, the class times are on
\textbf{Saturdays and Sundays 7:00pm to 9:30pm}.
%There are two sections:
%\begin{enumerate}[nosep]
%\item Saturdays 4:10pm--6:40pm
%\item Sundays 1:20pm--3:50pm
%\end{enumerate}

\section*{Outline}
The course is divided into two parts, and covers all the topics in the two
calculus-based exams:
\begin{itemize}[nosep,leftmargin=15pt]
\item Mechanics
  \begin{enumerate}[nosep]
  \item Kinematics
  \item Dynamics
  \item Work and energy
  \item Momentum, impulse and collisions
  \item Center of mass
  \item General circular motion
  \item Rotational motion
  \item Harmonic motion (oscillations)
  \item Universal gravitation and planetary motion
  \item\textbf{Practice AP Physics C: Mechanics exam}
  \end{enumerate}
\item Electricity and Magnetism (``E\&M'')
  \begin{enumerate}[nosep,resume]
  \item Electrostatics
  \item Capacitance
  \item Magnetism
  \item Inductance
  \item Circuit analysis (RC, RL, LC and RLC circuits)
  \item Maxwell's equations and electromagnetic wave
  \item Hall Effect
  \item\textbf{Practice AP Physics C: E\&M exam}
  \end{enumerate}
\end{itemize}

\section*{Prerequisites}
Before taking the course, you should be comfortable and competent in the
material covered in:
\begin{itemize}[nosep,leftmargin=15pt]
\item\textbf{Grades 11 \& 12 Physics:} Student should be comfortable in all the
  topics covered in the high-school level physics courses. Most topics from
  both Grades 11 and 12 Physics are covered more in-depth in this course.
\item\textbf{Calculus:} Both C exams are calculus based, and students will be
  required to perform basic differentiation and integration, as well as solving
  basic differential equations.
\item\textbf{Vectors:} Students need to have basic understanding of vector
  operations, including addition and subtraction, as well as dot products and
  cross products.
\end{itemize}


\section*{Course Material}
\begin{itemize}[nosep,leftmargin=15pt]
\item Textbook is not required. However, lecture slides, handouts and any
  additional resources are posted on the school website each week (please check
  and download)
\item Homework questions are generally based on university-level textbook
  problems, and past AP C exams. There is a homework problem set for each
  topic, and they are posted on the school website as well as on the online
  platform Classkick
%\item Students are expected to bring the following to each class:
%  \begin{itemize}[nosep]
%  \item A pen/pencil for note-taking
%  \item Paper/notebook/binder
%  \item A scientific calculator for working in-class example problems
%  \end{itemize}
\end{itemize}


\section*{Classroom Expectations}
Students attending this course will be expected to:
\begin{itemize}[nosep,leftmargin=15pt]
  %\item Be in your seat and ready to learn and participate during class
\item Be ready to learn and participate during class
\item Stay on task without disturbing or distracting others
\item For online classes: Post your questions in the Zoom chat window
\item For in-person classes: Raise your hand if you have any questions or
  comments and wait to be called.
\item Don't wait too long before you ask a question
\item If you need to leave the class early, please inform your teacher and the
  school %your parent needs to pick you up at
%  the classroom door
\item Be respectful for yourself and others. You may not be an adult yet, but
  you should know how to act like one%, and the facilities; act in a
  %  responsible manner in everything you do
\item Students who are removed due to bad behaviour will not be allowed back to
  the class.
\end{itemize}


\section*{Homework Expectation}
Homework is assigned after every topic is finished (approximately every
week during the fall/winter session, and twice a week during the summer),
depending on the length and difficulty of the material for that topic
\begin{itemize}[nosep,leftmargin=15pt]
\item Homework questions, like the AP exams themselves, will include both
  multiple-choice questions and free-response questions
\item Late homework is always accepted. However, the longer you wait, the less
  meaningful they will be in your learning
\item For free-response questions:
  \begin{itemize}[nosep,leftmargin=15pt]
  \item Show \emph{all} work by providing complete and organized steps. In AP
    exams, only part marks are awarded for the correct answer; most of the
    marks are for applying the correct concepts and diligent work. (Hint:
    Answer all questions as if the reader is learning the concept from you, not
    as if they already understands it.)
  \item If a question requires you to \emph{explain}, please do so using
    short short complete sentences with sufficient supporting detail.
  \item Proper math format should be used, e.g.\ proper use of ``='' sign,
    units, etc.
  \item Circle or box all your final answers.
  \end{itemize}
\item Due to time constraints, the homework questions will be taken up in
  separate videos outside of class time. The links to the videos will be posted
  on the school website.
%\item Some of the more difficult questions will be taken up during class.
%  However, this does \emph{not} mean you don't need to do your homework at
%  home. Always do your best.
\end{itemize}


\section*{Additional Notes}
To help students with getting comfortable with the format of the AP exams, all
effort have been made to ensure that
\begin{itemize}[nosep,leftmargin=15pt]
\item Whenever possible, the course material uses American spelling (e.g.\
  meter instead of \emph{metre}, color instead of \emph{colour})
\item The homework questions are formatted to look like the actual in-person
  AP exams
\end{itemize}
\end{document}


