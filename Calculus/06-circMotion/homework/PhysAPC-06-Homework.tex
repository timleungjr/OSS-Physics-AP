\documentclass{../../../oss-apphys}

\begin{document}
\genheader

\gentitle{C}{ROTATIONAL MOTION}

\genmultidirections

\gengravity

\raggedcolumns
\begin{multicols}{2}
  \begin{enumerate}[leftmargin=18pt]

  \item A girl stands on a rotating merry-go-round without holding on to a rail.
    The force that keeps her moving in a circle is the
    \begin{enumerate}[noitemsep,topsep=0pt,leftmargin=18pt,label=(\Alph*)]
    \item frictional force on the girl directed away from the center of the
      merry-go-round
    \item frictional force on the girl directed toward the center of the
      merry-go-round
    \item normal force on the girl directed away from the center of the
      merry-go-round
    \item normal force on the girl directed toward the center of the
      merry-go-round
    \item weight of the girl
    \end{enumerate}

    % This problem needs some fixing. None of the answers are correct
  \item A \SI{.5}{\kilo\gram} ball on the end of a \SI{.5}{\metre} long string
    is swung in a horizontal circle. What would the speed of the ball have to
    be for the tension in the string to be \SI{9.}{\newton}?
    \begin{center}
      \vspace{-.1in}
      \pic{.4}{ball-horizontal1.png}
    \end{center}
    \begin{enumerate}[noitemsep,topsep=0pt,leftmargin=18pt,label=(\Alph*)]
    \item\SI{1}{\metre\per\second}
    \item\SI{3}{\metre\per\second}
    \item\SI{6}{\metre\per\second}
    \item\SI{9}{\metre\per\second}
    \item\SI{12}{\metre\per\second}
    \end{enumerate}
    \columnbreak
    
  \item A ball of mass $m$ is swung in a vertical circle of radius $R$. The
    speed of the ball at the bottom of the circle is $v$. The tension in the
    string at the bottom of the circle is
    \begin{enumerate}[noitemsep,topsep=0pt,leftmargin=18pt,label=(\Alph*)]
    \item $\displaystyle mg$
    \item $\displaystyle mg+\frac{mv^2}{R}$
    \item $\displaystyle mg-\frac{mv^2}{R}$
    \item $\displaystyle \frac{mv^2}{R}$
    \item zero
    \end{enumerate}
%    \columnbreak
    
  \item A car of mass $m$ drives on a flat circular track of radius $R$. To
    maintain a constant speed $v$ on the track, the coefficient of friction
    $\mu$ between the tires and the road must be
    \begin{enumerate}[noitemsep,topsep=0pt,leftmargin=18pt,label=(\Alph*)]
    \item $\displaystyle mg$
    \item $\displaystyle mg+\frac{mv^2}{R}$
    \item $\displaystyle mg-\frac{mv^2}{R}$
    \item $\displaystyle \frac{v^2}{gR}$
    \item $\displaystyle \sqrt{\frac{v^2}{gR}}$
    \end{enumerate}
    
  \item A ball on the end of a string is swung in a circle of radius
    \SI{2}{\metre} according to the equation $\theta = 4t^2+3t$, where $\theta$
    is in radians and $t$ is in seconds. The angular acceleration of the ball
    is
    \begin{enumerate}[noitemsep,topsep=0pt,leftmargin=18pt,label=(\Alph*)]
    \item\SI{6}{rad\per\second^2}
    \item $4t^2 + 3t$ \si{rad\per\second^2}
    \item $8t +3$ \si{rad\per\second^2}
    \item $\displaystyle\frac{3}{4} t^3 + 3t^2$ \si{rad\per\second^2}
    \item \SI{8}{rad\per\second^2}
    \end{enumerate}
    \columnbreak
    
  \item The linear speed $v$ of the ball (in the previous question) at
    $t=\SI{3}{\second}$ is
    \begin{enumerate}[noitemsep,topsep=0pt,leftmargin=18pt,label=(\Alph*)]
    \item\SI{27 }{\metre\per\second}
    \item\SI{54 }{\metre\per\second}
    \item\SI{108}{\metre\per\second}
    \item\SI{135}{\metre\per\second}
    \item\SI{210}{\metre\per\second}
    \end{enumerate}
  
  \item Two wheels are attached to each other and fixed so that they can only
    turn together. The smaller wheel has a radius of $r$ and the larger wheel
    has a radius of $3r$. The two wheels can rotate together on a frictionless
    axle. Three forces act tangentially on the edge of the wheels as shown.
    The magnitude of the net torque acting on the system of wheels is
    \begin{center}
      \pic{.25}{2wheels.png}
    \end{center}
    \begin{enumerate}[noitemsep,topsep=0pt,leftmargin=18pt,label=(\Alph*)]
    \item$Fr$
    \item$2Fr$
    \item$3Fr$
    \item$4Fr$
    \item$6Fr$
    \end{enumerate}
    \columnbreak
    
  \item A belt is wrapped around two wheels as shown. The smaller wheel has
    a radius $r$, and the larger wheel has a radius $2r$. When the wheels turn,
    the belt does not slip on the wheels, and gives the smaller wheel an
    angular speed $\omega$. The angular speed of the larger wheel is
    \begin{center}
      \pic{.3}{wheels.png}
    \end{center}
    \begin{enumerate}[noitemsep,topsep=0pt,leftmargin=18pt,label=(\Alph*)]
    \item $\displaystyle \omega$
    \item $\displaystyle 2\omega$
    \item $\displaystyle \frac12\omega$
    \item $\displaystyle \frac14\omega$
    \item $\displaystyle 4\omega$
    \end{enumerate}
  \end{enumerate}
\end{multicols}

%\newpage
%\begin{center}
%  {\Large
%    \textbf{AP\textsuperscript{\textregistered} Physics 1 \&C: Circular Motion\\
%      Student Answer Sheet for Multiple-Choice Section}
%  }
%  
%%  \begin{minipage}[t]{.3\textwidth}
%  \vspace{.2in}
%  \bgroup
%  \begin{tabular}{>{\centering}m{1.3cm} >{\centering}m{1.7cm}}
%    No. & Answer
%  \end{tabular}\\
%  \def\arraystretch{1.5}
%  \begin{tabular}{|>{\centering}m{1.3cm}|>{\centering}m{1.7cm}|}
%    \hline
%    1 & \\ \hline
%    2 & \\ \hline
%    3 & \\ \hline
%    4 & \\ \hline
%    5 & \\ \hline
%    6 & \\ \hline
%    7 & \\ \hline
%    8 & \\ \hline
%    9 & \\ \hline
%    10 & \\ \hline
%    11 & \\ \hline
%    12 & \\ \hline
%    13 & \\ \hline
%    14 & \\ \hline
%    15 & \\ \hline
%    16 & \\ \hline
%    17 & \\ \hline
%    18 & \\ \hline
%    19 & \\ \hline
%    20 & \\ \hline
%    21 & \\ \hline
%    22 & \\ \hline
%%    23 & \\ \hline
%%    24 & \\ \hline
%%    25 & \\ \hline
%  \end{tabular}
%  \egroup
%%  \end{minipage}
%%  \begin{minipage}[t]{.3\textwidth}
%%  \vspace{.2in}
%%  \bgroup
%%  \begin{tabular}{>{\centering}m{1.3cm} >{\centering}m{1.7cm}}
%%    No. & Answer
%%  \end{tabular}\\
%%  \def\arraystretch{1.5}
%%  \begin{tabular}{|>{\centering}m{1.3cm}|>{\centering}m{1.7cm}|}
%%    \hline
%%    26 & \\ \hline
%%    27 & \\ \hline
%%    28 & \\ \hline
%%    29 & \\ \hline
%%    30 & \\ \hline
%%    31 & \\ \hline
%%    32 & \\ \hline
%%    33 & \\ \hline
%%    34 & \\ \hline
%%    35 & \\ \hline
%%    36 & \\ \hline
%%    37 & \\ \hline
%%    38 & \\ \hline
%%  \end{tabular}
%%  \egroup
%%  \end{minipage}
%\end{center}
\newpage

\genfreetitle{1 \& C}{CIRCULAR MOTION AND SIMPLE HARMONIC MOTION}{6}

\genfreedirections{10}

% TAKEN FROM THE 2014 AP PHYSICS C FREE-RESPONSE QUESTION MECH 2
\begin{center}
  \pic{.7}{ramp-circle}
\end{center}
\begin{enumerate}[leftmargin=15pt]
\item A small block of mass $m$ starts from rest at the top of a frictionless
  ramp, which is at a height $h$ above a horizontal tabletop, as shown in the
  side view above. The block slides down the smooth ramp and reaches point $P$
  with a speed $v_0$. After the block reaches point $P$ at the bottom of the
  ramp, it slides on the tabletop guided by a circular vertical wall with
  radius $R$, as shown in the top view. The tabletop has negligible friction,
  and the coefficient of kinetic friction between the block and the circular
  wall is $\mu$.
  \begin{enumerate}[leftmargin=15pt]
  \item Derive an expression for the height of the ramp $h$. Express your
    answer in terms of $u_0$, $m$, and fundamental constants, as appropriate.
    \vspace{.75in}
  \end{enumerate}
  A short time after passing point $P$, the block is in contact with the wall
  and moves with a speed of $u$.
  \begin{enumerate}[leftmargin=15pt,resume]
  \item
    \begin{enumerate}[leftmargin=15pt]
    \item Is the vertical component of the net force on the block upward,
      downward, or zero?

      \vspace{.1in}
      \underline{\hspace{.3in}} Upward\hspace{.2in}
      \underline{\hspace{.3in}} Downward\hspace{.2in}
      \underline{\hspace{.3in}} Zero
      
      \vspace{.1in}Justify your answer.\vspace{.4in}

    \item On the figure below, draw an arrow starting on the block to indicate
      the direction of the horizontal component of the net force on the moving
      block when it is at the position shown.
      \begin{center}
        \pic{.25}{ramp-circle-top}
      \end{center}
    \end{enumerate}
  \end{enumerate}
  \newpage
  Express your answers to the following in terms of $v_0$, $v$, $m$, $R$, $m$,
  and fundamental constants, as appropriate.
  \begin{enumerate}[leftmargin=15pt,resume]
  \item Determine an expression for the magnitude of the normal force $N$
    exerted on the block by the circular wall as a function of $v$.
    \vspace{1in}
  \item Derive an expression for the magnitude of the tangential acceleration
    of the block at the instant the block has attained a speed of $v$.
    \vspace{1in}
  \item Derive an expression for $v(t)$, the speed of the block as a function
    of time $t$ after passing point $P$ on the track.
  \end{enumerate}
%  its end, as shown in the figure above.
%  \begin{enumerate}[leftmargin=15pt]
%  \item Using integral calculus, derive the rotational inertia for the rod
%    around its end to show that it is $ML^2/3$.
%  \end{enumerate}
%  \begin{center}
%    \pic{.4}{thinrod2}
%  \end{center}
%  The rod is fixed at one end and allowed to fall from the horizontal
%  position $A$ through the vertical position $B$.
%  \begin{enumerate}[leftmargin=15pt,resume]
%  \item Derive an expression for the velocity of the free end of the rod at
%    position $B$. Express your answer in terms of $M$, $L$, and physical
%    constants, as appropriate.
%  \end{enumerate}
%  \newpage
%  An experiment is designed to test the validity of the expression found in
%  part (b). A student uses rods of various lengths that all have a uniform
%  mass distribution. The student releases each of the rods from the horizontal
%  position $A$ and uses photogates to measure the velocity of the free end at
%  position $B$. The data are recorded below.
%  \begin{center}
%    \def\arraystretch{1.45}
%    \begin{tabular}{|c|c|c|c|c|c|c|}
%      \hline
%      Length (m)     & 0.25 & 0.50 & 0.75 & 1.00 & 1.25 & 1.50\\\hline
%      Velocity (m/s) & 2.7  & 3.8  & 4.6  & 5.2  & 5.8  & 6.3 \\\hline
%      & & & & & & \\\hline
%      & & & & & & \\\hline
%    \end{tabular}
%    \def\arraystretch{1}
%  \end{center}
%
%  \begin{enumerate}[leftmargin=15pt,resume]
%  \item Indicate below which quantities should be graphed to yield a straight
%    line whose slope could be used to calculate a numerical value for the
%    acceleration due to gravity $g$.
%
%    \vspace{.1in}Horizontal axis: \underline{\hspace{1in}}
%
%    \vspace{.1in}Vertical axis: \underline{\hspace{1in}}
%
%    \vspace{.1in}Use the remaining rows in the table above, as needed, to
%    record any quantities that you indicated that are not given. Label each row
%    you use and include units.
%
%  \item Plot the straight line data points on the grid below. Clearly scale and
%    label all axes, including units as appropriate. Draw a straight line that
%    best represents the data.
%    \begin{center}
%      \pic{.9}{graph-paper}
%    \end{center}
%  \item
%    \begin{enumerate}[leftmargin=15pt]
%    \item Using your straight line, determine an experimental value for $g$.
%    \item Describe two ways in which the effects of air resistance could be
%      reduced.
%    \end{enumerate}
%  \end{enumerate}
%  \newpage
%
%\item A uniform sphere of mass $M$ and radius $R$ is free to rotate, without
%  friction, about a horizontal axis through its center. A string is wrapped
%  around the sphere and is attached to a body of mass $m$ as shown in the
%  figure below. Find
%  \begin{center}
%    \vspace{-.2in}
%    \begin{tikzpicture}[scale=.6]
%      \tikzstyle{balloon}=[ball color=red!80!gray!50];
%      \shade[balloon] (0,0) circle (2);% node[below right]{$m$};
%      \draw(0,0) circle(2);
%      \draw[fill=gray](0,0) circle(.2);
%      \draw[thick](2,0)--(2,-4);
%      \draw[fill=yellow!80!gray!50](1.5,-4) rectangle(2.5,-5)
%      node[midway,black]{$m$};
%      \draw[ultra thick,red!80!black,->](2,-4)--(2,-1.8)
%      node[pos=.9,right,black]{$T$};
%    \end{tikzpicture}
%  \end{center}
%  \begin{enumerate}[itemsep=1in,leftmargin=15pt]
%  \item the acceleration of the body, and
%  \item the tension in the string.
%  \end{enumerate}
%  \vspace{1in}
%
%\item A uniform cylinder of mass $M$ and radius $R$ has a string wrapped around
%  it. The string is held fixed, and the cylinder falls vertically as shown in
%  the figure below. Find
%  \begin{center}
%    \vspace{-.3in}
%    \begin{tikzpicture}[scale=.5]
%      \draw[fill=yellow!80!gray!50](0,0) circle(2);
%      \draw[thick](2,0)--(2,5);
%%      \draw[fill=yellow!80!gray!50](1.5,-4) rectangle(2.5,-5)
%%      node[midway,black]{$m$};
%      \draw[ultra thick,red!80!black,->](2,0)--(2,3) node[pos=.9,right]{$T$};
%    \end{tikzpicture}
%  \end{center}
%  \begin{enumerate}[itemsep=1in,leftmargin=15pt]
%  \item\vspace{-.3in} the acceleration of the body, and
%  \item the tension in the string.
%  \end{enumerate}
%  \newpage
%
%  \begin{center}
%    \pic{.65}{disc1}
%    \end{center}
%\item A large circular disk of mass $m$ and radius $R$ is initially stationary
%  on a horizontal icy surface. A person of mass $m/2$ stands on the edge of the
%  disk. Without slipping on the disk, the person throws a large stone of mass
%  $m/20$ horizontally at initial speed $v_0$ from a height $h$ above the ice in
%  a radial direction, as shown in the figures above. The coefficient of
%  friction between the disk and the ice is $\mu$. All velocities are measured
%  relative to the ground. The time it takes to throw the stone is negligible.
%  Express all algebraic answers in terms of $m$, $R$, $v_0$, $h$, $m$, and
%  fundamental constants, as appropriate.
%  \begin{enumerate}[leftmargin=15pt]
%  \item Derive an expression for the length of time it will take the stone to
%    strike the ice.
%  \item Assuming that the disk is free to slide on the ice, derive an
%    expression for the speed of the disk and person immediately after the stone
%    is thrown.
%  \item Derive an expression for the time it will take the disk to stop sliding.
%  \end{enumerate}
%  \newpage
%  \begin{center}
%    \pic{.25}{disc2}
%  \end{center}
%  The person now stands on a similar disk of mass $m$ and radius $R$ that has a
%  fixed pole through its center so that it can only rotate on the ice. The
%  person throws the same stone horizontally in a tangential direction at
%  initial speed $u_0$, as shown in the figure above. The rotational inertia of
%  the disk is $mR^2/2$.
%  \begin{enumerate}[leftmargin=15pt,resume]
%  \item Derive an expression for the angular speed $\omega$ of the disk
%    immediately after the stone is thrown.
%    \vspace{1.5in}
%  \item The person now stands on the disk at rest R 2 from the center of the
%    disk. The person now throws the stone horizontally with a speed $v_0$ in
%    the same direction as in part (d). Is the angular speed of the disk
%    immediately after throwing the stone from this new position greater than,
%    less than, or equal to the angular speed found in part (d) ?
%
%    \vspace{.1in}
%    \underline{\hspace{.3in}} Greater than\hspace{.2in}
%    \underline{\hspace{.3in}} Less than\hspace{.2in}
%    \underline{\hspace{.3in}} Equal to
%    
%    \vspace{.1in}Justify your answer.
%  \end{enumerate}
%  \newpage
%  
%\item A uniform ball of radius $r$ rolls without slipping along the
%  loop-the-loop track in the figure below. The ball starts at rest at a height
%  of $h$ above the bottom of the loop.
%  \begin{center}
%    \pic{.55}{roll-ball.png}
%  \end{center}
%  \begin{enumerate}[itemsep=1.7in,topsep=0pt,leftmargin=15pt]
%  \item\vspace{-.3in}If it is not to leave the track at the top of the loop,
%    what is the least value $h$ can have (in terms of radius $R$ of the loop)?
%  \item What would $h$ have to be if, instead of rolling, the ball slides
%    without friction?
%  \end{enumerate}
%  
%\item A mass $m$ is hung on a string that is wrapped around a disk of radius
%  $R$ and rotational inertia $I$. The mass is released from rest and
%  accelerates downward with an acceleration $a$.
%  \begin{enumerate}[noitemsep]
%  \item Determine the tension in the string as the mass accelerates downward
%    in terms of the given quantities.
%  \item In terms of the tension $T$ and the other given quantities, determine
%    the rate of change of the angular speed of the disk.
%  \end{enumerate}
%  \pic{.3}{mass-disk.png}
%  \vspace{.5in}
%  
%\item A disk having a rotational inertia of \SI{2.}{\kilo\gram.\metre^2}
%  rotates about a fixed axis through its center. The disk begins from rest at
%  $t=0$, and at time $t=\SI{2}{\second}$, its angular velocity is \SI{2}{rad\per\second}.
%  \begin{enumerate}[noitemsep]
%  \item Determine the angular momentum of the disk at $t=\SI{2}{\second}$.
%  \item What is the angular acceleration of the disk between $t=0$ and
%    $t=\SI{2}{\second}$?
%  \item What is the kinetic energy of the disk at $t=\SI{2}{\second}$?
%  \end{enumerate}
%  \pic{.3}{rotDisk.png}
%  \newpage
%\item A mass $m$ oscillates on an ideal spring of spring constant $k$ on a
%  frictionless horizontal surface. The mass is pulled aside to a distance $A$
%  from its equilibrium position, and released.
%  \begin{center}
%    \begin{tikzpicture}
%      \fill [pattern=north east lines] (5,0)--(0,0)--(0,2)--(-0.2,2)
%      --(-0.2,-0.2)--(5,-0.2)--cycle;
%      \draw[ultra thick] (5,0)--(0,0)--(0,2)--(-0.5,2);
%      \draw[decoration={aspect=0.3,segment length=2mm, amplitude=2.5mm, coil},
%        decorate] (0,0.5)--(1.5,0.5);
%      \draw[fill=gray!70](1.5,0) rectangle(2.5,1);
%      \draw[thick](2.5,0)--(2.5,-0.3) node[pos=1,below]{O};
%      \draw[thick](4,0)--(4,-0.3) node[pos=1,below]{A};
%    \end{tikzpicture}
%  \end{center} 
%  \begin{enumerate}[noitemsep]  
%  \item In terms of the given quantities, at what distance from the equilibrium
%    position is the potential energy of the mass equal to its kinetic energy?
%  \item In terms of the given quantities, what is the acceleration of the mass
%    when it is at the amplitude $A$?
%  \end{enumerate}
%%  \vspace{3in}
%  \newpage
%  
%\item Show that for the situations in the figures below, the object of mass
%  $m$ oscillates with a frequency of
%  $\displaystyle f=\frac{1}{2\pi}\sqrt{\frac{k_\mathrm{eff}}{m}}$
%  where $k_\mathrm{eff}$ is given by (a) $k_\mathrm{eff}=k_1+k_2$ and (b)
%  $\displaystyle\frac{1}{k_\mathrm{eff}}=\frac{1}{k_1}+\frac{1}{k_2}$. Hint:
%  find the net force on the mass and write $F=-k_\mathrm{eff}x$. Note that in
%  (b), the springs stretch by different amounts, the sum of which is $x$.
%  
%  (a)\hspace{5pt}\begin{tikzpicture}[scale=.5]
%    \fill[gray!50](0,0) rectangle(12,-.75);
%    \fill[gray!50](-.75,-.75) rectangle(0,3);
%    \fill[gray!50](12,-.75) rectangle(12.75,3);
%    \fill[yellow!80!gray](5.25,0) rectangle(6.75,1.5) node[midway,black]{$m$};
%    \draw[decoration={aspect=0.3,segment length=2mm, amplitude=1.25mm, coil},
%      decorate] (0,.75)--(5.25,.75) node[midway,above]{$k_1$};
%    \draw[decoration={aspect=0.3,segment length=2mm, amplitude=1.25mm, coil},
%      decorate] (6.75,.75)--(12,.75) node[midway,above]{$k_2$};
%    \draw[very thick](0,3)--(0,0)--(12,0)--(12,3);
%  \end{tikzpicture}
%
%  (b)\hspace{5pt}\begin{tikzpicture}[scale=.5]
%    \fill[gray!50](0,0) rectangle(12.75,-.75);
%    \fill[gray!50](-.75,-.75) rectangle(0,3);
%    \fill[yellow!80!gray](10.5,0) rectangle(12,1.5) node[midway,black]{$m$};
%    \draw[decoration={aspect=0.3,segment length=2mm, amplitude=1.25mm, coil},
%      decorate] (0,.75)--(6,.75) node[midway,above]{$k_1$};
%    \draw[decoration={aspect=0.3,segment length=2mm, amplitude=1.25mm, coil},
%      decorate] (6,.75)--(10.5,.75) node[midway,above]{$k_2$};
%    \fill[black](6,.75) circle(.15);
%    \draw[very thick](0,3)--(0,0)--(12.75,0);
%  \end{tikzpicture}
%  \vspace{3in}
%  
%\item A simple pendulum of length $L$ is released from rest from an angle of
%  $\theta_0$.
%  \begin{enumerate}[itemsep=.8in,topsep=0pt,leftmargin=15pt]
%  \item Assuming the motion of the pendulum to be simple harmonic motion, find
%    its speed as it passes through $\theta=0$.
%  \item Using the conservation of energy, find this speed exactly.
%  \item Show that your results for (a) and (b) are the same when $\theta_0$ is
%    small.
%  \item Find the difference in your results for $\theta_0=\SI{.20}{rad}$ and
%    $L=\SI{1}{\metre}$.
%  \end{enumerate}
% 
\end{enumerate}
\end{document}
