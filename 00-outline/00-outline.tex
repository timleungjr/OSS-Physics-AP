\documentclass[11pt]{article}
%\usepackage[left=.9in,right=.9in,top=0.35in,bottom=1in]{geometry}
\usepackage[margin=.7in,letterpaper]{geometry}

\usepackage{times}
\usepackage{enumitem}
\usepackage{titlesec}
\renewcommand{\familydefault}{\sfdefault}

\titleformat*{\section}{\large\bfseries}

\title{\vspace{-0.2in}\textbf{Advanced Placement Physics}}
\author{Instructor: Dr.\ Timothy Leung\\
  E-mail: \texttt{tleung@olympiadsmail.ca}}
\date{Fall 2018}

\begin{document}
\maketitle

\subsection*{Class Time}
\begin{itemize}[itemsep=.1em,leftmargin=12pt] %[noitemsep,topsep=0pt]
\item Saturdays 4:20pm--6:50pm (Starts November 3, 2018)
%Saturdays \& Sundays 7:00--10:00pm (July and August)\\
%Class time in September TBA
\end{itemize}

\subsection*{Course Material}
\begin{itemize}[itemsep=.1em,leftmargin=12pt] %[noitemsep,topsep=0pt]
\item No textbook required
\item Course outline, presentation slides, and homework assignments are
  downloadable from school website
\item Please bring
  \begin{itemize}[noitemsep,topsep=0pt]
  \item A pen/pencil for note-taking
  \item A scientific calculator for working in-class example problems
  \end{itemize}
\end{itemize}

\subsection*{Pre-requisites}
  \begin{itemize}[itemsep=.1em,leftmargin=12pt]
  \item\textbf{Physics 11 and 12:} Student will need to be competent in all the
    topics covered in the high-school level courses. Many topics from Phyiscs
    11 and 12 are covered more in-depth in this course.
    covered
  \item\textbf{Calculus:} The two ``C'' exams are calculus based, and students
    are required to perform basic differentiation and integration.
  \item\textbf{Vectors:} Students need to have basic understanding of vector
    operations, including addition and subtraction, as well as dot products and
    cross products.
  \end{itemize}
\subsection*{Course Outline}
%The course covers all topics in AP Physics 1, 2, and C exams.
%\begin{center}
%  \begin{tabular}{|c|l|c|}
%    \textbf{Class} & \textbf{Topic} & Exams \\ \hline
%    1 & Kinematics using calculus & 1 \& C (Mechanics)\\
%    2 & Dynamics & 1 \& C (Mechanics)\\
%    3,4 & Momentum, impulse and energy &  1 \& C (Mechanics)\\
%    4,5  & Center of mass & 1 \& C (Mechanics)\\
%    5,6 & General circular motion &  1 \& C (Mechanics)\\
%    7,8 & Universal gravitation, Kepler's laws and planetary motion &
%  1 \& C (Mechanics)\\
%  9 & Practice Test \# 1 (AP Physics C: Mechanics) & \\ \hline
%  10 & Electrostatics & 2 \& C (Electricity \& Magnetism)\\
%  11 & Gauss's law \& capacitors & 2 \& C (Electricity \& Magnetism)\\
%  12,13 & Circuits analysis with resistors and capicators &
%  2 \& C (Electricity \& Magnetism)\\
%  14,15 & Magnetism & 2 \& C (Electricity \& Magnetism)\\
%  16 & Maxwell's laws of electromagnetism & 2 \& C (Electricity \& Magnetism)\\
%  17 & Practice Test \# 2 (AP Physics C: Electricity and Magnetism) &\\
%  \hline
%  18
%  \end{tabular}
%\end{center}
\begin{enumerate}[itemsep=.1ex,leftmargin=15pt]
\item Topics in \emph{AP Physics C: Mechanics}
  \begin{enumerate}[itemsep=0pt,leftmargin=18pt]
  \item Kinematics
  \item Dynamics
  \item Momentum, impulse and energy
  \item Center of mass
  \item General circular motion and angular momentum
  \item Simple harmonic motion)--general equation of oscillatory systems,
    pendulums and spring-mass systems
  \item Universal gravitation and planetary motion
  \item\textbf{Practice AP Physics C: Mechanics exam}
  \end{enumerate}
\item Topics in \emph{AP Physics C: Electricity and Magnetism} (``E\&M'')
  \begin{enumerate}[itemsep=0pt,leftmargin=18pt]
  \item Electrostatics
  \item Gauss's law
  \item Capacitance
  \item Magnetism
  \item Inductance
  \item Circuit analysis (RC, RL, LC and RLC circuits)
  \item Maxwell's equations and electromagnetic wave
  \item\textbf{Practice AP Physics C: E\&M exam}
  \end{enumerate}
\item Additional topics in \emph{AP Physics 1} and \emph{AP Physics 2}
  \begin{enumerate}[itemsep=0pt,leftmargin=18pt]
  \item Fluid dynamics
  \item Thermal physics
  \item Mechanical waves
  \item Light and optics
  \item Special relativity
  \item Quantum mechanics
  \item\textbf{Practice AP Physics 2 Exam}
  \end{enumerate}
\end{enumerate}
\end{document}


