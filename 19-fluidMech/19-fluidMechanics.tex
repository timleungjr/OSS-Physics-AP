\documentclass[12pt,aspectratio=169]{beamer}

\mode<presentation>
{
  \usetheme{Singapore}
 %\setbeamersize{text margin left=.6cm,text margin right=.6cm}
%  \setbeamertemplate{navigation symbols}{} % suppress nav bar
%  \setbeamercovered{transparent}
}
\usefonttheme{professionalfonts}
\usepackage{graphicx}
\usepackage{tikz}
\usepackage{amsmath}
\usepackage{mathpazo}
\usepackage[scaled]{helvet}
\usepackage{xcolor,colortbl}
\usepackage{siunitx}
\usepackage{hyperref}

\sisetup{detect-all}

\title{Classes 19: Fluid Mechanics}
\subtitle{AP Physics}
\author[TML]{Dr.\ Timothy Leung}
\institute{Olympiads School}
\date{March 2018}

\newcommand{\pic}[2]{\includegraphics[width=#1\textwidth]{#2}}
\newcommand{\mb}[1]{\mathbf{#1}}
\newcommand{\eq}[2]{\vspace{#1}{\Large\begin{displaymath}#2\end{displaymath}}}
%\newcommand{\protip}[1]{
%  \begin{center}
%    \fbox{
%      \begin{minipage}{.95\textwidth}
%        {\footnotesize
%          \textbf{Protip: }#1
%        }
%      \end{minipage}
%    }
%  \end{center}
%}

\begin{document}

\begin{frame}
  \maketitle
\end{frame}


%\section[Intro]{Introduction}

\begin{frame}
  \frametitle{Files for You to Download}
  Download from the school website:
  \begin{enumerate}
  \item\texttt{19-fluidMechanics.pdf}---This
    presentation. If you want to print the slides on paper, I recommend
    printing 4 slides per page.
  \item\texttt{20-Homework.pdf}---Homework assignment for Classes 19 and 20,
    which cover Fluid Mechanics and Thermodynamics
  \end{enumerate}

  \vspace{.2in}Please download/print the PDF file before each class. When you
  are taking notes, pay particular attention to things I say that aren't
  necessarily on the slides.
\end{frame}


\begin{frame}
  \frametitle{Disclaimer}
  \framesubtitle{Use of Calculus}
  Fluid mechanics is part of the AP Physics 2 Exam, which does not require
  calculus. However, in the interest in completeness, \emph{some} calculus will
  still be used when deriving equations.
\end{frame}


\begin{frame}
  \frametitle{What is a Fluid}

  \begin{itemize}
  \item\textbf{The simple explanstion:} anything that flows, which covers
    most \emph{gases} and \emph{liquids}
  \item\textbf{The scientific explanation:} Any substancs that deform
    \emph{continuously} under oblique stress
  \end{itemize}
\end{frame}

\begin{frame}
  \frametitle{Properties of Fluids}
  \framesubtitle{Density}
\end{frame}


\begin{frame}
  \frametitle{Continuity}
  A fluid is considered to be continuous in space.
\end{frame}


\begin{frame}
  \frametitle{Properties of Fluids}
  \framesubtitle{Viscosity}
\end{frame}


\begin{frame}
  \frametitle{Hydrostatics}
\end{frame}


\begin{frame}
  \frametitle{Buoyancy}
  \framesubtitle{Everything Floats a Little}
  When an object is submerged inside a fluid (e.g.\ water, air, etc), the fluid
  exerts a pressure at the surface of the object. We can integrate the pressure
  over the entire surface area $S$ to find the total force $\mb{B}$ the fluid
  exerts on the object.
  \begin{center}
    \pic{.35}{rock_fbvectors.jpg}
  \end{center}
\end{frame}



\begin{frame}
  \frametitle{Derivation of Buoyance Force}
  Integrate the pressure$p$ over the entire surface $S$ to find the total force
  $\mb{B}$, or take some knowledge of vector calculus (divergence theorem):

  \eq{-.1in}{
    \mb{B}
    =-\oint_S p\mb{n}dS
    =-\iiint \nabla p dV
  }
  
  Since pressure $p=\rho gz$ is a function in $z$ only, the gradient easy to
  compute: $\nabla p=\partial p/\partial z=\rho g\hat{\mb{k}}$, giving us

  \eq{-.1in}{
    \mb{B}
    =\rho_\mathrm{fluid}g\hat{\mb{k}}\iiint dV = \rho_\mathrm{fluid}gV\hat{\mb{k}}
  }
\end{frame}

\begin{frame}
  \frametitle{Derivation of Buoyance Force}
  Although the derivation required a lot of calculus, the expression of
  buoyance force is straightforward:
  
  \eq{-.2in}{
    \boxed{\mb{B} = \rho_\mathrm{fluid}gV\hat{\mb{k}}=
      m_\mathrm{fluid}g\hat{\mb{k}}}
  }
  
  where $\rho_\mathrm{fluid}$ is the density of the displaced fluid, and
  $V$ is the volume displaced. This equation is known as
  \textbf{Archimedes' principle}.
  
  \vspace{.25in}\textbf{Buoyance force has a magnitude that equals to the
    weight of the fluid displaced by the submerged object, pointing upward.}
\end{frame}

\begin{frame}
  \frametitle{An Easier Explanation of Buoyancy}
  \framesubtitle{Not Much Calculus}
  \begin{columns}
    \column{.7\textwidth}
    There is a simpler way to find the buoyance force, by taking an
    infinitesimal ``tube'' of the object, and finding the pressure difference
    between the top and bottom of the tube:

    \vspace{-.5in}{\Large
      \begin{align*}
        \mb{B}&=\int (p_2-p_1)dA\\
        &= \rho_\mathrm{fluid} g\int(z_2-z_1)dA\\
        &=\rho_\mathrm{fluid} g V
      \end{align*}
    }

    \vspace{-.2in}which is the same expression that we got with calculus.

    \column{.3\textwidth}
    \pic{1}{buoyancy.jpg}
  \end{columns}
\end{frame}


\begin{frame}
  \frametitle{Buoyancy}

%  Buoyancy depends on:
%  \begin{itemize}
%  \item the density of the (displaced) fluid $\rho_\mathrm{fluid}$
%  \item the volume of the fluid displaced $V$, and
%  \item the local acceleration due to gravity $g$
%  \end{itemize}
  Note that buoyancy does not depend on:
  \begin{itemize}
  \item the mass of the immersed object, or
  \item the density of the immersed object
  \end{itemize}
%\end{frame}
%
%\begin{frame}
%  \frametitle{Buoyancy}
  \vspace{.15in}Objects immersed in a fluid have an ``apparent weight''
  $\mb{W}'$ that is reduced by the buoyance force:

  \eq{-.2in}{
    \mb{W}' = \mb{W}-\mb{B}=\rho'\mb{g}V
  }
  
  where $\rho'=\rho_{\textrm{obj}}-\rho_{\textrm{fluid}}$ is the relative density
\end{frame}

%\begin{frame}
%  \frametitle{Buoyancy}
%  For a submerged object:
%  \begin{center}
%    \begin{tabular}{c|c|c|c}
%      \rowcolor{pink}
%      Densities	&
%      $B>W_{\textrm{obj}}$ &
%      $B=W_{\textrm{obj}}$ &
%      $B<W_{\textrm{obj}}$ \\\hline
%      $\rho_{\textrm{obj}}<\rho_{\textrm{fluid}}$ & object rises & float on surface & \\
%      $\rho_{\textrm{obj}}=\rho_{\textrm{fluid}}$ & & neutral buoyancy & \\
%      $\rho_{\textrm{obj}}>\rho_{\textrm{fluid}}$ & & & object sinks
%    \end{tabular}
%  \end{center}
%\end{frame}


\begin{frame}
  \frametitle{How Submarines Work}
  \framesubtitle{Like this?}
  \begin{center}
    \pic{.7}{EbHMOXk.jpg}
  \end{center}
\end{frame}


\begin{frame}
  \frametitle{How Submarines Work}
  Like most ships, a submarine does not naturally sink because of the buoyance
  force. When a submarine submerges, water needed to be pumped inside
  ``ballast tanks'' to make the ship heavier.
  \begin{center}
    \pic{1}{risinglemur.jpg}
  \end{center}
\end{frame}


\begin{frame}
  \frametitle{Stable? Or unstable?}
  \begin{itemize}
  \item Buoyance force $\mb{B}$ acts at the \emph{center of buoyancy} (CB) of
    the submerged object
    \begin{itemize}
    \item The CB is the CG \emph{if the object has constant density} and is
      fully submerged
    \item The actual CG of the object may be at a different position
    \item Sometimes the object is not fully submerged
    \end{itemize}
  \item $\mb{F}_g$ and $\mb{B}$ may act at different points, creating
    a torque/moment on the object
  \end{itemize}
  \begin{center}
    \vspace{-.15in}
    \pic{.6}{stable-unstable.jpg}
  \end{center}
\end{frame}


\begin{frame}
  \frametitle{Fluid Flow: Continuity}
  In a fixed volume (known as a ``control volume'', or CV) we can quantify how
  fluid mass changes in the CV:
  \begin{center}
    \textbf{Rate of decrease in mass in the CV = mass flux out of the CV}
  \end{center}

  On the left hand side, the fluid mass in the CV is the integral of density
  over the volume:

  \eq{-.35in}{ \int_{CV}\rho dV }
  
  The rate of decrease is therefore the negative of the time derivative:
  
  \eq{-.25in}{
    -\frac{\partial}{\partial t}\int_{CV}\rho dV %=
%    -\int_CV\frac{\partial\rho}{\partial t}dV
  }
\end{frame}


\begin{frame}
  \frametitle{Fluid Flow: Continuity}
  The mass flux out of the surfaces of the control volume the volume flux
  multiplied by the fluid density at the surface:

  %Applying the divergence theorem, we can convert this surface interested

  \eq{-.2in}{
    \int_{CS}\rho\mb{v}\cdot d\mb{A} %=
    %\int_{CV}\nabla\cdot(\rho\mb{v}dV
  }
  
  Combining the LHS and RHS terms, we have the \emph{integral} form of the
  continuity equation:

  \eq{-.2in}{\boxed{
      \int_{CV}\frac{\partial\rho}{\partial t}dV +
      \int_{CS}\rho\mb{v}\cdot d\mb{A}=0
  }}
    %\int_{CV}\nabla\cdot(\rho\mb{v}dV=0
  %The   
\end{frame}


\begin{frame}
  \frametitle{Fluid Flow: Continuity}
  With some clever use of vector calculus, we get the \emph{differential form}
  of the continuity equation:

  \eq{-.2in}{
    \frac{\partial\rho}{\partial t} + \nabla\cdot(\rho\mb{v})=0
  }

  \ldots which is still too difficult. So in AP Physics we usually only look at
  simple cases where
  \begin{itemize}
  \item Steady flow (time independent)
  \item Constant density
  \item Flow perpendicular to control surfaces
  \end{itemize}
\end{frame}

\begin{frame}
  \frametitle{Inlet Outlet Flow}
  \begin{center}
    Diagram
  \end{center}
  In this example, the mass flowing at the inlet is the same as the flow out of
  it:

  \eq{-.2in}{
    \rho_1 v_1A_1=\rho_2 v_2A_2
  }
  And if density is constant, the $\rho$ terms will cancel.
\end{frame}

      
\begin{frame}
  \frametitle{Example: Multiple Inlet \& Outlets}
    examples
\end{frame}

\begin{frame}
  \frametitle{Governing Equations for Fluid Dynamics}

  To properly describe fluid flows, there are three conservation equations:
  \begin{itemize}
  \item continuity
  \item momentum, and
  \item energy
  \end{itemize}
\end{frame}

\begin{frame}
  \frametitle{Fluid Flow: Momentum Equation}
  In the momentum equation, the rate of decrease of total momentum inside the
  control volume CV is the net momentum flux of the fluid out of the control
  volume plus the normal and shear forces acting on the fluid:

  \eq{-.2in}{
    \frac{\partial(\rho\mb{v})}{\partial t} +
    \rho(\mb{v}\cdot\nabla)\mb{v} =
    -\nabla p +\mb{f}+\mu\nabla^2\mb{v}
  }
  (That's pretty hard, so thankfully you won't need this equation for AP
  Physics.)
\end{frame}

\begin{frame}
  \frametitle{Fluid Flow: Energy Equation}
  The energy equation follows a similar thought process as the previous two
  equations, but the terms are even more complicated:


\end{frame}


\begin{frame}
  \frametitle{Navier-Stokes Equations}
  The three conservation equations combined together is called the
  \textbf{Navier-Stokes equations}. In differential form, they are usually
  written as:


  Even for a 2nd-year engineering student with lots of experience with
  calculus, solving the N-S equation is still a daunting task.
\end{frame}


\begin{frame}
  \frametitle{Let's Make Some Assumptions}
  If we can make these assumptions:
  \begin{itemize}
  \item the flow is steady
    \begin{itemize}
    \item all derivatives w.r.t.\ time are zero
    \end{itemize}
  \item the flow is inviscid
    \begin{itemize}
    \item ``inviscid'' means zero viscosity
    \item no friction
    \item no shear stresses
    \item Only forces are pressure at the surface
    \end{itemize}
  \item there is \textbf{no shaft work} done along the streamline
  \item there is \textbf{no heat transfer} along the streamline
  \end{itemize}
  Then the N-S equations reduces to the
  \textbf{Bernoulli equation}
  
  \eq{-.1in}{\boxed{
      p_1+\frac{1}{2}\rho v_1^2 + \rho gz_1=
      p_2+\frac{1}{2}\rho v_2^2 + \rho gz_2
  }}

\end{frame}

%\begin{fame}
%  \frametitle{Flow Continuity}
%  Usually the continuity equation is written in \emph{differential form}, by
%  applying the \emph{divergence theorem} to the flux term:
%
%  \vspace{-.3in}{
%    \begin{align*}
%      \int_{CV}\frac{\partial\rho}{\partial t}dV+
%      \int_{CS}\rho\mb{v}\cdot d\mb{A}=&0\\
%    \end{align*}
%  }
%\end{frame}
%  \eq{-.15in}{
%    \boxed{\Phi_\mathrm{V}=\int\mb{V}\cdot d\mb{A}}
%  }
%    
%  \vspace{-.1in}where $\mb{V}$ is the velocity (vector field) at the surface,
%  and $d\mb{A}$ is the infinitesimal area pointing \textbf{outwards}. We can
%  also expressed volume flux using the outward normal unit vector
%  $\hat{\mb{n}}$:
%
%  \eq{-.15in}{
%    \boxed{\Phi_\mathrm{V}=\int\mb{V}\cdot\hat{\mb{n}}dA}
%  }
%\end{frame}

\begin{frame}
  \frametitle{Bernoulli Equation}

  \eq{-.01in}{\boxed{
      p_1+\frac{1}{2}\rho v_1^2 + \rho gz_1=
      p_2+\frac{1}{2}\rho v_2^2 + \rho gz_2
  }}
  The term $\displaystyle\frac{1}{2}\rho v^2 $ is called ``dynamic pressure''
\end{frame}


\begin{frame}
  \frametitle{Bernoulli Equation}

  \eq{-.01in}{\boxed{
      p_1+\frac{1}{2}\rho v_1^2 + \rho gz_1=
      p_2+\frac{1}{2}\rho v_2^2 + \rho gz_2
  }}

%  Bernoulli's equation is valid when
%  \begin{itemize}
%  \item the flow is \textbf{steady} (independent of time)
%  \item the flow is \textbf{incompressible}--compressibility (i.e. changes in
%    density of the fluid) effects are negligible for Mach number $M<0.30$
%  \item the flow \textbf{along a single streamline}
%  \item there is \textbf{no shaft work} done along the streamline between 1 and
%    2
%  \item there is \textbf{no heat transfer} along the streamline between 1 and 2
%  \end{itemize}
%\end{frame}



\begin{frame}
  \frametitle{Bernoulli Equation}

  Regions where Bernoulli equation is valid:
  \begin{center}
    \pic{.8}{bernoulli.jpg}
  \end{center}
\end{frame}



\begin{frame}
  \frametitle{How Does A Wing Work?}

  When air flows past a wing, a force is generated
\end{frame}

\end{document}
