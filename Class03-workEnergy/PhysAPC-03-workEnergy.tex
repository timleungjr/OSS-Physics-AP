\documentclass[12pt,compress,aspectratio=169]{beamer}
\usetheme{metropolis}
\setbeamersize{text margin left=.5cm,text margin right=.5cm}
\usepackage[lf]{carlito}
\usepackage{siunitx}
\usepackage{tikz}
\usepackage{mathpazo}
\usepackage{bm}
\usepackage{mathtools}
\usepackage[ISO]{diffcoeff}
\diffdef{}{ op-symbol=\mathsf{d} }
\usepackage{xcolor,colortbl}


\usetikzlibrary{decorations.pathmorphing,patterns}


\title{Class 3: Work and Energy}
\subtitle{Advanced Placement Physics C}
\author[TML]{Dr.\ Timothy Leung}
\institute{Olympiads School}
\date{Updated: Summer 2022}

\newcommand{\pic}[2]{
  \includegraphics[width=#1\textwidth]{#2}
}
\newcommand{\eq}[2]{
  \vspace{#1}{\Large
    \begin{displaymath}
      #2
    \end{displaymath}
  }
}
%\newcommand{\iii}{\ensuremath\hat{\bm{\imath}}}
%\newcommand{\jjj}{\ensuremath\hat{\bm{\jmath}}}
%\newcommand{\kkk}{\ensuremath\hat{\bm{k}}}
%\newcommand{\iii}{\ensuremath\hat x}
%\newcommand{\jjj}{\ensuremath\hat y}
%\newcommand{\kkk}{\ensuremath\hat z}



\begin{document}

\begin{frame}
  \maketitle
\end{frame}



\begin{frame}{Work and Energy}
  We start with some definition at are (unfortunately) not very useful:
  \begin{itemize}
    \item \textbf{Energy} is the ability to do work.
    \item \textbf{Work} is the mechanism in which energy is transformed.
  \end{itemize}
  Luckily, we can also use equations to define these concepts.
\end{frame}


\section{Mechanical Work}

\begin{frame}{Mechanical Work}
  \textbf{Mechanical work} $W$ is done when a force $\vec F(\vec x)$ displaces
  an object by $\dl\vec x$. When the force moves an object along a path
  $\mathcal C$, the total work done by the force is defined as:

  \eq{-.1in}{
    \boxed{
      W=\int_{\mathcal C}\dl W= \int_{\mathcal C}\vec F(\vec x)\cdot\dl\vec x
    }
  }

  \begin{itemize}
  \item Work is a scalar quantity
  \item Work (generally) depends on path $\mathcal C$
  \item No work done if the force is perpendicular to displacement, when
    $\vec F\cdot\dl\vec x=0$ (i.e.\ the force does not cause the displacement)
  \item No work done if there is no displacement ($\dl\vec x=\vec 0$)
  \item Work can be positive or negative depending on the dot product
  \item The SI unit for work is a \emph{joule} where
    $\SI1\joule=\SI1{\kilogram.\metre^2\per\second^2}$
  \end{itemize}
\end{frame}



\begin{frame}{In One Dimension}
  For motion confined to one direction in a 1D coordinate system (which is
  common for AP Physics C), the equation for work simplifies to:
  
  \eq{-.1in}{
    W=\int_{x_0}^{x_1} F(x)\dl x
  }

  (Direction still matters in 1D, i.e.\ there is still a positive and a
  negative direction.)
\end{frame}



\begin{frame}{Work by Constant Force}
  For a constant force, if the object moves along straight path, the integral
  simplifies to the dot product of the two vectors:

  \eq{-.1in}{
    \boxed{
      W=\vec F\cdot\Delta\vec x
    }
  }

  We can express this also in a scalar form that is more familiar in Grades
  11/12 Physics:

  \eq{-.1in}{
    \boxed{
      W=\vec F\Delta\vec x\cos\theta
    }
  }

  where $\theta$ is the angle between the force and displacement vectors
\end{frame}



\begin{frame}{Definition of Work}
  \textbf{Work done by a force}
  \begin{itemize}
  \item The work done by \emph{one specific force}
  \item Example: A boy pushes a cart forward. The ``work done by the boy'' is
    the work done by the applied force
  \end{itemize}

  \vspace{.15in}\textbf{Work done on an object}
  \begin{itemize}
  \item There may be more than one force acting on an object
  \item The \emph{sum} of all the work done on the object by each force
  \item The work done by the net force
  \item Also called the \textbf{net work} $W_\text{net}$
  \end{itemize}
\end{frame}



\section{Kinetic Energy \& Work-Energy Theorem}

\begin{frame}{Kinetic Energy}
  When a net force accelerates an object\footnote{We assume that the object has
  a constant mass}, the resulting work done on the object (net work
  $W_\text{net}$) is given by:

  \eq{-.1in}{
    W_\text{net}
    =\int_{x_0}^{x_1} F_\text{net}(x)\dl x
    =\int ma\dl x
    =m\int\diff vt\dl x
  }

  Both $v$ and $x$ are continuously differentiable in time, we can switch the
  order of differentiation:
  
  \eq{-.1in}{
    =m\int\diff xt\dl v=m\int_{v_0}^{v_1}v\dl v
  }
  where $v_0=v(x_0)$ and $v_1=v(x_1)$.
  \vspace{.3in}
\end{frame}



\begin{frame}{Kinetic Energy}
  This integral, when integrated from $v_0$ to $v_1$, becomes:

  \eq{-.1in}{
    =m\int_{v_0}^{v_1}v\dl v
    =\frac12mv^2\Big|^{v_1}_{v_0}
    =\frac12mv_1^2-\frac12mv_0^2
    =\Delta K
  }
  
  where $K$ is defined as the \textbf{translational kinetic energy}:

  \eq{-.1in}{
    \boxed{
      K=\frac12mv^2
    }
  }

  Later in the course we will discuss \emph{rotational} kinetic energy.
\end{frame}



\begin{frame}{Work-Energy Theorem}
  The \emph{definition} of kinetic energy came from this integration, in that
  work equals to the change in \emph{something}, and we define that
  \emph{something} as kinetic energy. This is the
  \textbf{work-energy theorem}\footnote{Also known as the
  \textbf{work-energy principle}}:

  \eq{-.1in}{
    \boxed{
      W_\text{net}=\Delta K
    }
  }
  \begin{itemize}
  \item $K$ increases ($\Delta K>0$) if net work is positive ($W_\text{net}>0$),
    and
  \item $K$ decreases ($\Delta K<0$) if net work is negative ($W_\text{net}>0$)
  \item It does not matter \emph{what} the net force is composed of
  \end{itemize}
\end{frame}



\begin{frame}{Example}
  \textbf{Example:} A force $F=4x$, measured in newtons, acts on an object of
  mass \SI2{\kilo\gram} as it moves along the $x$-axis from $x=1$ to
  $x=\SI5\metre$. Given that the object is at rest at $x=1$, calculate
  \begin{enumerate}[a.]
  \item the net work on the object
  \item the final speed of the object
  \end{enumerate}
\end{frame}



\section{Potential Energy}

\begin{frame}{Gravitational Force \& Gravitational Potential Energy}
  Consider an object that is free-falling under the force of gravity over a
  distance of $\Delta x$:
  \begin{center}
    \begin{tikzpicture}[scale=.6]
      \draw[thick,fill=cyan!10] (7.75,0) arc (75:105:30);
      \draw[mass] (0,4) circle (.2) node[right=2.5]{$m$};
      \draw[vectors,red] (0,4)--(0,2) node[below=-1]{$\vec F_g$};
    \end{tikzpicture}
  \end{center}
  \begin{itemize}
  \item When $\Delta\vec x$ is small, $\vec g$ can be considered constant
  \item The work done by the gravity ($W_g$) is \emph{positive}, and
    therefore, there is an increase in kinetic energy, and the object speeds up

    \eq{-.2in}{
      W_g=F_g\Delta x=mg\Delta x=\Delta K > 0
    }
  \end{itemize}
\end{frame}



\begin{frame}{Gravitational Potential Energy}
  The work done by gravity can also be expressed in terms of the change in
  height. Using ground as the reference level (i.e.\ $h=0$):
  \begin{center}
    \begin{tikzpicture}[scale=.55,thick]
      \draw[fill=cyan!10] (7.75,0) arc (75:105:30);
      \draw[mass] (0,6) circle (.25);
      \draw[vectors,magenta] (0,6)--(0,3) node[midway,right]{$\Delta x$};
      \draw[->|,blue] (-.4,1)--(-.4,6) node[midway,left]{$h_0$};
      \draw[->|,blue] (0,1)--(0,3) node[midway,right]{$h_1$};
    \end{tikzpicture}
  \end{center}

  \vspace{-.2in}{\large
    \begin{align*}
      W_g &
      = mg{\color{magenta}\Delta x}= mg{\color{magenta}(h_0-h_1)}
      ={\color{magenta}-}mg{\color{magenta}(h_1-h_0)}\\
      &=-(mgh_1-mgh_0)= -\Delta U_g
    \end{align*}
  }
\end{frame}



\begin{frame}{Gravitational Potential Energy}
  Defining the \textbf{gravitational potential energy} $U_g$ as:

  \eq{-.1in}{
    \boxed{U_g=mgh}\quad{\text{or}}\quad
    \boxed{\Delta U_g=mg\Delta h}
  }

  The work done by gravity is related to this potential energy by:
  
  \eq{-.1in}{
    \boxed{
      W_g=-\Delta U_g
    }
  }

  \fcolorbox{black}{yellow!10}{
    \begin{minipage}{.95\textwidth}
      \begin{itemize}
      \item \emph{Positive} work by gravity decreases gravitational potential
        energy, while
      \item \emph{Negative} work by gravity increases gravitational potential
        energy
      \item $W_g$ depends on the end points $h_0$ and $h_1$, but not \emph{how}
        it went from $h_0\rightarrow h_1$
      \item Only work done by gravity can affect gravitational potential energy
      \end{itemize}
    \end{minipage}
  }
\end{frame}



\begin{frame}{Spring Force \& Elastic Potential Energy}
  The spring force $\vec F_s$ is the force that a compressed/stretched spring
  exerts on the object connected to it.  An \emph{ideal} spring obeys Hooke's
  law:
    
  \eq{-.1in}{
    \boxed{\vec F_s=-k\vec x}
  }

  \vspace{-.1in}The spring force acts in the opposite direction to the spring's
  displacement, and is proportional to the amount of compression/stretching.

  \begin{center}
    \begin{tikzpicture}
      \draw[thick,fill=cyan!40] (5,.5) rectangle (6,1.5);
      \draw[thick,
        decoration={aspect=.6,segment length=5mm, amplitude=2.5mm, coil},
        decorate] (0,1)--(5,1);
      \fill[pattern=north east lines](-.2,0) rectangle (0,2);
      \draw[thick] (0,0)--(0,2);
      \fill[red] (5.5,1) circle (.06);
      \draw[vectors,red] (5.5,1)--(4,1) node[above]{$\vec F_s$};
      \draw[dashed] (3,0)--(3,2) node[above]{equilibrium/unstretched position};
      \draw[vectors] (3,.3)--(5,.3) node[midway,below]{$\vec x$};
    \end{tikzpicture}
    \hspace{.2in}
    \begin{tikzpicture}
      \draw[thick,gray!40,fill=gray!20] (5,.5) rectangle (6,1.5);
      \draw[thick,gray!20,
        decoration={aspect=.6,segment length=5mm, amplitude=2.5mm, coil},
        decorate] (0,1)--(5,1);
      \fill[pattern=north east lines](-.2,0) rectangle (0,2);
      \draw[thick] (0,0)--(0,2);
      \fill[gray!30] (5.5,1) circle (.06);
      \draw[vectors,gray!30] (5.5,1)--(4,1) node[above]{$\vec F_s$};
      \draw[dashed] (3,0)--(3,2);
      \draw[vectors,gray!30] (3,.3)--(5,.3)node[midway,below]{$\vec x$};
      \draw[thick,fill=cyan!40] (1.5,.5) rectangle (2.5,1.5);
      \draw[thick,
        decoration={aspect=.3,segment length=1.5mm, amplitude=2.5mm, coil},
        decorate] (0,1)--(1.5,1);
      \draw[vectors] (3,.3)--(1.5,.3) node[midway,below]{$\vec x$};
      \fill[red] (2,1) circle (.06);
      \draw[vectors,red] (2,1)--(3,1) node[above]{$\vec F_s$};
    \end{tikzpicture}
  \end{center}
\end{frame}



\begin{frame}{Elastic Potential Energy}
  \vspace{-.15in}The work done by the spring force $W_s$ as it pushes any
  masses that are connected to a compressed/stretched spring is therefore:

  \eq{-.1in}{
    W_s=\int_{x_0}^{x_1}F_s\dl x =-k\int_{x_0}^{x_1} x\dl x
    =-\frac12kx^2\Big|^{x_1}_{x_0}=-\Delta U_s
  }

  where $U_s$ is the  \textbf{elastic potential energy}, defined as:
  
  \eq{-.1in}{
    \boxed{
      U_s=\frac12kx^2
    }
  }
\end{frame}



\begin{frame}{Elastic Potential Energy}
  The the work done by the spring force can be related to the elastic
  potential energy by:
  
  \eq{-.1in}{
    \boxed{  W_s=-\Delta U_s }
  }

  \fcolorbox{black}{yellow!10}{
    \begin{minipage}{.95\textwidth}
      \begin{itemize}
      \item \emph{Positive} work by the spring decreases spring potential
        energy, while
      \item \emph{Negative} work by the spring increases spring potential energy
      \item $W_s$ depends on the end points $x_0$ and $x_1$, but not \emph{how}
        it went from $x_0$ to $x_1$
      \item Only work done by the spring force can affect elastic potential
        energy
      \end{itemize}
    \end{minipage}
  }
\end{frame}



\begin{frame}{Conservative Forces}
  These forces are called \textbf{conservative forces}
  \begin{itemize}
  \item Gravitational force $\vec F_g$
  \item Spring force $\vec F_s$
  \item Electrostatic force $\vec F_q$
  \item Magnetic force $\vec F_m$
  \item Nuclear forces
  \end{itemize}
  Because they shared these common properties:
  \begin{itemize}
  \item The work done by these forces relate to a change of a potential energy
    \begin{itemize}
    \item Positive work decreases this related potential energy
    \item Negative work increases this related potential energy
    \end{itemize}
  \item The work done by a conservative force is \emph{path independent}
    (depends only on end points)
  \end{itemize}
\end{frame}



\begin{frame}{Conservative Forces}
  By the fundamental theorem of calculus, any conservative forces $\vec F$
  must be the negative gradient of the potential energies:

  \eq{-.1in}{
    \boxed{
      \vec F=-\nabla U=
      -\diffp Ux\hat\imath-\diffp Uy\hat\jmath-\diffp Uz\hat k
    }
  }

  It is very unlikely that you will need to know the above expression in an
  AP exam, but you will need to know the one-dimension simplification:

  \eq{-.1in}{
    \boxed{
      F=-\diff Ux
    }
  }

  The direction of a conservative force \emph{always} decreases the potential
  energy. (Pay attention to the negative sign.)
\end{frame}



\begin{frame}{Energy Diagrams}
  When you plot of potential energy ($U$) vs.\ position ($x$) for a
  conservative force, you get a diagram like this:
  \begin{center}
    \begin{tikzpicture}[scale=.8]
      \draw[axes] (0,0)--(10,0) node[right]{$x$};
      \draw[axes] (0,0)--(0,5) node[right]{$U(x)$};
      \draw[very thick] (.2,4.5)
      to[out=-70,in=180] (1.5,2.5)
      to[out=0,in=180] (2.5,1)
      --(5.5,1) node[midway,below]{\scriptsize Neutral equilibrium}
      to[out=0,in=180] (7,3.5)
      to[out=0,in=180] (8,2.5)
      to[out=0,in=250] (9.5,4.5);
      \fill (1.5,2.5) circle (.07) node[below]{$A$};
      \fill (7,3.5) circle (.07) node[below]{$B$};
      \fill (8,2.5) circle (.07) node[above]{$C$};
      \begin{scope}[<-]
        \draw (1.6,2.7)--(2,3.3) node[right]{\scriptsize Unstable equilibrium};
        \draw (6.9,3.6)--(6.4,4.2) node[left]{\scriptsize Unstable equilibrium};
        \draw (8,2.4)--(8,1.5) node[below]{\scriptsize Stable equilibrium};
      \end{scope}
    \end{tikzpicture}
  \end{center}
\end{frame}

%\begin{frame}{Work and Potential Energy}
%  The expressions for potential energies also come from integrating the work
%  equation, in that work equals to the change in \emph{something}, and we
%  called that potential energy. Therefore:
%
%  \eq{-.2in}{
%    \boxed{
%      W_c=-\Delta U
%    }
%  }
%  \begin{itemize}
%  \item\vspace{-.15in}$\Delta U$ can be positive or negative depending on the
%    direction of the (conservative) force
%  \item Positive work \emph{decreases} the related potential energy
%  \item Negative work \emph{increases} the related potential energy
%  \end{itemize}
%\end{frame}



%\begin{frame}{Conservation of Mechanical Energy}
%  Positive work done by conservative forces on an object does two things:
%  \begin{enumerate}[1.]
%  \item Decrease its potential energy, while
%  \item Increase its kinetic energy by the same amount
%  \end{enumerate}
%  Mathematically, this shows that mechanical energy must \emph{always} be
%  conserved when there are only conservative forces:
%
%  \eq{-.1in}{
%    W_c=-\Delta U = \Delta K \quad\longrightarrow\quad
%    \Delta K + \Delta U =0
%  }
%
%  That's why those forces are called conservative forces, and they form the
%  basis for conservation of energy.
%\end{frame}


\section{Non-Conservative Forces}

\begin{frame}{Examples of Non-Conservative Force}
  The majority of forces are \textbf{non-conservative}. The common forces
  discussed in the previous class are generally non-conservative:
  \begin{itemize}
  \item Applied force
  \item Tension force
  \item Normal force
  \item Friction%\footnote{but sometimes it can also do positive work too.}
  \item Drag (fluid resistance)
  \end{itemize}
  The work-energy theorem still applies for non-conservative forces
\end{frame}



\begin{frame}{Work by Non-Conservative Forces}
  The work done by non-conservative forces differs from conservative forces in
  that:
  \begin{itemize}
  \item There is \textbf{no related potential energies}: the work done by a
    non-conservative force transform energy from one form of kinetic energy to
    another
  \item The work is \textbf{path dependent}
  \end{itemize}
\end{frame}



\begin{frame}{Work by Friction, an Illustration}
  Work done by friction is illustrated below. Two blocks ($m_1$ and $m_2$),
  stacked vertically, move to the right by external force.
  % $\vec F$ applied to $m_1$.
  \begin{center}
    \begin{tikzpicture}[scale=.75]
      \fill[pattern=north east lines] rectangle (7,-.2);
      \draw[thick] (0,0)--(7,0);
      \draw[thick] (1,0) rectangle (4,1.2) node[midway]{$m_1$};
      \draw[thick] (1.7,1.2) rectangle (3.3,2) node[midway]{$m_2$};
      \draw[vectors] (4,.6)--(6,.6) node[right]{$F$};
      \draw[<-] (0.95,0.05) to[out=150,in=0] (0,.5) node[left]{frictionless};
      \draw[<-] (3.35,1.25) to[out=20,in=180] (4.5,1.8) node[right]{$\mu$};
    \end{tikzpicture}
  \end{center}
  The FBDs of the blocks are shown below. (The forces highlighted in the same
  color are action-reaction pairs.)
  \begin{center}
    \begin{tikzpicture}[scale=.9,vectors]
      \fill circle (.08);
      \draw (0,0)--+(0,-1) node[below]{$m_2\vec g$};
      \draw (0,0)--+(0, 1) node[above,fill=pink!20]{$\vec N_{12}$};
      \draw (0,0)--+(1.5,0) node[right,fill=yellow!20]{$\vec f_{12}$};

      \fill (6,0) circle (.08);
      \draw (5.97,0)--+(0,-1) node[below left]{$m_1\vec g$};
      \draw (6.03,0)--+(0,-1) node[below right,fill=pink!20]{$\vec N_{12}$};
      \draw (6,0)--+(0,1) node[above]{$\vec N_1$};
      \draw (6,0)--+(-1.5,0) node[left,fill=yellow!20]{$\vec f_{12}$};
      \draw (6,0)--+(2,0) node[right]{$\vec F$};
    \end{tikzpicture}
  \end{center}
\end{frame}



\begin{frame}{Work by Friction}
  \begin{columns}
    \column{.25\textwidth}
    \centering
    \begin{tikzpicture}[scale=1.2,vectors]
      \fill circle (.08);
      \draw (0,0)--(0,-1) node[below]{$m_2\vec g$};
      \draw (0,0)--(0, 1) node[above]{$\vec N_{12}$};
      \draw (0,0)--(1.5,0) node[right,fill=yellow!20]{$\vec f_{12}$};
    \end{tikzpicture}

    \column{.75\textwidth}
    On the top block $m_2$, when it moves to the right by $\Delta x$
    \begin{itemize}
    \item Static friction $\vec f_{12}$ is the only force doing work
    \item The work done by $\vec f_{12}$ is positive
    \item Mass $m_2$ gains kinetic energy
    \end{itemize}
  \end{columns}
  \begin{columns}
    \column{.6\textwidth}
    On the bottom block $m_1$, when it moves to the right
    \begin{itemize}
    \item Applied force $\vec F$ does \emph{positive} work on $m_1$, while
    \item Static friction $\vec f_{12}$ does \emph{negative} work on $m_1$
    \item Therefore $\vec f_{12}$ decreases the kinetic energy of $m_1$ (i.e.\
      $m_1$ would have gone faster if it didn't have friction)
    \end{itemize}
    
    \column{.4\textwidth}
    \centering
    \begin{tikzpicture}[scale=1.2,vectors]
      \fill circle (.08);
      \draw (0,0)--(0,-1) node[below]{$m_1\vec g+\vec N_{12}$};
      \draw (0,0)--(0,1) node[above]{$\vec N_1$};
      \draw (0,0)--(-1.5,0) node[left,fill=yellow!20]{$\vec f_{12}$};
      \draw (0,0)--(2,0) node[right]{$\vec F$};
    \end{tikzpicture}    
  \end{columns}
\end{frame}



\begin{frame}{Work by Friction}
  The work done by friction is
  \begin{itemize}
  \item positive on one object ($m_2$)
  \item negative on another ($m_1$)
  \end{itemize}
  Therefore, work by non-conservative forces transforms energy from the kinetic
  energy of one object into the kinetic energy of another object.
\end{frame}




\section{Internal Energy}

\begin{frame}{Internal Energy}
  \begin{columns}
    \column{.28\textwidth}
    \begin{tikzpicture}
      \draw[very thick,fill=gray!10] (-.2,-.2) rectangle (2.2,2.2);
      \draw[vectors] (0,2.5)--(2,2.5) node[right]{$v$};
      \draw[very thick,|<-|] (-.5,1)--(-.5,-3) node[midway,left]{$h$};
      \foreach \i in {1,...,40}{
       \fill[red](rand+1,rand+1) circle (.06);
      }
    \end{tikzpicture}

    \column{.72\textwidth}
    Consider a container of gas of mass $M$ moving at speed $v$ at a height $h$
    above Earth. It has a \emph{bulk kinetic energy} of

    \eq{-.1in}{
      K=\dfrac12 Mv^2
    }
    
    and a gravitational potential energy of

    \eq{-.1in}{ U_g=Mgh }

    \vspace{-.1in}But the random motion of the air molecules also contribute to
    additional energy, called the \textbf{internal energy} $E_\text{int}$, or
    \textbf{thermal energy}.
  \end{columns}
\end{frame}



\begin{frame}{Internal Energy}
  Internal energy of a system of molecules is the sum of all their kinetic and
  potential energies at the microscopic level:
  
  \eq{-.1in}{
    \boxed{ E_\text{int}=K_\text{micro} + U_\text{micro} }
  }

  It is a function of the molecules' \textbf{absolute temperature}. This is
  part of a major field within physics called thermodynamics that is not
  covered in this course\footnote{The introductory dicussion is covered in AP
  Physics 2}; we are only including them here for completeness.\footnote{For
  those who are interested, for ideal and monatomic gases,
  $E_\text{int}=\frac32NkT$; for diatomic gases, $E_\text{int}\approx\frac52NkT$;
  and for solids, $E_\text{int}\approx3NkT$. }
\end{frame}




\section{Conservation of Energy}

\begin{frame}{Law of Conservation of Energy}
  The \textbf{law of conservation of energy} states that \emph{the change in
  the total energy of a system is equal to the external work done to it}:

  \eq{-.1in}{
    \boxed{
      \Delta E_\text{sys}=W_\text{ext}
    }
  }

  A \textbf{system} is a predefined collection of objects that apply forces
  on each other (and may or may not do work on each other). A system can be
  \begin{itemize}
  \item\textbf{Isolated}: objects in the system do work \emph{only} on
    each other
  \item\textbf{Open}: objects in the system do work on each other as
    well as to objects in its surroundings
  \end{itemize}
\end{frame}



\begin{frame}{Law of Conservation of Energy}
  The energy of a system includes the kinetic energies of all the objects, the
  potential energies stored between the objects, and the internal energies of
  the objects:

  \eq{-.1in}{
    \boxed{
      E_\text{sys} = \underbrace{\sum K + \sum U_i}_\text{mechanical} +
      \underbrace{\sum E_\text{int}}_\text{thermal}
    }
  }
  
  Not suprisingly, the \emph{change} in the system enenrgy is the net change
  in all these energies:
  
  \eq{-.1in}{
    \boxed{
      \Delta E_\text{sys}=\sum\Delta K + \sum\Delta U + \sum\Delta E_\text{int}
      = W_\text{ext}
    }
  }

  \vspace{-.1in}In an isolated system that does not interact with the outside
  (and therefore no external work can be done), conservation of energy reduces
  to:

  \eq{-.1in}{
    \boxed{
      \sum\Delta K + \sum\Delta U + \sum\Delta E_\text{int}= 0
    }
  }
\end{frame}



\begin{frame}{Law of Conservation of Energy}
  In almost all of the problem encountered in AP Physics C, there will be no
  change in the internal energy of the system, and conservation of energy
  reduces to:
  
  \eq{-.2in}{
    \boxed{\sum\Delta K + \sum\Delta U = W_\text{ext} }\quad\rightarrow\quad
    \boxed{\sum U + \sum K + W_\text{ext} = \sum U' + \sum K' }
  }

  External work $W_\text{ext}$ is
  \begin{itemize}
  \item\textbf{Positive} if work is done \text{to} the system
  \item\textbf{Negative} if work is done \text{by} by the system to the
    surrounding
  \end{itemize}
\end{frame}



%\begin{frame}{Isolated Systems and the Conservation of Energy}
%  An \textbf{isolated system} is a system of objects that does not interact with
%  the surrounding. Think of an isolated system as a bunch of objects inside an
%  insulated box.
%  \begin{center}
%    \begin{tikzpicture}[scale=.7]
%      \fill[pattern=north east lines] rectangle (5,4);
%      \draw[thick] rectangle (5,4);
%      \draw[thick,fill=blue!5] (.2,.2) rectangle (4.8,3.8);
%      \draw[thick,
%        decoration={aspect=.3,segment length=2mm, amplitude=2.5mm, coil},
%        decorate] (2.5,3.8)--(2.5,2.2) node[midway,right=3.5]{$k$};
%      \draw[mass] (2,2.25) rectangle (3,1.25) node[midway]{$m$};
%    \end{tikzpicture}
%  \end{center}
%  Since the system is isolated from the surrounding environment, the
%  environment can't do any work on it. Likewise, the energy inside the system
%  cannot escape either.
%\end{frame}


\begin{frame}{Example: Gravity}
  When there are no friction and drag, a free-falling object forms an isolated
  system with Earth:
  \begin{center}
    \begin{tikzpicture}[scale=.7]
      \draw[thick,fill=cyan!10] (7.75,0) arc (75:105:30);
      \draw[mass] (0,3) circle (.2) node[right=3]{$m$};
      \draw[vectors,red] (0,3)--(0,1.25) node[below]{$\vec F_g$};
    \end{tikzpicture}
  \end{center}
  The sum of the kinetic energy of the mass ($K$) and the gravitational
  potential energy stored between the tow masses ($U_g$) is constant
    
  \eq{-.1in}{
    K+U_g=\text{constant}
  }
\end{frame}



\begin{frame}{Example: Gravity}
  Energy is also conserved for an object sliding down an arbitrarily-shaped
  ramp, assuming that there is no friction or drag:
  \begin{center}
    \begin{tikzpicture}[scale=.7]
      \draw[thick] (0,4) to[out=-30,in=180] (3,1) to[out=0,in=180] (5,3)
      to[out=0,in=170] (8,0) to[out=-10,in=180] (10,0);
      \draw[mass,rotate around={-60.5:(1,2.93)}] (1,2.93) rectangle +(.6,.6);
      \fill[red] (1.4,2.8) circle (.08);
      \draw[vectors,red] (1.4,2.8)--+(0,-1.5) node[below]{$\vec F_g$};
      \draw[vectors,red,rotate around={30:(1.4,2.8)}]
      (1.4,2.8)--+(1.4,0) node[right]{$\vec F_N$};
    \end{tikzpicture}
  \end{center}
  \begin{itemize}
  \item Both normal force $\vec F_N$ and gravity $\vec F_g$ act on the object,
    but only gravity does mechanical work ($\vec F_N$ is perpendicular to
    motion)
  \item The sum of the kinetic energy ($K$) and gravitational potential energy
    is constant
    
    \eq{-.15in}{
      K + U_g = \text{constant}
    }
  \end{itemize}
\end{frame}




\begin{frame}{Example: Horizontal Spring-Mass System}
  When there are no friction, drag or other damping forces present, a
  horizontal spring-mass system is an isolated system. It consists of the
  mass and the spring:
  \begin{center}
    \begin{tikzpicture}[scale=1.15]
      \draw[mass] (5,.5) rectangle (6,1.5);
      \draw[thick,
        decoration={aspect=.3,segment length=2mm, amplitude=2.5mm, coil},
        decorate] (0,1)--(5,1);
      \fill[pattern=north east lines] (6.5,.5)--(6.5,.3)--(-.2,.3)
      --(-.2,2)--(0,2)--(0,.5)--cycle;
      \draw[very thick] (0,2)--(0,.5)--(6.5,.5);
      \draw[vectors,red] (5.5,1)--(5.5,0) node[below]{$\vec F_g$};
      \draw[vectors,red] (5.5,1)--(5.5,2) node[above]{$\vec F_N$};
      \draw[vectors,red] (5.5,1)--(4.5,1) node[above]{$\vec F_s$};
    \end{tikzpicture}
  \end{center}
  The sum of the kinetic energy of the mass ($K$) and the elastic potential
  energy stored in the spring ($U_s$) is constant

  \eq{-.25in}{
    K+U_s=\text{constant}
  }
\end{frame}



\begin{frame}{Example: Vertical Spring-Mass System}
  \begin{columns}
    \column{.25\textwidth}
    \centering
    \begin{tikzpicture}
      \draw[mass] (.5,1.5) rectangle (1.5,2.5);
      \draw[thick,
        decoration={aspect=.4,segment length=2mm, amplitude=2.5mm, coil},
        decorate] (1,5)--(1,2.5); 
      \fill[pattern=north east lines] (0,5) rectangle (2,5.2);
      \draw[very thick] (0,5)--(2,5);
      \draw[vectors,red] (1,2)--(1,1)node[right]{$\vec F_g$};
      \draw[vectors,red] (1,2)--(1,3)node[right]{$\vec F_s$};
      \fill[red] (1,2) circle (.05);      
    \end{tikzpicture}

    \column{.75\textwidth}
    Assuming that there are no friction, drag or other damping forces in the
    spring, the vertical spring-mass system (consists of the mass, the spring
    and Earth) is a closed system.

    \eq{-.1in}{
      K + U_g + U_s=\text{constant}
    }
    
    The sum of the kinetic energy of the mass ($K$), the gravitational
    potential energy stored between the mass and Earth ($U_g$), and the elastic
    potential energy stored in the spring ($U_s$) is constant.
  \end{columns}
\end{frame}



\begin{frame}{Simple Pendulum System}
  \begin{columns}
    \column{.7\textwidth}
    Assuming that there are no friction, drag or other damping forces in the
    spring, the simple pendulum system (consists of the mass and Earth) is a
    closed system.
    \begin{itemize}
    \item Gravity ($\vec F_g$), which is conservative, is the only force that
      does work
    \item Tension ($\vec F_T$) does not do work because it is \emph{always}
      perpendicular to the motion of the pendulum bob
    \end{itemize}
    The sum of the kinetic energy of the mass ($K$), the gravitational
    potential energy stored between the mass and Earth ($U_g$) is constant:

    \eq{-.2in}{
      K + U_g =\text{constant}
    }
    
    \column{.3\textwidth}
    \centering
    \begin{tikzpicture}
      \fill[pattern=north east lines] (-1,0) rectangle (1,0.2);
      \draw[thick] (-1,0)--(1,0);
      \begin{scope}[rotate=15]
        \draw[thick] (0,0)--(0,-5);
        \shade[ball color=cyan] (0,-5) circle (.2) node[right=3]{$m$};
        \begin{scope}[red,vectors]
          \draw (0,-5)--(0,-3.5) node[left]{$\vec F_T$};
          \draw[rotate around={-15:(0,-5)}] (0,-5)--(0,-6.3)
          node[below]{$\vec F_g$};
        \end{scope}
      \end{scope}
      \draw[dashed,thin] (0,0)--(0,-5);
      \draw[->] (0,-2) arc (270:285:2) node[midway,below]{$\phi$};
    \end{tikzpicture}
  \end{columns}
\end{frame}


\begin{frame}{Isolated System with Changing Internal Energy}
  Energy is always conserved as long as your system is defined properly. In
  this case, the system consists of a mass, a spring, Earth and all the air
  molecules inside the box:
  \begin{center}
    \begin{tikzpicture}[scale=.66]
      \fill[pattern=north east lines] rectangle (5,4);
      \draw[thick] rectangle (5,4);
      \draw[thick,fill=blue!5] (.2,.2) rectangle (4.8,3.8);
      \draw[thick,
        decoration={aspect=.3,segment length=2mm, amplitude=2.5mm, coil},
        decorate] (2.5,3.8)--(2.5,2.25) node[midway,right=4.5]{$k$};
      \draw[mass] (2,2.25) rectangle (3,1.25) node[midway]{$m$};
    \end{tikzpicture}
  \end{center}
  The energies of this system include
  \begin{itemize}
  \item Kinetic energy of the mass ($K$)
  \item Gravitational potential energy ($U_g$) between the mass and Earth
  \item Elastic potential energy ($U_s$) stored in the spring
  \item Internal energy ($E_\text{int}$) of the air molecules
  \end{itemize}
\end{frame}



\begin{frame}{Isolated System with Changing Internal Energy}
  As the mass vibrates, friction and drag with air slows it down, converting the
  kinetic energy of the mass into the internal energy of the air. Total energy
  is conserved even as the mass stops moving

  \vspace{.2in}
  \begin{columns}
    \column{.4\textwidth}
    \centering
    \begin{tikzpicture}[scale=.6]
      \fill[pattern=north east lines] rectangle (5,4);
      \draw[thick] rectangle (5,4);
      \draw[thick,fill=blue!5] (.2,.2) rectangle (4.8,3.8);
      \draw[thick,
        decoration={aspect=.3,segment length=2mm, amplitude=2.5mm, coil},
        decorate] (2.5,3.8)--(2.5,2.25) node[midway,right=4.5]{$k$};
      \draw[mass] (2,2.25) rectangle (3,1.25) node[midway]{$m$};
    \end{tikzpicture}

    \column{.6\textwidth}
    \eq{0pt}{
      K + E_\text{int}+U_g+U_s=\text{constant}
    }
  \end{columns}
%
%  \vspace{.2in}Non-conservative forces doing work are \emph{internal} to the
%  system, and therefore energy is still conserved. (Work done by friction
%  transform from the kinetic energy of the mass to the kinetic energy of the
%  air molecules.)
\end{frame}



\begin{frame}{Isolated System vs.\ Open System}
  Accounting for the change in the internal energy of the air molecules is not
  always practical, especially when the air molecules are not confined to a box.
  \begin{center}
    \begin{tikzpicture}[scale=.6]
      \draw[very thick] (-3,4)--(8,4);
      \fill[blue!5] (-3,0) rectangle (8,4);
      \draw[thick,
        decoration={aspect=.3,segment length=2mm, amplitude=2.5mm, coil},
        decorate] (2.5,4)--(2.5,2.25) node[midway,right=5]{$k$};
      \draw[mass] (2,2.25) rectangle (3,1.25) node[midway]{$m$};
    \end{tikzpicture}
  \end{center}
  The solution:
  \begin{itemize}
  \item Take the air molecule out of the \emph{system}
  \item No longer an isolated system
  \item Treat the negative work done by kinetic friction and drag as
    \emph{external work} between initial and final states

    \eq{-.15in}{
      K + U_g+ U_s + W_f= K' + U_g' + U_s'
    }
  \end{itemize}
\end{frame}

%\begin{frame}{Conservation of Energy}
%  If \emph{only} conservative forces are doing work, mechanical energy (i.e.\
%  $K+U$) is always conserved:
%
%  \eq{-.2in}{
%    \boxed{K+U =K'+U'}
%  }
%  
%  When external non-conservative forces are also doing work, instead of
%  \emph{trying} to isolate the system, we can instead calculate the work done
%  by them $W_{nc}$ and add it to the total energy of the system
%    
%  \eq{-.2in}{
%    \boxed{K+U+W_{nc}=K'+U'}
%  }
%\end{frame}



\begin{frame}{Example}
  \textbf{Example:} A mass $m$ is dropped from a height of $h$ above the
  equilibrium position of a spring. Set up the equation that determines the
  spring's compression $d$ when the object is instantaneously at rest.
  \begin{center}
    \pic{.35}{spring-example1}
  \end{center}
\end{frame}


%\begin{frame}{Example}
%  \textbf{Example 3:} A mass $m$ is pulled a distance $d$ up an incline (angle
%  of elevation $\theta$) at constant speed using a rope that is parallel to
%  the incline. The coefficient of friction is $\mu_k$.
%  \begin{enumerate}[(a)]
%  \item What is the magnitude of the tension force in the rope?
%  \item What is the magnitude of the normal force?
%  \item What is the work done by the normal force?
%  \item What is the work done by friction?
%  \item What is the work done by the tension force?
%  \item What is the net work?
%  \item What is the change in total mechanical energy?
%  \item Show that $\Delta E_{mech}=W_{nc}$.
%  \end{enumerate}
%\end{frame}



\section{Power \& Efficiency}

\begin{frame}{Power}
  \textbf{Power} is the \emph{rate} at which work is done, i.e.\ the rate at
  which energy is being transformed:

  \eq{-.1in}{
    \boxed{P(t) = \diff Wt}\quad\quad
    \boxed{\overline P = \frac W{\Delta t}}
  }
  \begin{center}
    \begin{tabular}{l|c|c}
      \rowcolor{pink}
      \textbf{Quantity}  & \textbf{Symbol} & \textbf{SI Unit} \\ \hline
      Instantaneous and average power & $P$, $\overline P$ & \si\watt \\
      Work done          & $W$ & \si\joule \\
      Time interval      & $\Delta t$ & \si\second
    \end{tabular}
  \end{center}
  In engineering, power is often more critical than the actual amount of work
  done.
\end{frame}



\begin{frame}{Power}
  If a force is used to push an object, the instantaneous power produced by the
  force is:
  
  \eq{-.1in}{
    P(t)=\diff Wt=\frac{F\cdot\dl\vec x}{\dl t}
    =\vec F\cdot\diff{\vec x}t \quad\longrightarrow\quad
    \boxed{P(t)=\vec F(t)\cdot\vec v(t)}
  }
  
  Application: aerodynamics
  \begin{itemize}
  \item When an object moves through air, the applied force must overcome air
    resistance (drag force), which is proportional with $v^2$
    \item Therefore ``aerodynamic power'' must scale with $v^3$ (i.e.\ doubling
      your speed requires $2^3=8$ times more power)
    \item Important when aerodynamic forces dominate
  \end{itemize}
\end{frame}



\begin{frame}{Efficiency}
  \textbf{Efficiency} is the ratio of useful energy or work output to the total
  energy or work input

  \eq{-.1in}{
    \boxed{ \eta = \frac{E_o}{E_i}\times\SI{100}\percent }\quad
    \boxed{ \eta = \frac{W_o}{W_i}\times\SI{100}\percent }
  }
  \begin{center}
    \begin{tabular}{l|c|c}
      \rowcolor{pink}
      \textbf{Quantity} & \textbf{Symbol} & \textbf{SI Unit} \\ \hline
      Useful output energy & $E_o$  & \si\joule \\
      Input energy         & $E_i$  & \si\joule \\
      Useful output work   & $W_o$  & \si\joule \\
      Input work           & $W_i$  & \si\joule \\
      Efficiency           & $\eta$ & no units
    \end{tabular}
  \end{center}
  Efficiency is always $0\leq\eta<\SI{100}\percent$
\end{frame}
\end{document}
