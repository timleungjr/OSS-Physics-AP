\documentclass[12pt,aspectratio=169]{beamer}
\usetheme{metropolis}
\setbeamersize{text margin left=.5cm,text margin right=.5cm}
\usepackage[lf]{carlito}
\usepackage{siunitx}
\usepackage{tikz}
\usepackage{mathpazo}
\usepackage{bm}
\usepackage{mathtools}
\usepackage[ISO]{diffcoeff}
\diffdef{}{ op-symbol=\mathsf{d} }
\usepackage{xcolor,colortbl}

\setmonofont{Ubuntu Mono}
\setlength{\parskip}{0pt}
\renewcommand{\baselinestretch}{1}

\sisetup{
  inter-unit-product=\cdot,
  per-mode=symbol
}

\tikzset{
  >=latex
}

%\newcommand{\iii}{\hat{\bm\imath}}
%\newcommand{\jjj}{\hat{\bm\jmath}}
%\newcommand{\kkk}{\hat{\bm k}}


\title{Class 21: Circuit Analysis, Part 2}
\subtitle{AP Physics C}
\author[TML]{Dr.\ Timothy Leung}
\institute{Olympiads School}
\date{Updated: Summer 2022}

\newcommand{\pic}[2]{
  \includegraphics[width=#1\textwidth]{#2}
}
\newcommand{\eq}[2]{
  \vspace{#1}{\Large
    \begin{displaymath}
      #2
    \end{displaymath}
  }
}
%\newcommand{\iii}{\ensuremath\hat{\bm{\imath}}}
%\newcommand{\jjj}{\ensuremath\hat{\bm{\jmath}}}
%\newcommand{\kkk}{\ensuremath\hat{\bm{k}}}
\newcommand{\iii}{\ensuremath\hat\imath}
\newcommand{\jjj}{\ensuremath\hat\jmath}
\newcommand{\kkk}{\ensuremath\hat k}



\begin{document}

\begin{frame}
  \maketitle
\end{frame}



\section{Capacitors in Circuit}

\begin{frame}{Capacitors in Parallel}
  \begin{center}
    \begin{tikzpicture}[scale=1.2,thick]
      \draw (0,1) to[short,o-] (1,1) to[C=$C_1$] (1,0) to[short,-o] (0,0);
      \draw (1,1)--(3,1) to[C=$C_2$] (3,0)--(1,0);
      \draw (3,1)--(5,1) to[C=$C_3\ldots$] (5,0)--(3,0);
    \end{tikzpicture}
  \end{center}
  From the voltage law, we know that the voltage across all the capacitors are
  the same, i.e.\ $V_1=V_2=V_3=\cdots=V$. We can express the total charge
  $Q_\text{tot}$ stored across all the capacitors in terms of capacitance and
  this common voltage $V$: 

  \eq{-.1in}{
    Q_\text{tot}=Q_1+Q_2+Q_3+\cdots=C_1V+C_2V+C_3V+\cdots
  }
  
  Factoring out $V$ from each term gives us the equivalent capacitance:

  \eq{-.1in}{
    \boxed{C_p=\sum_i C_i}
  }
\end{frame}



\begin{frame}{Capacitors in Series}
  Likewise, we can do a similar analysis to capacitors connected in series.
  \begin{center}
    \begin{tikzpicture}[scale=1.2,thick]
      \draw (0,0) to[C=$C_1$,o-] (1.25,0) to[C=$C_2$] (2.5,0)
      to[C=$C_3$,-o] (3.75,0);
    \end{tikzpicture}
  \end{center}
  The total voltage across these capacitors are the sum of the voltages across
  each of them, i.e.\ $V_\text{tot}=V_1+V_2+V_3+\cdots$
  
  \vspace{.1in}The charge stored on all the capacitors must be the same! The
  total voltage in terms of capacitance and charge is:

  \eq{-.1in}{
    V_\text{tot}=\frac Q{C_1}+\frac Q{C_2}+\frac Q{C_3}+\cdots
  }
\end{frame}




\begin{frame}{Equivalent Capacitance in Series}
  The inverse of the equivalent capacitance for $N$ capacitors connected in
  series is the sum of the inverses of the individual capacitance.

  \eq{-.1in}{
    \boxed{ \frac1{C_s}=\sum_i\frac1{C_i} }
  }
  
  Make sure we don't confuse ourselves with resistors.
\end{frame}



\begin{frame}{How Do We Know That Charges Are The Same?}
  It's simple to show that the charges across all the capacitors are the same
  \begin{center}
    \begin{tikzpicture}[scale=1.5]
      \draw[thick] (0,0) to[C=$C_1$,o-] (1.25,0) to[C=$C_2$,-o] (2.5,0);
      \draw[dashed] (0.625,-.75) rectangle (1.875,1);
    \end{tikzpicture}
  \end{center}
  The capacitor plates and the wire connecting them are really one piece of
  conductor. There is nowhere for the charges to leave the conductor, therefore
  when charges are accumulating on $C_1$, $C_2$ must also have the same charge
  because of conservation of charges.
\end{frame}



\section{RC Circuits}

%\begin{frame}{Circuits with Resistors and Capacitors}
%  An \textbf{RC circuit} is one that has both resistors and capacitors. The
%  simplest form is a resistor and capacitor connected in series, and then
%  connect to a voltage source.
%  \begin{center}
%    \begin{tikzpicture}[american voltages,thick]
%      \draw (0,0) to[battery,l=$V$] (0,2) to[R=$R$] (2,2)
%      to[C=$C$] (2,0)--(0,0);
%    \end{tikzpicture}
%  \end{center}
%  Because of the nature of capacitors, the current through the circuit will not
%  be steady as were the case with only resistors.
%\end{frame}



\begin{frame}{Discharging a Capacitor}
  An \textbf{RC circuit} has both resistors and capacitors. In its simplest
  form, a resistor and a capacitor are connected in series.
  \begin{columns}
    \column{.3\textwidth}
    \centering
    \begin{tikzpicture}[scale=1.4,thick]
      \draw (.7,0) to[short,o-] (0,0) to[R=$R$] (0,2)--(2,2) to[C=$C$]
      (2,0)--(1.2,0) to[short,-o] (.78,-.3);
    \end{tikzpicture}
    
    \column{.7\textwidth}
    \begin{itemize}
    \item The capacitor is initially charged to $Q_0$
    \item Initial voltage across the capacitor $V_c=\dfrac{Q_0}C$
    \end{itemize}
    At $t=0$, the switch is closed
    \begin{itemize}
    \item As current starts to flow, the charge on the capacitor decreases
      as a function of time: $Q=Q(t)$
      \begin{itemize}
      \item Over time, capacitor is fully discharged, and current stops flowing
      \end{itemize}
    \item The voltage across a capacitor is decreases as a function of time:
      $V_c(t)=Q(t)/C$
    \item The current in the circuit \emph{decreases} the total charge in
      the capacitor, i.e. $I=-\diff Q/t$,
    \end{itemize}
  \end{columns}
\end{frame}



\begin{frame}{Discharging a Capacitor}
  \begin{center}
    \begin{tikzpicture}[thick]
      \draw (0,0) to[R=$R$] (0,2)--(2,2) to[C=$C$] (2,0)--(0,0);
    \end{tikzpicture}
  \end{center}
  Applying the voltage law for the circuit, we get a first-order differential
  equation for $Q$:
  
  \eq{-.1in}{
    V_c-\underbrace{V_R}_{IR}=0\quad\longrightarrow\quad
    \frac QC+R\diff Qt=0
  }
\end{frame}



\begin{frame}{Discharging a Capacitor}
  After separating the variable:

  \eq{-.1in}{
    \frac{\dl Q}Q =-\frac{\dl t}{RC}
  }
  
  We can integrate both sides:% and ``exponentiate'':

  \eq{-.1in}{
    \int\frac{\dl Q}Q =-\int\frac{\dl t}{RC}
    \quad\rightarrow\quad
    \ln Q=-\frac t{RC} + K
    \quad\rightarrow\quad
    Q(t)=e^Ke^{-t/RC}
  }

  The constant of integration $K$ (really, $e^k$) is the initial charge on the
  capacitor $Q_0$:

  \eq{-.13in}{
    e^K=Q_0
  }
\end{frame}



\begin{frame}{Discharging a Capacitor}
  The charge $Q(t)$ across the capacitor is an exponential decay:

  \eq{-.1in}{
    \boxed{Q(t)=Q_0e^{-t/\tau}}
  }

  \vspace{-.1in}where $\tau=RC$ is the \textbf{time constant}. To find the
  current through the circuit, we take the time derivative of $Q(t)$:

  \eq{-.1in}{
    \boxed{
      I(t)=-\diff Qt=I_0e^{-t/\tau}
    }
  }

  where the initial current %$I_0=I(0)$ is given by
  $I_0=\dfrac{Q_0}\tau=\dfrac{Q_0}{RC}=\dfrac{V_0}R$, consistent with Ohm's law.
\end{frame}



\begin{frame}{Charging a Capacitor}
  \begin{columns}[T]
    \column{.28\textwidth}
    \centering
    \begin{tikzpicture}[american voltages,scale=1.25,thick]
      \draw (0,0) to[battery,l=$\mathcal E$] (0,2) to[R=$R$] (2,2)
      to[C=$C$] (2,0)--(0,0);
    \end{tikzpicture}
    
    \column{.72\textwidth}
    In order to discharge a capacitor, we first have to charge it. This can be
    done using the circuit on the left.
    \begin{itemize}
    \item The capacitor is initially uncharged, i.e. $Q(0)=0$ and therefore
      $V_c(0)=0$
    \end{itemize}
    At $t=0$, the switch is closed, and current flows
    \begin{itemize}
    \item Current flow $I$ \emph{increases} the charge in the capacitor, i.e.\
      $I=+\diff Q/t$
    \item Voltage across the capacitor $V_c=Q/C$ also increases
    \end{itemize}
  \end{columns}
\end{frame}



\begin{frame}{Charging a Capacitor}
  \begin{center}
    \begin{tikzpicture}[american voltages,thick]
      \draw (0,0) to[battery,l=$\mathcal E$] (0,2) to[R=$R$] (2,2)
      to[C=$C$] (2,0)--(0,0);
    \end{tikzpicture}
  \end{center}
  Applying the voltage law, and then substitute the expression for current and
  voltage across the capacitor, we arrive at a similar first-order differential
  equation:

  \eq{-.1in}{
    \mathcal E-\underbrace{V_R}_{IR}-\underbrace{V_c}_{Q/C}=0
    \quad\longrightarrow\quad
    \mathcal E-R\diff Qt-\frac QC=0
  }
\end{frame}


\begin{frame}{Charging a Capacitor}
  Again, separating variables and integrating, we get:
  
  \eq{-.1in}{
    \int\frac{\dl Q}{C\mathcal E-Q}=\int\frac{\dl t}{RC}
    \quad\rightarrow\quad-\ln(C\mathcal E-Q)=\frac t{RC}+K
  }
  
  ``Exponentiating'' both sides, we have
  
  \eq{-.1in}{
    C\mathcal E-Q=e^Ke^{-t/RC}
  }

  \vspace{-.1in}To find the constant of integration $K$, we note that at $t=0$,
  the charge across the capacitor is $0$, and we get

  \eq{-.13in}{
    e^K=C\mathcal E=Q_\text{tot}
  }

  \vspace{-.15in}which is the charge stored in the capacitor as
  $t\rightarrow\infty$.
  
\end{frame}




\begin{frame}{Charging a Capacitor}
  Substitute this back into the equation, we find the charge in the capacitor:

  \eq{-.1in}{
    \boxed{
      Q(t)=Q_\text{tot}(1-e^{-t/RC})=Q_\text{tot}(1-e^{-t/\tau})
    }
  }

  Charging a capacitor has the same time constant as during discharge. We can
  also differentiate to find the current through the circuit:

  \eq{-.1in}{
    \boxed{I(t)=\diff Qt=I_0e^{-t/\tau}}
  }

  \begin{itemize}
  \item The initial current is
    $I_0=\dfrac{Q_\text{tot}}\tau=\dfrac{Q_\text{tot}}{RC}=\dfrac{\mathcal E}R$.
    This makes sense because $V_C(0)=0$, all the energy must be dissipated
    through the resistor
  \item As $t\rightarrow\infty$, $I_\infty=0$
  \end{itemize}
\end{frame}



\begin{frame}{Two Small But Very Important Notes}
  \begin{enumerate}
  \item When a capacitor is uncharged, there is no voltage across the plate,
    it acts like a short circuit.
  \item When a capacitor is charged, there is a voltage across it, but no
    current flows \emph{through} it. Effectively it acts like an open circuit.
  \end{enumerate}
\end{frame}



\begin{frame}{Practical RC Circuits}
  While we can easily analyze a simple RC circuit, practical configurations are
  nearly impossible to solve without difficult calculations. These circuits may
  have resistor(s) and capacitor(s) arranged in series or in parallel. For
  example:
  \begin{center}
    \begin{tikzpicture}[scale=1.2,american voltages,thick]
      \draw (0,0) to[battery1,l=12<\volt>] (0,2) to[R=4<\ohm>] (2,2)
      to[short,-*] (2.46,2.3);
      \draw (2.5,2) to[short,*-] (3,2) to[short] (4.5,2)
      to[C=6<\micro\farad>] (4.5,0)--(0,0);
      \draw (3,0) to[R=8<\ohm>] (3,2);
    \end{tikzpicture}
  \end{center}
  The capacitor in the circuit is initially uncharged. At $t=0$, the switch is
  closed. Find the current through the battery and the resistors:
  \begin{enumerate}[(a)]
  \item Immediately after the switch is closed
  \item A long time after the switch is closed
  \end{enumerate}
\end{frame}



\begin{frame}{Practical RC Circuits: Example}
  \begin{columns}[T]
    \column{.5\textwidth}
    At $t=0$, the capacitor is uncharged, and it acts like a short circuit.
    There is no current through the \SI8{\ohm} resistor.
    \begin{center}
      \begin{tikzpicture}[american voltages,thick]
        \draw (0,0) to[battery1,l=12<\volt>] (0,2) to[R=4<\ohm>] (2,2)
        --(4.5,2)--(4.5,0)--(0,0);
        \draw (3,0) to[R=8<\ohm>] (3,2);
      \end{tikzpicture}
    \end{center}
    
    \column{.5\textwidth}
    As $t\rightarrow\infty$, the capacitor becomes fully charged, and it acts
    like an open circuit. It has the same voltage as the \SI8{\ohm} resistor.
    \begin{center}
      \begin{tikzpicture}[american voltages,thick]
        \draw (0,0) to[battery1,l=12<\volt>] (0,2) to[R=4<\ohm>] (2,2)--(4.5,2)
        to[short,-o] (4.5,1.2);
        \draw (4.5,.8) to[short,o-](4.5,0)--(0,0);
      \draw (3,0) to[R=8<\ohm>] (3,2);
      \end{tikzpicture}
    \end{center}
  \end{columns}
\end{frame}



\section{LR Circuits}

\begin{frame}{Circuits with Inductors}
  \begin{itemize}
  \item Coils and solenoids in circuits are known as ``inductors'' and have
    large self inductance $L$
  \item Self inductance prevents currents rising and falling instantaneously
  \end{itemize}
  A \textbf{LR circuit} consists of both resistors and inductors. In its
  simplest form, a resistor ($R$) and inductor ($L$) are connected to a voltage
  source ($\mathcal E$) in series:
  \begin{center}
    \begin{tikzpicture}[american voltages,scale=1.2,thick]
      \draw (0,0) to[battery,l=$\mathcal E$] (0,2) to[short,-*] (0.5,2);
      \draw (0.51,2.3) to[short,*-*] (1,2)--(1.5,2) to [R=$R$] (3,2)
      to [L=$L$] (3,0)--(0,0);
    \end{tikzpicture}
  \end{center}
\end{frame}



\begin{frame}{Analyzing LR Circuits}
  \begin{columns}
    \column{.33\textwidth}
    \begin{tikzpicture}[american voltages,scale=1.2,thick]
      \draw (0,0) to[battery,l=$\mathcal E$] (0,2) to[short,-*] (0.5,2);
      \draw (.51,2.3) to[short,*-*] (1,2)--(1.5,2) to [R=$R$] (3,2)
      to [L=$L$] (3,0)--(0,0);
    \end{tikzpicture}

    \column{.67\textwidth}
    Applying the voltage law:

    \eq{-.3in}{
      \mathcal E-V_R-V_L=0\quad\rightarrow\quad
      \mathcal E-IR-L\diff It=0
    }

    \vspace{-.1in}where $V_R$ is the voltage across the resistor, and $V_L$ is
    the voltage across the inductor. This is a first-order differential
    equation similar to those for RC circuits. Separating variables and
    integrating:

    \eq{-.1in}{
      \int\frac{\dl t}L=\int\frac{\dl I}{\mathcal E-IR}
    }
    
    and after a bit of work\ldots
  \end{columns}
\end{frame}



\begin{frame}{Analyzing LR Circuits}
  \begin{columns}
    \column{.33\textwidth}
    \begin{tikzpicture}[american voltages,scale=1.2,thick]
      \draw (0,0) to[battery,l=$\mathcal E$] (0,2) to[short,-*] (0.5,2);
      \draw (.51,2.3) to[short,*-*] (1,2)--(1.5,2) to [R=$R$] (3,2)
      to [L=$L$] (3,0)--(0,0);
    \end{tikzpicture}

    \column{.67\textwidth}
    We find the current through the circuit:

    \eq{-.1in}{
      I(t)=\frac{\mathcal E}R\left(1-e^{-t/\tau}\right)
      \quad\text{\normalsize where}\quad
      \tau=\frac LR
    }
  \end{columns}
  \vspace{.2in}
  \begin{itemize}
  \item At $t=0$, when the switch is closed, the back emf prevents current from
    flowing, therefore the inductor acts like an open circuit
  \item As $t\rightarrow\infty$, magnetic flux no longer changes with time, and
    the inductor (which has no resistance) acts like a short circuit
  \end{itemize}
\end{frame}



\begin{frame}{Practical LR Circuits}
  Like the RC circuit, while we can analytically solve for $I(t)$ in a simple
  LR circuit, practical configurations (where multiple inductors and resistors
  are in series or in parallel) are complex problems. For example, suppose we
  replaced the capacitor in the previous example with an inductor:
  \begin{center}
    \begin{tikzpicture}[scale=1.2,american voltages,thick]
      \draw (0,0) to[battery1,l=12<\volt>] (0,2) to[R=4<\ohm>] (2,2)
      to[short,-*] (2.46,2.3);
      \draw (2.5,2) to[short,*-] (3,2) to[short] (4.5,2)
      to[L=6<\henry>] (4.5,0)--(0,0);
      \draw (3,0) to[R=8<\ohm>] (3,2);
    \end{tikzpicture}
  \end{center}  
\end{frame}



\begin{frame}{Practical RC Circuits: Example}
  \begin{columns}[T]

    \column{.5\textwidth}{\footnotesize
      At $t=0$, magnetic flux through the inductor is rapidly changing,
      generating a back \emph{emf}, therefore it acts like an open circuit. The
      back \emph{emf} is equal to the voltage across the \SI8{\ohm}
      resistor.\par}
    \begin{center}
      \begin{tikzpicture}[american voltages,thick]
        \draw (0,0) to[battery1,l=12<\volt>] (0,2) to[R=4<\ohm>] (2,2)--(4.5,2)
        to[short,-o] (4.5,1.2);
        \draw (4.5,.8) to[short,o-](4.5,0)--(0,0);
      \draw (3,0) to[R=8<\ohm>] (3,2);
      \end{tikzpicture}
    \end{center}
    
    \column{.5\textwidth}{\footnotesize
      As $t\rightarrow\infty$, magnetic flux through the inductor is no longer
      channging, and no back \emph{emf} is generated. The inductor acts like a
      short circuit. There is no current through the \SI8{\ohm} resistor.\par}
    \begin{center}
      \begin{tikzpicture}[american voltages,thick]
        \draw (0,0) to[battery1,l=12<\volt>] (0,2) to[R=4<\ohm>] (2,2)
        --(4.5,2)--(4.5,0)--(0,0);
        \draw (3,0) to[R=8<\ohm>] (3,2);
      \end{tikzpicture}
    \end{center}
  \end{columns}
\end{frame}


\section{LC Circuit}

\begin{frame}{LC Circuit}
  The final type of circuit in AP Physics is the \textbf{LC circuit}, which
  consists of inductors and capacitors (but no resistors). In its simplest
  form, the circuit has an inductor and capacitor (initially charged to $V_c$)
  connected in series:
  \begin{center}
    \begin{tikzpicture}[american voltages,thick]
      \draw (0,0) to[L=$L$] (0,2)--(3,2) to[C=$C$] (3,0)--(0,0);
    \end{tikzpicture}
  \end{center}
  We apply the Kirchhoff's voltage law:
  
  \eq{-.2in}{
    V_C-V_L=0
    \quad\longrightarrow\quad
    \frac QC+L\diff It=0
    \quad\longrightarrow\quad
    \diff{}t\left(\frac QC+L\diff It=0\right)
  }
  
  Since both sides are continuously differentiable, we can differentiate the
  entire equation w.r.t.\ time\ldots
\end{frame}



\begin{frame}{LC Circuits}
  Since both sides are continuously differentiable, we can differentiate the
  equation w.r.t.\ time, which gives us a second-order ordinary differential
  equation with constant coefficients:

  \eq{-.1in}{
    \frac1C\diff Qt+L\diff[2]It=0
    \quad\longrightarrow\quad
    \frac1{LC}I+\diff[2]It=0
  }

  The solution to such an equation is the simple harmonic motion\footnote{This
  was studied in AP Physics C Mechanics for spring-mass and pendulum systems.
  Of couse there is no reason that the same math won't appear in E\&M!}.
  Current flows as an alternating current (AC); the natural frequency
  $\omega_0$ is based on the inductance and capacitance:

  \eq{-.1in}{
    I(t)=I_0\sin(\omega_0 t)\quad\text{\normalsize where}\quad
    \omega_0=\frac1{\sqrt{LC}}
  }
\end{frame}
\end{document}
