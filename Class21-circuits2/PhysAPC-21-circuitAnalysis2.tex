\documentclass[12pt,aspectratio=169]{beamer}
\usetheme{metropolis}
\setbeamersize{text margin left=.5cm,text margin right=.5cm}
\usepackage[lf]{carlito}
\usepackage{siunitx}
\usepackage{tikz}
\usepackage{mathpazo}
\usepackage{bm}
\usepackage{mathtools}
\usepackage[ISO]{diffcoeff}
\diffdef{}{ op-symbol=\mathsf{d} }
\usepackage{xcolor,colortbl}

\setmonofont{Ubuntu Mono}
\setlength{\parskip}{0pt}
\renewcommand{\baselinestretch}{1}

\sisetup{
  inter-unit-product=\cdot,
  per-mode=symbol
}

\tikzset{
  >=latex
}

%\newcommand{\iii}{\hat{\bm\imath}}
%\newcommand{\jjj}{\hat{\bm\jmath}}
%\newcommand{\kkk}{\hat{\bm k}}


\title{Class 21: Circuit Analysis, Part 2}
\subtitle{AP Physics C}
\author[TML]{Dr.\ Timothy Leung}
\institute{Olympiads School}
\date{Updated: Summer 2022}

\newcommand{\pic}[2]{
  \includegraphics[width=#1\textwidth]{#2}
}
\newcommand{\eq}[2]{
  \vspace{#1}{\Large
    \begin{displaymath}
      #2
    \end{displaymath}
  }
}
%\newcommand{\iii}{\ensuremath\hat{\bm{\imath}}}
%\newcommand{\jjj}{\ensuremath\hat{\bm{\jmath}}}
%\newcommand{\kkk}{\ensuremath\hat{\bm{k}}}
\newcommand{\iii}{\ensuremath\hat\imath}
\newcommand{\jjj}{\ensuremath\hat\jmath}
\newcommand{\kkk}{\ensuremath\hat k}



\begin{document}

\begin{frame}
  \maketitle
\end{frame}



\section{Capacitors in Circuit}

\begin{frame}{Capacitors in Parallel}
  \begin{center}
    \begin{tikzpicture}[scale=1.2,thick]
      \draw (0,1) to[short,o-] (1,1) to[C=$C_1$] (1,0) to[short,-o] (0,0);
      \draw (1,1)--(3,1) to[C=$C_2$] (3,0)--(1,0);
      \draw (3,1)--(5,1) to[C=$C_3\ldots$] (5,0)--(3,0);
    \end{tikzpicture}
  \end{center}
  From the voltage law, we know that the voltage across all the capacitors are
  the same, i.e.\ $V_1=V_2=V_3=\cdots=V$. We can express the total charge
  $Q_\text{tot}$ stored across all the capacitors in terms of capacitance and
  this common voltage $V$: 

  \eq{-.1in}{
    Q_\text{tot}=Q_1+Q_2+Q_3+\cdots=C_1V+C_2V+C_3V+\cdots
  }
  
  Factoring out $V$ from each term gives us the equivalent capacitance:

  \eq{-.1in}{
    \boxed{C_p=\sum_i C_i}
  }
\end{frame}



\begin{frame}{Capacitors in Series}
  Likewise, we can do a similar analysis to capacitors connected in series.
  \begin{center}
    \begin{tikzpicture}[scale=1.2,thick]
      \draw (0,0) to[C=$C_1$,o-] (1.25,0) to[C=$C_2$] (2.5,0)
      to[C=$C_3$,-o] (3.75,0);
    \end{tikzpicture}
  \end{center}
  The total voltage across these capacitors are the sum of the voltages across
  each of them, i.e.\ $V_\text{tot}=V_1+V_2+V_3+\cdots$
  
  \vspace{.1in}The charge stored on all the capacitors must be the same! The
  total voltage in terms of capacitance and charge is:

  \eq{-.1in}{
    V_\text{tot}=\frac Q{C_1}+\frac Q{C_2}+\frac Q{C_3}+\cdots
  }
\end{frame}




\begin{frame}{Equivalent Capacitance in Series}
  The inverse of the equivalent capacitance for $N$ capacitors connected in
  series is the sum of the inverses of the individual capacitance.

  \eq{-.1in}{
    \boxed{ \frac1{C_s}=\sum_i\frac1{C_i} }
  }
  
  Make sure we don't confuse ourselves with resistors.
\end{frame}



\begin{frame}{How Do We Know That Charges Are The Same?}
  It's simple to show that the charges across all the capacitors are the same
  \begin{center}
    \begin{tikzpicture}[scale=1.5]
      \draw[thick](0,0) to[C=$C_1$,o-] (1.25,0) to[C=$C_2$,-o] (2.5,0);
      \draw[dashed](0.625,-.75) rectangle (1.875,1.2);
    \end{tikzpicture}
  \end{center}
  The capacitor plates and the wire connecting them are really one piece of
  conductor. There is nowhere for the charges to leave the conductor, therefore
  when charges are accumulating on $C_1$, $C_2$ must also have the same charge
  because of conservation of charges.
\end{frame}



\section{RC Circuits}

\begin{frame}{Circuits with Resistors and Capacitors}
  An \textbf{RC circuit} is one that has both resistors and capacitors. The
  simplest form is a resistor and capacitor connected in series, and then
  connect to a voltage source.
  \begin{center}
    \begin{tikzpicture}[american voltages,thick]
      \draw (0,0) to[battery,l=$V$] (0,2) to[R=$R$] (2,2)
      to[C=$C$] (2,0)--(0,0);
    \end{tikzpicture}
  \end{center}
  Because of the nature of capacitors, the current through the circuit will not
  be steady as were the case with only resistors.
\end{frame}



\begin{frame}{Discharging a Capacitor}
  \begin{columns}
    \column{.3\textwidth}
    \centering
    \begin{tikzpicture}[scale=1.4,thick]
      \draw (0,0) to[R=$R$] (0,2)--(2,2) to[C=$C$] (2,0)--(0,0);
    \end{tikzpicture}
    
    \column{.7\textwidth}
    The analysis starts with something simpler. There is no voltage source,
    and the capacitor is already charged to $V_c=Q_\text{tot}/C$. What happens
    when the current begin to flow?
    
    \vspace{.15in}As current starts to flow, the charge on the capacitor
    decreases. Over time the current decreases, until the capacitor is fully
    discharged, and current stops flowing.
  \end{columns}

  \vspace{.15in}Now we apply the voltage law for the circuit. In this case,
  as the current flow in the circuit \emph{decreases} the total charge in the
  capacitor, $I=-\diff Q/t$, while the voltage across a capacitor is
  $V_c=Q/C$:

  \eq{-.1in}{
    V_c-IR=0\quad\rightarrow\quad
    \frac QC+R\diff Qt=0}
\end{frame}


\begin{frame}{Discharging a Capacitor}
  Separating the variable gives the first-order linear differential equation:

  \eq{-.1in}{
    \frac{\dl Q}Q = \frac{-\dl t}{RC}
  }
  
  which we can now integrate and ``exponentiate'':

  \eq{-.1in}{
    \int\frac{\dl Q}Q = \int\frac{-\dl t}{RC}
    \quad\rightarrow\quad
    \ln Q=\frac{-t}{RC} + K
    \quad\rightarrow\quad
    Q=e^Ke^{-t/RC}
  }

  The constant of integration $K$ is the initial charge on the capacitor
  $Q_\text{tot}$:

  \eq{-.1in}{
    e^K=Q_\text{tot}
  }
\end{frame}



\begin{frame}{Discharging a Capacitor}
  The expression of charge across the capacitor is time-dependent:

  \eq{-.1in}{
    \boxed{Q(t)=Q_0e^{-t/\tau}}
  }

  where $Q_0=Q_\text{tot}$ is the initial charge on the capacitor, and $\tau=RC$
  is called the \textbf{time constant}. Taking the time derivative of $Q(t)$
  gives us the current through the circuit:

  \eq{-.1in}{
    \boxed{
      I(t)=\diff Qt=I_0e^{-t/\tau}
    }
  }

  where the initially current at $t=0$ is given by
  $I_0=Q_\text{tot}/\tau$.
\end{frame}



\begin{frame}{Charging a Capacitor}
  \begin{columns}
    \column{.3\textwidth}
    \centering
    \begin{tikzpicture}[american voltages,scale=1.25,thick]
      \draw (0,0) to[battery,l=$\mathcal E$] (0,2) to[R=$R$] (2,2)
      to[C=$C$] (2,0)--(0,0);
    \end{tikzpicture}
    
    \column{.7\textwidth}
    In charging up the capacitor, we go back to our original circuit, and apply
    the voltage law, then substitute the expression for current and voltage
    across the capacitor:

    \eq{-.1in}{
      \mathcal E-R\diff Qt-\frac QC=0
    }
  \end{columns}
  \vspace{.2in}Again, separating variables, and integrating, we get:

  \eq{-.1in}{
    \int\frac{\dl Q}{C\mathcal E-Q}=\int\frac{\dl t}{RC}
    \quad\rightarrow\quad-\ln(C\mathcal E-Q)=\frac t{RC}+K
  }
 \end{frame}



\begin{frame}{Charging a Capacitor}
  ``Exponentiating'' both sides, we have
  
  \eq{-.1in}{
    C\mathcal E-Q=e^Ke^{-t/RC}
  }

  To find the constant of integration $K$, we note that at $t=0$, the charge
  across the capacitor is $0$, and we get

  \eq{-.1in}{
    e^K=C\mathcal E=Q_\text{tot}
  }

  which is the charge stored in the capacitor at the end. Substitute this back
  into the equation:

  \eq{-.1in}{
    \boxed{Q(t)=Q_\text{tot}(1-e^{-t/RC})}
  }
\end{frame}




\begin{frame}{Capacitors}

  \eq{-.1in}{\boxed{Q(t)=Q_\text{tot}(1-e^{-t/\tau})}}

  Charging a capacitor has the same time constant $\tau=RC$ as during
  discharge. We can also differentiate to find the current through the circuit;
  it is identical to the equation for discharge:

  \eq{-.1in}{
    \boxed{I(t)=\diff Qt=I_0e^{-t/\tau}}
  }

  where the initial current is given by $I_0=Q_\text{tot}/\tau=\mathcal E/R$.
  This makes sense because $V_C(t=0)=0$, and all of the energy must be
  dissipated through the resistor. Similarly, $I_\infty=0$.
\end{frame}



\begin{frame}{Two Small Notes}
  \begin{enumerate}
  \item When a capacitor is uncharged, there is no voltage across the plate,
    it acts like a short circuit.
  \item When a capacitor is charged, there is a voltage across it, but no
    current flows \emph{through} it. Effectively it acts like an open circuit.
  \end{enumerate}
\end{frame}



\begin{frame}{A Slightly More Difficult Problem}
  \begin{center}
    \begin{tikzpicture}[scale=1.2,american voltages,thick]
      \draw (0,0) to[battery1,l=12<\volt>] (0,2) to[R=4<\ohm>] (2,2)
      to[short,-*] (2.46,2.3);
      \draw (2.5,2) to[short,*-] (3,2) to[short] (4,2)
      to[C=6<\micro\farad>] (4,0)--(0,0);
      \draw (3,0) to[R=8<\ohm>] (3,2);
    \end{tikzpicture}
  \end{center}
  \textbf{Example:} The capacitor in the circuit is initially uncharged. Find
  the current through the battery
  \begin{enumerate}
  \item Immediately after the switch is closed
  \item A long time after the switch is closed
  \end{enumerate}
\end{frame}



\section{LR Circuits}

\begin{frame}{Circuits with Inductors}
  \begin{itemize}
  \item Coils and solenoids in circuits are known as ``inductors'' and have
    large self inductance $L$
  \item Self inductance prevents currents rising and falling instantaneously
  \item A basic circuit containing a resistor and an inductor is called an
    \textbf{LR circuit}:
    \begin{center}
      \begin{tikzpicture}[american voltages,scale=1.2,thick]
        \draw (0,0) to[battery,l=$\mathcal E$] (0,2) to[short,-*] (0.5,2);
        \draw (0.51,2.3) to[short,*-*] (1,2)--(1.5,2) to [R=$R$] (3,2)
        to [L=$L$] (3,0)--(0,0);
      \end{tikzpicture}
    \end{center}
  \end{itemize}
\end{frame}



\begin{frame}{Analyzing LR Circuits}
  \begin{columns}
    \column{.35\textwidth}
    \begin{tikzpicture}[american voltages,scale=1.2,thick]
      \draw (0,0) to[battery,l=$\mathcal E$] (0,2) to[short,-*] (0.5,2);
      \draw (.51,2.3) to[short,*-*] (1,2)--(1.5,2) to [R=$R$] (3,2)
      to [L=$L$] (3,0)--(0,0);
    \end{tikzpicture}

    \column{.65\textwidth}
    Applying Kirchhoff's voltage law:

    \eq{-.1in}{
      \mathcal E-IR-L\diff It=0
    }

    Following the same procedure as charging a capacitor, the time-dependent
    current is found to be:

    \eq{-.1in}{
      I(t)=\frac{\mathcal E}R\left(1-e^{-t/\tau}\right)
    }

    Where the time constant $\tau$ is defined as:
    
    \eq{-.1in}{
      \tau=\frac LR
    }
  \end{columns}
\end{frame}



\section{LC Circuit}

\begin{frame}{LC Circuit}
  The final type of circuit in AP Physics is the L-C circuit. In its simplest
  form, the circuit has an inductor and capacitor connected in series:
  \begin{center}
    \begin{tikzpicture}[american voltages,scale=1.2,thick]
      \draw (0,0) to[L=$L$] (0,2)--(2,2) to[C=$C$] (2,0)--(0,0);
    \end{tikzpicture}
  \end{center}
  We apply the Kirchhoff's voltage law:
  
  \eq{-.2in}{
    -V_L-V_C=0
    \quad\rightarrow\quad
    L\diff It+\frac QC=0
  }
\end{frame}



\begin{frame}{LC Circuits}
  Since both terms are continuously differentiable, we can differentiate both
  sides of the equation, which gives:

  \eq{-.1in}{
    L\diff[2]It+\frac1C\diff Qt=0
  }

  In fact, the above equation a second-order ordinary differential equation
  with constant coefficients.

  \eq{-.2in}{
    \diff[2]It+\frac1{LC}I=0
  }

  The solution to such an equation is the simple harmonic motion.

  \eq{-.1in}{
    I(t)=I_0\sin(\omega t)\quad\text{where}\quad
    \omega=\frac1{\sqrt{LC}}
  }
\end{frame}
\end{document}
