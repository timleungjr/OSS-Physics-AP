\documentclass[12pt,aspectratio=169]{beamer}
\usetheme{metropolis}
\setbeamersize{text margin left=.5cm,text margin right=.5cm}
\usepackage[lf]{carlito}
\usepackage{siunitx}
\usepackage{tikz}
\usepackage{mathpazo}
\usepackage{bm}
\usepackage{mathtools}
\usepackage[ISO]{diffcoeff}
\diffdef{}{ op-symbol=\mathsf{d} }
\usepackage{xcolor,colortbl}

\setmonofont{Ubuntu Mono}
\setlength{\parskip}{0pt}
\renewcommand{\baselinestretch}{1}

\sisetup{
  inter-unit-product=\cdot,
  per-mode=symbol
}

\tikzset{
  >=latex
}

%\newcommand{\iii}{\hat{\bm\imath}}
%\newcommand{\jjj}{\hat{\bm\jmath}}
%\newcommand{\kkk}{\hat{\bm k}}


\title{Class 22B: Maxwell's Equations}
\subtitle{Advanced Placement Physics C}
\author[TML]{Dr.\ Timothy Leung}
\institute{Olympiads School}
\date{Updated: Summer 2022}

\newcommand{\pic}[2]{
  \includegraphics[width=#1\textwidth]{#2}
}
\newcommand{\eq}[2]{
  \vspace{#1}{\Large
    \begin{displaymath}
      #2
    \end{displaymath}
  }
}
%\newcommand{\iii}{\ensuremath\hat{\bm{\imath}}}
%\newcommand{\jjj}{\ensuremath\hat{\bm{\jmath}}}
%\newcommand{\kkk}{\ensuremath\hat{\bm{k}}}
\newcommand{\iii}{\ensuremath\hat\imath}
\newcommand{\jjj}{\ensuremath\hat\jmath}
\newcommand{\kkk}{\ensuremath\hat k}


\begin{document}

\begin{frame}
  \maketitle
\end{frame}


\begin{frame}{Making Amp\`{e}re's Law Better}
  Amp\`{e}re's law, as we know it, only applies to \emph{steady} currents $I_c$:

  \eq{-.1in}{
    \boxed{
      \oint_C \vec B\cdot\dl\vec\ell=\mu_0 I_c
    }
  }
  However,
  \begin{itemize}
  \item Current are usually not steady in RC, RL, LC or RLC circuits
  \item Applying Amp\`{e}re's law at a charging/discharging capacitor gives an
    ambiguous answer
  \end{itemize}
\end{frame}



\begin{frame}{Modifying Amp\`{e}re's Law for Unsteady Current}
  \begin{center}
    \pic{.3}{mag_displacement_fig3}
  \end{center}
  Four surfaces bounded by the same circular Amperian loop
  (think blowing a soap bubble). Surfaces \numlist{1;2;4} have currents
  penetrating through them, but surface 3 does not.
\end{frame}



\begin{frame}{Modifying Amp\`{e}re's Law}
  This might give a better view of what the ``soap bubble'' looks like
  \begin{center}
    \pic{.5}{bubble}
  \end{center}
  There is no current through the surface $A_2$ (same as surface \num{3} in the
  last slide), but there is definitely a changing \emph{electric flux}
\end{frame}



\begin{frame}{Maxwell's Modification to Amp\`{e}re's Law}
  James Clerk Maxwell, in 1860, proposed a modification to Amp\`{e}re's Law
  to make it work with unsteady current as well

  \eq{-.1in}{
    \boxed{
      \oint\vec B\cdot\dl\vec\ell=
      \mu_0\left( I + \epsilon_0 \diff{\Phi_q}t \right)
    }
  }


  Maxwell called the correction term $\displaystyle\epsilon_0\diff{\Phi_q}t$
  \textbf{displacement current}.
  \begin{itemize}
  \item The word ``displacement'' has historical roots, but no physical meaning
  \item However, ``current'' means that the effect of changing the electric
    flux is indistinguishable from real currents in producing magnetic field
  \end{itemize}
\end{frame}



\begin{frame}{Maxwell's Equations}
  \begin{itemize}
  \item Maxwell recognized the relationship between electricity and
    magnetism in \textbf{Gauss's law}, \textbf{Faraday's law} and
    \textbf{Amp\`{e}re's law}
  \item Combined them into a unified set of equations, now known as
    \textbf{Maxwell's equations} for electrodynamics
  \end{itemize}
\end{frame}



\begin{frame}{Maxwell's Equations in Integral Form}
  Maxwell's equations can be expressed in its integral form, which is how we
  have studied the equations in the first place:

  \vspace{-.2in}{\large
    \begin{align*}
      \oint\vec E\cdot\dl\vec A &=\frac q{\epsilon_0} &
      \text{\normalsize (Gauss, for $\vec E$)}\\
      \oint\vec B\cdot\dl\vec A &= 0 &
      \text{\normalsize (Gauss, for $\vec B$)}\\
      \oint\vec E\cdot\dl\vec\ell &=-\diff{\Phi_m}t &
      \text{\normalsize (Faraday)}\\
      \oint\vec B\cdot\dl\vec\ell &
      =\mu_0I+\mu_0\epsilon_0\diff{\Phi_q}t &
      \text{\normalsize (Amp\`{e}re-Maxwell)}
    \end{align*}
  }
\end{frame}



\begin{frame}{Maxwell's Equations in a Vacuum}
  \begin{columns}
    \column{.35\textwidth}
    \large
    \begin{align*}
      \oint\vec E\cdot\dl\vec A &= 0 \\
      \oint\vec B\cdot\dl\vec A &= 0 \\
      \oint\vec E\cdot\dl\vec\ell &=-\diff{\Phi_m}t \\
      \oint\vec B\cdot\dl\vec\ell &=\mu_0\epsilon_0\diff{\Phi_q}t\\
    \end{align*}

    \column{.65\textwidth}
    In a vacuum, we can remove all references to matter in the equation, and
    Maxwell's equations simplifies.
    \begin{itemize}
    \item The equations show ``symmetry''
    \item Magnetic and electric fields are on equal footing
    \item In a vacuum where charges are currents are absent, the only source of
      either field is a change in the other field
    \end{itemize}
  \end{columns}
\end{frame}



\begin{frame}{Maxwell's Equations in Differential Form}
  \begin{columns}
    \column{.25\textwidth}
    \large
    \begin{align*}
      \nabla\cdot\vec E &= 0\\
      \nabla\cdot\vec B &= 0\\
      \nabla\times\vec E &=-\diffp{\vec B}t \\
      \nabla\times\vec B &=\mu_o\epsilon_o\diffp{\vec E}t
    \end{align*}

    \column{.7\textwidth}
    \begin{itemize}
    \item Maxwell's equations are usually expressed in \emph{differential} form,
      which is obtained using vector calculus. Follow
      [\underline{\href{https://www.wikihow.com/Convert-Maxwell\%27s-Equations-into-Differential-Form}{this link}}] if you want to see how it's done.
    \item The differential form shows how the \emph{time derivatives} of
      $\vec E$ and $\vec B$ are related to the \emph{spatial derivatives}
      of the other field
    \item The last two equations (Faraday's and Amp\`{e}re's laws) together
      represent two set of second order partial differential equations (one for
      each field), the solution of which represents a traveling wave
    \end{itemize}
  \end{columns}
\end{frame}



\begin{frame}{Electromagnetic (EM) Wave}
  Maxwell's equations show that an ``electromagnetic wave'' must exist. In a
  simple case where electric and magnetic fields only vary in $x$ and time $t$
  only, i.e.\ $E=E(x,t)$ and $B=B(x,t)$, Faraday's and Amp\`{e}re's laws reduce
  to:

  \eq{-.1in}{
    \diffp Ex=-\diffp Bt \quad\quad
    \diffp Bx=-\mu_0\epsilon_0\diffp Et
  }

  (A negative sign appears on the right-hand side because we have ignored some
  of the vector operations.) Taking the spatial derivative of $E$ with respect
  to $x$ on both side of Faraday's law, and switch the order of
  differentiation, we get:

  \eq{-.1in}{
    \diffp*{\left(\diffp Ex\right)}x =-\diffp*{\left(\diffp Bt\right)}x
    \quad\rightarrow\quad
    \diffp[2]Ex = -\diffp*{\left(\diffp Bx\right)}t
  }
\end{frame}



\begin{frame}{Electromagnetic (EM) Wave}
  But we already have an expression for $\partial B/\partial x$ from
  Amp\`{e}re's law:

  \eq{-.1in}{
    \diffp[2] Ex = \diffp*{\left(\diffp Bx \right)}t
    =-\diffp*{\left(-\mu_0\epsilon_0\diffp Et\right)}t
  }

  Rearranging the terms on the right hand side, we get

  \eq{-.1in}{
    \diffp[2] Ex = \mu_0\epsilon_0\diffp[2] Et
  }
  
  This is the standard form of the \textbf{wave equation} (a second-order
  partial differential equation):

  \eq{-.1in}{
    \diffp[2]{\Psi}x = \frac1{v^2}\diffp[2]{\Psi}t
  }
\end{frame}



\begin{frame}{Electromagnetic (EM) Wave}
  \begin{itemize}
  \item ``Second-order'' means that the equation deals with second derivatives,
    in this case, in $x$ and in $t$.
  \item ``Partial'' means the equation involves partial derivatives (i.e.\
    when a function has more than one variables, and you only differentiate
    against one variable)
  \item We can also repeat the exercise by first differentiating Amp\`{e}re's
    law to get

    \eq{-.1in}{
      \diffp[2] Bx = \mu_0\epsilon_0\diffp[2] Bt
    }
  \end{itemize}
\end{frame}



\begin{frame}{Electromagnetic (EM) Wave}
  The wave equation shows that disturbances in electric and magnetic fields
  propagate as an electromagnetic wave (``EM wave'') with a universal speed
  generally referred to as the \textbf{speed of light}.

  \eq{-.1in}{
    v=c_0=\frac1{\sqrt{\mu_0\epsilon_0}}=\SI{299792458}{\metre\per\second}
  }

  The simplified 1D example cannot (because we have ignored the cross-product)
  that $\vec E$ and $\vec B$ are actually perpendicular to each other
\end{frame}



\begin{frame}{Electromagnetic (EM) Wave}
  A EM wave is considered to be \textbf{polarized} if both $\vec E$
  (and therefore) $\vec B$ of the wave are confined to a single plane. The
  direction of the polarization is the direction of $\vec E$.
  \begin{center}
    \pic{.4}{em-20field}
  \end{center}
\end{frame}



\begin{frame}{``Failure'' of Maxwell's Equation}
  A peculiar feature of Maxwell's equation:
  \begin{itemize}
  \item When applying \emph{Galilean transformation} (our classical equation for
    \emph{relative motion}) to Maxwell's equations, they seem to ``fail''
  \item Gauss's law for magnetism break down: magnetic field lines appear to
    have beginnings/ends
  \item So does that mean that in \emph{some} inertial frames of reference,
    Maxwell's equations are valid, but in others, they are not?
  \item Physicists theorized that, perhaps, there is/are actually
    \emph{preferred} inertial frame(s) of references
  \item This violate the long-standing \emph{principle of relativity}, which
    says that \emph{the laws of physics are equal in all inertial frames of
    reference}
  \end{itemize}
\end{frame}



\begin{frame}{Making The Equations Work Again}
  Maxwell's equations didn't ``fail''; it was our understanding of space and
  time that needed to change
  \begin{itemize}
  \item Albert Einstein believed in the principle of relativity, and rejected
    the concept of a preferred frame of reference
  \item In Maxwell's equations, the speed of an electromagnetic wave (speed of
    light) is independent of the frame of reference
  \item In order to make the equations to work again, Einstein revisited the
    most basic concepts involved in our understanding of physics: space and
    time
  \end{itemize}
\end{frame}



\begin{frame}{Einstein and Special Relativity}
  Einstein's Postulates of Special Relativity:
  \begin{enumerate}
  \item\textbf{Principle of relavity:} All laws of physics must apply equally
    in all inertial frames of reference.
  \item\textbf{Principle of invariant speed of light:} As measured in any
    inertial frame of reference, light always propagated in a vacuum with a
    definite velocity $c_0$ that is independent of the state of motion of the
    emitting body.
  \end{enumerate}
  Published in 1905 in the article \emph{On the Electrodynamics of Moving
    Bodies} when Einstein was 26 years old working as a patent clerk in
  Switzerland
\end{frame}
\end{document}
