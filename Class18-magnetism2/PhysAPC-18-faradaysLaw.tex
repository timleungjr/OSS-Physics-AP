\documentclass[12pt,aspectratio=169]{beamer}
\usetheme{metropolis}
\setbeamersize{text margin left=.5cm,text margin right=.5cm}
\usepackage[lf]{carlito}
\usepackage{siunitx}
\usepackage{tikz}
\usepackage{mathpazo}
\usepackage{bm}
\usepackage{mathtools}
\usepackage[ISO]{diffcoeff}
\diffdef{}{ op-symbol=\mathsf{d} }
\usepackage{xcolor,colortbl}

\setmonofont{Ubuntu Mono}
\setlength{\parskip}{0pt}
\renewcommand{\baselinestretch}{1}

\sisetup{
  inter-unit-product=\cdot,
  per-mode=symbol
}

\tikzset{
  >=latex
}

%\newcommand{\iii}{\hat{\bm\imath}}
%\newcommand{\jjj}{\hat{\bm\jmath}}
%\newcommand{\kkk}{\hat{\bm k}}


\title{Classes 18 \& 19: Faraday's Law and Magnetic Induction}
\subtitle{Advanced Placement Physics C}
\author[TML]{Dr.\ Timothy Leung}
\institute{Olympiads School}
\date{Updated: Summer 2022}

\newcommand{\pic}[2]{
  \includegraphics[width=#1\textwidth]{#2}
}
\newcommand{\eq}[2]{
  \vspace{#1}{\Large
    \begin{displaymath}
      #2
    \end{displaymath}
  }
}
%\newcommand{\iii}{\ensuremath\hat{\bm{\imath}}}
%\newcommand{\jjj}{\ensuremath\hat{\bm{\jmath}}}
%\newcommand{\kkk}{\ensuremath\hat{\bm{k}}}
\newcommand{\iii}{\ensuremath\hat\imath}
\newcommand{\jjj}{\ensuremath\hat\jmath}
\newcommand{\kkk}{\ensuremath\hat k}



\begin{document}

\begin{frame}
  \maketitle
\end{frame}


\section{Magnetic Flux}


\begin{frame}{Magnetic Flux}
  \begin{columns}
    \column{.4\textwidth}
    \pic1{flux2}
  
    \column{.6\textwidth}
    Magnetic flux is defined as:
    
    \eq{-.1in}{
      \boxed{\Phi_m=\int_\mathcal{S}\vec B\cdot\dl\vec A}
    }
    
    where $\vec B$ is the magnetic field, and $\dl\vec A$ is the infinitesimal
    area pointing \textbf{outwards}. Note that magnetic flux can also be
    expressed as:

    \eq{-.1in}{
      \boxed{\Phi_m=\int_\mathcal{S}(\vec B\cdot\hat n)\dl A}
    }

    where $\hat n$ is the outward normal direction
  \end{columns}
\end{frame}



\begin{frame}{Magnetic Flux Over a Closed Surface}
  The SI unit for magnetic flux is a \textbf{weber} (\si\weber), after
  German physicist Wilhelm Weber, who invented the electromagnetic telegraph
  with Carl Gauss. The unit is defined as:

  \eq{-.15in}{
    \SI1\weber=\SI1{\tesla\metre\squared}
  }
  
  The magnetic flux over any closed surface $\mathcal S$ must always be zero.
  This is \textbf{Gauss's law for magnetism}:

  \eq{-.1in}{
    \boxed{\oint_{\mathcal S}\vec B\cdot\dl\vec A=0}
  }

  Since magnetic field lines only exist as a loop, that means there should be
  equal amount of ``flux'' flowing out of a closed surface as entering the
  surface.
\end{frame}



\begin{frame}{Changing Magnetic Flux}
  Magnetic flux can change due to any of these three reasons:
  \begin{enumerate}
  \item A time-dependent magnetic field:
    \begin{displaymath}
      \vec B=\vec B(t)
    \end{displaymath}
  \item A time-dependent surface area:
    \begin{displaymath}
      \mathcal S=\mathcal S(t)
    \end{displaymath}
  \item The orientation of the magnetic field with the surface is
    time-dependent:
    \begin{displaymath}
      \dl\Phi_m(t)=%\vec B\cdot\dl\vec A=
      B\sin\left(\theta(t)\right)\dl A
    \end{displaymath}
  \end{enumerate}
\end{frame}



\section{Faraday's Law}

\begin{frame}{Faraday's Law}
  \textbf{Faraday's law} states that the time rate of change of magnetic flux
  induces electric field $\vec E$ in a circuit, and therefore an
  induced electromotive force, or \emph{emf}, $\mathcal E$ (voltage gain):

  \eq{-.1in}{
    \boxed{
      \mathcal E=-\oint\vec E\cdot\dl\vec\ell={\color{red}-}
      \diff{\Phi_m}t
    }
  }
  
  The negative sign {\color{red}highlighted in red} is the result of
  \textbf{Lenz's law}, which is related to the conservation energy
\end{frame}



\begin{frame}{Example}
  \vspace{.1in}
  A loop of wire is place inside a uniform time-dependent magnetic field
  $\vec B_\text{in}(t)$ into the page that is increasing in magnitude in time.
  The plane of the loop is perpendicular to $\vec B$.

  \vspace{.1in}
  \begin{tikzpicture}[scale=.35]
    \foreach \xx in {-4.5,-3,...,4.5}{
      \foreach \yy in {-4.5,-3,...,4.5}
      \node at (\xx,\yy) {\color{cyan}$\times$};
    }
    \node[cyan] at (2.75,5.4) {$\vec B_\text{in}$};
    \draw[ultra thick] circle (3);
    \uncover<2->{
      \draw[ultra thick,fill=magenta!50,opacity=.5] circle (3);
      
      \node[thick,draw=red,fill=red!5,text width=4.8cm]
      (flux) at (-14,2.8){\color{red}
        Magnetic flux through the wire loop into the page is increasing, and
        it is a function of time:
        
        \vspace{-.25in}\begin{displaymath}
          \Phi_m(t)=\int_{\mathcal S}\vec B(t)\cdot\dl\vec A=B(t)A
        \end{displaymath}\par
      };
      \draw[thick,->,red] (flux) to[out=0,in=110] (-.75,1.5)
      node[below=-2]{\scriptsize$\mathcal S$};
    }
    
    \uncover<3->{
      \node[thick,draw=green!60!black,fill=green!5,text width=4.8cm]
      (emf) at (-14,-4.8){\color{green!60!black}
        An electric field $\vec E$ is generated inside the wire, resulting in
        an induced \emph{emf} $\mathcal E$:
        
        \vspace{-.07in}\begin{displaymath}
          \mathcal E=\left|\diff{\Phi_m}t\right|
        \end{displaymath}\par
      };
      %\draw[thick,->,green!60!black] (flux)--(emf);
      \foreach \x in {0,45,90,...,360}
      \draw[rotate=\x,vectors,green!60!black] (3,0)--(3,2.5);
    }
    \uncover<4->{
      \node[thick,draw=blue!70!black,fill=blue!5,text width=4.7cm]
      (curr) at (14,0){\color{blue!70!black}
        An induced current runs in the wire in direction of the electric
        field (Ohm's law):
        
        \vspace{-.05in}\begin{displaymath}
          I=\frac{\mathcal E}R
        \end{displaymath}\par
        $R$ is the internal resistance of the wire loop
      };
      %\draw[thick,->,blue!70!black] (emf) to[out=0,in=270] (curr);
      
      \draw[vectors,blue!70!black] (2.7,0) arc (0:300:2.7) node[above]{$I$};
    }

    \uncover<5>{
      \node[text width=9.6cm,font=\normalsize] at (8,-7.5){
        One remaining (and very important) question: how do we know the
        direction of the electric field $\vec E$ and current $I$?
      };
    }
  \end{tikzpicture}
\end{frame}



\begin{frame}{Lenz's Law}
  To find the direction of the current, we apply \textbf{Lenz's law}:
  \begin{center}
    \fcolorbox{black}{yellow!10}{
      \begin{minipage}{.7\textwidth}
        \textbf{LENZ'S LAW:} The induced \emph{emf} and induced current are in
        such are direction as to oppose the change that produces them
      \end{minipage}
    }
  \end{center}
  The law comes from the conservation of energy, which disallows perpetual
  motion machines.
  %So basically, the conservation of energy
\end{frame}



%\begin{frame}{When Magnetic Flux is Changing}
%  \begin{itemize}
%  \item When the magnetic flux $\Phi_m$ is changing, an electromotive force
%    (\emph{emf}, $\mathcal E$) is created in the wire.
%  \item Unlike in a circuit, where the \emph{emf} is concentrated at the
%    terminals of the battery, the induced \emph{emf} is spread across the
%    entire wire.
%  \end{itemize}
%  \begin{columns}
%    \column{.3\textwidth}
%    \centering
%    \begin{tikzpicture}[scale=.35]
%      \foreach \xx in {-4.5,-3,...,4.5}{
%        \foreach \yy in {-4.5,-3,...,4.5}
%        \node at (\xx,\yy) {\color{cyan}$\times$};
%      }
%      \draw[gray,ultra thick] circle (2.9);
%      \draw[axes,rotate=45] (0,0)--(2.9,0) node[midway,left]{$r$};
%      \foreach \x in {0,90,...,360} {
%        \draw[rotate=\x,vectors,red] (2.9,0)--(2.9,2.5)
%        node[pos=0,above left]{$\vec E$};
%      }
%    \end{tikzpicture}
%    
%    \column{.7\textwidth}
%    \begin{itemize}
%    \item Since \emph{emf} is work per unit charge, that means that there is an
%      electric field inside the wire to move the charges.
%    \item In this example:
%      \begin{itemize}
%      \item Magnetic field $\vec B$ into the page
%      \item The direction of the electric field $\vec E$ corresponds to an
%        \emph{increase} in magnetic flux
%      \end{itemize}
%    \end{itemize}
%  \end{columns}
%\end{frame}



\section{AC Generator}

\begin{frame}{AC Generator}
  A simple AC (alternating current) generator makes use of the fact that a 
  coil rotating against a fixed magnetic field has a changing magnetic flux.
  \begin{center}
    \pic{.5}{generator}
  \end{center}
  Let's say the permanent magnets produce a uniform magnetic field $B$, and the
  coil between them has $N$ turns, and an area $A$. Now let's say that the coil
  is rotating with an angular frequency $\omega$.
\end{frame}



\begin{frame}{AC Generator}
  \begin{columns}
    \column{.35\textwidth}
    \pic1{generator}

    \column{.65\textwidth}
    When the coil is turning with an angular frequency $\omega$, the angle
    between the coil and the magnetic field is:
    
    \eq{-.15in}{
      \theta=\omega t+\theta_0
    } 

    \vspace{-.15in}The initial angle $\theta_0$ is not important. The magnetic
    flux through the coil is given by:
    
    \eq{-.3in}{
      \Phi_m = NAB\cos\theta = NAB\cos(\omega t+\theta_0)
    }

    \vspace{-.25in}as the generator turns. The induced \emph{emf} is the rate
    of change of magnetic flux:

    \eq{-.1in}{
      \mathcal E(t) =\left|\diff{\Phi_m}t\right| =
      \underbracket{NAB\omega}_{\mathcal E_\text{max}}\sin(\omega t+\theta_0)
    }
  \end{columns}
\end{frame}



%\begin{frame}{Motional EMF: What happens when I slide the rod to the right?}
%  \begin{columns}
%    \column{.45\textwidth}
%    \pic1{motional-emf-1}
%
%    \column{.55\textwidth}
%    When sliding the rod to the right with speed $v$, the magnetic flux through
%    the loop (and its rate of change) is:
%
%    \vspace{-.3in}{\large
%      \begin{align*}
%        \Phi_m &=BA=B\ell x\\
%        \mathcal E &= \diff{\Phi_m}r = B\ell\diff xt=B\ell v
%      \end{align*}
%    }
%    
%    We can use the magnetic force on the charges on the rod to find the
%    direction of the induced current $I$, and its magnitude is:
%
%    \eq{-.15in}{
%      I=\frac{\mathcal E}R
%      }
%  \end{columns}
%\end{frame}


\section{Motional EMF}

\begin{frame}{Motional EMF}
  What happens when I slide the rod to the right?
  \begin{center}
    \begin{tikzpicture}
      \node {\pic{.45}{motional-emf-1}};
      \uncover<2->{
        \fill[magenta!60,opacity=.6] (-2.75,-1.37) rectangle (.97,1.17);

        \node[thick,draw=red,fill=pink!30,text width=4cm]
        (phi) at (-4,-3.3){\color{red}
          The magnetic flux through the circuit is given by
          
          \vspace{-.15in}
          \begin{displaymath}
            \Phi_m=BA=B\ell x
          \end{displaymath}\par
        };
        \draw[axes,red] (phi)--(-1,-.7);
      }

      \uncover<3->{
        \node[thick,draw=green!60!black,fill=green!10,text width=6.4cm]
        (phi) at (2.5,-3.3){\color{green!60!black}
          When sliding the rod to the right at speed $v$, the rate
          of change in magnetic flux is

          \vspace{-.1in}
          \begin{displaymath}
            \mathcal E = \left|\diff{\Phi_m}t\right| = B\ell\diff xt=B\ell v
          \end{displaymath}\par
        };
      }

      \uncover<4->{
        \node[thick,draw=blue!70!black,fill=blue!10,text width=2.5cm]
        (i) at (5,0){\color{blue!70!black}
          This emf induces a current of:

          \vspace{-.1in}
          \begin{displaymath}
            I=\frac{\mathcal E}R
          \end{displaymath}\par
        };
      }
    \end{tikzpicture}
  \end{center}
%    \column{.55\textwidth}
%    
%    We can use the magnetic force on the charges on the rod to find the
%    direction of the induced current $I$, and its magnitude is:
%

%  \end{columns}
\end{frame}



%\begin{frame}{Motional EMF}
%  \begin{columns}
%    \column{.22\textwidth}
%    \begin{tikzpicture}[scale=1.5,thick]
%      \draw (1.5,0) to[battery1,l_=$\mathcal E$] (1.5,1.5)--(0,1.5)
%      to[R,l_=$R$] (0,0)--(1.5,0);
%    \end{tikzpicture}
%    
%    \column{.78\textwidth}
%    \begin{itemize}
%    \item An equivalent circuit is shown on the left
%    \item The amount of current can be found using Ohm's law: $V=IR$
%    \item Note that the ``motional emf'' produced is spread over the entire
%      circuit
%      \begin{itemize}
%      \item In constrast, in a voltaic cell (or battery), the \emph{emf} is
%        concentrated between the two terminals.
%      \end{itemize}
%    \end{itemize}
%  \end{columns}
%\end{frame}



\begin{frame}{Lenz's Law}
  Something very interesting happens when the current starts running on the
  wire.
  \begin{center}
    \pic{.35}{motional-emf-2}
  \end{center}
  It produces an ``induced magnetic field'' out of the page, in the opposite
  direction as the field that generated the current in the first place, just
  as we expect from Lenz's law.
\end{frame}







\begin{frame}{Solving Problems with Faraday's Law \& Lenz's Law}
  \centering
  \vspace{.2in}
  \begin{tikzpicture}[auto,node distance=105, thick]
    \node[fill=pink!40,font=\large] (a) {
      $\displaystyle\Phi_m=\int_{\mathcal S}\vec B\cdot\dl\vec A$};
    \node[fill=green!20,font=\large] (b) [right of=a] {
      $\displaystyle\left|\diff{\Phi_m}t\right|=\mathcal E$};
    \node[fill=blue!20,font=\large] (c) [right of=b] {
      $I=\dfrac{\mathcal E}R$};
    \node[fill=orange!20,font=\large] (d) [right of=c]{
      $P=I^2R=\dfrac{\mathcal E^2}R$};
    \draw[->] (a)--(b);
    \draw[->] (b)--(c);
    \draw[->] (c)--(d);
    \draw[<-,red] (a)--+(0,-1.3)
    node[draw=red,fill=pink!30,text width=2.7cm,below,font=\scriptsize]{
      Find the loop of circuit and calculate magnetic flux $\Phi_m$ through it
      as a function of time 
    };

    \draw[<-,green!60!black] (b)--+(0,-1.3) node[
      draw=green!60!black,fill=green!10,text width=2.8cm,below,font=\scriptsize
    ]{
      Take time derivative to find the induced \emph{emf} $\mathcal E$
      generated in the circuit loop
    };

    \draw[<-,blue!80!black] (c)--+(0,-1.3) node (cc) [
      draw=blue!80!black,fill=blue!10,text width=2.9cm,below,font=\scriptsize
    ]{
      Apply Ohm's law to find the induced current in the circuit. $R$ can be the
      internal resistance, or a resistor in the circuit
    };

    \draw[<-,orange] (d)--+(0,-1.3) node[
      draw=orange,fill=orange!10,text width=3.2cm,below,font=\scriptsize
    ]{
      Calculate power dissipated in the circuit, and compare to the
      power input
    };

    \node[fill=gray!20,draw=black,font=\scriptsize,text width=7.2cm]
    (I) at (1.9,-3.9) {Direction of induced current $I$ must produce a magnetic
      field that opposes the change in magnetic flux.
      \begin{itemize}
      \item If $\Phi_m$ is increasing, magnetic field must decrease it
      \item If $\Phi_m$ is decreasing, magnetic field must increase it
      \end{itemize}\par
    };
    \draw[->] (cc) to[out=270,in=0] (I);
  \end{tikzpicture}
\end{frame}



\section{Inductance}

\begin{frame}{Back \emph{emf}}
  Consider a very simple circuit consisting of a voltage source and a coil
  \begin{center}
    \begin{tikzpicture}[american voltages,scale=1.3,thick]
      \draw (0,0) to[battery1,l=$\mathcal E$] (0,1.5)
      to[short,-*] (.5,1.5);
      \draw (.55,1.7) to[short,*-*] (1,1.5)--(1.5,1.5) to[L] (1.5,0)--(0,0);
    \end{tikzpicture}
  \end{center}
  \begin{itemize}
  \item When the switch is closed and current begins to flow, the coil
    begins to generate a magnetic flux inside
  \item As the current changes (initially increasing with time), it
    self-induces a ``back \emph{emf}'' that opposes the change in current
  \item A current can't jump from zero to some value (or from some value to
    zero) instantaneously
  \end{itemize}
\end{frame}



\begin{frame}{Back \emph{emf}}
  \begin{center}
    \begin{tikzpicture}[american voltages,scale=1.3,thick]
      \draw (0,0) to[battery1,l=$\mathcal E$] (0,1.5)
      to[short,-*] (.5,1.5);
      \draw (.55,1.7) to[short,*-*] (1,1.5)--(1.5,1.5) to[L] (1.5,0)--(0,0);
    \end{tikzpicture}
  \end{center}
  \begin{itemize}
  \item Breaking the circuit causes the magnetic flux to change very rapidly
  \item The rapid change of $\Phi_m$ creates a large induced back \emph{emf}
    that is proportional to the time rate of change of magnetic flux
    $\Phi_m'(t)$
  \item The back \emph{emf} creates a large voltage drop across the switch
  \item Large voltage across two metal contact produces a very strong electric
    field--strong enough to tear electrons away from air molecules
    (``dielectric breakdown'')
  \item Air conducts electricity in the form of a ``spark''
  \end{itemize}
\end{frame}



\begin{frame}{Self Inductance}
  A solenoid carrying a current generates a magnetic field $B$; its magnitude at
  the core is proportional to the current $I$:

  \eq{-.2in}{
    B=\left[\frac{\mu_0N}\ell\right]I
  }

  Since $\vec B\propto I$, the magnetic flux through the core of the solenoid
  (really $\Phi_m=NBA$, where $A$ is the cross-sectional area of the solenoid
  and $N$ is the number of coils) is therefore also proportional to $I$, i.e.

  \eq{-.1in}{
    \boxed{\Phi_m=LI}
  }

  where $L$ is the called the \textbf{self inductance} of the coil.
\end{frame}



\begin{frame}{Self Inductance}
  For a solenoid, we can see that the self inductance is given by:

  \eq{-.1in}{
    \boxed{L=\frac{\Phi_m}I=\mu_0 n^2A\ell}
  }

  \begin{center}
    \begin{tabular}{l|c|c}
      \rowcolor{pink}
      \textbf{Quantity} & \textbf{Symbol} & \textbf{SI Unit} \\ \hline
      Self inductance & $L$ & \si\henry\\
      Vacuum permeability & $\mu_0$ & \si{\tesla.\metre\per\ampere} \\
      Number of coils per unit length & $n$ & \si\ampere \\
      Cross-section area of the solenoid & $A$ & \si{\metre\squared} \\
      Length of the solenoid & $\ell$ & \si\metre
    \end{tabular}
  \end{center}
  Note that $A\ell$ is the \emph{enclosed volume} of the solenoid. The SI unit
  for self inductance is a \textbf{henry} (\si\henry).
\end{frame}



\begin{frame}{Self Inductance and Induced EMF}
  If the current changes, the magnetic flux changes as well, therefore inducing
  an electromotive force in the circuit According Faraday's law:

  \eq{-.1in}{
    \boxed{\mathcal E=-\diff {\Phi_m}t=-L\diff It}
  }

  The self-induced \emph{emf} is proportional to the rate of change of current
\end{frame}



\begin{frame}{Magnetic Energy}
  At any instant, the magnitude of the induced emf is

  \eq{-.1in}{
    \mathcal E=L\diff It
  }
  so the power absorbed by the inductor is

  \eq{-.1in}{
    P(t)=\mathcal EI=LI\diff It
  }

  Integrating in time gives the magnetic (potential) energy that is stored in
  the magnetic field:

  \eq{-.2in}{
    U_m
    =\int_0^t P(t)\dl t
    =L \int I\diff It \dl t
    =L \int_0^I I\dl I=\frac12LI^2
  }
\end{frame}



\begin{frame}{Magnetic Energy}
  Just as a capacitor stores energy in its electric field, an inductor coil
  carrying a current $I$ stores energy in its magnetic field, given by:

  \eq{-.1in}{
    \boxed{U_m=\frac12LI^2}
  }

  We can also define a \textbf{magnetic energy density}:

  \eq{-.1in}{
    \boxed{\eta_m=\frac{B^2}{2\mu_0}}
  }
\end{frame}
\end{document}
