\documentclass[12pt,compress,aspectratio=169]{beamer}
\usetheme{metropolis}
\setbeamersize{text margin left=.5cm,text margin right=.5cm}
\usepackage[lf]{carlito}
\usepackage{siunitx}
\usepackage{tikz}
\usepackage{mathpazo}
\usepackage{bm}
\usepackage{mathtools}
\usepackage[ISO]{diffcoeff}
\diffdef{}{ op-symbol=\mathsf{d} }
\usepackage{xcolor,colortbl}

\setmonofont{Ubuntu Mono}
\setlength{\parskip}{0pt}
\renewcommand{\baselinestretch}{1}

\sisetup{
  inter-unit-product=\cdot,
  per-mode=symbol
}

\tikzset{
  >=latex
}

%\newcommand{\iii}{\hat{\bm\imath}}
%\newcommand{\jjj}{\hat{\bm\jmath}}
%\newcommand{\kkk}{\hat{\bm k}}


\usetikzlibrary{decorations.pathmorphing,patterns}

\title{Class 2: Dynamics}
\subtitle{Advanced Placement Physics C}
\author[TML]{Dr.\ Timothy Leung}
\institute{Olympiads School}
\date{Updated: Summer 2022}

\newcommand{\pic}[2]{
  \includegraphics[width=#1\textwidth]{#2}
}
\newcommand{\eq}[2]{
  \vspace{#1}{\Large
    \begin{displaymath}
      #2
    \end{displaymath}
  }
}
%\newcommand{\iii}{\ensuremath\hat{\bm{\imath}}}
%\newcommand{\jjj}{\ensuremath\hat{\bm{\jmath}}}
%\newcommand{\kkk}{\ensuremath\hat{\bm{k}}}
\newcommand{\iii}{\ensuremath\hat\imath}
\newcommand{\jjj}{\ensuremath\hat\jmath}
\newcommand{\kkk}{\ensuremath\hat k}


\begin{document}

\begin{frame}
  \maketitle
\end{frame}



\begin{frame}{Dynamics}
  While \textbf{kinematics} describes the motion of any object mathematically,
  \textbf{dynamics} describes \emph{what} causes motion to change
\end{frame}



\section{Laws of Motion}

\begin{frame}{First Law of Motion}
  \begin{center}
    \fbox{
      \begin{minipage}{.7\textwidth}
        \textbf{Law \#1: An object will remain in its state of rest or uniform
          motion, until a net external force is applied to it.}
      \end{minipage}
    }
  \end{center}
  \begin{itemize}
  \item Uniform motion means constant velocity; an object ``at rest'' is also
    in uniform motion with $\vec v=\vec 0$
  \item As long as an object moves in uniform motion, it must be that
    $\vec F_\text{net}=\vec 0$
  \item Common examples:
    \begin{itemize}
    \item A hockey puck sliding on very smooth ice has gravity and normal
      force, but the net force is zero
    \item A car traveling on a highway at \SI{100}{\kilo\metre\per\hour}
      has many forces acting on it, but the net force is zero 
    \end{itemize}
  \item{\color{red!80!black}This is a special case that assumes a constant
    mass}
  \end{itemize}
\end{frame}

\begin{frame}{Aristotle}
  Prior to Galileo, the understanding of motion came primarily from Greek
  philosopher Aristotle, who said that any body in motion will come its
  natural state: at rest
  \begin{itemize}
  \item Aristotle's idea ``makes sense''. Example: slide a book across a table,
    it will eventually stop
  \item Galileo pointed out that Aristotle's philosophy is ambiguous: absolute
    rest cannot be determined because all motion is relative
  \end{itemize}
\end{frame}



\begin{frame}{There is No Absolute Rest}
  \begin{center}
    \fbox{\textbf{\large Absolute rest cannot be determined}}
  \end{center}
  \begin{columns}
    \column{.35\textwidth}
    \pic{1.1}{coffeecup}
    
    \column{.65\textwidth}
    \vspace{-.3in}
    \begin{itemize}
    \item Objects that is stationary in one coordinate system can be moving
      relative to another (see \emph{relative motion} from last class)
    \item A cup of water is on the tray table on an airplane. The cup is
      stationary relative to the passengers, but moving relative to anyone
      outside
    \end{itemize}
  \end{columns}
\end{frame}



\begin{frame}{There is No Absolute Rest}
  \begin{center}
    \fbox{\textbf{\large Uniform motion is indistinguishable from rest}}
  \end{center}
  \begin{columns}
    \column{.25\textwidth}
    \pic{1.1}{Aging-Self-2}
    
    \column{.75\textwidth}
    \vspace{-.3in}
    \begin{itemize}
    \item If I drop a tennis ball while standing on the ground, the ball falls
      straight down with an acceleration due to gravity
    \item If I repeat this while on an airplane traveling in uniform motion
      (constant velocity) in the air, I will observe exactly the same thing
    \item The motion of the ball gives me no information about whether I am
      at rest on in uniform motion
    \end{itemize}
  \end{columns}
\end{frame}




\begin{frame}{Translational Equilibrium}
  If the net force on an object is zero ($\sum\vec F=\vec 0$) then the
  object is in a \emph{state of translational equilibrium}
  \begin{itemize}
  \item Dynamic equilibrium: the object is moving relative to us
  \item Static equilibrium: the object is not moving relative to us
  \end{itemize}
\end{frame}



\begin{frame}{Second Law of Motion}
  \begin{center}
    \fbox{
      \begin{minipage}{.7\textwidth}
        \textbf{Law \#2: The acceleration of an object is proportional to,
          and along the direction of, the net external force.}
      \end{minipage}
    }
  \end{center}
  \begin{itemize}
  \item\textcolor{red!80!black}{Like the first law, this is also a
    ``special case'' that assumes a constant mass.}
  \item The first two laws of motion can be summarized in the equation:

    \eq{-.1in}{
      \boxed{\vec F_\text{net}=\sum\vec F=m\vec a}
    }
  \item For non-constant mass, net force is the rate of change of momentum
    $\vec p$:

    \eq{-.1in}{
      \boxed{
        \vec F_\text{net}=\diff{\vec p}t
      }
    }
  \end{itemize}
\end{frame}



\begin{frame}{Mass}
  What is \textbf{mass} then? It is
  \begin{itemize}
  \item the property of an object that relates its acceleration to the force
    applied to it. Literally, it means:

    \eq{-.1in}{
      m\equiv\frac{F_\text{net}}a
    }
  \item Intrinsic to the object itself
  \item This is explicitly referred to as the object's \textbf{inertial mass}
  \end{itemize}
\end{frame}



\begin{frame}{Third Law of Motion}
  \begin{center}
    \fbox{
      \begin{minipage}{.9\textwidth}
        \textbf{Law \#3: For every action there is always an opposite and
          equal reaction; the mutual actions of two objects on each other are
          always equal, and directed to the other object.}
      \end{minipage}
    }
  \end{center}

  Whenever object (B) exerts any force on object (A), (A) will immediately
  exert a reaction force on (B) that is equal in magnitude but opposite in
  direction:

  \eq{-.1in}{
    \boxed{\vec F_\text{AB} = -\vec F_\text{BA}}
  }
  \begin{itemize}
  \item The action and reaction forces act on different objects!
  \item Third law is an application of the first law. Action/reaction forces
    are \emph{internal} forces to the system of objects.
  \end{itemize}
\end{frame}



\begin{frame}{Forces}
  A \textbf{force} is the interaction between the objects.
  \begin{itemize}
  \item When there is interaction, then forces are created
  \item A ``push'' or a ``pull''
  \end{itemize}
  There are two broad categories of forces:
  \begin{itemize}
  \item\textbf{Contact forces} act between two objects that are in contact
    with one another
  \item\textbf{Non-contact forces} act between two objects without them
    touching each other. They are also called ``action-at-a-distance'' force
  \end{itemize}
\end{frame}



\section{Common Forces}

\begin{frame}{Common Forces}
  Common forces that we encounter in AP Physics C include:
  \begin{itemize}
  \item Weight (gravitational force) $\vec F_g$
  \item Normal force $\vec F_N$
  \item Friction force (static $\vec f_s$ and kinetic $\vec f_k$)
  \item Tension force $\vec F_T$
  \item Applied force $\vec F_a$
  \item Spring force $\vec F_s$
  \item Aerdynamic drag $\vec F_D$ (a.k.a.\ fluid resistance)
  \item Buoyant force $\vec F_B$ (discussed in fluid mechanics, in AP Physics 2)
  \item Electrostatic force $\vec F_q$ (discussed in E \& M)
  \item Magnetic force $\vec F_m$ (discussed E \& M)
  \end{itemize}
\end{frame}



\begin{frame}{Gravity}
  Gravity is the force of attraction between all objects with mass
    
  \eq{-.1in}{
    \boxed{\vec F_g=m\vec g}
  }
  \begin{itemize}
  \item Near surface of Earth, use $g=\SI{9.81}{\metre\per\second\squared}$ (or
    $g=\SI{10}{\metre\per\second\squared}$ for your AP exam)
  \item $\vec F_g$ always points \emph{down}
  \item Based on the law of universal gravitation:

    \eq{-.1in}{
      \boxed{F_g=\frac{Gm_1m_2}{r^2}}\;\;
      \text{\normalsize where}\;\;
      G=\SI{6.674e-11}{\newton\metre\squared\per\kilo\gram\squared}
    }
  \end{itemize}    
\end{frame}



\begin{frame}{Normal Force}
  \begin{columns}
    \column{.22\textwidth}
    \centering
    \begin{tikzpicture}
      \draw[thick] (-1,0)--(2.5,0);
      \draw[mass] rectangle (1.5,1);
      \fill (.75,.5) circle (2pt);
      \draw[vectors] (.75,.5)--(.75,-.5) node[below]{$\vec F_g$};
      \draw[vectors] (.75,.5)--(.75,1.5) node[above]{$\vec F_N$};
    \end{tikzpicture}
    %$\vec F_g=-\vec F_N$\\(special case)
    
    \column{.78\textwidth}
    \begin{itemize}
    \item A force a surface exerts on another object that it is in contact with
    \item Always \textbf{perpendicular} to the contact surface
    \item\textbf{Special case:} When an object is on a horizontal surface
      with no additional applied force, the magnitude of the normal force is
      equal to the magnitude of the weight of the object, i.e.\ $N=mg$
    \end{itemize}
  \end{columns}
\end{frame}



\begin{frame}{Normal Force on a Slope}
  The normal force remains perpendicular to the support surface even when it is
  at an angle:
  \begin{center}
    \begin{tikzpicture}[scale=.9]
      \draw[thick] (0,0)--(5,0);
      \draw[axes] (1.7,0) arc (0:30:1.7) node[pos=.6,right] {$\theta$};
      \begin{scope}[rotate=30]
        \draw[thick] (0,0)--(5,0);
        \draw[mass] (2,0) rectangle (4,1.5);
        \draw[vectors,rotate around={-30:(3,.75)}]
        (3,.75)--(3,-1) node[right]{$\vec F_g$};
        \draw[vectors,red] (3,.75)--(3,2.25) node[left=-1]{$\vec F_N$};
        \fill (3,.75) circle (.06);
      \end{scope}
    \end{tikzpicture}
  \end{center}
  \textbf{Important note:} It is not always clear what the magnitude of the
  normal force is. Finding the magnitude of the normal force is part of the
  analysis process.
\end{frame}



%\begin{frame}{Normal Force on a Stationary Slope}
%  For this case, we define the $x$-axis to be along the slope, and $y$-axis to
%  be perpendicular to the slope.
%  \begin{columns}
%    \column{.35\textwidth}
%    \centering
%    \begin{tikzpicture}[scale=.9]
%      \draw[thick] (-1,0)--(3,0);
%      \draw[thick,->](.1,0) arc(0:37:1.1) node[pos=0,below]{$\theta$};
%      \begin{scope}[rotate around={37:(-1,0)}]
%        \draw[->](3.5,.5)--(4,.5)  node[right]{$x$};
%        \draw[->](3.5,.5)--(3.5,1) node[above]{$y$};
%        \draw[thick] (-1,0)--(4,0);
%        \draw[fill=cyan!80,thick] rectangle (3,2);
%        \fill(1.5,1) circle (2pt);
%        \draw [->,very thick,dotted,red] (1.5,1)--(1.5,-.5)
%        node[right]{$mg\cos\theta$};
%        \draw [->,very thick,dotted,red] (1.5,1)--(.45,1)
%        node[left]{$mg\sin\theta$};
%        \draw [->,very thick,red] (1.5,1)--(.45,-.5)
%        node[below]{$\vec F_g$};
%        \draw [->,very thick] (1.5,1)--(1.5,2.5) node[above]{$\vec F_N$};
%      \end{scope}
%    \end{tikzpicture}
%    
%    \column{.65\textwidth}
%    \begin{itemize}
%    \item On a stationary slope: $N=mg\cos\theta$
%      \begin{itemize}
%      \item $N$ decreases as ramp angle $\theta$ increases
%      \end{itemize}
%    \item Weight has a component along the ramp ($mg\sin\theta$) that wants
%      to slide the block down.
%    \end{itemize}
%  \end{columns}
%\end{frame}



\begin{frame}{Friction}
  \begin{itemize}
  \item A force that opposes the sliding of two surface against one another
  \item Always act in a direction that opposes motion or attempted motion
  \item Depends on:
    \begin{itemize}
    \item Normal force $N$: The force the two surfaces are pressed against
      each other
    \item Coefficients of friction ($\mu_s$ and $\mu_k$): Smoothness of the
      surfaces, which itself depends on
      \begin{itemize}
      \item The material(s) the surfaces are made of
      \item The use of lubricants
      \end{itemize}
    \end{itemize}
  \end{itemize}
  \begin{center}
    \vspace{-.1in}
    \pic{.5}{graphics/friction}
  \end{center}
\end{frame}



\begin{frame}{Static Friction}
  When there is no relative motion between the two surface, the friction is
  called the \textbf{static friction}
%  \textbf{Static friction} between the two surfaces is when there is no
%  relative motion between them
  \begin{itemize}
  \item Increases with increasing applied force
  \item Maximum when the object is just about to move
  \end{itemize}

  \eq{-.1in}{
    \boxed{f_s\leq\mu_sN}
  }
  \begin{center}
    \begin{tabular}{l|c|c}
      \rowcolor{pink}
      \textbf{Quantity} & \textbf{Symbol} & \textbf{SI Unit} \\ \hline
      Magnitude of static friction & $f_s$ & \si\newton \\
      Coefficient of static friction & $\mu_s$ & no units \\
      Magnitude of normal force    & $N$ & \si\newton
    \end{tabular}
  \end{center}
\end{frame}



\begin{frame}{Kinetic Friction}
  When the two surfaces are moving relative to each other, the friction is
  called \textbf{kinetic friction} $f_k$. $f_k$ is approximately constant along
  the path of movement as long  as $\vec F_N$ stays constant
%  \textbf{Kinetic friction} between two surfaces is when they are moving
%  relative to each other.

  \eq{-.1in}{
    \boxed{f_k = \mu_kN}
  }
  \begin{center}
    \begin{tabular}{l|c|c}
      \rowcolor{pink}
      \textbf{Quantity} & \textbf{Symbol} & \textbf{SI Unit} \\ \hline
      Magnitude of kinetic friction & $f_k$ & \si\newton \\
      Coefficient of kinetic friction & $\mu_k$ & no units \\
      Magnitude of normal force & $N$ & \si\newton
    \end{tabular}
  \end{center}
\end{frame}



\begin{frame}{Static and Kinetic Coefficients of Friction}    
  Coefficient of kinetic friction is always lower than the coefficient for
  static friction, otherwise nothing will ever move:
    
  \eq{-.1in}{
    \mu_k\leq\mu_s
  }

  \vspace{-.2in}Consider a simple case of a box being pulled along a level
  floor. The free-body diagram is simple (left). How do the magnitudes of the
  applied force $F_a$ and friction $f$ compare?

  \begin{columns}
    \column{.4\textwidth}
    \centering
    \begin{tikzpicture}
      \draw[thick] (-1,0)--(2.5,0);
      \draw[mass] rectangle (1.5,1);
      \fill (.75,.5) circle (2pt);
      \draw[vectors] (.75,.5)--(.75,-.5) node[below]{$\vec F_g$};
      \draw[vectors] (.75,.5)--(.75,1.5) node[above]{$\vec F_N$};
      \draw[vectors] (.75,.5)--(2.,.5) node[right]{$\vec F_a$};
      \draw[vectors] (.75,.5)--(.1,.5) node[left]{$\vec f$};
    \end{tikzpicture}

    \column{.6\textwidth}
    \centering
    \begin{tikzpicture}[scale=.5,vectors]
      \draw (0,0)--(10,0) node[right]{$F_a$};
      \draw (0,0)--(0,5) node[left]{$f$};
    \end{tikzpicture}
  \end{columns}
\end{frame}



\begin{frame}{Tires}
  Most people associate friction as the force that slows down things, but very
  often friction is what accelerates things. \textbf{Example:} the forward
  acceleration of a car is caused by the static friction between the tires and
  the road.
  \begin{center}
    \pic{.38}{graphics/all-season-tires}
  \end{center}
  Tires also generated a force called \emph{rolling resistance} as it rolls
  along a road because the weight of the car deforms the tires.
\end{frame}




\begin{frame}{Drag}
  When an object moves through a fluid (most gases and liquids), it experiences
  a fluid resistance force called \textbf{drag} $\vec F_D$.
  \begin{center}
    \pic{.338}{graphics/boeing787}
    \pic{.35}{graphics/ganna}
    \pic{.263}{graphics/submarine}
  \end{center}
  Unlike kinetic friction, drag force scales with the square of the speed of the
  object relative to the fluid that it is moving in:
  
  \eq{-.15in}{
    F_D=\frac12\rho v_\infty^2C_DA
  }

  \vspace{-.1in}In AP Physics C you don't need to know the drag equation,
  just that $F_D\propto v^2$, and $F_D\propto A$
\end{frame}



\begin{frame}{Drag}
  \eq{-.1in}{
    \boxed{
      F_D=\frac12\rho v_\infty^2C_DA
    }
  }  
  \begin{center}
    \begin{tabular}{l|c|c}
      \rowcolor{pink}
      \textbf{Quantity} & \textbf{Symbol} & \textbf{SI Unit} \\ \hline
      Magnitude of drag force & $F_D$     & \si\newton \\
      Density of the fluid    & $\rho$    & \si{\kilo\gram\per\metre\cubed}\\
      Free-stream velocity    & $v_\infty$ & \si{\metre\per\second}\\
      Reference area          & $A$       & \si{\metre\squared}\\
      Drag coefficient        & $C_D$     & (no unit)
    \end{tabular}
  \end{center}
  Drag coefficient $C_D$ depends on the shape and surface smoothness of the
  object. For blunt objects (``bluff bodies'') $A$ is the frontal area; for
  streamlined objects $A$ is the planform (top-view) area
\end{frame}



\begin{frame}{Terminal Velocity}
  When we take drag force into account, we understand that the drag force
  increases as an object speeds up, and therefore a free-falling object does
  \emph{not} accelerate infinitely. Instead it reaches a
  \textbf{terminal velocity}.

  \begin{columns}
    \column{.33\textwidth}
    
    {\footnotesize There is no air resistance just as the object \emph{begins}
      to fall. Acceleration is due to gravity alone.\par}

    \vspace{-.15in}
    \begin{center}
      \begin{tikzpicture}[scale=.9,vectors]
        \draw (0,0)--(0,-1.5) node[below]{$\vec F_g$};
        \draw[black!2] (0,0)--(0,1.5) node[above]{$\vec F_D$};
        \fill circle (.07);
      \end{tikzpicture}
    \end{center}
    
    \column{.33\textwidth}

    {\footnotesize Drag increases as $v$ increases. Magnitude of acceleration
      decreases, but the object continues to gather speed\par}

    \vspace{-.15in}
    \begin{center}
      \begin{tikzpicture}[scale=.9,vectors]
        \draw (0,0)--(0,-1.5) node[below]{$\vec F_g$};
        \draw[black!2] (0,0)--(0,1.5) node[above]{$\vec F_D$};
        \draw[green] (0,0)--(0,.6) node[above]{$\vec F_D$};
        \fill circle (.07);
      \end{tikzpicture}
    \end{center}
    
    \column{.33\textwidth}
    
    {\footnotesize Terminal velocity is reached when the drag force equals the
      object's weight. Not net force; no acceleration.\par}

    \vspace{-.15in}
    \begin{center}
      \begin{tikzpicture}[scale=.9,vectors]
        \draw (0,0)--(0,-1.5) node[below]{$\vec F_g$};
        \draw[red] (0,0)--(0,1.5) node[above]{$\vec F_D$};
        \fill circle (.07);
      \end{tikzpicture}
    \end{center}
        
  \end{columns}
\end{frame}




%\begin{frame}{Example Problem}
%  \textbf{Example 8:} To move a \SI{45}{kg} wooden crate across a wooden floor
%  ($\mu=0.20$), you tie a rope onto the crate and pull on the rope. While you
%  are pulling the rope with a force of \SI{115}{N}, it makes an angle of
%  \ang{15}
%  with the horizontal. How much time elapses between the time at which the
%  crate just starts to move and the time at which you are pulling it with a
%  velocity of \SI{1.4}{m/s}?
%  \begin{center}
%    \pic{.5}{graphics/pull-box}
%  \end{center}
%\end{frame}
%
%\begin{frame}{Example Problem}
%  \textbf{Example 9:} You are holding an \SI{85}{kg} trunk at the top of a ramp
%  that slopes from a moving van to the ground, making an angle of \ang{35} with
%  the ground. You lose your grip and the trunk begins to slide.
%  \begin{itemize}
%  \item If the coefficient of friction between the trunk and the ramp is
%    $0.42$, what is the acceleration of the trunk?
%  \item If the trunk slides \SI{1.3}{m} before reaching the bottom of the ramp,
%    for what time interval did it slide?
%  \end{itemize}
%\end{frame}



%\begin{frame}{Example: Vertical Motion}
%  \textbf{Example 10:} A \SI{55}{kg} person is standing on a scale in an
%  elevator. If
%  the scale is calibrated in \emph{newtons}, what is the reading on the scale
%  when the elevator is not moving? If the elevator begins to accelerate upward
%  at \SI{.75}{m/s^2}, what will be the reading on the scale?
%\end{frame}



\begin{frame}{Tension Force}
  \textbf{Tension} $\vec F_T$ is the force that is transmitted through objects
  that can be stretched, e.g.\ a rope that is being pulled
  \begin{center}
    \pic{.6}{graphics/3-rope}
  \end{center}
  \begin{itemize}
  \item Examples: ropes, cables, strings, etc.
  \item Tension force can only be transmitted if the cable is fully extended
  \item You can't push on a rope
  \item Can be used with pulleys to change the direction of force
  \end{itemize}
\end{frame}



\begin{frame}{Spring Force}
  The spring force $\vec F_s$ is the force that a compressed/stretched spring
  exerts on the object connected to it.  An \emph{ideal} spring obeys Hooke's
  law:
    
  \eq{-.1in}{
    \boxed{\vec F_s=-k\vec x}
  }

  \vspace{-.1in}The spring force acts in the opposite direction to the spring's
  displacement, and is proportional to the amount of compression/stretching.

  \vspace{.1in}
  \begin{center}
    \begin{tikzpicture}
      \draw[mass] (5,.5) rectangle (6,1.5);
      \draw[thick,
        decoration={aspect=.6,segment length=5mm, amplitude=2.5mm, coil},
        decorate] (0,1)--(5,1);
      \fill[pattern=north east lines] (-.2,0) rectangle (0,2);
      \draw[thick] (0,.0)--(0,2);
      \fill[red] (5.5,1) circle (.06);
      \draw[vectors,red] (5.5,1)--(4,1) node[above]{$\vec F_s$};
      \draw[dashed] (3,0)--(3,2) node[above]{\scriptsize Equilibrium position};
      \draw[vectors] (3,.3)--(5,.3) node[midway,below]{$\vec x$};
    \end{tikzpicture}
    \hspace{.2in}
    \begin{tikzpicture}
      \draw[thick,gray!40,fill=gray!20] (5,.5) rectangle (6,1.5);
      \draw[thick,gray!20,
        decoration={aspect=.6,segment length=5mm, amplitude=2.5mm, coil},
        decorate] (0,1)--(5,1);
      \fill[pattern=north east lines](-.2,0) rectangle (0,2);
      \draw[thick](0,.0)--(0,2);
      \fill[gray!30] (5.5,1) circle (.06);
      \draw[vectors,gray!30] (5.5,1)--(4,1) node[above]{$\vec F_s$};
      \draw[dashed](3,0)--(3,2);
      \draw[vectors,gray!30](3,.3)--(5,.3)node[midway,below]{$\vec x$};
      \draw[mass] (1.5,.5) rectangle (2.5,1.5);
      \draw[thick,
        decoration={aspect=.3,segment length=1.5mm, amplitude=2.5mm, coil},
        decorate] (0,1)--(1.5,1);
      \draw[vectors] (3,.3)--(1.5,.3) node[midway,below]{$\vec x$};
      \fill[red] (2,1) circle (.06);
      \draw[vectors,red] (2,1)--(3,1) node[above]{$\vec F_s$};
    \end{tikzpicture}
  \end{center}
\end{frame}


\section{Free-Body Diagrams}

\begin{frame}{Free-Body Diagrams}
  \begin{itemize}
  \item Acceleration (if there is going to be any at all) depends
    on net force $\vec F_\text{net}$
  \item Without a vector sum of all the forces, we cannot determine the
    magnitude, direction of the acceleration, or how acceleration will evolve
    in time
  \item We use \textbf{free-body diagrams} (FBD) to represent all the forces.
    \begin{itemize}
    \item Very important in solving any dynamics problems
    \item Don't try to save this step, even if the problem does not ask for it
    \item Always draw FBD for solving classical mechanics problem
    \end{itemize}
  \end{itemize}
\end{frame}



\begin{frame}{Free-Body Diagrams}
  For \emph{rectilinear}, or \emph{translational} motion, FBDs are usually drawn
  by assuming that all forces acting at the center of mass (``CM''),
  represented by the ``big dot''. For example:
  \begin{center}
    \begin{tikzpicture}[scale=.8]
      \fill (1.5,1) circle (.1);
      \draw[vectors] (1.5,1)--(1.5,-.5) node[below]{$\vec F_g$};
      \draw[vectors] (1.5,1)--(1.5,2.5) node[above]{$\vec F_N$};
      \draw[vectors] (1.5,1)--(3.1,1) node[right]{$\vec F_a$};
      \draw[vectors] (1.5,1)--(.75,1) node[left] {$\vec f$};
    \end{tikzpicture}
  \end{center}
  However, for motion where \emph{rotation} is (at least) a possibility, we
  must note that:
  \begin{itemize}
  \item Gravitational force $\vec F_g$ acts at the CM, but
  \item Normal force $\vec F_N$, friction $\vec f$ and applied force $\vec F_a$
    act at the point of contact
  \end{itemize}
\end{frame}



\begin{frame}{Free-Body Diagrams}
  In those cases, forces should be drawn where they are applied. For example,
  a sphere rolling down a ramp should have weight $\vec F_g$, normal force
  $\vec F_N$ and static friction $\vec f_s$ acting on it:
  \begin{center}
    \begin{tikzpicture}[scale=1.1,rotate=-30]
      \shade[ball color=red!20] circle (1);
      \draw[thick] (-2,-1)--(2,-1);
      \fill circle (.05);
      \draw[vectors,rotate=30] (0,0)--(0,-1.5) node[below]{$\vec F_g$};
      \draw[vectors] (0,-1)--(0,.3) node[above]{$\vec F_N$};
      \draw[vectors] (0,-1)--(-.5,-1) node[left]{$\vec f_s$};
      \draw[axes] (2,0)--(2.5,0) node[right]{$x$};
      \draw[axes] (2,0)--(2,.5) node[above]{$y$};
      \draw[axes] (0,1.2) arc (90:60:1.2) node[pos=.3,right]{$\omega$};
    \end{tikzpicture}
  \end{center}
  Once the FBD is drawn, decide on the axes to help you solve the motion. One of
  the axes should line up with the direction of motion. This guarantees that
  the \emph{other} axis will not have any net force.
\end{frame}



\begin{frame}{Example Problem}
  A more difficult static problem may involve two surfaces with two different
  friction coefficients. For example, a ladder leaning on a wall. This problem
  cannot be solved without first understanding rotational motion, but we can
  still draw a FBD.
  \vspace{.2in}
  \begin{columns}
    \column{.75\textwidth}
    \textbf{Example:} A uniform ladder is \SI{5.0}{\metre} long and weighs
    \SI{400}\newton. The ladder rests against a slippery vertical wall, as
    shown in the figure. The inclination angle between the ladder and the rough
    floor is \ang{53}. Find the reaction forces from the floor and
    from the wall on the ladder and the coefficient of static friction $\mu_s$
    at the interface of the ladder with the floor that prevents the ladder from
    slipping.

    \column{.25\textwidth}
    \pic1{graphics/ladder}
  \end{columns}
\end{frame}



\section{Multi-Body Problems}

\begin{frame}{Applying Third Law on Connected Bodies}
  \begin{center}
    \pic{.7}{graphics/worldslongestroadtrainwithpowertrailer8}
  \end{center}
  \begin{itemize}
  \item The objects are connected by a cable or a solid linkage with negligible
    mass
  \item All objects (usually) have the same acceleration
  \item Require multiple free-body diagrams
  \end{itemize}
\end{frame}



\begin{frame}{Solving Connected-Bodies Problems}
  To solve a connected-bodies problem, you can follow these procedures:
  \begin{enumerate}
  \item Draw a FBD on each of the objects
  \item Sum all the forces on all the objects along the direction of motion
    \begin{itemize}
    \item Direction of motion is usually very obvious
    \item All internal forces should cancel and do not figure into the
      acceleration of the system
    \end{itemize}
  \item Compute the acceleration of the entire system using second law of motion
    \begin{itemize}
    \item Remember that (usually) every object has the same acceleration!
    \end{itemize}
  \item Go back to the FBD of each of the objects and compute the unknown
    forces (usually tension)
  \end{enumerate}
\end{frame}



%\begin{frame}{Connected Bodies: Example}
%  \textbf{Example:} A tractor-trailer pulling two trailers starts from rest
%  and accelerates with an acceleration $a$ on a straight, level road. The mass
%  of the truck (T) is \SI{5450}{\kilo\gram}, the mass of the first trailer (A)
%  is \SI{31500}{\kilo\gram}, and the mass of the second trailer (B) is
%  \SI{19600}{\kilo\gram}.
%  \begin{enumerate}
%  \item What magnitude of force must the truck generate in order to accelerate
%    the entire vehicle?
%  \item What magnitude of force must each of the trailer hitches withstand
%    while the vehicle is accelerating?
%  \end{enumerate}
%  Assume that frictional forces are negligible in comparison with the forces
%  needed to accelerate the large masses.
%\end{frame}



\begin{frame}{Different Types of Connected Bodies}
  Multiple objects pressed against one another. There may not be friction, but
  there are definitely action/reaction forces between the blocks.
  \begin{center}
    \begin{tikzpicture}[scale=.8]
      \draw[thick] (-3,0)--(4,0);
      \draw[thick] rectangle (1,1) node[midway]{$m$};
      \draw[thick] (1,0) rectangle (3,1.5) node[midway]{$M$};
      \draw[vectors] (-1.5,.5)--(0,.5) node[pos=0,left]{$\vec F_a$};
    \end{tikzpicture}
  \end{center}
  Or multiple objects stacked on top of one another. The contact surface between
  $M$ and the floor may (or may not) have friction, while the surface between
  $M$ and $m$ must have a friction coefficient $\mu$.
  \begin{center}
    \begin{tikzpicture}[scale=.8]
      \draw[thick] (-1,0)--(5.5,0);
      \draw[thick] rectangle (3,1.5) node[midway]{$M$};
      \draw[thick] (.75,1.5) rectangle (2.25,2.5) node[midway]{$m$};
      \draw[vectors] (3,.5)--(4.5,.5) node[right]{$\vec F_a$};
    \end{tikzpicture}
  \end{center}
\end{frame}



\section{Pulley Problems}

\begin{frame}{Example Problem: Atwood Machine}
  \begin{columns}
    \column{.25\textwidth}
    \centering
    \begin{tikzpicture}[scale=.7]
      \draw[ultra thick,brown] (-1,-1.6)--(-1,0);
      \draw[ultra thick,brown] (1,0)--(1,-3);
      \draw[thick,fill=gray] circle (1.05);
      \draw[thick,fill=gray!40] circle (.95);
      \draw[line width=5.5] (0,-.15)--(0,2);
      \draw[very thick] (-2,2)--(2,2);
      \fill[white] circle (.1);
      \draw[mass] (-1.5,-1.6) rectangle +(1,-1) node[midway]{$M$};
      \draw[mass] (.5,-3) rectangle +(1,-1.4) node[midway]{$m$};
    \end{tikzpicture}   

    \column{.7\textwidth}
    An \textbf{Atwood machine} is made of two objects connected by a rope that
    runs over a pulley. The pulley allows the direction of force and direction
    of motion to change between two objects.
    
    \vspace{.2in}\textbf{Example:} The object on the left has a mass of $M$ and
    the object on the right has a mass of $m$.
    \begin{enumerate}[(a)]
    \item What is the acceleration of the masses?
    \item What is the tension in the rope?
    \end{enumerate}
  \end{columns}
\end{frame}
%\begin{frame}{Example Problem: Atwood Machine}
%  An \textbf{Atwood machine} is made of two objects connected by a rope that
%  runs over a pulley. The pulley allows the direction of force and direction
%  of motion to change between two objects.
%  \begin{columns}
%    \column{.35\textwidth}
%    \centering
%    \pic1{graphics/pulley_prob_2}
%
%    \column{.65\textwidth}
%    \textbf{Example:} The object on the left has a mass of $M$ and the object
%    on the right has a mass of $m$.
%    \begin{itemize}
%    \item What is the acceleration of the masses?
%    \item What is the tension in the rope?
%    \end{itemize}
%  \end{columns}
%\end{frame}



%\begin{frame}{A Difficult Problem!}
%  \begin{columns}
%    \column{.6\textwidth}
%    \textbf{Example:} Two blocks of mass $m$ and $M$ are connected via pulley
%    with a configuration as shown on the right. The coefficient of static
%    friction between the left block and the surface is $\mu_{s,1}$, and the
%    coefficient of static friction between the right block and the surface is
%    $\mu_{s,2}$. Formulate a mathematical inequality for the condition that no
%    sliding occurs. There may be more than one inequality. 
%    
%    \column{.4\textwidth}
%    \pic{1}{graphics/pulley_prob_6}
%  \end{columns}
%\end{frame}
%
%
%
%\begin{frame}{Multiple Pulleys}
%  When there are multiple pulleys involved, we have to remember that tension
%  force is distributed evenly along the cable.
%
%  \vspace{.2in}
%  \begin{columns}
%    \column{.6\textwidth}
%    \textbf{Example:} A block of mass $m$ is pulled, via two pulleys as shown,
%    at constant velocity along a surface inclined at angle $\theta$. The
%    coefficient of kinetic friction is $\mu_k$, between block and surface.
%    Determine the pulling force $F$. Ignore the mass of the pulleys. 
%    
%    \column{.4\textwidth}
%    \pic{1}{graphics/pulley_prob_7}
%  \end{columns}  
%\end{frame}


%\begin{frame}{One More!}
%  \begin{columns}
%    \column{.75\textwidth}
%    \textbf{Example:} A block of mass $M$ is lifted at constant velocity, via
%    an arrangement of pulleys as shown. Determine the pulling force $F$. Ignore
%    the mass of the pulleys. 
%
%    \uncover<2>{
%      \vspace{.2in}\textbf{Example:} The pulling force is replaced by a $10M$
%      mass, and was let go. What are the accelerations of the $M$ and the
%      $10M$ mass?
%    }
%    
%    \column{.25\textwidth}
%    \pic{1}{graphics/pulley_prob_9}
%  \end{columns}
%\end{frame}


\begin{frame}{A More Typical Problem}
  More typically, an Atwood machine problem is one where two objects are
  sliding on a surface. These surfaces may have (or may not) have friction. In
  this example, two blocks are connected by a massless string over a
  frictionless pulley as shown in the diagram.
  \begin{center}
    \begin{tikzpicture}[scale=1.2]
      \draw[thick,brown] (-4,.4)--(.1,.4);
      \draw[thick] (0,0)--(-5.5,0) node[midway,below]{$\mu$};
      \draw[thick,fill=magenta!20] (-4,0) rectangle (-5,.75) node[midway]{$m$};
      \begin{scope}[rotate=-30,thick]
        \draw[brown] (1,.4)--(-.05,.4);
        \draw (0,0)--(3,0) node[midway,below left]{$\mu$};
        \draw[mass] (1,0) rectangle (2.5,1) node[midway]{$M$};
      \end{scope}
      \begin{scope}[rotate=-15]
        \draw[thick,fill=gray] (0,.3) circle (.15);
        \draw[thick,fill=lightgray] (0,.3) circle (.1);
        \draw[ultra thick] (0,0)--(0,.3);
        \fill (0,.3) circle (.04);
      \end{scope}
      \draw[thick,gray!70] (0,0)--(0,-1.5);
      \draw[axes] (0,-.5) arc (270:330:.5) node[midway,below]{$\phi$};
    \end{tikzpicture}
  \end{center}
  \begin{enumerate}[(a)]
  \item Determine the acceleration of the blocks.
  \item Calculate the tension in the string.
  \end{enumerate}
\end{frame}
\end{document}
