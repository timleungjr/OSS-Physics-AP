\documentclass[12pt,compress,aspectratio=169,dvipsnames]{beamer}
\usetheme{metropolis}
\setbeamersize{text margin left=.5cm,text margin right=.5cm}
\usepackage[lf]{carlito}
\usepackage{siunitx}
\usepackage{tikz}
\usepackage{mathpazo}
\usepackage{bm}
\usepackage{mathtools}
\usepackage[ISO]{diffcoeff}
\diffdef{}{ op-symbol=\mathsf{d} }
\usepackage{xcolor,colortbl}

\setmonofont{Ubuntu Mono}
\setlength{\parskip}{0pt}
\renewcommand{\baselinestretch}{1}

\sisetup{
  inter-unit-product=\cdot,
  per-mode=symbol
}

\tikzset{
  >=latex
}

%\newcommand{\iii}{\hat{\bm\imath}}
%\newcommand{\jjj}{\hat{\bm\jmath}}
%\newcommand{\kkk}{\hat{\bm k}}

\usepackage{mathtools}
  
\title{Class 1: Kinematics}
\subtitle{Advanced Placement Physics C}
\author[TML]{Dr.\ Timothy Leung}
\institute{Olympiads School}
\date{Updated: Summer 2022}

\newcommand{\pic}[2]{
  \includegraphics[width=#1\textwidth]{#2}
}
\newcommand{\eq}[2]{
  \vspace{#1}{\Large
    \begin{displaymath}
      #2
    \end{displaymath}
  }
}
%\newcommand{\iii}{\ensuremath\hat{\bm{\imath}}}
%\newcommand{\jjj}{\ensuremath\hat{\bm{\jmath}}}
%\newcommand{\kkk}{\ensuremath\hat{\bm{k}}}
\newcommand{\iii}{\ensuremath\hat\imath}
\newcommand{\jjj}{\ensuremath\hat\jmath}
\newcommand{\kkk}{\ensuremath\hat k}


\begin{document}

\begin{frame}
  \centering\Large
  \textbf{WELCOME TO AP PHYSICS C}
\end{frame}



%\begin{frame}{Prerequisites}
%  \begin{itemize}
%  \item\textbf{Physics 11 and 12:} You will need to be comfortable with the
%    topics covered in high-school level physics courses.
%  \item\textbf{Calculus:} Physics C exams are calculus based, and you will be
%    required to do basic differentiation and integration. You don't need to be
%    an expert, but basic knowledge is required. Differentiation and integration
%    in the course are generally not difficult, but there are occasional
%    challenges.
%  \item\textbf{Vectors:} You need to be comfortable with vector operations,
%    including addition and subtraction, multiplication/division by constants,
%    as well as dot products and cross products.
%  \end{itemize}
%\end{frame}



\begin{frame}{AP Physics C Exams}
  There are two calculus-based AP Physics C exams, which are usually taken
  together on the same day, in the first or second week of May of each year.
  \begin{itemize}
  \item Mechanics
  \item Electricity and Magnetism
  \end{itemize}
\end{frame}



\begin{frame}
  \titlepage
\end{frame}



%\begin{frame}{Files for You to Download}
%  There are a considerable number of files for you to download at the beginning
%  of the course:
%  \begin{itemize}
%  \item\texttt{PhysAPC-courseOutline.pdf}--The course outline
%  \item\texttt{PhysAPC-equationSheet.pdf}--Equation sheet for your exams
%  \item\texttt{PhysAPC-01-kinematics.pdf}
%  \item\texttt{PhysAPC-01a-vectorCalculus.pdf}--Vectors and calculus handout
%  \item\texttt{PhysAPC-01b-kinematicsHandout.pdf}--Basic kinematics, expanded
%    version
%  %\item\texttt{PhysAPC-01c-motionGraphs.pdf}--Handout on motion graphs
%  \item\texttt{PhysAPC-01d-projectileMotion.pdf}--Handout on projectile motion
%  %\item\texttt{PhysAPC-02-dynamics.pdf}
%  \item\texttt{PhysAPC-01-Homework.pdf}--Homework problems for Topic 1
%  %\item\texttt{PhysAPC-02-Homework.pdf}--Homework problems for Topic 2
%  \end{itemize}
%\end{frame}

\section{Kinematics}

%\begin{frame}{Kinematics}
%  \textbf{Kinematics} is a discipline within mechanics concerning the
%  motion of bodies. It describes the mathematical relationship between 
%  \begin{itemize}
%  \item<alert@1> Position
%  \item<alert@1> Displacement
%  \item Distance 
%  \item<alert@1> Velocity
%  \item Speed
%  \item<alert@1> Acceleration
%  \end{itemize}
%  Kinematics does not deal with the causes of motion. The quantities that are
%  highlighted are vectors.
%\end{frame}



\begin{frame}{Position}
  \textbf{Position} ($\vec r(t)$) is the location of an object in a coordinate
  system as a function if time. It is measured in \textbf{meters}
  (\si\metre).
  \begin{columns}
    \column{.7\textwidth}

    \eq{-.1in}{
      \boxed{\vec r(t)=x(t)\iii + y(t)\jjj + z(t)\kkk}
    }
    \begin{itemize}
    \item\vspace{-.1in}\textbf{IJK notation}: $\iii$, $\jjj$ and $\kkk$ are
      \textbf{basis vectors} indicating the directions of the $x$, $y$ and $z$
      axes. Basis vectors are \textbf{unit vectors} (i.e.\ length 1)
    \item The IJK notation does not explicitly give the magnitude or the
      direction of the vector (needs to be calculated using the Pythagorean
      theorem)
    \end{itemize}

    \column{.3\textwidth}
    \begin{tikzpicture}[scale=.45]
      \draw[axes] (0,0)--(5,0) node[right]{$x$};
      \draw[axes] (0,0)--(0,5) node[above]{$y$};
      \draw[axes] (0,0)--(-2,-2) node[below]{$z$};
      \fill[red] (4,3) circle(.1);
      \draw[vectors,red] (0,0)--(4,3) node[midway,above]{$\vec r$}
      node[above]{$(x,y,z)$};
    \end{tikzpicture}
  \end{columns}
\end{frame}



\begin{frame}{Displacement}
  \textbf{Displacement} $\Delta\vec r(t)$ is the change in position from the
  initial position $\vec r_0$ within the same coordinate system. It is also a
  function of time.

  \eq{-.1in}{
    \boxed{
      \Delta\vec r(t)=\vec r(t)-\vec r_0
      =[x(t)-x_0]\iii + [y(t)-y_0]\jjj + [z(t)-z_0]\kkk
    }
  }

  where the initial position is give by:

  \eq{-.2in}{
    r_0=x_0\iii+y_0\jjj + z_0\kkk
  }
  
\end{frame}



\begin{frame}{Distance}
  \textbf{Distance} $s(t)$ is a quantity that is \emph{related} to displacement.
  \begin{columns}
    \column{.7\textwidth}
    \begin{itemize}
    \item The \emph{length of the path} taken by an object when it travels from
      $\vec r_0$ to $\vec r(t)$
    \item A scalar quantity
    \item Non-negative, i.e.\ $s\geq 0$
    \item Although the magnitude of the displacement vector is also a scalar,
      it is not necessarily the same as distance
    \item $s\geq |\Delta\vec r|$
    \end{itemize}
    
    \column{.3\textwidth}
    \begin{tikzpicture}[scale=.5]
      \draw[axes] (0,0)--(6,0) node[right]{$x$};
      \draw[axes] (0,0)--(0,8) node[above]{$y$};
      \draw[vectors,red] (0,0)--(4,1) node[midway,above]{$\vec r_0$};
      \draw[vectors,red] (0,0)--(2,6) node[midway,left]{$\vec r$};
      \draw[vectors,blue] (4,1)--(2,6) node[midway,right]{$\Delta\vec r$};
      \draw[thick,dash dot] (4,1) ..controls (6,5) and (5,7).. (2,6)
      node[midway,right]{$s$};
    \end{tikzpicture}
  \end{columns}
\end{frame}



\begin{frame}{Instantaneous \& Average Velocity}
  \textbf{Instantaneous velocity} $\vec v(t)$ is the rate of change of
  position\footnote{Or the rate of change of displacement. The two quantities
  only differ by the constant $\vec r_0$.} with respect to time. It is related
  to position $\vec r(t)$ by:

  \eq{-.13in}{
    \boxed{
      \vec v(t)= \diff{\vec r}t
      \quad\quad
      \vec r(t)=\int\vec v(t)\dl t + \vec r_0
    }
  }

  $\vec r_0$ is the constant of integration evaluated at $t=t_0$. Likewise, the
  \textbf{average velocity} $\vec v_\text{avg}(t)$ is the displacement (change
  in position) $\Delta\vec r(t)$ over a finite time interval $\Delta t$:

  \eq{-.1in}{
    \boxed{
      \vec v_\text{avg}(t) = \frac{\Delta\vec r(t)}{\Delta t}=
      \frac{\vec r(t)-\vec r_0}{t-t_0}
      = \frac{\int_{t_0}^t\vec v\dl t}{t-t_0}
    }
  }

  The SI unit of velocity is \textbf{meters per second} (\si{\metre\per\second})
  \vspace{.2in}
\end{frame}




\begin{frame}{Instantaneous \& Average Speed}
  \textbf{Instantaneous speed} $v$ is the time rate of change of
  \emph{distance}. It is the \emph{magnitude} of instantaneous velocity
  (i.e.\ $v=|\vec v|$). Since $s\geq 0$, instantaneous speed must be positive,
  i.e.\ $v\geq 0$.

  \eq{-.1in}{
    \boxed{
      v=\diff st \geq 0
    }
  }

  \textbf{Average speed} $v_\text{avg}(t)$ is the distance $s(t)$ traveled over
  a finite time interval $\Delta t$:
  
  \eq{-.1in}{
    \boxed{
      v_\text{avg}(t)=\frac{s(t)}{\Delta t}=\frac{\int_{t_0}^t v(t)\dl t}{t-t_0}
    }
  }

  The SI unit of speed is also measured in \si{\metre\per\second}.
\end{frame}



%\begin{frame}{Path}
%  Sometimes instead of explicitly describing the position $x=x(t)$ and $y=y(t)$,
%  the path of an object can be given in terms of $x$ coordinate $y=y(x)$, while
%  giving the $x$ (or $y$) coordinate as a function of time.
%  \begin{itemize}
%  \item In this case, substitute the expression for $x(t)$ into $y=y(x)$ to
%    get an expression of $y=y(t)$
%  \item Take derivative using chain rule to get $v_y=v_y(t)$
%  \end{itemize}
%\end{frame}


\begin{frame}{Instantaneous \& Average Acceleration}
  In the same way that velocity is the time rate of change in position,
  \textbf{instantaneous acceleration} $\vec a(t)$ is the time rate of change of
  instantaneous velocity:

  \eq{-.1in}{
    \boxed{
      \vec a(t)= \diff{\vec v}t=\diff[2]{\vec r}t
      \quad\quad
      \vec v(t)=\int\vec a(t)\dl t+\vec v_0
    }
  }

  The SI unit for acceleration is \textbf{meters per second squared}
  (\si{\metre\per\second\squared}). \textbf{Average acceleration}
  $\vec a_\text{avg}(t)$ is the finite change in velocity $\Delta\vec v(t)$ over
  a finite time interval $\Delta t$:

  \eq{-.1in}{
    \boxed{
      \vec a_\text{avg}(t)=
      \frac{\Delta\vec v(t)}{\Delta t}
      =\frac{\vec v(t)-\vec v_0}{t-t_0}
      =\frac{\int_{t_0}^t\vec a(t)\dl t}{t-t_0}
    }
  }

  \textbf{Important note:} Aacceleration only requires a \emph{change} in
  velocity; it does \emph{not} necessarily mean an object speeds up or slows
  down (e.g.\ uniform circular motion)
\end{frame}



\begin{frame}{Special Notation When Differentiating With Time}
  Physicists and engineers often use a special notation when the derivative is
  taken with respect to \emph{time}, by writing a dot above the variable. For
  example:

  \vspace{-.3in}{\large
    \begin{align*}
      v &= \dot r \\
      a &= \dot v =\ddot r
    \end{align*}
  }

  We will use this notation whenever it is convenient
\end{frame}


\begin{frame}{Linear Independence}
  The $x$, $y$ and $z$ components of $\vec r$ along the $\iii$, $\jjj$ and
  $\kkk$ directions are \emph{linearly independent}, therefore time
  derivatives and integrals can be separated into components:

  \vspace{-.2in}{\large
    \begin{align*}
      \vec v(t) &=
      \diff xt\iii + \diff yt\jjj + \diff zt\kkk = v_x\iii + v_y\jjj + v_z\kkk\\
      \vec a(t) &=
      \diff{v_x}t\iii + \diff{v_y}t\jjj + \diff{v_z}t\kkk =
      a_x\iii + a_y\jjj + a_z\kkk
    \end{align*}
  }
\end{frame}



\begin{frame}{If You Are Curious}
  The time derivative of acceleration is called \textbf{jerk}, with a unit
  of \si{\metre\per\second\cubed}:

  \eq{-.1in}{
    \vec j(t)=\diff{\vec a}t=\diff[2]{\vec v}t=\diff[3]{\vec r}t
  }

  The time derivative of jerk is \textbf{jounce}, or \textbf{snap}, with a
  unit of \si{\metre\per\second^4}:
  
  \eq{-.1in}{
    \vec s(t)=\diff{\vec j}t
    =\diff[2]{\vec a}t=\diff[3]{\vec v}t=\diff[4]{\vec r}t
  }
  
  The next two derivatives of snap are called \textbf{crackle} and
  \textbf{pop}, but these higher derivatives of position vector are rarely used.
  We will \emph{not} be using them for AP Physics.
\end{frame}



\begin{frame}{Acceleration as Functions of Velocity and Position}
  Acceleration may be expressed as functions of velocity and position rather
  than of time, if motion is driven by these forces that you have learned from
  Physics 11, 12 or AP Physics 1 \& 2:
  \begin{itemize}
  \item Gravitational or electrostatic forces:
    $a(r)=\dfrac{Gm_s}{r^2}\quad a(r)=\dfrac{kq_1q_2}{mr^2}$
  \item Spring force: $a(r)=-\dfrac kmr$
  \item Damping force: $a(v)=-bv$
  \item Aerodynamic drag:
    $a(v)=\left[\dfrac{\rho C_L A}{2m}\right] v^2$
  \end{itemize}

  \vspace{.1in}In these cases, solving for $r(t)$, $v(t)$ and $a(t)$ will
  require solving a differential equation. % (see handout).
\end{frame}



\section{Kinematic Equations}

\begin{frame}{Kinematic Equations}
  While kinematic problems in AP Physics C exams often require calculus, these
  basic kinematic equations for \underline{constant acceleration} are still a
  powerful tool:

  \vspace{-.25in}{\large
    \begin{align*}
      x &= x_0+ v_0t + \frac12at^2\\
      v &= v_0+at\\
      v^2 &= v_0^2+ 2a(x-x_0)
    \end{align*}
  }

  For AP Physics, you are only given three kinematic equations. You are
  required to integrate when acceleration is not constant.
\end{frame}



\section{Motion Graphs}

\begin{frame}{Motion Graphs}
  You should already be familiar with the basic motion graphs for 1D motion:
  \begin{itemize}
  \item Position vs.\ time graph
  \item Velocity vs.\ time graph
  \item Acceleration vs.\ time graph
  \end{itemize}

  However, depending on the situation, it may be more useful to plot motion
  using other quantities as well.
\end{frame}



\begin{frame}{Uniform Motion: Constant Velocity}
  \begin{center}
    \begin{tikzpicture}[scale=.5]
      \draw[axes] (0,0)--(4.5,0) node[right]{$t$};
      \draw[axes] (0,0)--(0,4.5) node[above]{$x$};
      \draw[functions] (0,1)--(4,4) node[pos=0,left,black]{$x_0$};
    \end{tikzpicture}
    \hspace{.15in}
    \begin{tikzpicture}[scale=.5]
      \draw[axes] (0,0)--(4.5,0) node[right]{$t$};
      \draw[axes] (0,0)--(0,4.5) node[above]{$v$};
      \draw[functions] (0,2)--(4,2) node[pos=0,left,black]{$v$};
    \end{tikzpicture}
    \hspace{.15in}
    \begin{tikzpicture}[scale=.5]
      \draw[axes] (0,0)--(4.5,0) node[right]{$t$};
      \draw[axes] (0,0)--(0,4.5) node[above]{$a$};
      \draw[functions] (0,0)--(4,0);
    \end{tikzpicture}
  \end{center}
  \begin{itemize}
  \item Constant velocity has a straight line in the $x-t$ graph
  \item The slope of the $x$-$t$ graph is the velocity $v$, which is constant
  \item The slope of the $v$-$t$ graph is the acceleration $a$, which is zero
    in this case
  \end{itemize}
\end{frame}



\begin{frame}{Uniform Acceleration: Constant Acceleration}
  \begin{center}
    \begin{tikzpicture}[scale=.5]
      \draw[axes] (0,0)--(4.5,0) node[right]{$t$};
      \draw[axes] (0,-1.5)--(0,4.5) node[above]{$x$};
      \draw[smooth,samples=20,domain=0:4,functions]
      plot(\x,{.35*(\x-1)*(\x-1)+1});
      \node[left] at (0,1.3){$x_0$};
    \end{tikzpicture}
    \hspace{.15in}
    \begin{tikzpicture}[scale=.5]
      \draw[axes] (0,0)--(4.5,0) node[right]{$t$};
      \draw[axes] (0,-1.5)--(0,4.5) node[above]{$v$};
      \draw[functions] (0,-1)--(4,2.5) node[pos=0,left,black]{$v_0$};
    \end{tikzpicture}
    \hspace{.15in}
    \begin{tikzpicture}[scale=.5]
      \draw[axes] (0,0)--(4.5,0) node[right]{$t$};
      \draw[axes] (0,-1.5)--(0,4.5) node[above]{$a$};
      \draw[functions] (0,1)--(4,1) node[pos=0,left,black]{$a$};
    \end{tikzpicture}
  \end{center}
  \begin{itemize}
  \item The $x$-$t$ graph for motion with constant acceleration is part of a
    \emph{parabola}
    \begin{itemize}
    \item If the parabola opens up, then acceleration is positive
    \item If the parabola opens down, then acceleration is negative
    \end{itemize}
  \item The $v$-$t$ graph is a straight line; its slope (a constant) is the
    acceleration
  \end{itemize}
\end{frame}



\begin{frame}{Simple Harmonic Motion}
  For \textbf{harmonic motions}, neither position, velocity nor acceleration
  are constant:

  \vspace{-.1in}
  \begin{center}
    \begin{tikzpicture}[scale=.5]
      \draw[axes] (0,0)--(4.5,0) node[right]{$t$};
      \draw[axes] (0,-2.5)--(0,2.5) node[above]{$x$};
      \draw[smooth,samples=50,domain=0:4,functions] plot(\x,{2*cos(150*\x)});
    \end{tikzpicture}
    \hspace{.15in}
    \begin{tikzpicture}[scale=.5]
      \draw[axes] (0,0)--(4.5,0) node[right]{$t$};
      \draw[axes] (0,-2.5)--(0,2.5) node[above]{$v$};
      \draw[smooth,samples=50,domain=0:4,functions] plot(\x,{-2*sin(150*\x)});
    \end{tikzpicture}
    \hspace{.15in}
    \begin{tikzpicture}[scale=.5]
      \draw[axes] (0,0)--(4.5,0) node[right]{$t$};
      \draw[axes] (0,-2.5)--(0,2.5) node[above]{$a$};
      \draw[smooth,samples=50,domain=0:4,functions] plot(\x,{-2*cos(150*\x)});
    \end{tikzpicture}
  \end{center}
  Bottom line: regardless of the type motion,
  \begin{itemize}
  \item The $v$-$t$ graph is the slope of the $x$-$t$ graph
  \item The $a$-$t$ graph is the slope of the $v$-$t$ graph
  \end{itemize}
\end{frame}



\begin{frame}{Area Under Motion Graphs}
  \begin{columns}
    \column{.25\textwidth}
    \begin{center}
      \begin{tikzpicture}[scale=.65]
        \draw[pink!40,fill=pink!40] (0,0)--(0,-1)--(1,0)--cycle;
        \draw[blue!20,fill=blue!20] (1,0)--(3.5,0)--(3.5,2.5)--cycle;
        \draw[functions] (0,-1)--(4,3);
        \draw[axes] (0,0)--(4.5,0) node[right]{$t$};
        \draw[axes] (0,-1.5)--(0,4.5) node[above]{$v$};
      \end{tikzpicture}
    \end{center}
    
    \column{.75\textwidth}
    The area under the $v-t$ graph is the displacement $x-x_0$.
    \begin{itemize}
    \item Area {\color{blue!20}\emph{above}} the time axis: $+$
      displacement
    \item Area {\color{red!40}\emph{below}} the time axis: $-$ displacement
    \end{itemize}
    \vspace{.2in}Likewise, the area under the $a-t$ graph is the change in
    velocity $v-v_0$.
  \end{columns}
\end{frame}



\begin{frame}{Velocity Squared vs.\ Displacement}
  If velocity information is given as a function of position\footnote{Depends
    on experimental set up} then a motion graph can be plotted using this
  kinematic equation:

  \eq{-.1in}{
    \underbracket{v^2}_y=\underbracket{v_0^2}_b+\underbracket{2a}_m
    \underbracket{(x-x_0)}_x
  }

  by plotting $v^2$ on the $y$-axis and displacement $\Delta x=x-x_0$ on the
  $x$-axis. The slope of the graph is $m=2a$. The square of the initial
  velocity ($v_0^2$) is the $y$-intercept.
\end{frame}



\begin{frame}{Graphing ``Linear'' Functions}
  This concept extends to graphing other physical quantities not relating to
  motion:
  \begin{itemize}
  \item To find the index of refraction of a material using Snell's law, plot
    $\sin\theta_i$ vs.\ $\sin\theta_2$ (rather than $\theta_1$ vs.\ $\theta_2$).
    The slope is the index $n$:

    \eq{-.1in}{
      \underbracket{\sin\theta_1}_y=\underbracket{n}_m
      \underbracket{\sin\theta_2}_x
    }
  \item To relate the period of oscillation of a simple pendulum to the length
    of the pendulum, plot $T^2$ vs.\ $L$:

    \eq{-.2in}{
      \underbracket{T^2}_y=\underbracket{\frac{4\pi^2}g}_m
      \underbracket{L}_x
    }
  \end{itemize}
\end{frame}



\section{Projectile Motion}

\begin{frame}{Projectile Motion}
  A \textbf{projectile} is an object that is launched with an initial velocity
  of $\vec v_0$ along a parabolic trajectory and accelerates only due to
  gravity.
  \begin{center}
    \begin{tikzpicture}[scale=1.1]
      \draw[axes] (0,0)--(2,0) node[right]{$x$};
      \draw[axes] (0,0)--(0,2) node[above]{$y$};
      \draw[axes] (1,0) arc (0:atan{1.2}:1) node[pos=.6,right]{$\theta$};
      \draw[vectors,rotate=atan(1.2)] (0,0)--(2,0) node[above]{$\vec v_0$};
      \draw[vectors,red] (0,0)--(0,2*sin{atan{1.2}}) node[left]{$v_y$};
      \draw[vectors,blue] (0,0)--(2*cos{atan{1.2}},0) node[below]{$v_x$};
      \draw[dotted,domain=0:4,thick] plot(\x, {1.2*\x-.2*\x*\x});
    \end{tikzpicture}
  \end{center}
  \begin{itemize}
  \item $x$-axis is the \emph{horizontal} direction, with the ($+$) direction
    pointing \emph{forward}
  \item $y$-axis is the \emph{vertical} direction, with the ($+$) direction
    pointing \emph{up}
  \item The reference point is where the projectile is launched
  \item Consistent with the right-handed Cartesian coordinate system
  \item The launch angle $\theta$ is measured above the the horizontal.
  \end{itemize}
\end{frame}



\begin{frame}{Horizontal ($x$) Direction}
  There is no acceleration (i.e.\ $a_x=0$) along the horizontal direction,
  therefore horizontal velocity is constant. The kinematic equations reduce to:

  \eq{-.1in}{
    x(t)=v_xt=\left[v_0\cos\theta\right]t
  }

  where $x(t)$ is the horizontal position at time $t$, $v_0$ is the
  magnitude of the initial velocity, $v_x=v_0\cos\theta$ is its horizontal
  component.
\end{frame}




\begin{frame}{Vertical ($y$) Direction}
  Constant acceleration due to gravity alone along the vertical direction, i.e.\
  $a_y=-g$. (Acceleration is \emph{negative} due to the way we defined the
  coordinate system.) The important equation is this one:

  \eq{-.1in}{
    y(t) = \left[v_0\sin\theta\right]t-\frac12gt^2
  }

  These two kinematic equations may also be useful:

  \vspace{-.2in}{\large
    \begin{align*}
      v_y &= \left[v_0\sin\theta\right] -gt\\
      v_y^2&=\left[v_0\sin\theta\right]^2-2gy
    \end{align*}
  }
\end{frame}



\begin{frame}{Solving Projectile Motion Problems}
  Horizontal and vertical motions are independent of each other, but there are
  variables that are shared in both directions, namely:
  \begin{itemize}
  \item Time $t$
  \item Launch angle $\theta$ (above the horizontal)
  \item Initial speed $v_0$
  \end{itemize}
  
  \vspace{.2in}When solving any projectile motion problems
  \begin{itemize}
  \item \emph{Two} equations with \emph{two} unknowns
  \item If an object lands on an incline, there will be a third equation
    describing the relationship between $x$ and $y$
  \end{itemize}
\end{frame}



\begin{frame}{Symmetric Trajectory}
  A projectile's trajectory is symmetric if the object lands at the same height
  as when it launched. These equations are \emph{not} provided in the AP Exam
  equation sheet, but it can save you a lot of time if you can use them, instead
  of deriving them during the exam.
  \begin{itemize}
  \item Time of flight
    \eq{-.1in}{T=\frac{2v_0\sin\theta}g}
  \item Range
    \eq{-.1in}{R=\frac{v_0^2\sin(2\theta)}g}
  \item Maximum height
    \eq{-.1in}{H=\frac{v_0^2\sin^2\theta}{2g}}
  \end{itemize}
\end{frame}



\begin{frame}{Maximum Range}
  \eq{-.1in}{
    R=\frac{v_0^2\sin(2\theta)}g
  }
  
  \begin{itemize}
  \item Maximum range occurs at $\theta=\ang{45}$
  \item For a given initial speed $v_0$ and range $R$, launch angle $\theta$ is
    given by:
    
    \eq{-.1in}{
      \theta_1=\frac12\sin^{-1}\left(\frac{Rg}{v_0^2}\right)
    }

    But there is another angle that \emph{gives the same range}!

    \eq{-.1in}{
      \theta_2=\ang{90}-\theta_1
    }
  \end{itemize}
\end{frame}



\section{Relative Motion}

\begin{frame}{Relative Motion}
  
  \begin{block}{}
    \textbf{All motion quantities must be measured relative to a
      \emph{frame of reference}}
  \end{block}

  \vspace{.2in}
  \begin{itemize}
  \item\textbf{Frame of reference:} the \emph{coordinate system} from which all
    physical measurements are made.
  \item In \emph{classical} mechanics, the coordinate system is the
    Cartesian system
  \item There is no absolute motion/rest: all motions are relative
  \item\textbf{Principle of Relativity:} All laws of physics are equal in all
    inertial (non-accelerating) frames of reference
  \end{itemize}
\end{frame}  



\begin{frame}{Relative Motion}
  \begin{columns}
    \column{.4\textwidth}
    \begin{tikzpicture}[scale=1.3,axes]
      \draw (0,0)--(-.5,-.5) node[left]{$x'$}
      node[pos=0,above left]{$\mathcal C$};
      \draw (0,0)--(1,0) node[right]{$y'$};
      \draw (0,0)--(0,1) node[above]{$z'$};

      \draw (2,3)--(1.5,2.5) node[left]{$x$}
      node[pos=0,above left]{$\mathcal B$};
      \draw (2,3)--(3,3) node[right]{$y$};
      \draw (2,3)--(2,4) node[above]{$z$};

      \fill[red] (3,1) circle(.03) node[right]{$A$};
    \end{tikzpicture}

    \column{.6\textwidth}
    Two frames of reference
    \begin{itemize}
    \item $\mathcal B$ with axes $x,y,z$
    \item $\mathcal C$ with axes $x',y',z'$
    \end{itemize}
    The two reference frames may (or may not) be moving relative to each other.
    The motion of the two reference frames affect how motion of
    $\color{red} A$ is calculated.
  \end{columns}
\end{frame}


\begin{frame}{Relative Motion}
  \begin{columns}
    \column{.4\textwidth}
    \begin{tikzpicture}[scale=1.3,axes]
      \draw (0,0)--(-.5,-.5) node[left]{$x'$}
      node[pos=0,above left]{$\mathcal C$};
      \draw (0,0)--(1,0) node[right]{$y'$};
      \draw (0,0)--(0,1) node[above]{$z'$};

      \draw (2,3)--(1.5,2.5) node[left]{$x$}
      node[pos=0,above left]{$\mathcal B$};
      \draw (2,3)--(3,3) node[right]{$y$};
      \draw (2,3)--(2,4) node[above]{$z$};

      \draw[vectors,blue] (0,0)--(3,1) node[midway,below]{$\vec r_{AC}$};
      \draw[vectors,orange] (2,3)--(3,1) node[midway,right]{$\vec r_{AB}$};
      \fill[red] (3,1) circle (.03) node[right]{$A$};
    \end{tikzpicture}

    \column{.6\textwidth}
    The position of $\color{red} A$ can be described by
    \begin{itemize}
    \item $\color{orange}\vec r_{AB}(t)$ (relative to frame $B$)
    \item $\color{blue}\vec r_{AC}(t)$ (relative to frame $C$)
    \end{itemize}
    It is obvious that $\color{orange}\vec r_{AB}(t)$ and
    $\color{blue}\vec r_{AC}(t)$ are different vectors
  \end{columns}
\end{frame}



\begin{frame}{Relative Motion}
  \begin{columns}
    \column{.4\textwidth}
    \begin{tikzpicture}[scale=1.3,axes]
      \draw (0,0)--(-.5,-.5) node[left]{$x'$}
      node[pos=0,above left]{$\mathcal C$};
      \draw (0,0)--(1,0) node[right]{$y'$};
      \draw (0,0)--(0,1) node[above]{$z'$};

      \draw (2,3)--(1.5,2.5) node[left]{$x$}
      node[pos=0,above left]{$\mathcal B$};
      \draw (2,3)--(3,3) node[right]{$y$};
      \draw (2,3)--(2,4) node[above]{$z$};

      \draw[vectors,violet] (0,0)--(2,3) node[midway,left]{$\vec r_{BC}$};
      \draw[vectors,blue!30] (0,0)--(3,1) node[midway,below]{$\vec r_{AC}$};
      \draw[vectors,orange!30] (2,3)--(3,1) node[midway,right]{$\vec r_{AB}$};
      \fill[pink] (3,1) circle (.03) node[right]{$A$};
    \end{tikzpicture}

    \column{.6\textwidth}
    The position vector of the origins of the two reference frames is given by
    $\color{violet}\vec r_{BC}$
    \begin{itemize}
    \item The vector pointing from the origin of frame $C$ to the origin of
      frame $B$
    \item If the two frames are moving relative to each other, then
    $\color{violet}\vec r_{BC}$ is also a function of time
    \end{itemize}
    Even without using vector notations, the relationship between the vectors
    is obvious:

    \eq{-.1in}{
      \boxed{
        {\color{blue}\vec r_{AC}} = {\color{orange}\vec r_{AB}} +
        {\color{violet}\vec r_{BC}}
      }
    }
  \end{columns}
\end{frame}



\begin{frame}{Relative Motion}
  \begin{columns}
    \column{.4\textwidth}
    \begin{tikzpicture}[scale=1.3,axes]
      \draw (0,0)--(-.5,-.5) node[left]{$x'$}
      node[pos=0,above left]{$\mathcal C$};
      \draw (0,0)--(1,0) node[right]{$y'$};
      \draw (0,0)--(0,1) node[above]{$z'$};

      \draw (2,3)--(1.5,2.5) node[left]{$x$}
      node[pos=0,above left]{$\mathcal B$};
      \draw (2,3)--(3,3) node[right]{$y$};
      \draw (2,3)--(2,4) node[above]{$z$};

      \draw[vectors,blue] (0,0)--(3,1) node[midway,below]{$\vec r_{AC}$};
      \draw[vectors,orange] (2,3)--(3,1) node[midway,right]{$\vec r_{AB}$};
      \draw[vectors,violet] (0,0)--(2,3) node[midway,left] {$\vec r_{BC}$};
      \fill[red] (3,1) circle(.03) node[right]{$A$};
    \end{tikzpicture}

    \column{.6\textwidth}
    Starting from the definition of \textbf{relative position}:

    \eq{-.13in}{
      \boxed{
        {\color{blue}\vec r_{AC}} =
        {\color{orange}\vec r_{AB}} + {\color{violet}\vec r_{BC}}
      }
    }
    
    \vspace{-.13in}Differentiating all terms with respect to time, we get the
    equation for \textbf{relative velocity}:

    \eq{-.13in}{
      \boxed{
        {\color{blue}\vec v_{AC}} =
        {\color{orange}\vec v_{AB}}+ {\color{violet}\vec v_{BC}}
      }
    }

    \vspace{-.13in}Differentiating with respect to time again, and we obtain
    the equation for \textbf{relative acceleration}:

    \eq{-.13in}{
      \boxed{
        {\color{blue}\vec a_{AC}} =
        {\color{orange}\vec a_{AB}} + {\color{violet}\vec a_{BC}}
      }
    }
  \end{columns}
\end{frame}



\begin{frame}{Relative Velocity}
  In classical mechanics, the equation for relative velocities follows the
  \textbf{Galilean velocity addition rule}, which applies to speeds much less
  than the speed of light:

  \eq{-.1in}{
    \vec v_{AC}=\vec v_{AB}+\vec v_{BC}
  }

  The velocity of $A$ relative to reference frame $C$ is the velocity of $A$
  relative to reference frame $B$, plus the velocity of $B$ relative to $C$. If
  we add another reference frame $D$, the equation becomes:

  \eq{-.1in}{
    \vec v_{AD}=\vec v_{AB}+\vec v_{BC}+\vec v_{CD}
  }
\end{frame}



\begin{frame}{Typical Problems}
  In an AP Physics C exam, questions involving only kinematics usually appear
  in the multiple-choice section. The problems themselves are not very different
  compared to the Grade 12 Physics problems, but:
  \begin{itemize}
  \item You have to solve problems faster because of time constraint
  \item You can use $g=\SI{10}{\metre\per\second\squared}$
    in your calculations to make your lives simpler
  \item Many problems are \emph{symbolic}, which means that they deal with
    the equations, not actual numbers
  \item Will be coupled with other types (e.g.\ dynamics and rotational) in
    the free-response section
  \item You \emph{will} be given an equation sheet
  \end{itemize}
\end{frame}
\end{document}
