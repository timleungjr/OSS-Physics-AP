\documentclass{../../oss-handout}
\usepackage{newtxtext}
\usepackage{enumitem}
\usepackage{titlesec}
\usepackage{xcolor,color,colortbl}

\setlength{\parindent}{0pt}
\setlength{\parskip}{6pt}
\setlength{\headheight}{26pt}

\titleformat*{\section}{\bfseries\large}
\titlespacing\section{0pt}{10pt plus 4pt minus 2pt}{4pt plus 12pt minus 2pt}

\pagestyle{plain}

\renewcommand{\institution}{Meritus Academy}
\renewcommand{\coursetitle}{AP Physics C}
\renewcommand{\term}{Fall 2024 to Winter 2025}

\title{AP PHYSICS C -- COURSE OUTLINE}
\author{Dr. Timothy Leung (\texttt{t.leung@meritusacademy.ca})}

\begin{document}
\thispagestyle{title}
\gentitle


\section{Course Objectives}
\begin{itemize}[nosep]
\item Develop analytical skills, strategies, and habits of mind required for
  scientific inquiry, including critical thinking and inferring
\item Develop communicative skills, strategies, and habits required for
  scientific inquiry
\item Learn fundamental concepts of introductory high-school physics
\end{itemize}



%\section{Teacher Information}
%Teacher name \& contact information: \underline{\hspace{4.5in}}



\section{Class \& Tutorial Times}
The AP Physics C course at Meritus Academy runs for a total of 60 hours
(twenty-four 2.5-hour classes)
%, twice a week, and then once a week in September
%and October) online
from Fall 2024 to Winter 2025. There is also a weekly online
drop-in tutorial (shared with AP Physics 1 and 2) for students to ask
course-related questions. Zoom link for the tutorial can be found in the
student account.

\bgroup
\def\arraystretch{1.25}
\begin{tabular}{|p{1.9in}|c|c|c|}
  \rowcolor{lightgray}
  \hline
  & \textbf{Day} & \textbf{Time} & \textbf{Dates} \\
  \hline\hline
  
  \textbf{Regular class time} &
  \hspace{.1in}Sundays\hspace{.1in} &
  \hspace{.1in}1:20 to 3:50 pm\hspace{.1in} &
  October 6, 2024 to March 30, 2025 \\
  \hline
  
  \textbf{Drop-in tutorial (optional)} &
  Fridays & 7:30 to 8:30 pm &  October 11, 2024 to May 2, 2025 \\
  \hline
\end{tabular}
\egroup


\section{Course Material}
No textbook is required. Presentation slides, handouts and homework sets are
downloadable from the school website; students are expected to download them
prior to class. Please have a pen/pencil (or tablet for online classes) for
note-taking, and a scientific calculator for working out in-class example
problems.



%\section{Course Pre-requisites}
%When taking this course, you should be familiar with the physics units of
%\begin{itemize}[nosep]
%\item Grade 10 Science (optics)
%\item Grade 8 Science (forces)
%\end{itemize}
%as well as topics in math including: functions, basic trigonometry



\section{Homework}
Homework questions are assigned for every class based on the topics covered.
There are usually 15 to 20 questions, consisting of multiple-choice, general
problem-solving and AP-style ``free-response'' questions. The questions are
based on university-level problems, as well as past AP and IB exams. Homework
sets are posted on the school's website as well as on \texttt{classkick.com},
and normally due at the end of the following class. Homework questions will
\emph{not} be reviewed during class time. Instead, links to additional
homework take-up videos are posted on the school website regularly.
%homework take-up tutorial on Monday nights (see Section~\ref{tutorial}).


\section{Tests}
There are \underline{two} in-class practice tests for assessment:
\begin{itemize}[nosep,leftmargin=15pt]
\item A \textbf{mechanic} test on Class 13, covering mechanics topics from
  Classes 1 to 12
\item An \textbf{electricity and magnetism} test on Class 24 (last day),
  covering material from Classes 14 to 23
\end{itemize}




%\section{Online Tutorial}
%\label{tutorial}
%In addition to the regular class time, there is a drop-in tutorial session
%(shared with Physics 11 students) for students to ask course-related questions.
%%There are two online tutorials. Homework questions from the previous class will
%%be taken-up during the Monday session, while the Thursday evening drop-in
%%sessions is for students to ask questions.
%Zoom link for the tutorial can be found in the student account.
%\begin{center}
%  \bgroup
%  \def\arraystretch{1.25}
%  \begin{tabular}{|p{2.1in}|c|c|c|}
%    \rowcolor{lightgray}
%    \hline
%    & \textbf{Day} & \textbf{Time} & \textbf{Dates} \\
%    \hline\hline
%    \textbf{Drop-in tutorial} &
%    Thursday &
%    7:00 -- 8:00 pm & 
%    January 25 -- May 23 \\
%    \hline
%  \end{tabular}
%  \egroup
%\end{center}




%\section{Academic Integrity}
%Meritus Academy values academic integrity. Students are encouraged to complete
%their homework and tests using knowledge gained from class. Handing in work
%that are copied from the internet, or generated by AI is not allowed. If a
%student is suspected to have copied, cheated or plagiarized, they will be
%temporarily assigned a 0\% or incomplete, and the issue will be brought to the
%school administration.


%\section{Homework Standard}
%\begin{itemize}
%\item You do not need to show work for multiple-choice questions. For online
%  classes, please answer them directly on selection box on Classkick. For
%  in-person classes, simply circle the correct letter.
%\item For problem-solving questions, you must show \emph{all} work by providing
%  complete and organized steps.
%  \begin{itemize}[nosep]
%  \item Always write out the relevant equations \emph{before} substituting
%    numerical values
%  \item If you introduce any new variables, explain what they are
%  \item Draw diagrams whenever necessary, or whenever it helps to explain your
%    work
%  \item Only use standard variable names in you calculations (e.g.\ when
%    calculating speed, always use $v$ instead of ``$X$'')
%  \item Show all algebraic steps and numerical calculations
%  \item Circle or box all your final answers
%  \item Proper math format must be observed (e.g.\ proper use of ``='' sign,
%    units, etc.)
%  \end{itemize}
%  In short, answer the questions as if the reader is learning the concept from
%  you, not as if they already understands it.
%\item If a question requires you to \emph{explain}, please do so using short
%  complete sentences with supporting detail. There is no need to write full
%  paragraphs.
%\end{itemize}


\section{Class Schedule}
The course covers all the topics in the AP Physics C curriculum. The dates for
each class are tentative, and may change if the teacher is away.\\
\bgroup
\def\arraystretch{1.1}
\begin{center}
  \vspace{-.5in}
  \begin{tabular}{|c|c|p{4in}|c|}
    \hline
    \rowcolor{lightgray}
    \textbf{Class} & \textbf{Date} & \textbf{Description} &
    \textbf{Homework Due} \\
    \hline\hline
    1 & October 6 & Kinematics & --- \\
    \hline
    2 & October 13 & Dynamics   & HW 1\\
    \hline
    3 & October 20 & Work and Energy & HW 2 \\
    \hline
    4 & Octoer 27 & Momentum, Impulse and Collisions & HW 3 \\
    \hline
    5 & November 3 & Centre of Mass & --- \\
    \hline
    6 & November 10 & Circular Motion & HW 5 \\
    \hline
    7 & November 17 & Rotational Motion, Part 1 & HW 6 \\
    \hline
    8 & December 1 & Rotational Motion, Part 2 & HW 7 \\
    \hline
    9 & December 8 & Rotational Motion, Part 3 & HW 8 \\
    \hline
    10 & December 15 & Harmonic Motion & HW 9 \\
    \hline
    11 & December 22 & Universal Gravitation & HW 10 \\
    \hline
    12 & January 12 & Planetary Motion & HW 11 \\
    \hline
    \rowcolor{lightgray!50}
    13 & January 19 & \textbf{Practice Mechanics Test} & HW 12 \\
    \hline
    14 & January 26 & Electrostatics & --- \\
    \hline
    15 & February 2 & Gauss's Law & HW 14 \\
    \hline
    16 & February 9 & Capacitors & HW 15 \\
    \hline
    17 & February 16 & Magnetism  & HW 16\\
    \hline
    18 & February 23 & Faraday's Law \& Magnetic Induction & HW 17 \\
    \hline
    19 & March 2 & Faraday's Law \& Magnetic Induction, Part 2 & HW 18 \\
    \hline
    20 & March 9 & Circuit Analysis (resistive) & HW 19 \\
    \hline
    21 & March 16 & Circuit Analysis (RC, LR, LC) & HW 20 \\
    \hline
    22 & March 23 & Hall Effect \& Electromagnetic Waves & HW 21 \\
    \hline
    23 & March 30 & Review & HW 22 \\
    \hline
    \rowcolor{lightgray!50}
    24 & April 7 & \textbf{Practice E\&M Test} & --- \\
    \hline
  \end{tabular}
\end{center}
\egroup
There is no class on \textbf{November 24, 2024} due to a scheduling conflict.
\newpage

\section{Classroom Expectations (Online)}
Students attending this course will be expected to:
\begin{itemize}[nosep]
\item Log into the Zoom meeting a few minutes before the start of the class
\item Have your full name displayed on the screen
\item Have your camera turned on showing your face
\item Be ready to learn and participate during class
\item Type in all your in-class questions and comments into the Zoom chat
  window. Don't wait too long before you ask a question; the longer you wait,
  the less effective it will be to answer your questions
\item Be \emph{specific} with your questions. As skilled as the teachers are,
  vague statements and questions like ``I don't understand, can you explain it
  again'' are impossible to answer
\item Please inform your teacher if you have to leave the class early for
  whatever reason
\item Be respectful to yourself, your teacher, and your fellow students. You
  are expected to act maturely and responsibly
\end{itemize}



\section{Homework Expectation}
\begin{itemize}[nosep]
\item For multiple-choice questions, please answer them directly on the web
  interface on \texttt{classkick.com} %, which is the online platform that
  %Meritus uses for homework submissions.
\item For free-response questions:
  \begin{itemize}[nosep]
  \item Show \emph{all} work by providing complete and organized steps. In
    \emph{all} AP exams, only part marks are awarded for the correct answer;
    most of the marks are for applying the correct concepts and showing your
    diligent work.
  \item If a question requires you to \emph{explain} or \emph{justify} your
    answers, do so using \underline{short} complete sentences with sufficient
    supporting detail. There is no need/reason to write long paragraphs; you
    will not have enough time during an exam, and it will not add value to
    your work.
  \item Proper math format must be used, e.g.\ proper use of ``='' sign, units,
    etc.
  \item Circle or box all your final answers for clarity.
  \end{itemize}
\item Late homework is always accepted. However, the longer you wait, the less
  meaningful they will be for your learning.
%\item Some of the questions will be taken up during class. However, this does
%  \emph{not} mean you don't need to do your homework at home. Always do your
%  best.
\end{itemize}

%\section{Tests}
%There are three practice tests in the course. They roughly follow the format
%for AP exams, with multiple-choice questions and free-response questions.
%\begin{itemize}[nosep]
%\item A take-home practice test for Class 8. This test covers material from the
%  the first 7 classes (kinematics and dynamics)
%\item A take-home practice test during Class 13. This test covers material from
%  the entire course, but with emphasis on the material from Classes 8 to 12
%  (work and energy)
%\item An in-class practice test during Class 16. This test covers all material
%  in the course, but with emphasis on topics in Classes 11 to 15 (circular
%  motion, gravity, rotational motion and harmonic motion)
%\end{itemize}
\end{document}


