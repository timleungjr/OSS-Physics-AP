\documentclass[11pt]{article}
\usepackage{bm}
\usepackage[margin=1.7cm,top=2.5cm,bottom=2.5cm,letterpaper]{geometry}
\usepackage{enumitem}
\usepackage{tikz,graphicx,wrapfig}
\usepackage{amsmath}
%\usepackage{pgfplots}
\usepackage{times}
\usepackage[document]{ragged2e}
\usepackage[none]{hyphenat}
\usepackage{siunitx}
\usepackage{multicol}
\usepackage{fancyhdr}

\newcommand{\mb}[1]{\mathbf{#1}}
\newcommand{\iii}{\bm{\hat{\imath}}}
\newcommand{\jjj}{\bm{\hat{\jmath}}}
\newcommand{\pic}[2]{\includegraphics[width=#1\textwidth]{#2}}

\setlength{\parskip}{15pt}

\sisetup{
  %  mode=text,
  detect-all,
  %number-math-rm=\mathnormal,
  %  text-sf=\sffamily,
  %  text-rm=\sffamily,
  inter-unit-product =\ensuremath{\cdot{}},
  per-mode=symbol,
  group-separator={,},
}

\pagestyle{fancy}
\chead{}
\lhead{Olympiads School}
\rhead{Advanced Placement Physics}
\lfoot{Solutions to Free Response Questions}
\cfoot{-\thepage-}
\rfoot{}


\begin{document}

\fbox{
  \begin{minipage}{.97\linewidth}
    \textbf{Free Response Question 4:} A steel ball is dropped from a point with
    $(x,y)$ coordinate of $(\SI{8}{\metre},\SI{16}{\metre})$. At the same time,
    another ball is launched from the origin with a speed of
    \SI{20}{\metre\per\second} at an angle of \ang{30}.
    \begin{enumerate}[noitemsep,topsep=0pt]
    \item Find the minimum distance of separation occur of the two balls.
    \item At what time does this separation occur?
    \item Give the coordinates of the two balls for the minimum separation.
    \end{enumerate}
  \end{minipage}
}

There are two ways to solve the problem. The first method \emph{looks} easy on
first glance, but will require a lot of calculus. The second method, on the
other hand, is a straightforward geometry problem that requires a bit of
ingenuity.

\textbf{\underline{Method 1 (not recommended)}}

The most straightforward approach is to express the distance between the steel
balls as a function of time, and then take the derivative with respect to time
to find out when it occurs $t$ and minimum value of $d$.

Let's call the steel ball being dropped $A$, and the ball that is launched
$B$. Their respective position in the coordinate system are expressed as
functions of time using kinematic equations\footnote{For the $x$ direction,
  $x=v_xt$ and for the $y$ direction, $y=y_0+v_{y0}t-\frac12gt^2$.}, and using
$g=\SI{10}{\metre\per\second^2}$ for both cases\footnote{which is acceptable
  for all AP exams} for simplicity.
\begin{align}
  \mb{x}_A &= 8\iii + (16-5t^2)\jjj \\
  \mb{x}_B &=20\cos\ang{30}t\iii+(20\sin\ang{30}t-5t^2)\jjj
\end{align}
The ``displacement'' vector between $A$ and $B$ can be expressed as:
\begin{equation}
  \Delta\mb{x}
  =\mb{x}_A-\mb{x}_B
  =(8-20\cos\ang{30}t)\iii + (16-20\sin\ang{30}t)\jjj
  \label{no-g}
\end{equation}
Not surprisingly, the gravitational acceleration term $\frac12gt^2=5t^2$ term
cancels, because both $A$ and $B$ are free-falling objects with the same
downward acceleration. The square of the \emph{distance} between $A$ and $B$ are
expressed as:
\begin{equation}
  d^2 =(8-20\cos\ang{30}t)^2 + (16-20\sin\ang{30}t)^2
\end{equation}
What we will need to do now is to take the time derivative of $d^2$ with
respect to time, and to find $d$ and $t$. This process is laborious and
tedious (and prone to error for someone uncomfortable with using chain rule)
and therefore generally not recommended. We will instead try a completely
different approach.

\textbf{\underline{Method 2}}

However, we have already noted that in Equation~\ref{no-g}, acceleration due to
gravity terms cancel, which means that the minimum separation distance $d$ does
not depend on $g$! We instead consider an observer who is falling alongside
steel ball $A$ (which is equivalent to effectively treating $g=0$.) In this
case, the observer sees that $A$ remains stationary while $B$ travels in a
straight line instead of the parabolic path of a projectile. The observer's
point of view is shown in Fig.~\ref{falling}. The minimum distance of
separation occurs at $C$ in this frame of reference, with a value of $d$.
\begin{figure}[ht]
  \begin{center}
    \begin{tikzpicture}[scale=.4]
      \draw[->](0,0)--(17,0) node[pos=1,right]{$x$};
      \draw[->](0,0)--(0,17) node[pos=1,above]{$y$};
      \fill(8,16) circle(.18) node[above]{$A$};
      \fill(8,4.62) circle(.1) node[above left]{$E$};
      \fill(8,0) circle(.1) node[below]{$D$};
      \draw[dashed](8,16)--(8,0);
      \draw[dashed](12.9,7.46)--(12.9,0) node[below]{$C'$};
      \draw[->](3,0) arc(0:30:3) node[midway,right]{\ang{30}};
      \draw[->](8,13) arc(270:300:3) node[midway,below]{\ang{30}};
      \begin{scope}[rotate=30]
        \draw[very thick](0,0)--(18,0);
        \fill(18,0) circle(.18) node[right]{$B$};
      \end{scope}
      \draw[dashed](8,16)--(12.9,7.46) node[midway,above right]{$d$};
      \fill(12.9,7.46) circle(.1) node[below right]{$C$};
    \end{tikzpicture}
  \end{center}
  \caption{Observing the steel balls while free-falling}
  \label{falling}
\end{figure}
 Using basic geometry, we can find distance $DE$ and $AE$:
 \begin{align*}
   DE&=8\tan\ang{30}=\SI{4.6}{\metre}\\
   AE&=16-DE=\SI{11.4}{\metre}
 \end{align*}
Using basic trigonometry again, we can now find the minimum distance of
separation:
\begin{displaymath}
  d=AE\cos\ang{30}=\boxed{\SI{9.9}{\metre}}
\end{displaymath}
The second part is slightly trickier, because we have (so far) ignored
acceleration due to gravity. However, there is no acceleration in the $x$
direction, i.e.\ the horizontal distance that $B$ travels is the same. We
need to compute $DC'$ which is just
\begin{displaymath}
  DC'=d\sin\ang{30}=\SI{4.9}{\metre}
\end{displaymath}
which means that in the time to reach minimum separation distance, the steel
ball $B$ has travelled $8+4.9=\SI{12.9}{\metre}$ horizontally. We can now use
the kinematic equation in the $x$-direction to compute $t$:
\begin{displaymath}
  t=\frac{\Delta x}{v_x}=\frac{12.9}{20\cos\ang{30}}=\boxed{\SI{.75}{\second}}
\end{displaymath}
Finally, we substitute $t$ back into the \emph{actual} position of $A$ and $B$
to compute their \emph{actual} location when the minimum separation occurs:
\begin{align*}
  \mb{x}_A &= 8\iii + (16-5t^2)\jjj = \boxed{8\iii+13.2\jjj} \\
  \mb{x}_B &=20\cos\ang{30}t\iii+(20\sin\ang{30}t-5t^2) \jjj
  =\boxed{12.9\iii+1.7\jjj}
\end{align*}
\newpage

\fbox{
  \begin{minipage}{.97\linewidth}
    \textbf{Free Response Question 6:} A trail bike take off from a ramp with
    velocity $\mb{v}_0$ at angle $\theta$ to clear a ditch of width $x$ and
    land on the other side, which is elevated at a height $H$.
    \begin{center}
      \pic{.4}{../homework/trail-bike.jpg}
    \end{center}
    \begin{enumerate}[noitemsep,topsep=0pt]
    \item For a given angle $\theta$ and distance $x$, what is the upper limit
      for $H$ such that the bike has an chance of making the jump?
    \item For $H$ less than this upper limit, what is the minimum take-off speed
      $v_0$ necessary for a successful jump?
    \end{enumerate}
    Neglect the size of the trail bike, and assume that covering a horizontal
    distance $x$ and a vertical distance $H$ is sufficient to clear the ditch.
  \end{minipage}
}

To solve this problem, we must break down the initial velocity $\mb{v}_0$ into
its horizontal ($\iii$) and vertical ($\jjj$) directions, i.e.:
\begin{displaymath}
  \mb{v}_0=v_o\cos\theta\iii+v_0\sin\theta\jjj
\end{displaymath}
Ignoring air resistance, the only acceleration will be due to gravity (which is
constant), in the $\jjj$ direction. Also assuming that $H$ is the maximum
height that the dirt bike reaches. We can use the kinematic equations to obtain
an expression for $H$ in terms of $v_0$ and $\theta$:
\begin{equation}
  v_y^2=v_{y0}^2-2gH\quad\longrightarrow\quad
  0=v_o^2\sin^2\theta-2gH\quad\longrightarrow\quad
  \boxed{H=\frac{v_o^2\sin^2\theta}{2g}}
  \label{HH}
\end{equation}
Note that this is the same calculation we used to obtain the maximum height
for a symmetric trajectory projectile motion. The time it takes to reach this
height is also straightforward to obtain:
\begin{equation}
  v_y=v_{y0}-gt\quad\longrightarrow\quad
  0=v_0\sin\theta-gt\quad\longrightarrow\quad
  \boxed{t=\frac{v_0\sin\theta}{g}}
\end{equation}
Likewise, this the exactly \emph{half} of the value for a symmetric projectile.
In the $\iii$ direction, there is no acceleration, and the width of the ditch
$x$ can be related to the initial velocity as:
\begin{equation}
  x=v_xt=v_0\cos\theta\left(\frac{v_0\sin\theta}{g}\right)
  =\boxed{\frac{v_0^2\sin\theta\cos\theta}{g}}
  \label{halfrange}
\end{equation}
For completeness, Eq.~\ref{halfrange} is also \emph{half} the range of a
symmetric projectile. Substituting the expression for $H$ from Eq.~\ref{HH}
into Eq.~\ref{halfrange}, and solving for $H$ obtains the answer to the first
part of the question:
\begin{equation}
  x=\frac{2H\cos\theta}{\sin\theta}
  \quad\longrightarrow\quad
  \boxed{H=\frac12x\tan\theta}
  \label{solution1}
\end{equation}
For the second part of the question, we assume that the width of the ditch
$x$ and the angle of the ramp $\theta$ are constant, and that $H$ is the only
thing that will change. Therefore, we want to express $v_0$ needed to clear $H$
in terms of $x$ and $\theta$. Then any $v_0$ less than this value will clear a
height less than $H$. Equating the expression for $H$ in Eq.~\ref{solution1} to
Eq.~\ref{HH}, and rearranging terms:
\begin{align}
  \frac{v_0^2\sin^2\theta}{2g}&=\frac12x\tan\theta
  =\frac12 x\frac{\sin\theta}{\cos\theta}\\
  v_0^2&=\frac{gx}{\sin\theta\cos\theta}=\frac{2gx}{\sin(2\theta)}
\end{align}
Finally, solving for $v_0$ we have the expression for the velocity to clear
\begin{equation}
  \boxed{v_0\geq\sqrt{\frac{2gx}{\sin(2\theta)}}}
\end{equation}
\end{document}
