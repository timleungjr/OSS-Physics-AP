\documentclass{../../../oss-handout}
\usepackage{amsmath,bm}
\usepackage{enumitem}
\usepackage{tikz}
\usepackage{siunitx}
\usepackage{titlesec}

\usepackage[ISO]{diffcoeff}

\titleformat{\section}{\fontsize{13}{15}\bfseries}{\thesection}{1em}{}
\titlespacing{\section}{0pt}{14pt}{2pt}

\titleformat{\subsection}{\bfseries}{\thesubsection}{1em}{}
\titlespacing{\subsection}{0pt}{6pt}{-3pt}

\sisetup{
  per-mode=symbol
}

\setlength{\parindent}{0pt}
\setlength{\parskip}{8pt}
\setlength{\headheight}{26pt}

\newcommand{\mb}[1]{\ensuremath\mathbf{#1}}
\newcommand{\iii}{\ensuremath\hat{\bm{\imath}}}
\newcommand{\jjj}{\ensuremath\hat{\bm{\jmath}}}
\newcommand{\kkk}{\ensuremath\hat{\bm{k}}}
\newcommand{\pic}[2]{\includegraphics[width=#1\textwidth]{#2}}

% Set the page style for the document
\pagestyle{plain}

% Course & handout information
\renewcommand{\institution}{Olympiads School, Toronto, ON, Canada}
\renewcommand{\coursetitle}{Advanced Placement Physics C}
\renewcommand{\term}{Summer 2021}

\title{Basic Vector and Calculus That You Need to Know}
\author{Dr.\ Timothy Leung}
\date{\today}

\begin{document}
\thispagestyle{title}
\gentitle


Both AP Physics C exams (Mechanics and Electricity \& Magnetism) are calculus
based, and will use vectors extensively.
Students should be familiar with the material in this handout, however, it is
likely that the calculus and vector operations used in the exams will be much
simpler. If these concepts are difficult, this should be a good time to grab a
calculus textbook and review them.

\section{Vectors}
Vectors are used extensively in physics. They are an integral part of a larger
discipline within mathematics called \textbf{linear algebra}. For the purpose
of AP Physics, it is sufficient to think of vectors as
``a number with a direction''.

\subsection{Notation}
In keeping with the convention used in \emph{most} technical journals and
university-level textbooks, vectors are \emph{printed} (e.g.\ on the slides and
handouts) using a bold face font in this course:
\begin{equation*}
  \bm{v}\quad\bm{F}_g\quad\bm{p}\quad\bm{I}
\end{equation*}
while the ``arrow on top'' notation is used when \emph{writing} (e.g.\ on the
blackboard)\footnote{Although this format is still used in \emph{some}
  introductory level physics textbooks in universities}:
\begin{equation*}
  \vec{v}\quad\vec{F}_g\quad\vec{p}\quad\vec{I}
\end{equation*}
The magnitude of vectors are expressed either with the absolute-value symbol:
\begin{equation*}
  |\bm{v}|\quad|\bm{F}_g|\quad|\bm{p}|\quad|\bm{I}|
\end{equation*}
or as a scalar quantity (afterall, the magnitude of a vector is indeed a scalar
with a positive value):
\begin{equation*}
  v\quad F_g\quad p \quad I
\end{equation*}


\subsection{Writing Vectors}
In Grades 11 and 12 Physics, vectors are usually written by separating the
magnitude from the direction. For example, a velocity vector would usually
written as:
\begin{equation*}
  \bm{v}=\SI{4.5}{\metre\per\second}\text{ [N \ang{55} E]}
\end{equation*}
This approach is based on the \textbf{polar coordinate system}, which is the
preferred coordinate system for circular motion. In general, this approach is
very intuitive for describing \emph{one} vector in
two dimensions (that's why it is used extensively in high-school level
physics courses), but it is more complicated when extended into 3D; the
coordinate system needs to be extended to \textbf{spherical coordinate system}
or the \textbf{cylindrical coordinate system}. Moreover, it is difficult to
perform vector arithmetic for \emph{rectilinear} motion.

Instead, for rectilinear motion, vectors in 2D/3D Cartesian space are generally
written in their $x$, $y$ and $z$ components using the \textbf{IJK notation}:
\begin{equation*}
  \bm{A}=A_x\bm{\hat{\imath}} + A_y\bm{\hat{\jmath}} + A_z\bm{\hat{k}}
\end{equation*}
The vectors $\bm{\hat{\imath}}$, $\bm{\hat{\jmath}}$ and $\bm{\hat{k}}$ are
\textbf{basis vectors} indicating the directions of the $x$, $y$ and $z$ axes.
Basis vectors are \textbf{unit vectors} (i.e.\ length $1$). Note that the
IJK notation does not give the magnitude of the vector, which needs to be
calculated using the general form of the pythagorean theorem:
\begin{equation*}
  A=|\bm{A}|=\sqrt{A_x^2 + A_y^2 + A_z^2}
\end{equation*}


\subsection{Vector Addition and Subtraction}

Adding and subtracting vectors is straightforward:
\begin{equation*}
  \bm{A}\pm\bm{B}=
  (A_x\pm B_x)\bm{\hat{\imath}} +
  (A_y\pm B_y)\bm{\hat{\jmath}} +
  (A_z\pm B_z)\bm{\hat{k}}
\end{equation*}
All that is required is to add or subtract each component of the vector in the
$\iii$, $\jjj$ and $\kkk$ directions.


\subsection{Dot Product}
The vector \textbf{dot product} (or \textbf{inner product} for general vectors)
is the \emph{scalar} multiplication of two vectors. This vector operation had
been used throughout Grades 11 and 12 Physics courses (although without
explicitly referring using this notation), for example, when calculating
mechanical work. It is determined by the magnitude of the two vectors and the
cosine of the angle $\theta$ between them:
\begin{equation*}
  C=\bm{A}\cdot\bm{B}=\bm{B}\cdot\bm{A}=|\bm{A}||\bm{B}|\cos\theta
\end{equation*}
In the dot product, $C$ is the \emph{projection} of the vector $\bm{A}$ onto
$\bm{B}$, or the component of $\bm{A}$ along $\bm{B}$. Note that
$\iii\cdot\iii=1$, $\jjj\cdot\jjj=1$, and $\kkk\cdot\kkk=1$. For general
vectors written in IJK notation, where the magnitude and direction of vectors
are not immediately known, the dot product is the sum of the products of
individual components of $\bm{A}$ and $\bm{B}$:
\begin{equation*}
  C=\bm{A}\cdot\bm{B}=A_xB_x+A_yB_y+A_zB_z
\end{equation*}


\subsection{Cross Products}
The vector \textbf{cross product} is the \emph{vector multiplication} of two
vectors:
\begin{equation*}
  \bm{C}=\bm{A}\times\bm{B}
\end{equation*}
The magnitude of the cross product is determined by the magnitude of $\bm{A}$
and $\bm{B}$ and the angle $\theta$ between them:
\begin{equation*}
  C=AB\sin\theta
\end{equation*}
The cross product $\bm{C}$ is perpendicular to \emph{both} $\bm{A}$ and
$\bm{B}$; its direction given by the right hand rule. Cross products are used
extensively in rotational motion and in electromagnetism.
\begin{figure}[ht]
  \centering
  \pic{.3}{cross-product.png}
  \caption{Vector cross product.}
  \label{fig:cross1}
\end{figure}
Note that unlike the dot product, the order of the cross product is important.
(This is why you have to get the right hand rule correctly.)
\begin{equation*}
  \bm{A}\times\bm{B}=-\bm{B}\times\bm{A}
\end{equation*}
In general, the cross product of any two vectors in 3D space is the determinant
of this $3\times 3$ matrix:
\begin{equation*}
  \bm{A}\times\bm{B}=
  \left|
  \begin{matrix}
    \bm{\hat{\imath}} & \bm{\hat{\jmath}} & \bm{\hat{k}}\\
    A_x & A_y & A_z\\
    B_x & B_y & B_z
  \end{matrix}
  \right|
  =(A_yB_z-A_zB_y)\bm{\hat{\imath}} +
  (A_zB_x-A_xB_z)\bm{\hat{\jmath}} +
  (A_zB_y-B_yA_x)\bm{\hat{k}}
\end{equation*}
although such notation is extremely rare in any AP exams. Most cross product
applications in AP C exams are much simpler, so we
only have to remember the circle shown in Figure~\ref{fig:cross2}.
\begin{figure}[ht]
  \centering
  \pic{.12}{cross-product-circle.png}
  \caption{Cross product circle that you will likely see in Physics C exams}
  \label{fig:cross2}
\end{figure}

The direction of the arrow gives the index of the cross product (e.g.\
$\bm{\hat{\imath}}\times\bm{\hat{\jmath}}=\bm{\hat{k}}$); going against the
direction of the arrow gives the negative of the next index (e.g.\
$\bm{\hat{k}}\times\bm{\hat{\jmath}}=-\bm{\hat{\imath}}$)


\section{Calculus}
We cannot learn physics properly without calculus\footnote{You got away with it
  for long enough in grades 11 and 12!}. Calculus was a mathematical tool that
was ``invented'' so that we can understand motion, especially non-constant
velocities and accelerations. Even in your calculus class(es), you may have
already noticed that a lot of the word problems  are really physics problems.
In this course, both forms of calculus will be used:
\begin{itemize}[nosep]
\item\textbf{Differential calculus} concerns how quickly something is changing
  (the \emph{rate of change} of a physical quantity). In mathematics, this is
  represented by the slopes of functions; in physics, it is how quickly a
  physical quantity is changing in time and/or space. Example includes
  \emph{velocity} (how quickly position changes with time), \emph{acceleration}
  (how quickly velocity changes with time), \emph{power} (how quickly work is
  done), \emph{electric fields} (how electric potential changes in space)
\item\textbf{Integral Calculus} is the opposite of differentiation. It is used
  to compute the area under a curve, or summation of many small terms. For
  exsmple, the area under the velocity vs.\ time $v$-$t$ graph gives the
  \emph{displacement} of an object; the area under the force vs.\ time
  $F$-$t$ graph gives the \emph{impulse} acted on an object, and the area
  under the force vs.\ position $F$-$d$ graph gives the \emph{work} that is
  done by that force.
\end{itemize}


\subsection{Derivative}

For any arbitrary function $f(x)$, the derivative with respect to $x$ is:
\begin{equation*}
  f'(x)=\lim_{h\rightarrow 0}\frac{f(x+h)-f(x)}{h}
\end{equation*}
The ``limit as $h$ approaches $0$'' is the mathematical way of making $h$ a
very small non-zero number.


\subsection{Basic Rules for Differentiation}

The derivative of a constant $C$ with respect to any variable is zero. This
should be obvious, since the slope of any function $f(x)=C$ is zero.
\begin{equation*}
  \diff Cx=0
\end{equation*}
A constant multiple $a$ of any function $f$ can be factored outside the
derivative:
\begin{equation*}
  \diff{}x(af)=a\diff fx
\end{equation*}
The derivative of a sum of two functions is the sum of the derivatives of the
functions:
\begin{equation*}
  \diff{}t\left(f(t)+g(t)\right) = \diff ft+\diff gt
\end{equation*}
Power Rule:
\begin{equation*}
  \diff{}t\left(t^n\right) = nt^{n-1}\quad\text{for}\quad n\neq 0
\end{equation*}
Product Rule:
\begin{equation*}
  \diff{}x\left(f(x)g(x)\right)=f'(x)g(x)+f(x)g'(x)
\end{equation*}
Chain Rule:
\begin{equation*}
  \diff{}xf\left(g(x)\right)=f'(g(x))g'(x)
\end{equation*}
Quotient Rule is rarely used in physics tests in AP or first-year university,
but you should remember it anyway:
\begin{equation*}
  \diff{}x\left[\frac{f(x)}{g(x)}\right]=
  \frac{f'(x)g(x)-g'(x)f(x)}{\left(g(x)\right)^2}
\end{equation*}



\subsection{Elementary Derivatives}
When studying \textbf{harmonic motion} and \textbf{circular motion},
trigonometric and exponential functions are often used. We will also find out
the relationship between complex exponential functions and sine/cosine
functions. %The derivatives of sines and cosines are related:
\begin{align*}
    \diff{}t \sin t &= \cos t \quad\quad\quad\\
    \diff{}t \cos t &= -\sin t
\end{align*}
And the exponential function:
\begin{equation*}
  \diff{}t e^{at} = ae^{at}
\end{equation*}

\subsection{Partial Derivatives}
Some functions have many variables (multi-variable function). For example,
gravitational potential energy $U_g$ has three variables: masses $m_1$ and
$m_2$ and the distance $r$ between them:
\begin{equation*}
    U_g(m_1,m_2,r)=-\frac{Gm_1m_2}{r}
\end{equation*}
Differentiating with respect to one variable while holding others constant
gives its \textbf{partial derivative}. (We use the $\partial$ symbol). For
example, the partial derivative of $U_g$ with respect to $r$ is
\begin{equation*}
    \diffp{U_g}{r}=\frac{Gm_1m_2}{r^2}
\end{equation*}
In case you have not noticed: the derivative is the is the relationship between
gravitational potential energy $U_g$ and the magnitude of the gravitational
force $F_g$.


\subsection{Integration}

If $F(x)$ is the anti-derivative of $f(x)$, they are related this way:
\begin{equation*}
  \diff{}xF(x)=f(x)\quad\longrightarrow\quad F(x)=\int f(x)\dl x
\end{equation*}
The mathematical proof is the \textbf{fundamental theorem of calculus}.


\subsection{Common Integrals in Physics}
Integration, while often necessary, can be very daunting, but integrals in AP
Physics C  are generally straightforward. These rules should help in most cases.

Power rule in reverse:
\begin{equation*}
  \int x^n\dl x=\frac1{n+1}x^{n+1}+C
\end{equation*}
Natural logarithm:
\begin{equation*}
  \int \frac1x\dl x =\ln |x|+C 
\end{equation*}
Sines and cosines:
\begin{align*}
  \int\cos x\dl x&=\sin x+C\\
  \int\sin x\dl x&=-\cos x+C
\end{align*}


\subsection{Definite and Indefinite Integrals}
Integrals can be either \textbf{indefinite} or \textbf{definite}. An
``indefinite'' integral is another function, e.g.\ position $\bm{x}(t)$ as a
function of time is found by integrating velocity $\bm{v}(t)$:
\begin{equation*}
  \bm{x}(t)=\int\bm{v}(t)\dl t=\cdots+\bm{C}
\end{equation*}
A \textbf{constant of integration} $\bm{C}$ is added to the integral
$\bm{x}(t)$. It is obtained through applying ``initial condition'' to the
problem. On the other hand, a \textbf{definite integral} has lower and upper
bounds. For example, given $\bm{v}(t)$, the displacement between $t_0$ and
$t_1$ can be found by integrating between these limits:
\begin{equation*}
  \Delta\bm{x}=\int_{t_0}^{t_1} \bm{v}(t)\dl t
\end{equation*}
Once we have computed the integral, we evaluate the limits:
\begin{equation*}
  \Delta\bm{x} =
  \bm{x}(t_1)\Big|^t_{t_0}=
  \bm{x}(t_1)-\bm{x}(t_0)=
  \bm{x}_1-\bm{x}_0
\end{equation*}
The constant of integration $\bm{C}$ cancels when we evaluate the upper and
lower bounds.
\end{document}
