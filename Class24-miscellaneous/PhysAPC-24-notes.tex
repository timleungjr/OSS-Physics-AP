\documentclass[12pt,aspectratio=169]{beamer}
\usetheme{metropolis}
\setbeamersize{text margin left=.5cm,text margin right=.5cm}
\usepackage[lf]{carlito}
\usepackage{siunitx}
\usepackage{tikz}
\usepackage{mathpazo}
\usepackage{bm}
\usepackage{mathtools}
\usepackage[ISO]{diffcoeff}
\diffdef{}{ op-symbol=\mathsf{d} }
\usepackage{xcolor,colortbl}

\usepackage[final]{pdfpages}


\title{Class 24: Wrapping Things Up}
\subtitle{Advance Placement Physics C}
\author[TML]{Dr.\ Timothy Leung}
\institute{Olympiads School}
\date{Updated: Summer 2022}

\newcommand{\pic}[2]{
  \includegraphics[width=#1\textwidth]{#2}
}
\newcommand{\eq}[2]{
  \vspace{#1}{\Large
    \begin{displaymath}
      #2
    \end{displaymath}
  }
}
%\newcommand{\iii}{\ensuremath\hat{\bm{\imath}}}
%\newcommand{\jjj}{\ensuremath\hat{\bm{\jmath}}}
%\newcommand{\kkk}{\ensuremath\hat{\bm{k}}}
%\newcommand{\iii}{\ensuremath\hat x}
%\newcommand{\jjj}{\ensuremath\hat y}
%\newcommand{\kkk}{\ensuremath\hat z}



\begin{document}

\begin{frame}
  \maketitle
\end{frame}


\begin{frame}{Today's Class}
  There are just two objectives in today's class:
  \begin{enumerate}
  \item Go over the final instructions on how to do well in the exams
  \item Do a mock E\&M exam
  \end{enumerate}
  And then, the course is over!
\end{frame}


\section{Final Instructions}

\begin{frame}{AP Physics C Exams}
  There are two AP Physics C exams that will be held on the same day:
  \begin{itemize}
  \item Mechanics
  \item Electricity \& Magnetism
  \end{itemize}
  Each of these exams consist of 
  \begin{itemize}
  \item 45 minutes of multiple-choice questions (35 questions)
  \item 45 minutes of free-response questions (3 questions)
  \end{itemize}
  with a 15-minute break between the two sections. We have already covered all the topics in
  both exams.
\end{frame}



\begin{frame}{Multiple-Choice Questions}
The multiple-choice questions cover \emph{every} topic that we discussed in class.
\begin{itemize}
  \item 45 minutes to complete 35 questions
  \item The questions cover a range of difficulties (some are very simple!)
  \item Scan through all the questions first before working on them. The questions
    on the last page are not necessarily more difficult than on the first.
  \item You will not penalized for wrongs answers, therefore:
  	\begin{itemize}
  	\item Eliminate as many wrong answers as possible
  	\item Guess the rest of them (this will improve your odds)
  	\end{itemize}
\end{itemize}
\end{frame}


%\includepdf{advanced-placement-answer-sheet-2013.pdf}


\begin{frame}{Free-Response Questions}
  The topics that you are most likely to see in the free-response question.
  \begin{itemize}
  \item\textbf{Mechanics Exam:}
    \begin{itemize}
    \item Kinematics \& dymamics
    \item Work and energy
    \item Rotational Motion
    \end{itemize}
  \item\textbf{Electricity and Magnetism Exam}
    \begin{itemize}
    \item Gauss's law (finding and plotting electric field strength)
    \item Faraday's Law (finding magnetic flux, induced \emph{emf}, induced current
    \item Circuits with capacitors
    \end{itemize}
  \end{itemize}
\end{frame}



%\begin{frame}{Coulomb's Law for Electrostatic Force}
%  \begin{center}
%    \begin{tikzpicture}[scale=.65]
%      \begin{scope}[->,very thick]
%        \draw[red] (0,0)--(2,0) node[right]{$\vec F_{21}$};
%        \draw[blue](8,0)--(6,0) node[left] {$\vec F_{12}$};
%      \end{scope}
%      \shade[ball color=red] circle(.2) node[above]{$q_1$};
%      \shade[ball color=blue] (8,0) circle(.2) node[above]{$q_2$};
%      \draw[dashed] (0,0)--(0,-1.5);
%      \draw[dashed] (8,0)--(8,-1.5);
%      \draw[->,thick](0,-1.3)--(8,-1.3) node[midway,below]{$\vec r_{12}$};
%    \end{tikzpicture}
%  \end{center}
%  The \textbf{electrostatic force} (or \textbf{coulomb force}) is a mutually
%  repulsive/attractive force between all charged objects. The force that charge
%  $q_1$ exerts on $q_2$ is given by \textbf{Coulomb's law}:
%
%  \eq{-.1in}{
%    \boxed{\vec F_{12}=\frac{kq_1q_2}{|\vec r_{12}|^2}\hat r_{12}}
%  }
%\end{frame}
%
%
%
%\begin{frame}{Coulomb's Law for Electrostatic Force}
%  \eq{-.1in}{
%    \boxed{\vec F_{12}=\frac{kq_1q_2}{|\vec r_{12}|^2}\hat r_{12}}
%  }
%  \begin{center}
%    \begin{tabular}{l|c|c}
%      \rowcolor{pink}
%      \textbf{Quantity} & \textbf{Symbol} & \textbf{SI Unit} \\ \hline
%      Electrostatic force    & $\vec F_{12}$ & \si\newton \\
%      Coulomb's constant     & $k$          & \si{N.m^2/C^2} \\
%      Point charges 1 and 2  & $q_1$, $q_2$ &  \si\coulomb \\
%      Distance between point charges & $|\vec r_{12}|$ & \si\metre \\
%      Unit vector of direction between point charges & $\hat r_{12}$ &
%    \end{tabular}
%  \end{center}
%\end{frame}
%
%
%
%\begin{frame}{Coulomb's Constant}
%
%  \eq{-.1in}{
%    \boxed{\vec F_{12}=\frac{kq_1q_2}{|\vec r_{12}|^2}\hat r_{12}}
%  }
%
%  The constant $k$ in the Coulomb's law is called the
%  \textbf{coulomb's constant}, defined as:
%
%  \eq{-.1in}{
%    k=\dfrac1{4\pi\epsilon_0}=\SI{8.99e9}{N.m^2/C^2}
%  }
%
%  where $\epsilon_0$ is a fundamental constant called the
%  \textbf{permittivity of free space}, or \textbf{vacuum permittivity}. It
%  measures a vacuum's ability to resist the formation of an electric field:
%
%  \eq{-.1in}{
%    \epsilon_0=\SI{8.85e-12}{C^2/N.m^2}
%  }
%\end{frame}
%
%
%\begin{frame}{Coulomb's Law for Electrostatic Force}
%  \begin{center}
%    \begin{tikzpicture}[scale=.5]
%      \begin{scope}[->,very thick]
%        \draw[red] (0,0)--(2,0) node[right]{$\vec F_{12}$};
%        \draw[blue](8,0)--(6,0) node[left] {$\vec F_{21}$};
%      \end{scope}
%      \shade[ball color=red] circle(.2) node[above]{$q_1$};
%      \shade[ball color=blue] (8,0) circle(.2) node[above]{$q_2$};
%      \draw[dashed] (0,0)--(0,-1.5);
%      \draw[dashed] (8,0)--(8,-1.5);
%      \draw[->,thick](0,-1.3)--(8,-1.3) node[midway,below]{$\vec r_{12}$};
%    \end{tikzpicture}
%  \end{center}
%  \begin{itemize}
%  \item Third law of motion: If $q_1$ exerts an electrostatic force
%    $\vec F_{12}$ on $q_2$, then $q_2$ likewise exerts a force of
%    $\vec F_{21}=-\vec F_{12}$ on $q_1$. The two forces are equal in magnitude
%    and opposite in direction.
%  \item $q_1$ and $q_2$ are assumed to be \emph{point charges} that do not
%    occupy any space
%  \item The scalar form is often used as well, since the direction of $F_q$ can
%    easily be found:
%
%    \eq{-.1in}{
%      \boxed{F_q=\frac{kq_1q_2}{r^2}}
%    }
%  \end{itemize}
%\end{frame}
%
%
%
%\begin{frame}{More Than One Charge}
%  \begin{columns}
%    \column{.4\textwidth}
%    \begin{tikzpicture}[scale=.4]
%      \shade[ball color=red] circle(.72) node[white]{$Q$};
%
%      \begin{scope}[rotate=45]
%        \draw[->,very thick,blue](.72,0)--(2.5,0) node[right]{$\vec F_1$};
%        \shade[ball color=blue] (7,0) circle(1) node[white]{$q_1$};
%      \end{scope}
%      
%      \begin{scope}[rotate=105]
%        \draw[->,very thick,green](.72,0)--(2,0)
%        node[right]{$\vec F_2$};
%        \shade[ball color=green] (4,0) circle(.65) node[white]{$q_2$};
%      \end{scope}
%
%      \begin{scope}[rotate=190]
%        \draw[->,very thick,yellow](.7,0)--(3.5,0)
%        node[above]{$\vec F_3$};
%        \shade[ball color=yellow!70!black]
%        (6,0) circle(1.1) node[white]{$q_3$};
%      \end{scope}
%
%      \begin{scope}[rotate=260]
%        \draw[->,very thick,violet](.72,0)--(1.4,0)
%        node[right]{$\vec F_4$};
%        \shade[ball color=violet] (9,0) circle(.8) node[white]{$q_4$};
%      \end{scope}
%
%      \begin{scope}[rotate=125]
%        \draw[->,ultra thick](.72,0)--(3.5,0) node[left]{$\vec F$};
%      \end{scope}
%    \end{tikzpicture}
%
%    \column{.6\textwidth}
%    For a charge $Q$ that is subjected to the influence of multiple discrete
%    point charges $q_i$, the total electrostatic force that $Q$ experiences is
%    the vector sum of all the forces $\vec F_i$:
%    
%    \eq{-.1in}{
%      \boxed{\vec F
%        =\sum_i\vec F_i
%        =kQ\left(\sum_{i=1}^N\frac{q_i}{r_i^2}\hat r_i\right)
%      }
%    }
%  \end{columns}
%\end{frame}
%
%
%
%\begin{frame}{Continuous Distribution of Charges}
%  As $N\rightarrow\infty$, the summation becomes an integral, and can now be
%  used to describe the force from charges with \emph{spatial extend} i.e.\
%  charges that take up physical space (e.g.\ a continuous distribution of
%  charges):
%
%  \eq{-.1in}{
%    \boxed{\vec F
%      =\int\dl\vec F
%      =kQ\int\frac{\dl q}{r^2}\hat r
%    }
%  }
%\end{frame}
%
%
%
%\begin{frame}{Infinitesimal Charge $\dl q$}
%  The calculation for the infinitesimal charge $\dl q$ is similar to the
%  calculation for the infinitesimal mass $\dl m$ earlier in the course (See
%  Class 5: Center of Mass)
%  \begin{itemize}
%  \item Linear charge density (for 1D problems)
%
%    \eq{-.1in}{
%      \gamma = \diff qL\quad\rightarrow\quad \dl q =\gamma\dl L
%    }
%
%  \item Surface charge density (for 2D problems)
%
%    \eq{-.1in}{
%      \sigma=\diff qA\quad\rightarrow\quad \dl q=\sigma\dl A
%    }
%
%  \item Charge density (for 3D problems)
%
%    \eq{-.1in}{
%      \rho=\diff qV\quad\rightarrow\quad \dl q=\rho\dl V
%    }
%  \end{itemize}
%\end{frame}
%
%    
%
%\section{Electric Field}
%
%\begin{frame}{Electric Field}
%  The expression for \textbf{electric field} is obtained by repeating the same
%  procedure as with gravitational field, by grouping the variables in
%  Coulomb's law:
%
%  \eq{-.1in}{
%    F_q
%    =\underbrace{
%      \left[\frac{kq_1}{|\vec r_{12}|^2}\hat r\right]
%    }_{\vec E}q_2
%  }
%
%  The electric field $\vec E$ created by $q_1$ is a vector function (called a
%  \textbf{vector field}) that shows how it influences other charged particles
%  around it.
%\end{frame}
%
%
%
%\begin{frame}{Electric Field Near a Point Charge}
%  The electric field a distance $r$ away from a point charge $q$ is given by:
%
%  \eq{-.1in}{
%    \boxed{\vec E(q,\vec r)=\frac{kq}{r^2}\hat r}
%  }
%  \begin{center}
%    \begin{tabular}{l|c|c}
%      \rowcolor{pink}
%      \textbf{Quantity} & \textbf{Symbol} & \textbf{SI Unit} \\ \hline
%      Electric field intensity    & $\vec E$ & \si{\newton\per\coulomb}\\
%      Coulomb's constant          & $k$   & \si{N.m^2/C^2} \\
%      Source charge               & $q$   & \si\coulomb \\
%      Distance from source charge & $r=|\vec r|$ & \si\metre \\
%      Outward unit vector from point source & $\hat r$ &
%    \end{tabular}
%  \end{center}
%  The direction of $\vec E$ is radially outward from a positive point charge
%  and radially inward toward a negative charge. The SI unit for electric field
%  is \textbf{newton per coulomb} (\si{\newton\per\coulomb}) or
%  \textbf{volt per meter} (\si{\volt\per\metre}). Both units are equivalent,
%  and both are used in AP exams.
%\end{frame}
%
%
%
%\begin{frame}{More Than One Charge}
%  When multiple point charges are present, the total electric field at any
%  position $\vec r$ is the vector sum of all the fields $\vec E_i$:
%    
%  \eq{-.1in}{
%    \boxed{\vec E
%      =\sum_i\vec E_i
%      =k\left(\sum_{i=1}^N\frac{q_i}{r_i^2}\hat r_i\right)
%    }
%  }
%
%  As $N\rightarrow\infty$, the summation becomes an integral, and can now be
%  used to describe the electric field generated by charges with
%  \emph{spatial extend}:
%
%  \eq{-.1in}{
%    \boxed{
%      \vec E=\int\dl\vec E=k\int\frac{\dl q}{r^2}\hat r
%    }
%  }
%  
%  This integral may be difficult to compute if the geometry of is complicated,
%  but in general, in AP Physics C, there are usually symmetry that can be
%  exploited.
%\end{frame}
%
%
%
%\begin{frame}{Think Electric Field}
%  $\vec E$ itself \emph{doesn't do anything} until another charge interacts with
%  it. And when there is a charge $q$, the electrostatic force $\vec F_q$ that
%  the charge experiences is proportional to $q$ and $\vec E$, regardless of how
%  the electric field is generated:
%
%  \eq{-.1in}{
%    \boxed{\vec F_q=q\vec E}
%  }
%
%  A positive charge in the electric field experiences an electrostatic force
%  $\vec F$ in the same direction as $\vec E$.
%\end{frame}
%
%
%
%\begin{frame}{Electric Field Lines}
%  \textbf{Electric field lines} can be used to visualize the direction of the
%  electric field.
%  \begin{center}
%    \begin{tikzpicture}[scale=.5]
%      \shade[ball color=blue] circle(.35) node[white]{\tiny $+q$};
%      \foreach \theta in {15,30,...,360}{
%        \begin{scope}[rotate=\theta,red!70!black,thick]
%          \draw[->](.35,0)--(3,0);
%          \draw (2.8,0)--(4,0);
%        \end{scope}
%        }
%    \end{tikzpicture}
%    \hspace{.2in}
%    \begin{tikzpicture}[scale=.5]
%      \shade[ball color=blue] circle(.35) node[white]{\tiny $-q$};
%      \foreach \theta in {15,30,...,360}{
%        \begin{scope}[rotate=\theta,red!70!black,thick]
%          \draw (.35,0)--(2.5,0);
%          \draw[<-](2.3,0)--(4,0);
%        \end{scope}
%        }
%    \end{tikzpicture}
%  \end{center}
%\end{frame}
%
%
%
%\begin{frame}{Electric Field from Multiple Charges}
%  \begin{center}
%    \pic{.8}{field-lines}
%  \end{center}
%  \begin{itemize}
%  \item Electric field lines must begin and/or end at a charge
%  \item Field lines do not cross
%  \item Direction of the electric field is tangent to the field lines
%  \end{itemize}
%\end{frame}
%
%
%
%\begin{frame}{Lord of the Ring Charge}
%  Suppose you have been given \emph{The One Ring To Rule Them All}, and you
%  found out that it is charged! What is its electric field at point $P$ along
%  its axis?
%  \begin{center}
%    \pic{.5}{physicsbook_emism_graphik_35}
%  \end{center}
%  Note that calculating the electric field away from the axis is very
%  difficult.
%\end{frame}
%
%
%
%\begin{frame}{Electric Field Along Axis of a Ring Charge}
%  \begin{columns}
%    \column{.3\textwidth}
%    \vspace{.1in}
%    \pic{1.1}{Fig25}
%
%    \column{.7\textwidth}
%    \begin{itemize}
%    \item We can separate the electric field $\dl\vec E$ (generated by charge
%      $\dl q$) into axial ($\dl E_x$) and radial ($\dl E_\perp$) components
%    \item Based on symmetry, $\dl E_\perp$ doesn't contribute to anything; but
%      $\dl E_x$ is pretty easy to find:
%
%      \eq{-.2in}{
%        \dl E_x =\frac{k\dl q}{r^2}{\color{red}\cos\theta}
%        =\frac{k\dl q}{r^2}{\color{red}\frac xr}
%        =\frac{kx\dl q}{(x^2+a^2)^{3/2}}
%      }
%    \end{itemize}
%  \end{columns}
%
%  \vspace{.1in}Integrating this over all charges $\dl q$, we have:
%  
%  \eq{-.1in}{
%    E_x =\frac{kx}{(x^2+a^2)^{3/2}}\int \dl q=\boxed{\frac{kQx}{(x^2+a^2)^{3/2}}}
%  }
%\end{frame}
%
%
%
%\begin{frame}{Electric Field Along Axis of a Uniformly Charged Disk}
%  Let's extend what we know to a disk of radius $a$ and charge density $\sigma$
%
%  \vspace{.1in}
%  \begin{columns}
%    \column{.37\textwidth}
%    \pic1{serway}
%
%    \column{.63\textwidth}
%    We start with the solution from the ring problem, and replace $Q$ with
%    $\dl q=2\pi\sigma r\dl r$:
%
%    \eq{-.1in}{
%      \dl E_x =\frac{2\pi kr\sigma x}{(x^2+r^2)^{3/2}}\dl r
%    }
%
%    Integrating over the entire disk:
%
%    \eq{-.1in}{
%      E_x =\pi kx\sigma\int_0^a\frac{2r}{(x^2+r^2)^{3/2}}\dl r
%    }
%    
%    This is not an easy integral!
%  \end{columns}
%\end{frame}
%
%
%
%\begin{frame}{Eclectic Field Along Axis of a Uniformly Charged Disk}
%  \begin{columns}
%    \column{.35\textwidth}
%    \pic1{serway}
%    
%    \column{.65\textwidth}
%    Luckily for us, the integral is in the form of $\int u^ndu$,
%    with $u=x^2+r^2$ and $n=\frac{-3}2$. You can find the integral in any math
%    textbook:
%
%    \eq{-.1in}{
%      E_x =2\pi k\sigma\left(1-\frac x{\sqrt{x^2+a^2}}\right)
%    }
%  \end{columns}
%\end{frame}
%
%
%
%\section{Electric Potential Energy}
%
%\begin{frame}{Electric Potential Energy}
%  The electrostsatic force is a conservative force, therefore the work done
%  by $F_q$ is related to the \textbf{electric potential energy} $U_q$:
%  
%  \eq{-.1in}{
%    W=\int\vec F_q\cdot\dl\vec r
%    =kq_1q_2\int_{r_1}^{r_2}\frac{\dl r}{r^2}
%    =-\frac{kq_1q_2} r\Big|^{r_2}_{r_1}=-\Delta U_q
%  }
%
%  where
%    
%  \eq{-.1in}{
%    \boxed{U_q=\frac{kq_1q_2} r}
%  }
%  \begin{itemize}
%  \item $U_q$ can be ($+$) or ($-$), because charges can be either ($+$) or
%    ($-$)
%  \item Positive work done by $F_q$ decreases $U_q$, while
%  \item Negative work done by $F_q$ increases $U_q$
%  \item $W$ depends on $r_1$ and $r_2$ but not \emph{how} the charge moves from
%    $r_1\rightarrow r_2$
%  \end{itemize}
%\end{frame}
%
%
%
%\begin{frame}{How it Differs from Gravitational Potential Energy}
%  \begin{columns}
%    \column{.33\textwidth}
%    \centering
%    Two positive charges:
%
%    \eq{-.1in}{U_q>0}
%    
%    \column{.33\textwidth}
%    \centering
%    Two negative charges:
%
%    \eq{-.1in}{U_q>0}
%    
%    \column{.34\textwidth}
%    \centering
%    One positive and one negative charge:
%
%    \eq{-.3in}{U_q<0}
%  \end{columns}
%  \begin{itemize}
%  \item $U_q>0$ means positive work is done to bring two charges together from
%   $r=\infty$ to $r$ (both charges of the same sign)
%  \item $U_q<0$ means negative work (the charges are opposite signs)
%  \item For gravitational potential $U_g$ is always $<0$
%  \end{itemize}
%\end{frame}
%
%
%
%\begin{frame}{Relating $U_q$ to $\vec F_q$}
%  From the fundamental theorem calculus, we can relate electrostatic force
%  ($\vec F_q$) to electric potential energy ($U_q$) by the gradient operator:
%
%  \eq{-.1in}{
%    \Delta U_q=-\int\vec F_q\cdot\dl\vec r\quad\rightarrow\quad
%    \vec F_q(r)=-\nabla U_q=-\diffp{U_q}r\hat r
%  }
%
%  Electrostatic force $\vec F_q$ always points from high to low potential
%  energy (steepest descent direction)
%\end{frame}
%
%
%
%\section{Electric Potential}
%
%\begin{frame}{Electric Potential: Using Gravity as Example}
%  An object at a specific location inside a gravitational field has a
%  gravitational potential energy proportional to its mass, i.e.\
%
%  \eq{-.1in}{
%    U_g=V_gm
%  }
%  
%  This ``constant'' $V_g$ is called the \textbf{gravitational potential}, which
%  is the \emph{gravitational potential energy per unit mass}. In the trivial
%  case with a uniform gravitational field:
%
%  \eq{-.1in}{
%    V_g=\frac{U_g}m=gh
%  }
%
%  This also applies to the general case of the gravitational
%  potential energy:
%  
%  \eq{-.1in}{
%    V_g=\frac{U_g}m=-\frac{Gm}r
%  }  
%\end{frame}
%
%
%
%\begin{frame}{Electric Potential}
%  This is also true for moving a charged particle $q$ against an electric
%  electric field created by $q_s$, and the ``constant'' is called the
%  \textbf{electric potential}. The unit for electric potential is a \emph{volt}
%  which is \emph{one joule per coulomb}, i.e.\
%  $\SI1\volt=\SI1{\joule\per\coulomb}$
%
%  \eq{-.1in}{
%    \boxed{
%      V=\frac{U_q}q
%    }
%  }
%
%  The electric potential from a source point charge $q_s$ is therefore:
%
%  \eq{-.1in}{
%    \boxed{
%      V=\frac{kq_s}r
%    }
%  }
%\end{frame}
%
%
%
%\begin{frame}{Electric Potential from Multiple Charges}
%  When there are multiple point charges present, the electric potential is
%  given by the summation:
%
%  \eq{-.1in}{
%    \boxed{
%      V =k\sum_{i=1}^N\frac{q_i}{r_i}
%    }
%  }
%
%  As $N\rightarrow\infty$ the summation becomes an integral:
%
%  \eq{-.1in}{
%    \boxed{
%      V =k\int\frac{\dl q}r
%    }
%  }
%
%  where $r$ is the distance to the infinitesimal charge $\dl q$
%\end{frame}
%
%
%
%\section{Electric Potential Difference}
%
%\begin{frame}{Potential Difference}
%  \begin{columns}
%    \column{.45\textwidth}
%    \begin{tikzpicture}[scale=.75]
%      \shade[ball color=blue] circle(.3) node[white]{\footnotesize$Q$};
%      \foreach \theta in {30,60,...,360}{
%        \draw[rotate=\theta,->,gray,thick](.3,0)--({7-4*sin(\theta/2)},0);
%      }
%      \draw[red,ultra thick,dotted,<-]
%      (2.5*cos{30}+.23,-2.5*sin{30}) to[out=0,in=270](5*cos{15},5*sin{15}-.2);
%      \draw[thick,red,rotate=-70] (2.5,0) arc(0:100:2.5) node[above]{$V_2$};
%      \draw[thick,red,rotate=-60] (5,0)   arc(0:90:5)    node[above]{$V_1$};
%      \begin{scope}[rotate=-30]
%        \shade[ball color=red](2.5,0) circle(.23) node[white]{\footnotesize$q$};
%        \draw[<->,thick](.3,0)--(2.27,0) node[midway,above]{$r_2$};
%      \end{scope}
%      \begin{scope}[rotate=15]
%        \shade[ball color=red](5,0) circle(.23) node[white]{\footnotesize$q$};
%        \draw[<->,thick](.3,0)--(4.77,0) node[pos=2/3,below]{$r_1$};
%      \end{scope}
%    \end{tikzpicture}
%
%    \column{.55\textwidth}
%    When a charge is moved from $r_1$ to $r_2$, the change in electric potential
%    energy is related to the change in electric potential by:
%
%    \eq{-.1in}{
%      \Delta U_q=U_2-U_1=q\Delta V
%    }
%
%    where $\Delta V$ is called the \textbf{potential difference}
%  \end{columns}
%\end{frame}
%
%
%
%
%\begin{frame}{Electric Potential}
%  \begin{columns}
%    \column{.4\textwidth}
%    \centering
%    \begin{tikzpicture}[scale=.75]
%      \shade[ball color=blue] circle(.3) node[white]{\tiny$\bm{+q}$};
%      \foreach \theta in {20,40,...,360}{
%        \draw[rotate=\theta,->,gray,thick](.3,0)--(3,0);
%        \draw[rotate=\theta,gray,thick](2.8,0)--(4,0);
%      }
%      \foreach \r in {1,2,3}{
%        \draw[very thick,red] circle(\r);
%        \node[red,below] at (0,-\r+.09){\footnotesize $V_\r$};
%      }
%    \end{tikzpicture}
% 
%    \column{.6\textwidth}
%    For a point charge $q$, every point at a distance $r$ will have the same
%    electric potential $V(r)$.
%    \begin{itemize}
%    \item The red lines have the same electric potential; they are called
%      \textbf{equipotential lines}, or \textbf{equipotential contours}
%    \item Equipotential lines are perpendicular to the electric field lines
%    \item Electric field lines always points from higher $V$ toward lower $V$,
%      i.e.
%
%      \eq{-.1in}{
%        V_1>V_2>V_3
%      }
%    \end{itemize}
%  \end{columns}
%\end{frame}
%
%
%
%\begin{frame}{Electric Potential}
%  \begin{columns}
%    \column{.4\textwidth}
%    \centering
%    \begin{tikzpicture}[scale=.75]
%      \shade[ball color=blue] circle(.3) node[white]{\tiny$\bm{+q}$};
%      \foreach \theta in {20,40,...,360}{
%        \draw[rotate=\theta,->,gray,thick](.3,0)--(3,0);
%        \draw[rotate=\theta,gray,thick](2.8,0)--(4,0);
%      }
%      \foreach \r in {1,2,3}{
%        \draw[very thick,red] circle(\r);
%        \node[red,below] at (0,-\r+0.09){\footnotesize$V_\r$};
%      }
%      \shade[ball color=red,rotate=-20](2,0) circle(.23)
%      node[white]{\footnotesize$Q$};
%    \end{tikzpicture}
%    
%    \column{.6\textwidth}
%    A charge $Q$ that is placed inside this electric field will now have an
%    electric potential energy of:
%
%    \eq{-.1in}{
%      U_q=QV=Q\left[\frac{kq}r\right]
%    }
%
%    in agreement with equation for electric potential energy
%  \end{columns}
%\end{frame}

\begin{frame}{Most Problems are Symbolic}
  Most free-resonse questions will be symbolic. For example, this is from the
  2019 Mechanics exam:
  \begin{center}
    \pic{.75}{2019-mech-q3}
  \end{center}
  Be patient and careful with any differentiation, integration and algebraic
  operations. Don't substitute numerical values for physical constants like
  $g$, $G$, $k$, $\epsilon_0$, $\mu_0$, etc.
\end{frame}



\begin{frame}{Interpreting Experimental Data}
  Usually 1 free-response question will involve interpreting and
  plotting data in a straight-line graph.
  \begin{center}
    \pic{.65}{2016-em-q2}
  \end{center}
  In all of these problems, you will have use the slope to find a constant.
  Sometimes are you told what to plot; sometime you have to figure it out.
\end{frame}



\begin{frame}{Reminder on plotting straight line graphs}
\begin{center}
  \begin{tikzpicture}[scale=1.5]
    \draw[axes](0,0)--(2,0) node[right]{$t$};
    \draw[axes](0,0)--(0,2) node[above]{$x$};
    \draw[red,very thick,domain=0:2] plot(\x,{.4*\x^2+.3}) node[right]{$x=x_0+\dfrac12at^2$};
    \node[below] at (1,-.5){$x$ vs.\ $t$ is quadratic};
  \end{tikzpicture}
  \hspace{.3in}
  \begin{tikzpicture}[scale=1.5]
    \draw[axes](0,0)--(2,0) node[right]{$t^2$};
    \draw[axes](0,0)--(0,2) node[above]{$x$};
    \draw[red,very thick,domain=0:2] plot(\x,{.6*\x+.3}) node[right]{slope$=\dfrac12a$};
    \node[below] at (1,-.5){$x$ vs.\ $t^2$ is linear};
  \end{tikzpicture}
\end{center}
\begin{itemize}
\item This was discussed right from the beginning of the course, in Class 1, and we have done
  numerous problems like this before.
\item You have to be familiar with the equations.
\end{itemize}
\end{frame}


\begin{frame}{Questions with Numerical Values}
  Most of the AP Physics C (both exams) free-response will lean
  towards algebraic answer, but one of the questions will still likely to
  contain numerical values. For example: this is
  from the 2018 E\& M exam:
  \begin{center}
    \pic{.75}{2018-em-q2}
  \end{center}
  In these cases, you have to consider the number of significant figures in
  your answers. Your answer should have the same number of sig.\ figs.\ as
  the given information, which will be likely either 2 or 3.
\end{frame}



\begin{frame}{Drop-In Tutorial}
  There is a drop-in tutorial for all AP Physics students (including AP 1, AP 2
  and AP C) on Thursday nights from 7:30 to 8:30pm.
  \begin{itemize}
  \item Every Thursday except March 30th; that week is moved to Friday March 31st.
  \item Anyone who is in the waiting before will have a chance to ask your
    questions
  \item Please keep your questions specific (i.e.\ don't ask ``I didn't understand
    the entire E\& M part; can you explain it to me again?''
  \end{itemize}
  You can also e-mail me with question until the end the AP exam period.
\end{frame}

\end{document}
