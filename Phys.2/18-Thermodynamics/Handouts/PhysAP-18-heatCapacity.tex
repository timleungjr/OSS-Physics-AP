\documentclass[11pt]{article}

\usepackage[margin=.8in,letterpaper]{geometry}
\usepackage{amsmath,bm}
\usepackage{txfonts} % must be loaded after amsmath?
%\usepackage{times}
\usepackage{siunitx}
\usepackage{enumitem}
\usepackage{graphicx}
\usepackage{tikz}
%\usepackage{mathpazo}
%\usepackage{xcolor,colortbl}
%\usepackage{hyperref}
%\usepackage{cancel}
\usepackage{wrapfig}
\usepackage{authblk}

%\newcommand\vertarrowbox[2]{%
%    \begin{array}[t]{@{}c@{}} #1 \\
%    \rotatebox{90}{$\xrightarrow{\hphantom{abcdefgh}}$} \\[-1ex]
%    \mathclap{\scriptstyle\text{#2}}%
%    \end{array}}

\setlength{\parindent}{0pt}
\setlength{\parskip}{8pt}


%\usetikzlibrary{decorations.pathmorphing,patterns}

\sisetup{
  detect-all,
  per-mode=symbol
}

\title{Topic 18: Heat Capacity}
\author{Timothy M.\ Leung\thanks{Ph.D., \texttt{tleung@olympiadsmail.ca}}}
\affil{Olympiads School\\Toronto, Ontario, Canada}
\date{Revised: \today}

%\newcommand{\pic}[2]{\includegraphics[width=#1\textwidth]{#2}}
%\newcommand{\mb}[1]{\ensuremath\mathbf{#1}}

\begin{document}
\maketitle

%\textbf{Kinematics} is a discipline with in mechanics for describing the
%motion of points, bodies (objects), and systems of  bodies (groups of objects).
%It is the mathematical representation of the relationship between
%\emph{position}, \emph{displacement}, \emph{distance}, \emph{velocity},
%\emph{speed} and \emph{acceleration}. Note that kinematics does \emph{not}
%deal with what causes motion.

\section{Heat Capacity}

The \textbf{heat capacity} $C$ of an object is defined to be the amount of
heat needed to raise its temperature by \SI{1}{\kelvin}, i.e.:
\begin{equation}
  C=\frac{Q}{\Delta T}
  \label{eq:heat-capacity}
\end{equation}
Of course, this depends on \emph{what substance} the objects is made of, and
\emph{how much} of the sustance (i.e.\ its mass) there is in the object.
Therefore, a better quantity to use is the \textbf{specific heat capacity}
$c$,\footnote{Note that the symbol for heat capacity is in \emph{uppercase},
  while the symbol for specific heat capacity is in lowecase} which is defined
as:
\begin{equation}
  c=\frac{C}{m}=\frac{Q}{m\Delta T}
  \label{eq:specific-heat}
\end{equation}
Note that when heat is added to a system, by the first law of thermodynamics,
the internal energy can change ($\Delta U$), and mechanical work
$W$ may also be done as well.\footnote{In this handout, we assume that
  mechanical work is \emph{positive} when it is done from the surrounding
  \emph{to} the system, and \emph{negative} when it is done \emph{by} the
  system to the surroundings} so Equation~\ref{eq:heat-capacity} is best
expressed as:
\begin{equation}
  C=\frac{\Delta U-W}{\Delta T}
  \label{eq:with-1st-law}
\end{equation}
Since mechanical work can be \emph{anything} while heat is added, $C$ is not
necessarily a well defined property. In this case, we examine two situations:
\begin{itemize}[noitemsep,topsep=0pt]
\item Heat capacity at constant volume ($C_V$)
\item Heat capacity at constant pressure ($C_P$)
\end{itemize}



\section{Heat Capacity at Constant Volume}

At constant volume, no work is being done, and therefore all the heat added
to the thermodynamic system goes to the internal energy of the system.
Therefore, we define the \textbf{heat capacity at constant volume} as:
\begin{equation}
  C_V
  =\left(\frac{\Delta   U}{\Delta T}\right)_V
  =\left(\frac{\partial U}{\partial T}\right)_V
\end{equation}
The subscript in the symbol $C_V$ indicates that the partial derivative is taken
at constant volume $V$.

\section{Heat Capacity at Constant Pressure}

In everyday life, objects usually \emph{expand} as they are heated, pushing
against the atmosphere at constant pressure as it expands. The work done by
the system to the surrounding is therfore negative:
\begin{equation}
  W=-P\Delta V
\end{equation}
And subsequently, the Equation~\ref{eq:with-1st-law} can now be expressed for
the \textbf{heat capacity at constant pressure}:
\begin{equation}
  C_P=\frac{\Delta U-W}{\Delta T}=
  \frac{\Delta U-(-P\Delta T)}{\Delta T}=
  \boxed{\left(\frac{\partial U}{\partial T}\right)_P+
  P\left(\frac{\partial V}{\partial T}\right)_P}
\end{equation}
\end{document}
