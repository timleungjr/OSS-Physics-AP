\documentclass[12pt,compress,aspectratio=169]{beamer}
\usetheme{metropolis}
\setbeamersize{text margin left=.5cm,text margin right=.5cm}
\usepackage[lf]{carlito}
\usepackage{siunitx}
\usepackage{tikz}
\usepackage{mathpazo}
\usepackage{bm}
\usepackage{mathtools}
\usepackage[ISO]{diffcoeff}
\diffdef{}{ op-symbol=\mathsf{d} }
\usepackage{xcolor,colortbl}

\setmonofont{Ubuntu Mono}
\setlength{\parskip}{0pt}
\renewcommand{\baselinestretch}{1}

\sisetup{
  inter-unit-product=\cdot,
  per-mode=symbol
}

\tikzset{
  >=latex
}

%\newcommand{\iii}{\hat{\bm\imath}}
%\newcommand{\jjj}{\hat{\bm\jmath}}
%\newcommand{\kkk}{\hat{\bm k}}


\title{Class 5: Center of Mass}
\subtitle{Advanced Placement Physics C}
\author[TML]{Dr.\ Timothy Leung}
\institute{Olympiads School}
\date{Updated: Summer 2022}

\newcommand{\pic}[2]{
  \includegraphics[width=#1\textwidth]{#2}
}
\newcommand{\eq}[2]{
  \vspace{#1}{\Large
    \begin{displaymath}
      #2
    \end{displaymath}
  }
}
%\newcommand{\iii}{\ensuremath\hat{\bm{\imath}}}
%\newcommand{\jjj}{\ensuremath\hat{\bm{\jmath}}}
%\newcommand{\kkk}{\ensuremath\hat{\bm{k}}}
\newcommand{\iii}{\ensuremath\hat\imath}
\newcommand{\jjj}{\ensuremath\hat\jmath}
\newcommand{\kkk}{\ensuremath\hat k}


\begin{document}

\begin{frame}
  \maketitle
\end{frame}



\begin{frame}{Center of Mass}
  Finding an object's center of mass is important, because
  \begin{itemize}
  \item The laws of motion are formulated by treating an objects as point
    masses (for real-life objects, we let the forces apply to the center of
    mass)
  \item Objects can have \emph{rotational} motion in addition to
    \emph{translational} motion as well (we will examine that a bit more in a
    very-important topic later)
  \end{itemize}
\end{frame}



\begin{frame}{Start with a Definition}
  The \textbf{center of mass} (``CM'') is the \emph{weighted average of the
  masses in a system.} The ``system'' may be:
  \begin{itemize}
  \item A collection of individual particles
  \item A continuous distribution of mass with constant density. In this case,
    CM is also the geometric center (\textbf{centroid}) of the object
  \item A continuous distribution of mass with varying density
  \item If the masses are inside a \emph{uniform} gravitational field, then the
    CM is also its \textbf{center of gravity} (``CG'')
  \end{itemize}
\end{frame}


\section{Finding the Center of Mass}

\begin{frame}{Simple Example}
  Two equal masses $m$ along the $x$-axis, located at $x_1$ and $x_2$. Where is
  the center of mass of the system?
  \begin{center}
    \begin{tikzpicture}
      \draw[axes] (-1,0)--(8,0) node[right]{$x$};
      \draw[mass] (2,0) circle (.2) node{$m$};
      \draw[mass] (6,0) circle (.2) node{$m$};
      \draw[thick] (0,.85)--(0,-.45) node[below]{$O$};
      \fill (4,0) circle (.05) node[above]{cm};
      \begin{scope}[very thick,->|]
        \draw (0,.35)--(2,.35) node[pos=0,left]{$x_1$};
        \draw (0,.75)--(6,.75) node[pos=0,left]{$x_2$};
        \draw[violet] (0,-.3)--(4,-.3) node[pos=0,left]{$x_\text{cm}$};
      \end{scope}
    \end{tikzpicture}
  \end{center}
  The center of mass is at the half-way point between the masses:

  \eq{-.1in}{
    x_\text{cm}=\frac{x_1+x_2}2
    \quad\quad\text{or}\quad\quad
    x_\text{cm}=\frac{mx_1+mx_2}{2m}
  }
\end{frame}



\begin{frame}{Slightly More Challenging}
  What if one of the masses are increased to $2m$? This is still not a
  difficult problem; you can still \emph{guess} the right answer without
  knowing the equation for center of mass. 
  \begin{center}
    \begin{tikzpicture}
      \draw[axes] (-4,0)--(4,0) node[right]{$x$};
      \draw[mass] (-2.5,0) circle (.3) node{$m$};
      \draw[mass] (2.5,0) circle (.42) node{$2m$};
      \fill (5/6,0) circle (.05) node[below]{cm};
    \end{tikzpicture}
  \end{center}
  The answer is still simple. The center of mass is no longer half way between
  the two masses, but now $\frac13$ the total distance from the larger masses.
  We can show using a weighted average:
  
  \eq{-.1in}{
    x_\text{cm}=\frac{mx_1+(2m)x_2}{m+2m}
  }
\end{frame}



\begin{frame}{Many Point Masses}
  The weighted average concept can now be applied to cases when there are
  masses in two or more dimensions:
  \begin{center}
    \begin{tikzpicture}
      \draw[axes] (-3,0)--(3,0) node[right]{$x$};
      \draw[axes] (0,-1.5)--(0,1.5) node[above]{$y$};
      \draw[mass] (-1.3,1) circle (.4) node{$m_1$};
      \draw[mass] (-1.5,-.5) circle (.3) node{$m_2$};
      \draw[mass] (1,.3) circle (.25) node{$m_3$};
      \draw[mass] (0,.3) circle (.2) node{$m_4$};
      \draw[mass] (2,-1) circle (.25) node{$m_5$};
    \end{tikzpicture}
  \end{center}
\end{frame}


\begin{frame}{Center of Mass Equation}
  The center of mass is defined for discrete number of masses with the weighted
  average:

  \eq{-.1in}{
    \boxed{
      \vec x_\text{cm}=\frac{\sum \vec x_i m_i}{\sum m_i}
    }
  }
  \begin{center}
    \begin{tabular}{l|c|c}
      \rowcolor{pink}
      \textbf{Quantity} & \textbf{Symbol} & \textbf{SI Unit} \\ \hline
      Position of center of mass (vector) & $\vec x_\text{cm}$ & \si\metre \\
      Position of point mass $i$ (vector) & $\vec x_i$ & \si\metre \\
      Point mass $i$ & $m_i$ & \si{\kilo\gram}
    \end{tabular}
  \end{center}
  In components:

  \eq{-.1in}{
    x_\text{cm}=\frac{\sum x_i m_i}{\sum m_i}\quad\quad
    y_\text{cm}=\frac{\sum y_i m_i}{\sum m_i}\quad\quad
    z_\text{cm}=\frac{\sum z_i m_i}{\sum m_i}
  }
\end{frame}



\begin{frame}{An Example}
  \textbf{Example:} Consider the following masses and their coordinates
  which make up a ``discrete mass'' rigid body''
  \begin{align*}
    m_1&=\SI{5.0}{\kg} &\vec x_1&=3.0\iii-2.0\kkk\\
    m_2&=\SI{10.0}{\kg}&\vec x_2&=-4.0\iii+2.0\jjj+7.0\kkk\\
    m_3&=\SI{1.0}{\kg}&\vec x_3&=10.0\iii-17.0\jjj+10.0\kkk
  \end{align*}
  What are the coordinates for the center of mass of this system?
\end{frame}



\begin{frame}{Continuous Mass Distribution}
  When the number of masses approaches infinity, this becomes a continuous
  distribution of mass. Taking the limit of masses $N\rightarrow\infty$ gives
  the integral form of our equation:

  \eq{-.1in}{
    \boxed{\vec x_\text{cm}=\frac{\int\vec x\dl m}{\int\dl m}}
  }

  What is the infinitesimal mass $\dl m$ then?
\end{frame}



\begin{frame}{Densities}
  Linear mass density (for 1D problems)

  \eq{-.1in}{
    \gamma = \diff mL\quad\rightarrow\quad \dl m =\gamma\dl L
  }

  Surface mass density (for 2D problems)

  \eq{-.1in}{
    \sigma=\diff mA\quad\rightarrow\quad \dl m =\sigma\dl A
  }

  Volume density (for 3D problems)

  \eq{-.1in}{
    \rho=\diff mV\quad\rightarrow\quad \dl m =\rho\dl V
  }
  
  The densities do not have to be constant
\end{frame}



\begin{frame}{An Example with Integrals}
  \begin{columns}
    \column{.65\textwidth}
    \textbf{Example:} A triangular plate is placed in a Cartesian coordinate
    system with two of its edges along the $x$ and $y$-axis. The length of the
    edges along the axes are $a$ and $b$ respectively. Assuming that the
    surface area density $\sigma$ is uniform, determine the coordinate of its
    center of mass.

    \column{.35\textwidth}
    \begin{tikzpicture}[scale=.8]
      \draw[axes] (0,0)--(4,0) node[right]{$x$};
      \draw[axes] (0,0)--(0,5) node[above]{$y$};
      \draw[fill=lightgray] (0,0)--(3,0) node[midway,below]{$a$}
      --(0,4)--cycle node[midway,left]{$b$};
      \uncover<2->{
        \draw[fill=red] (1,0)--(1.2,0)--(1.2,2.4)--(1,2.67)
        node[midway,right]{$\dl m=y\dl x$}--cycle;
      }
    \end{tikzpicture}
  \end{columns}
\end{frame}



\begin{frame}{Centroid}
  For an object with a uniform mass distribution (i.e.\ constant density), the
  center of mass is also its geometric center, called the \textbf{centroid}.
  The locations of centroids can be found in most physics and math textbooks.
  \begin{center}
    \pic{.65}{eng130C9_11}
  \end{center}
\end{frame}



\begin{frame}{Compound Shapes}
  For compound shapes, the center of mass is the weighted average of the center
  of mass of each component. For example, for the T-beam below:
  \begin{center}
    \begin{tikzpicture}[scale=.7]
      \draw[axes] (0,0)--(4.5,0) node[right]{$x$};
      \draw[axes] (0,0)--(0,4.5) node[left]{$y$};
      \draw[blue!70!black,mass] (0,4)--(4,4)--(4,3)--(2.5,3)
      --(2.5,0)--(1.5,0)--(1.5,3)--(0,3)--(0,4);
    \end{tikzpicture}
    \hspace{.1in}
    \begin{tikzpicture}[scale=.7]
      \draw[lightgray,fill=gray!20,thick] (0,4)--(4,4)--(4,3)--(2.5,3)--(2.5,0)
      --(1.5,0)--(1.5,3)--(0,3)--(0,4);
      \draw[lightgray,dotted,thick] (1.5,3)--(2.5,3);
      \draw[axes] (0,0)--(4.5,0) node[right]{$x$};
      \draw[axes] (0,0)--(0,4.5) node[left]{$y$};
      \fill[blue!70!black] (2,3.5) circle (.08) node[right]{$\vec x_1$};
      \fill[blue!70!black] (2,1.5) circle (.08) node[right]{$\vec x_2$};
    \end{tikzpicture}
    \hspace{.1in}
    \begin{tikzpicture}[scale=.7]
      \draw[lightgray,thick] (0,4)--(4,4)--(4,3)--(2.5,3)
      --(2.5,0)--(1.5,0)--(1.5,3)--(0,3)--(0,4);
      \draw[axes] (0,0)--(4.5,0) node[right]{$x$};
      \draw[axes] (0,0)--(0,4.5) node[left]{$y$};
      \fill[blue!30] (2,3.5) circle (.08) node[right]{$\vec x_1$};
      \fill[blue!30] (2,1.5) circle (.08) node[right]{$\vec x_2$};
      \fill[red!70!black] (2,2.64) circle (.08) node[right]{$\vec x_\text{cm}$};
    \end{tikzpicture}
  \end{center}
\end{frame}


%\begin{frame}{A Difficult Example to Try at Home}
%  Not typically an AP problem, this example shows how we can use integral to
%  find the center of mass for something very complicated.
%  \begin{columns}
%    \column{.6\textwidth}
%    \textbf{Example 3:} Find the $x$-coordinate of the center of mass in the
%    shape bound by the two functions shown on the right.
%
%    \column{.4\textwidth}
%    \begin{tikzpicture}[scale=3]
%      \draw[->](0,0)--(1.25,0) node[right]{ $x$};
%      \draw[->](0,0)--(0,1.25) node[above]{ $y$};
%      \draw[fill=green!40]
%      plot[smooth,samples=40,domain=0:1] (\x,{\x*\x*\x})--
%      plot[smooth,samples=40,domain=1:0] (\x,{\x^(.5)});
%      
%      \draw[red!70,thick]  plot[smooth,samples=40,domain=0:1] (\x,{\x*\x*\x});
%      \draw[blue!70,thick] plot[smooth,samples=40,domain=0:1] (\x,{\x^(.5)});
%      \node at (.4,.85){\textcolor{blue!70}{ $y=\sqrt{x}$}};
%      \node at (.9,.3){\textcolor{red!70}{ $y=x^3$}};
%    \end{tikzpicture}
%  \end{columns}
%\end{frame}



\begin{frame}{Symmetric Configurations}
  \begin{itemize}
  \item Any plane of symmetry, mirror line, axis of rotation, point of inversion
    \emph{must} contain the center of mass.
  \item Caveat: only works if the density distribution is also symmetric
  \item Again: if density is uniform, CM is also geometric center (centroid)
  \end{itemize}
\end{frame}



\begin{frame}{``Negative Mass''}{A Mathematical Trick}
  \begin{itemize}
  \item Where there is a ``hole'' in the geometry, treat it as having negative
    mass density $-\sigma$ in that region.
  \item Negative masses don't exist, so this is really just a trick.
  \item\textbf{Example:} What is the center of mass of this shape?
    \begin{center}
      \begin{tikzpicture}
        \draw[thick,fill=lightgray] circle (2);
        \draw[thick,fill=black!2] (0,1) circle (1);
        \fill circle (.04);
        \draw[axes,rotate=45] (0,0)--(-2,0) node[midway,below]{$r$};
        \fill (0,1) circle (.04);
        \draw[axes,rotate around={-45:(0,1)}] (0,1)--(1,1) node[right]{$r/2$};
        \draw[axes](-2.4,0)--(2.4,0) node[right]{$x$};
        \draw[axes](0,-2.4)--(0,2.4) node[above]{$y$};
      \end{tikzpicture}
    \end{center}
  \end{itemize}
\end{frame}



\begin{frame}{Negative Mass Example}
  \begin{itemize}
  \item This is how we would think of it:
    \begin{center}
      \begin{tikzpicture}[scale=.6]
        \draw[thick,fill=lightgray] circle (2);
        \draw[thick,fill=black!2] (0,1) circle (1);
        \draw circle (.04);
        \draw[axes,rotate=45] (0,0)--(-2,0) node[midway,below]{$r$};
        \fill (0,1) circle(.04);
        \draw[axes,rotate around={-45:(0,1)}] (0,1)--(1,1) node[right]{$r/2$};
      
        \draw[thick,fill=lightgray] (6,0) circle (2) node{\large$A$};
        \draw[thick] (11,1) circle (1) node{\large$B$};
        \node at (3,0) {\huge=};
        \node at (9,0) {\huge -};
      \end{tikzpicture}
    \end{center}
  \item Let the origin of the coordinate system to located at the center of $A$
  \item Based on symmetry: $x_\text{cm}=0$; only have to find $y$-coordinate.
  \end{itemize}

  \eq{-.2in}{
    y_\text{cm}
    =\frac{\sum y_i m_i}{\sum m_i}
    =\frac{m_A(0) + m_B (r/2)}{m_A+m_B}
    =\frac{-\sigma\pi\left(r/2\right)^2(r/2)}
    {\sigma\pi r^2-\sigma\pi\left(r/2\right)^2}
    =\frac{-r}6
  }
\end{frame}



\section{Momentum and Center of Mass}

\begin{frame}{Velocity of the Center of Mass}
  Take time derivative of the equation for $\vec x_\text{cm}$ to get the
  velocity at the center of mass:

  \eq{-.1in}{
    \vec v_\text{cm}=\diff{\vec x_\text{cm}}t
    =\frac1m\diff{}t\left(\int\vec x\dl m\right)
    =\frac1m\int\diff{\vec x}t\dl m
    =\frac{\int\vec v\dl m}m
  }

  Or, in the form that is familiar, the velocity of the CM is the
  weighted sum of the velocities of the distribution of mass:
  
  \eq{-.1in}{
    \vec v_\text{cm} = \frac{\int\vec v\dl m}m
  }
\end{frame}



\begin{frame}{Velocity and Momentum}
  We can also rearrange the equation for the velocity of the center of mass to
  relate it to momentum, because the term $\int\vec v\dl m$ is the net momentum
  of the mass distribution $p_\text{net}$:
  
  \eq{-.05in}{
    \vec v_\text{cm} = \frac{\int\vec v\dl m}m
    \quad\longrightarrow\quad
    \vec p_\text{net}=m\vec v_\text{cm}
  }

  During a collision, there is no change in the net momentum\footnote{Because
  there are are no external forces}, the center of mass will continue to move
  at the same velocity before/after the collision, as if the collision never
  occurred.
\end{frame}



\begin{frame}{Center of Mass During Collision}
  \vspace{.1in}
  During a collision\footnote{As we have studied in conservation of momentum in
  Physics 12 and in our previous class}, there are no external forces,
  therefore the velocity of the CM remains constant. Consider this 1D inelastic
  collision in between two masses:
  \begin{center}
    \begin{tikzpicture}[scale=.8,thick]
      \begin{scope}[violet]
        \draw (1,0) rectangle (2,1) node[midway]{$m_1$};
        \draw (4,0) rectangle (5,1) node[midway]{$m_2$};
        \draw[vectors] (.8,1.3)--(2.5,1.3) node[right]{$v_1$};
        \draw[vectors] (4,1.3)--(5,1.3) node[right]{$v_2$};
        \node[above] at (3,1.5) {Before Collision};
      \end{scope}
      \draw[vectors] (2.7,.5)--(3.7,.5) node[above]{$v_\text{cm}$};
      \begin{scope}[orange]
        \draw (10,0) rectangle (11,1) node[midway]{$m_1$};
        \draw (11,0) rectangle (12,1) node[midway]{$m_2$};
        \draw[vectors] (10.2,1.3)--(11.8,1.3) node[right]{$v'$};
        \node[above] at (11,1.5) {After Collision};
      \end{scope}
      \draw (0,0)--(14,0);
      \draw (2.7,.5) circle (.15);
      \fill (2.7,.5)--(2.85,.5) arc (0:90:.15)--cycle;
      \fill (2.7,.5)--(2.55,.5) arc (180:270:.15)
      node[below=-2] (c){\scriptsize cm\par}--cycle; 

      \draw (11,.5) circle (.15);
      \fill (11,.5)--(11.15,.5) arc (0:90:.15)--cycle;
      \fill (11,.5)--(10.85,.5) arc (180:270:.15)
      node[below=-2] {\scriptsize cm\par}--cycle; 

      \node[text width=2.05in,draw,violet,below right] (a) at (-.5,-.6)
           {Using the definition of the velocity of the CM, we find
             that \emph{before} the collision:
             \vspace{-.1in}
             \begin{displaymath}
               v_\text{cm} = \frac{\sum m_iv_i}{\sum m_i}
               =\frac{m_1v_1+m_2v_2}{m_1+m_2}
             \end{displaymath}\par
           };
      \draw[axes,violet] (a)--(c);
           
      \node[text width=2.05in,draw,orange,below left] at (14.5,-.6)
           {Using momentum conservation, we find that the final
             velocity \emph{after} the collision is:
             \vspace{-.08in}
             \begin{displaymath}
               v'=\frac{m_1v_1+m_2v_2}{m_1+m_2}=v_\text{cm}
             \end{displaymath}
             \par
           };
    \end{tikzpicture}
    \vspace{.3in}
  \end{center}
\end{frame}


\begin{frame}{Acceleration of the Center of Mass}
  The rate of change of the net momentum\footnote{i.e\ applying the 2nd law of
  motion to this distribution of masses}:
  
  \eq{-.1in}{
    \diff{\vec p_\text{net}}t =
    \diff{}t (m\vec v_\text{cm})
  }

  If the system mass is constant, then this equation reduces to:

  \eq{-.1in}{
    \diff{\vec p_\text{net}}t
    =m\diff{\vec v_\text{cm}}t
    \quad\longrightarrow\quad
    \boxed{
      \vec F_\text{net}=m\vec a_\text{cm}
    }
  }
  
  We can see that when a net force is applied to an object (or a system of
  objects), the object's (or the system's)  acceleration is evaluated at the
  CM.
  \vspace{.3in}
\end{frame}



%\begin{frame}{What This All Means}
%  \begin{itemize}
%  \item Newton was right all along by treating all objects as point masses
%    located at the CM
%  \end{itemize}
%\end{frame}

\end{document}
