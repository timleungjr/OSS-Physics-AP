\documentclass[12pt,compress,aspectratio=169]{beamer}

\usetheme{metropolis}
\setbeamersize{text margin left=.5cm,text margin right=.5cm}
%\setbeamertemplate{navigation symbols}{} % suppress nav bar
\usefonttheme{professionalfonts}
\usepackage[lf]{carlito}
\usepackage{mathpazo}
\usepackage{xcolor,colortbl}
\usepackage{siunitx}
\usepackage{mhchem}

\setmonofont{Ubuntu Mono}
\setlength{\parskip}{0pt}
\renewcommand{\baselinestretch}{1}

\sisetup{
  %detect-all,
  number-math-rm=\mathnormal,
  inter-unit-product={\ensuremath{\cdot}},
  per-mode=symbol
}

\title{Topic 17: Nuclear Physics}
\subtitle{AP Physics 2}
\author{Dr.\ Timothy Leung}
\institute{Olympiads School, Toronto, ON, Canada}
\date{January 2021}


\newcommand{\pic}[2]{\includegraphics[width=#1\textwidth]{#2}}
\newcommand{\eq}[2]{
  \vspace{#1}{\Large\begin{displaymath}#2\end{displaymath}}
}

\begin{document}

\begin{frame}
  \titlepage
\end{frame}



\section{Nucleus of the Atom}

%\begin{frame}{The Atomic Model}
%  To begin our discussion on nuclear physics, we start with a model of the atom.
%  The \textbf{Rutherford-Bohr atomic model} looks like this:
%  \begin{center}
%    \pic{.5}{graphics/bohr-model-for-aluminium-clipart}
%  \end{center}
%\end{frame}
%


\begin{frame}{The Rutherford-Bohr Model}
  In the \textbf{Rutherford-Bohr atomic model}, an atom consists of
  \begin{itemize}
  \item A dense positively-charged nucleus
  \item Negatively-charged electrons ``orbiting'' the nucleus in predefined
    energy levels
  \item Most of the atom is empty space
  \end{itemize}
  Most of the atom is empty space, although almost all the mass is concentrated
  at the nucleus
\end{frame}



\begin{frame}{Nucleus of an Atom}
  The nucleus---where most of the mass of the atom is concentrated---are made
  up of \textbf{nucleons} which are:
  \begin{itemize}
  \item Positively charged \textbf{protons}, and 
  \item Electrically neutral \textbf{neutrons}
  \end{itemize}
\end{frame}



\begin{frame}{Balancing Fundamental Forces}
  \begin{center}
    \pic{.25}{graphics/Radioact}
  \end{center}
  The nucleus of an atom is held together by balancing two fundamental forces:
  \begin{itemize}
  \item\textbf{Electromagnetic force}: the repulsive force between protons
    \begin{itemize}
    \item The force drop off as the square of the distance (inverse square law)
    \end{itemize}
  \item\textbf{Nuclear strong force}: short-distance force
    holding nucleons together
    \begin{itemize}
    \item Attractive between nucleons at about $\SI1{\femto\metre}$
      (\SI{e-15}{\metre})
    \item Insignificant at distances beyond \SI{2.5}{\femto\metre}
    \item  Repulsive at distances below \SI{.7}{\femto\metre}
    \end{itemize}
  \end{itemize}
\end{frame}


  


\begin{frame}{Atomic Properties}
  The nucleus of an atom is identified by two numbers:
  \begin{center}
    \pic{.45}{graphics/helium-notation}
  \end{center}
  \begin{itemize}
  \item\textbf{Atomic number} ($Z$): the number of protons
    \begin{itemize}
    \item Determines what element it is, and 
    \item its chemical properties
    \end{itemize}
  \item\textbf{Mass number} ($A$): number of nucleons
  \item The number of neutrons ($N$) is therefore $N=A-Z$
  \end{itemize}
\end{frame}



\begin{frame}{Isotopes}
  Carbon has three common \textbf{isotopes}, all with the same atomic numbers,
  but different number of neutrons (called ``carbon-12'', ``carbon-13'' and
  ``carbon-14'' respectively). The atomic number is sometimes omitted.
  
  \vspace{-.3in}{\Huge
    \begin{displaymath}
      \ce{^{12}C}\quad\ce{^{13}C}\quad\ce{^{14}C}
    \end{displaymath}
  }

  Hydrogen also has three common isotopes (hydrogen, deuterium and tritium)
  
  \vspace{-.3in}{\Huge
    \begin{displaymath}
      \ce{^1H}\quad\ce{^2H}\quad\ce{^3H}
    \end{displaymath}
  }

  On average, there are \num{2.6} isotopes for each element.
\end{frame}


\begin{frame}{Mass of the Nucleus}
  The \textbf{unified atomic mass unit} (\si{u})\footnote{It replaces the
    ``atomic mass unit'' (\si{amu}) which is no longer used. The atomic mass
    value in the periodic table is a weighted average across all isotopes.} is
  defined as $1/12$ of the rest mass of a carbon-12 atom in its nuclear ground
  state:
  
  \eq{-.2in}{
    \SI1{u}=\frac1{12}m(\ce{^{12}C})\approx\SI{1.66054e-27}{\kilo\gram}
  }

  The rest mass of the elementary particles can be expressed in this unit:
  \begin{center}
    \begin{tabular}{l|l|l}
      \rowcolor{pink}
      \textbf{Particle} & \textbf{Mass} (\si{u}) & \textbf{Mass} (\si{kg})\\
      \hline
      Proton   & \num{1.007276} & \num{1.672614e-27} \\
      Neutron  & \num{1.008665} & \num{1.674920e-27} \\\hline
      Electron & \num{0.000549} & \num{9.10956e-31}
    \end{tabular}
  \end{center}
  \vspace{.3in}
\end{frame}



\section{Mass Defect \& Nuclear Binding Energy}

\begin{frame}{Mass-Energy Equivalence}
  One of the discoveries in special relativity\footnote{Einstein, A., ``Does
    the Inertia of a Body Depend Upon Its Energy Content?'', \emph{Annelen der
    Physik}, 18(13):639-641, 21 November, 1905.} is the fundamental equivalence
  of mass and energy. While the derivation is based on \emph{relativistic
    kinematics}, it also applies to \emph{quantum mechanics}, from which
  nuclear physics is based on.
  
  \eq{-.2in}{
    \boxed{E=mc^2}
  }
  \begin{center}
    \begin{tabular}{l|c|c}
      \rowcolor{pink}
      \textbf{Quantity} & \textbf{Symbol} & \textbf{SI Unit} \\ \hline
      Energy                  & $E$ & \si{\joule}\\
      Rest mass of a particle & $m$ & \si{\kilo\gram}\\
      Speed of light          & $c$ & \si{\metre\per\second}
    \end{tabular}
  \end{center}
  The speed of light is a universal constant:
  $c=\SI{2.998e8}{\metre\per\second}$.
  \vspace{.2in}
\end{frame}




\begin{frame}{Mass-Energy Equivalence}
  Any change in energy is \emph{always} accompanied by a change in mass, and
  vice versa.
  
  \eq{-.2in}{
    E=mc^2\quad\longrightarrow\quad\Delta E=\Delta mc^2
  }

  \begin{itemize}
  \item\vspace{-.15in}The concept is difficult to visualize in everyday life.
    Example: A baseball moving at \SI{160}{\kilo\metre\per\hour} is more
    massive than a baseball travelling at \SI{20}{\kilo\metre\per\hour}, but
    the $\Delta m$s is too small to measure.
  \item But in cases when
    \begin{itemize}
      \item speeds approaches the speed of light ($v>0.3c$), or when
      \item the masses are small (e.g.\ electrons, protons, neutrons)
    \end{itemize}
    the difference significant
  \end{itemize}
\end{frame}



\begin{frame}{Mass Defect}
  %When comparing the sum of the mass of the nucleons with the actual mass of
  %the nucleus, we find that
  The sum of the masses of the nucleons are always higher than the mass of the
  nucleus itself. \textbf{Example:} the mass of the alpha particle (helium
  nucleus) is lower than 2 protons plus 2 neutrons:
  \begin{center}
    \pic{.45}{graphics/nucbind}
  \end{center}

\end{frame}



%  We can express the mass of elementary particles in unified atomic mass unit:
%
%\end{frame}
%  The total mass of an atom is \emph{always} lower than the sum of the masses of
%  the proton and neutrons.
%  This discrepancy is called the \textbf{mass defect}.
%  The additional energy stored in the atom is exhibited as a change in mass.
%\end{frame}



\begin{frame}{Mass Defect}
  This difference in mass is called the \textbf{mass defect} $\Delta m$, which
  can be calculated with a simple equation:
  
  \eq{-.2in}{
    \boxed{
      \Delta m=\left[Zm_P+(A-Z)m_N\right]-m_A
    }
  }
  \begin{center}
    \begin{tabular}{l|c|c}
      \rowcolor{pink}
      \textbf{Quantity}        & \textbf{Symbol} & \textbf{SI Unit} \\ \hline
      Mass defect              & $\Delta m$ & \si{\kilo\gram}\\
      Atomic number and mass numbers & $Z$, $A$ & \\
      Rest mass of a proton and neutron & $m_P$, $m_N$ & \si{\kilo\gram}\\
      Rest mass of the nucleus & $m_A$ & \si{\kilo\gram}\\
    \end{tabular}
  \end{center}
\end{frame}



\begin{frame}{Mass Defect}
  Mass defect exists in any atoms because the nucleus of an atom is always
  \emph{in a lower energy state} than the individual nucleons alone
  
  \vspace{.25in}Similar examples:
  \begin{itemize}
  \item The total energy of a planet orbiting the Sun is lower than the two
    objects individually
  \item An electron orbiting the nucleus is at a lower energy state also
  \end{itemize}
\end{frame}



\begin{frame}{Nuclear Binding Energy}
  The amount of energy that is equivalent to the mass defect is called the
  \textbf{nuclear binding energy} $E_b$, defined as:

  \eq{-.2in}{
    \boxed{E_b=(\Delta m)c^2}
  }
  \begin{center}
    \begin{tabular}{l|c|c}
      \rowcolor{pink}
      \textbf{Quantity}      & \textbf{Symbol} & \textbf{SI Unit} \\ \hline
      Nuclear binding energy & $E_b$      & \si{\joule}\\
      Mass defect            & $\Delta m$ & \si{\kilo\gram}\\
      Speed of light         & $c$        & \si{\metre\per\second}
    \end{tabular}
  \end{center}
  \begin{itemize}
  \item The energy required to break up the nucleus into its individual
    nucleons
  \item Generally expressed in \emph{electron volts} (\si{\electronvolt})
    where $\SI1{\electronvolt}=\SI{1.602e-19}{\joule}$, rather than in joules
  \end{itemize}
\end{frame}



\begin{frame}{Nuclear Binding Energy}
  The nuclear binding energy is the amount of work required to separate the
  nucleons
  \begin{center}
    \pic{.5}{graphics/CNX_UPhysics_43_02_BindEnergy}
  \end{center}
  The higher the binding energy, the more \emph{tightly bound} a nucleus is.
\end{frame}



\begin{frame}{Nuclear Binding Energy \& Stability of the Nucleus}
  The nuclear binding energy is highest for iron-56
  ($E_b=\SI{492.275}{\mega\electronvolt}$, or \SI{8.7906}{\mega\electronvolt}
  per nucleon)

  \vspace{.1in}
  \begin{columns}
    \column{.5\textwidth}
    \pic{1}{graphics/Binding-energy-curve}
 
    \column{.5\textwidth}
    It means that the nucleus
    \begin{itemize}
    \item requires the most energy to separate the nucleons
    \item most tightly bound
    \item most stable
    \end{itemize}
    To achieve greater stability in the nucleus
    \begin{itemize}
    \item Heavier atoms can split into lighter nuclei, while
    \item Lighter atoms can combine into heavier nuclei
    \end{itemize}
  \end{columns}
\end{frame}



%\begin{frame}{Mass Defect \& Nuclear Binding Energy}
%  \textbf{Example:} Given that the mass of a lithium-7 atom is
%  $m=\SI{7.01600}{u}$, determine its mass defect and binding energy.
%\end{frame}



\section{Radioactivity}

%\begin{frame}{Radioactive Decay}
%  \begin{itemize}
%  \item Heavier elements: protons are further apart, and
%    the strong nuclear force is weakened by their distance
%  \item The nucleus can spontaneously disintegrate and releases energy
%  \end{itemize}
%\end{frame}



\begin{frame}{Radioactive Decay}
  \textbf{Radioactivity}\footnote{Or \textbf{radioactive decay}} is the
  spontaneously disintegration of a nucleus. There are \emph{three} types of
  radioactive decay:
  \begin{itemize}
  \item\textbf{Alpha decay}, or $\alpha$-decay
  \item\textbf{Beta decay}, or $\beta$-decay
  \item\textbf{Gamma decay}, or $\gamma$-decay
  \end{itemize}
\end{frame}



\begin{frame}{Alpha Decay}
  In $\alpha$-decay, an \textbf{alpha particle} (a helium-4 nucleus, with $2$
  protons and $2$ neutrons), is spontaneously emitted from the nucleus. Example:
  plutonium-240 nucleus decays into a uranium-236 nucleus, emitting an alpha
  particle
  \begin{center}
    \pic{.5}{graphics/alpha-decay}
  \end{center}

  \eq{-.3in}{
    \boxed{\ce{^{240}_{94}Pu -> ^{236}_{92}U + ^4_2He}}
  }
  
\end{frame}



\begin{frame}{General Formula for Alpha Decay}
  The general formula for an $\alpha$-decay is shown by the equation:
  
  \eq{-.15in}{
    \boxed{\ce{^A_ZX -> ^{A-4}_{Z-2}Y + ^4_2He}}
  }
  
  The \textbf{parent atom} ($X$) is the reactant, and the
  \textbf{daughter atom} ($Y$) is the product. This nuclear reaction is
  called a \textbf{transmutation}, because the atomic number changes during the
  decay, and a new element is formed.
  \begin{itemize}
  \item The nuclear binding energy of the daughter atom and the alpha
    particle are \emph{higher} than the parent atom.
  \item The combined mass of the daughter atom plus the alpha particle is
    \emph{less} than the parent atom, meaning that energy is released
  \end{itemize}
\end{frame}



\begin{frame}{Beta-Negative Decay}
  Beta decays involve the emission/capture of a \textbf{beta particle}
  (electron or a positron). In beta-negative decay ($\beta^-$-decay), a neutron
  spontaneously decays into a proton and an electron, and the electron is
  ejected from the nucleus. For example, the $\beta^-$-decay of a
  tritium%\footnote{Hydrogen atom with two neutrons}
  atom is:
  \begin{center}
    \pic{.6}{graphics/beta-negative}
  \end{center}

  \eq{-.3in}{
    \boxed{\ce{^3_1H -> ^3_2He + ^0_-1e}}
  }
\end{frame}



\begin{frame}{Beta-Negative Decay}
  The general form for $\beta^-$-decay is given by:
  
  \eq{-.15in}{
    \boxed{\ce{^A_ZX -> ^A_{Z+1}Y + ^0_{-1}e}}
  }

  The daughter atom has a higher nuclear binding energy than the parent atom.
\end{frame} 



\begin{frame}{Beta-Positive Decay}
  In beta-positive decay ($\beta^+$-decay), a proton spontaneously decays into
  a neutron and a positron. Example: decay of carbon-11 into boron-11:

  \begin{center}
    \pic{.65}{graphics/beta-positive1}
  \end{center}

  \eq{-.35in}{
    \boxed{\ce{^{11}_6C -> ^{11}_5B + ^0_{+1}e}}
  }
\end{frame}



\begin{frame}{Beta-Positive Decay}
  The general equation for $\beta^+$-decay is given by:
  
  \eq{-.15in}{
    \boxed{\ce{^A_ZX -> ^A_{Z-1}Y + ^0_{+1}e}}
  }

  The positron released in the decay will be immediately annihilated by an
  electron, releasing a pair of gamma particles, each with an energy of
  \SI{.511}{\mega\electronvolt}, moving in opposite directions:

  \eq{-.15in}{
    \boxed{\ce{^0_{+1}e + ^0_{-1}e -> 2(^0_0$\gamma$)}}
  }
\end{frame}



\begin{frame}{Electron Capture}
  \textbf{Electron capture} is a form of beta decay where an electron is
  absorbed by a nucleus and combines with a proton to form a neutron.
  For example:

  \begin{center}
    \pic{.55}{graphics/electron-capture}
  \end{center}

  \eq{-.35in}{
    \boxed{\ce{^{56}_{28}Ni + ^0_{-1}e -> ^{56}_{27}Co }}
  }
\end{frame}



\begin{frame}{Electron Capture}
  The general equation for electron capture is given by:
  
  \eq{-.15in}{
    \boxed{\ce{^A_ZX + ^0_{-1}e -> ^A_{Z-1}Y}}
  }
  
  The electron that is absorbed usually comes from the lowest energy shell
  ($n=1$, or K-shell) that is closest to the nucleus, so it is often called
  \emph{K-capture}.
\end{frame}



\begin{frame}{Gamma Radiation}
  \textbf{Gamma decay} ($\gamma$-decay) occurs after a nuclear reaction
  (e.g.\ $\alpha$ or $\beta$ decay)
  \begin{itemize}
  \item The daughter nucleus is in a high-energy (excited) state
  \item The nucleus spontaneously releases energy in a \textbf{gamma particle}
    to return to a lower (therefore more stable) energy state.
  \item A gamma particle:
    \begin{itemize}
    \item is a highly energetic form of electromagnetic radiation that is
      emitted as a \textbf{photon}
    \item Has zero mass
    \end{itemize}
  \end{itemize}
\end{frame}


\begin{frame}{Gamma Radiation}
  As an example, the $\gamma$-decay of a helium-3 atom is given by:
  \begin{center}
    \pic{.65}{graphics/gamma-decay}
  \end{center}

  \eq{-.25in}{
    \boxed{\ce{^3_2He^* -> ^3_2He + ^0_0$\gamma$}}
  }
\end{frame}



\begin{frame}{Gamma Radiation}
  The general equation for $\gamma$-decay is:

  \eq{-.15in}{
    \boxed{\ce{^A_ZX^* -> ^A_ZX + ^0_0$\gamma$}}
  }
  \begin{itemize}
  \item The parent and daughter nuclei are identical
  \item Only the energy level of the nucleus has changed
    \begin{itemize}
    \item The asterisk indicates that the parent an excited state
    \end{itemize}
  \item Notice that the mass number and atomic number of a gamma ray
    (\ce{^0_0$\gamma$}) are both zero.
  \end{itemize}
\end{frame}



\begin{frame}{Energies of Radiation}
  Radioactive particles post danger to living tissues, because
  \begin{itemize}
  \item they can ionize (or strip the electrons from) atoms
  \item $\alpha$ particles have strongest ionizing ability, but can only travel
    a relatively short distance before becoming absorbed
  \item $\beta$ particles and $\gamma$ rays have a greater penetrating range in
    air and must be shielded against
  \end{itemize}
  \begin{tabular}{l|l|c|l}
    \rowcolor{pink}
    \textbf{Type} & \textbf{Radiation} & \textbf{Charge} &
    \textbf{Penetrating Ability}\\ \hline
    $\alpha$-decay   & Alpha particle (He-4 nucleus) & $+2$ & Skin or paper\\
    $\beta^-$-decay  & Beta particle (electron)          & $-1$ &
    thin sheet of aluminum\\
    $\beta^+$-decay  & Beta particle (positron)          & $+1$ &
    thin sheet of aluminum\\
    $e^-$ capture & None                              & N/A  & N/A \\
    $\gamma$-decay   & Gamma particle (photon)           & $0$  &
    Few centimetres of lead
  \end{tabular}
\end{frame}



\begin{frame}{Half-Life}
  While the radioactive decay of a single atom seems to be \emph{random}, when
  there are a large number of atoms, the overall rate of decay is very
  predictable.
  
  \eq{-.2in}{
    \boxed{N=N_0\left(\frac12\right)^{\frac{t}\tau}}
  }
  \begin{center}
    \begin{tabular}{l|c|c}
      \rowcolor{pink}
      \textbf{Quantity}      & \textbf{Symbol} & \textbf{SI Unit} \\ \hline
      Amount of radioactive material & $N$  & \si{\kilogram}\\
      Initial sample amount          & $N_0$ & \si{\kilo\gram}\\
      Time                           & $t$  & \si{\second} \\
      Half-life                      & $\tau$ & \si{\second}
    \end{tabular}
  \end{center}
  \textbf{Half-life} is the time it requires for a radioactive amount to decay
  to half of its original value.
\end{frame}



\begin{frame}{Half-Life of Radioactive Isotopes}
  The half-life of radioactive substance can vary from a fraction of
  a second to billions of years.
  \begin{center}
    \begin{tabular}{l|c}
      \rowcolor{pink}
      \textbf{Substance} & \textbf{Half-Life} ($t_0$) \\ \hline
      Polonium-215 & \SI{.0018}{\second} \\
      Bismuth-212  & \SI{60.5}{\second} \\
      Sodium-24    & \SI{15}{\hour} \\
      Iodine-131   & \SI{8.07}{\day} \\
      Cobalt-60    & \SI{5.26}{yr} \\ 
      Radium-226   & \SI{1600}{yr} \\
      Uranium-238  & \SI{4.5e9}{yr}
    \end{tabular}
  \end{center}
\end{frame}



\section{Nuclear Fission}

\begin{frame}{Nuclear Fission}
  The release of energy in a nuclear reaction primarily comes from fission,
  where a heavier atomic is split into lighter atoms. For example, the fission
  reaction of uranium-235 splitting into krypton-92 and barium-141 atoms:
  \begin{center}
    \pic{.75}{graphics/fission1}
  \end{center}

  \eq{-.3in}{
    \boxed{\ce{^{235}_{92}U + ^1_0n -> ^{92}_{36}Kr + ^{141}_{56}Ba + 3(^1_0n) +
    energy}}
  }
\end{frame}


\begin{frame}{Nuclear Fission}
  \begin{center}
    \pic{.5}{graphics/fission1}
  \end{center}
  \begin{itemize}
  \item\vspace{-.25in}Fission reaction begins with the uranium nucleus
    capturing a neutron
  \item The additional neutron creates instability in the nucleus, causing it
    to split
  \item In splitting the atom, 3 neutrons are released from the nucleus
  \item The neutrons may be captured by other uranium-235 atoms, causing further
    reaction
%    \begin{itemize}
%    \item Generally the neutrons are too fast to be captured, so in a nuclear
%      reactor, they are slowed down by heavy water (water mulecules with
%      deuterium)
%    \end{itemize}
  \end{itemize}
\end{frame}

\begin{frame}{Example Problem}
  \textbf{Example:} What is the energy yield of the following fission reaction?

  \eq{-.2in}{
    \ce{^{235}_{92}U + ^1_0n -> ^{140}_{55}Cs + ^{93}_{37}Rb + 3(^1_0n)}
  }

  \begin{align*}
    m(\text{U-235})  &= \SI{235.044}{u}\\
    m(\text{Cs-140}) &= \SI{139.909}{u}\\
    m(\text{Rb-93})  &= \SI{92.922}{u}\\
    m(\text{n})      &= \SI{1.009}{u}
  \end{align*}
\end{frame}



\begin{frame}{Chain Reaction}
  A \textbf{chain reaction} is a series of reactions that can repeat over
  several cycles
  \begin{itemize}
  \item Reactions occur without requiring any material being added to the
    system %; the products of one reaction produce subsequent reactions
  \item The amount of fuel required to cause a chain reaction is called
    \textbf{critical mass}
  \end{itemize}
  \begin{center}
    \pic{.6}{graphics/chain-reaction}
  \end{center}
\end{frame}



\section{Nuclear Fusion}

\begin{frame}{Proton-Proton Reaction}
  \begin{columns}
    \column{.7\textwidth}
    The simplest fusion reaction occurs under high temperature about
    \SI{4e7}{\kelvin} is the proton-proton chain. In most the basic form, it is
    written as:

    \eq{-.35in}{
      \ce{4(^1_1H) -> ^4_2He + 2(^0_{+1}e) + \text{energy}}
    }
    
    \vspace{-.2in}In each p-p reaction, \SI{26.732}{\mega\electronvolt} is
    released. The exact mechanism for the p-p chain is shown in the right.
    
    \column{.3\textwidth}
    \pic{1}{graphics/Fusion_in_the_Sun}
  \end{columns}
\end{frame}



\begin{frame}{CNO Cycle}
  \begin{columns}
    \column{.6\textwidth}
    A fusion reaction that occurs in even higher temperatures (between 1.5 to
    \SI{1.7e7}{\kelvin}) is the carbon-nitrogen-oxygen (CNO) cycle. The total
    energy released in one cycle is \SI{26.73}{\mega\electronvolt}.
    \begin{itemize}
    \item The core temperature of the sun is about \SI{1.56e7}{\kelvin}, so
      only 1.7\% of helium-4 nuclei produced in the Sun are from this process
    \end{itemize}
    \column{.4\textwidth}
    \pic{1.1}{graphics/CNO-Cycle}
  \end{columns}
\end{frame}
\end{document}
