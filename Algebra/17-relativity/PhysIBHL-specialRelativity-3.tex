\documentclass[12pt,compress,aspectratio=169]{beamer}
\usetheme{metropolis}
\setbeamersize{text margin left=.5cm,text margin right=.5cm}
\usepackage[lf]{carlito}
\usepackage{siunitx}
\usepackage{tikz}
\usepackage{mathpazo}
\usepackage{bm}
\usepackage{mathtools}
\usepackage[ISO]{diffcoeff}
\diffdef{}{ op-symbol=\mathsf{d} }
\usepackage{xcolor,colortbl}

\setmonofont{Ubuntu Mono}
\setlength{\parskip}{0pt}
\renewcommand{\baselinestretch}{1}

\sisetup{
  inter-unit-product=\cdot,
  per-mode=symbol
}

\tikzset{
  >=latex
}

%\newcommand{\iii}{\hat{\bm\imath}}
%\newcommand{\jjj}{\hat{\bm\jmath}}
%\newcommand{\kkk}{\hat{\bm k}}


\setlength{\parskip}{0pt}
\renewcommand{\baselinestretch}{1}

\sisetup{
  inter-unit-product=\cdot,
  per-mode=symbol
}
\tikzset{>=latex}

\title{Topic 18: Special Relativity, Part 3}
\subtitle{AP and IBHL Physics}
\author[TML]{Dr.\ Timothy Leung}
\institute{Olympiads School}
\date{Updated: Summer 2022}

\newcommand{\pic}[2]{
  \includegraphics[width=#1\textwidth]{#2}
}
\newcommand{\eq}[2]{
  \vspace{#1}{\Large
    \begin{displaymath}
      #2
    \end{displaymath}
  }
}
%\newcommand{\iii}{\ensuremath\hat{\bm{\imath}}}
%\newcommand{\jjj}{\ensuremath\hat{\bm{\jmath}}}
%\newcommand{\kkk}{\ensuremath\hat{\bm{k}}}
\newcommand{\iii}{\ensuremath\hat\imath}
\newcommand{\jjj}{\ensuremath\hat\jmath}
\newcommand{\kkk}{\ensuremath\hat k}

\newcommand{\bigsqrt}{\ensuremath\sqrt{1-\left(\frac{v}{c}\right)^2}}
\newcommand{\lorentz}{\ensuremath\frac1{\bigsqrt}}



\begin{document}

\begin{frame}
  \maketitle
\end{frame}


\section{Relativistic Momentum}

\begin{frame}{Relativistic Momentum}
  In Grade 12 Physics, you were taught that momentum is mass times velocity.
  And in Grade 11 Physics, you were taught that velocity is displacement over
  time. \emph{These definitions have not changed.}

  \eq{-.2in}{
    \bm{p}=m\frac{d\bm{x}}{dt}
  }

  \vspace{-.1in}But now that you know $d\bm{x}$ and $dt$ are relativistic
  quantities that depend on motion, we can find a new expression for
  ``relativistic momentum'':
  
  \eq{-.2in}{
    \bm{p}=m\frac{d\bm{x}}{dt}
    =\frac{md\bm{x}}{\bigsqrt\;dt}
    =\boxed{\frac{m\bm{v}}{\bigsqrt}}=\gamma m\bm{v}
  }
\end{frame}


\section[Relative Mass]{Relativistic Mass}

\begin{frame}{Relativistic Mass}
  From the relativistic momentum expression, we see the relativistic aspect to
  mass as well. The \textbf{apparent mass} (or \textbf{relativistic mass}) $m'$
  as measured by a moving observer is related to its \textbf{rest mass} (or
  \textbf{intrinsic mass} or \textbf{invariant mass}) $m$ by the Lorentz factor:

  \eq{-.18in}{
    \boxed{m'=\frac{m}{\bigsqrt}=\gamma m}
  }
  
  The intrinsic mass of a moving object does not change, but a moving observer
  will observe that it behaves as if it is more massive. As $v\rightarrow c$,
  $m'\rightarrow\infty$.
\end{frame}



\section[Relativistic Energy]{Relativistic Energy}

\begin{frame}{Work and Energy}
  Einstein published a fourth paper in \emph{Annalen der Physik} on November
  21, 1905 (received Sept.\ 27) titled ``Does the Inertia of a Body Depend Upon
  Its Energy Content?'' (In German: Ist die Tr\"{a}gheit eines K\"{o}rpers von
  seinem Energieinhalt abh\"{a}ngig?)
  \begin{itemize}
  \item Einstein deduced the most famous of equations: $E=mc^2$
  \end{itemize}
\end{frame}


\begin{frame}{Work and Energy}
  In Grade 12 Physics, you were taught that force is the rate of change of
  momentum with respect to time. \emph{This definition has not changed.}

  \eq{-.2in}{
    \bm{F}=\frac{d\bm{p}}{dt}
  }

  \vspace{-.1in}and that work is the integral of the dot product between force
  and displacement vectors. \emph{This definition has not changed either.}

  \eq{-.2in}{
    W=\int\bm{F}\cdot d\bm{x}=\int\frac{d\bm{p}}{dt}\cdot \bm{dx}
  }

  \vspace{-.1in}Since we now have a relativistic expression for momentum, we
  substitute that new expression into the expression for force, and then
  integrate.
\end{frame}



\begin{frame}{Work and Energy}
  For 1D motion (for simplicity), we can rearrange the terms in the integral:

  \eq{-.2in}{
    W=\int Fdx=\int\frac{dp}{dt}dx=\int vdp
  }
  
  \vspace{-.1in}Assuming that both $v$ and $p$ are continuous in time, we can
  apply the chain rule to find the infinitesimal change in momentum ($dp$) with
  respect to $\gamma$ and $v$:
  
  \eq{-.25in}{
    p=\gamma mv \quad\rightarrow\quad dp= \gamma dv +vd\gamma
  }

  \vspace{-.1in}Substituting that back into the integral, we have:
  
  \eq{-.3in}{
    W=\int vdp=\int mv(\gamma dv +vd\gamma)=
    \int m\left(\gamma vdv +v^2d\gamma\right)
  }
\end{frame}



\begin{frame}{Work and Energy}
  One of the integral is with respect to $\gamma$, so we express $v$ and $dv$
  in terms of $\gamma$ using its definition:

  \eq{-.2in}{
    v^2=c^2\left[1-\left(\frac1{\gamma}\right)^2\right]
    \quad\rightarrow\quad
    dv=\frac{c^2}{\gamma^3v}d\gamma
  }
\end{frame}



\begin{frame}{Work and Energy}
  Putting everything together, we have

  \eq{-.25in}{
    W=\int m(\gamma vdv +v^2d\gamma)=\int m\left[\frac{c^2}{\gamma^2}+
      c^2\left(1-\frac1{\gamma^2}\right)\right]d\gamma
  }

  This is a surprisingly simple integral:

  \eq{-.2in}{
    W=\int_1^\gamma mc^2d\gamma
  }

  The limit of the integral is from $1$ because at $v=0$, $\gamma=1$
\end{frame}



\begin{frame}{Work and Kinetic Energy}
  The integral gives us this expression:
  
  \eq{-.25in}{
    W=\gamma mc^2-mc^2
  }

  \vspace{-.15in}We know from the work-kinetic energy theorem that the work $W$
  done is equal to the change in kinetic energy $K$, therefore
  
  \eq{-.2in}{ \boxed{K=m'c^2-mc^2} }

  \vspace{-.1in}
  \begin{center}
    \begin{tabular}{l|c|c}
      \rowcolor{pink}
      \textbf{Variable} & \textbf{Symbol} & \textbf{SI Unit}\\ \hline
      Kinetic energy of an object & $K$  & \si{\joule}\\
      Apparent mass (measured in moving frame) & $m'$ & \si{\kilo\gram}\\
      Rest mass (measured in stationary frame) & $m$  & \si{\kilo\gram}\\
      Speed of light              & $c$ & \si{\metre\per\second}
    \end{tabular}
  \end{center}
\end{frame}



\begin{frame}{Relativistic Energy}
  \framesubtitle{What This All Means}
  {\Large
    \begin{displaymath}
      \boxed{K=m'c^2-mc^2}
    \end{displaymath}
  }

  The minimum amount of energy that any object has, regardless of it's motion
  (or lack of) is its \textbf{rest energy}:
  
  \eq{-.4in}{ E_0=mc^2 }

  \vspace{-.2in}The \textbf{total energy} of an object has is
    
  \eq{-.3in}{
    E_T=m'c^2=\gamma mc^2
  }

  \vspace{-.2in}The difference between total energy and rest energy is the
  kinetic energy:

  \eq{-.3in}{
    K=E_T-E_0
  }
\end{frame}


\begin{frame}{Relativistic Energy}{What This All Means}
  
  \eq{-.2in}{
    \boxed{E=mc^2}
  }

  \textbf{Mass-energy equivalence}:
  \begin{itemize}
  \item Whenever there is a change of energy, there is also a change of mass
  \item ``Conservation of mass'' and ``conservation of energy'' must be
    combined into a single concept of \textbf{conservation of mass-energy}
  \item Mass-energy equivalence doesn't merely mean that mass can be converted
    into energy, and vice versa (although this is true), but rather, one can be
    converted into the other
    \emph{because they are fundamentally the same thing}
  \end{itemize}
\end{frame}



%\begin{frame}{Example Problem}
%  \textbf{Example 3:} An electron has a rest mass of \SI{9.11e-31}{\kilo\gram}.
%  In a detector, it behaves as if it has a mass of \SI{12.55e-31}{\kilo\gram}.
%  How fast is that electron moving relative to the detector?
%\end{frame}



\begin{frame}{Energy-Momentum Relation}
  The \textbf{energy-momentum relation} relates an object's rest (intrinsic)
  mass $m$, total energy $E$, and momentum $p$:

  \eq{-.2in}{
    \boxed{E^2=p^2c^2+m^2c^4}
  }
  \begin{center}
    \begin{tabular}{l|c|c}
      \rowcolor{pink}
      \textbf{Quantity} & \textbf{Symbol} & \textbf{SI Unit} \\ \hline
      Total energy   & $E$ & \si{\joule} \\
      Momentum       & $p$ & \si{\kilo\gram\metre\per\second}\\
      Rest mass      & $m$ & \si{\kilo\gram} \\
      Speed of light & $c$ & \si{\metre\per\second}
    \end{tabular}
  \end{center}
\end{frame}



\begin{frame}{Energy-Momentum Relation}
  This equation is derived by squaring the expression for relativistic momentum:

  \eq{-.2in}{
    p=\gamma mv=\frac{mv}{\bigsqrt}\quad\rightarrow\quad
    p^2=\gamma^2m^2v^2=\frac{m^2v^2}{1-\left(\frac{v}c\right)^2}
  }
\end{frame}



\begin{frame}{Energy-Momentum Relation}
  Solving for $v^2$ and substituting it back into the Lorentz factor, we
  obtain an alternative form for $\gamma$ in terms of momentum and mass:

  \eq{-.2in}{
    \gamma =\sqrt{1+\left(\frac{p}{mc}\right)^2}
  }
  Inserting this form of the Lorentz factor into the energy equation, we have

  \eq{-.2in}{
    E=mc^2 \sqrt{1+\left(\frac{p}{mc}\right)^2}
  }

  Which is the same equation as in the last slide.
\end{frame}



\begin{frame}{Energy-Momentum Relation}
  In the \textbf{stationary frame of reference}, (rest frame,
  center-of-momentum frame) the momentum is zero, so the equation simplifies to

  \eq{-.2in}{
    \boxed{ E=mc^2}
  }
  where $m$ is the rest mass of the object.

  \vspace{.2in}If the object is \textbf{massless}, as is the case for a
  \textbf{photon}, then the equation reduces to

  \eq{-.2in}{
    \boxed{E=pc}
  }
\end{frame}



\begin{frame}{Kinetic Energy--Classical vs.\ Relativistic}
  \begin{columns}
    \column{.5\textwidth}
    \textbf{Relativistic:}
    {\Large
      \begin{displaymath}
        K=\frac{mc^2}{\bigsqrt}-mc^2
      \end{displaymath}
    }
    
    \column{.5\textwidth}
    \textbf{Newtonian:}
    {\Large
      \begin{displaymath}
        K=\frac12mv^2
      \end{displaymath}
    }
  \end{columns}
  But are they really that different?
  \begin{itemize}
  \item If space and time are indeed relative quantities, then the relativistic
    equation for $K$ must apply to all velocities
  \item But we know that when $v\ll c$, the Newtonian expression works perfectly
  \item i.e.\ The Newtonian expression for $K$ must be a very good approximation
    for the relativistic expression for $K$ for $v\ll c$
  \end{itemize}
\end{frame}




\begin{frame}{Binomial Series Expansion}
  The \textbf{binomial series} is the Maclaurin series for the function
  $f(x)=(1+x)^\alpha$, given by:
  
  \eq{-.3in}{
    (1+x)^\alpha=\sum_{k=0}^\infty\left(
    \begin{matrix}
      \alpha\\
      k
    \end{matrix}
    \right)
    x^k=1+\alpha x + \frac{\alpha(\alpha-1)}{2!}x^2+\cdots
  }

  In the case of relativistic kinetic energy, we use:

  \eq{-.2in}{
    x=-\left(\frac{v}c\right)^2\quad\text{\normalsize and}\quad\quad
    \alpha=-\frac12
  }
\end{frame}



\begin{frame}{Binomial Series Expansion}
  Substituting these terms into the equation:
  
  \vspace{-.3in}{\Large
    \begin{align*}
      K &= mc^2
      \left(1+\frac12\frac{v^2}{c^2}+\frac38\frac{v^4}{c^4}+\cdots\right)-mc^2\\
      &\approx\frac12mv^2+\frac38m\frac{v^4}{c^2}+\cdots
      \end{align*}
  }
  
  For $v\ll c$, we can ignore the high-order terms. The leading term reduces to
  the Newtonian expression
\end{frame}


\begin{frame}{Comparing Classical and Relativistic Energy}
  \begin{columns}
    \column{.5\textwidth}
    In classical mechanics:
    {\Large
      \begin{displaymath}
        K=\frac12mv^2
      \end{displaymath}
    }
    In relativistic mechanics:
    {\Large
      \begin{displaymath}
        K=\gamma mc^2-mc^2
      \end{displaymath}
    }
    
    \column{0.5\textwidth}
    \pic{.85}{graphics/e_k}
  \end{columns}

  The classical expression is accurate for speeds up to $v\approx 0.3c$.
\end{frame}



%\begin{frame}{Example Problem}
%  \textbf{Example 4:} A rocket car with a mass of \SI{2.00e3}{kg} is accelerated
%  from rest to \SI{1.00e8}{m/s}. Calculate its kinetic energy:
%  \begin{enumerate}
%  \item Using the classical equation
%  \item Using the relativistic equation
%  \end{enumerate}
%\end{frame}

\end{document}
