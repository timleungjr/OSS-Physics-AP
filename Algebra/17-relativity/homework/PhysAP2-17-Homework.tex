\documentclass{../../../oss-ap12ibhl}

\begin{document}
\genheader
\gentitle{17}{SPECIAL RELATIVITY}


\begin{questions}
  \question At what speed does a clock move if it runs at a rate which is
  one-half the rate of a clock at rest?

  \question An atomic clock is placed in a jet airplane. The clock measures a
  time interval of \SI{3600}{\second} when the jet moves with speed
  \SI{400}{\metre\per\second}. How much larger a time interval does an
  identical clock held by an observer at rest on the ground measure?

  \question The muon is an unstable particle that spontaneously decays into an
  electron and two neutrinos. If the number of muons at $t=0$ is $N_0$, the
  number $N$ at time $t$ is
  \begin{equation*}
    N = N_0 e^{-t/\tau}
  \end{equation*}
  where $\tau=\SI{2.20}{\micro\second}$ is the mean lifetime of the muon.
  Suppose the muons move at speed $0.95c$.
  \begin{parts}
    \part What is the observed lifetime of the muons?
    \part How many muons remain after traveling a distance of
    \SI{3.}{\kilo\metre}? (The answer should be expressed in terms of $N_0$)
  \end{parts}
  \newpage

  \question A muon has a lifetime of \SI{2e-6}{\second} in its rest frame. It
  is created \SI{100}{\kilo\metre} above the earth and moves toward it at
  a speed of \SI{2.97e8}{\metre\per\second}. At what altitude does it decay?
  According to the muon, how far did it travel in its brief life?
  
  \question Two rockets of rest length $L_0$ are approaching the Earth from
  opposite directions at velocities $\pm c/2$, relative to Earth. How long does
  one of them appear to the other?

  \question A body quadruples its momentum when its speed doubles. What was the
  initial speed in units of $c$, i.e.\ what was $\varv/c$?
  \newpage

  \question A body of rest mass $m_0$ moving at speed $\varv$ collides with and
  sticks to an identical body at rest. What is the mass $M$ and momentum
  $p'$ of the final clump?

  \question The Starship Enterprise goes to a planet in a star system far away
  with a speed of $0.9c$, spends 6 months on the planet, and comes back with a
  speed of $0.95c$. The entire trip takes 5 years for the crew.
  \begin{parts}
    \part How far is the planet according to Earth observers?
    \part How long did it take the crew to get to the planet?
    \part How long did the entire trip take for the Earth observers?
  \end{parts}
  (Hints: For this kind of problems, instead of using SI units, it is much
  easier to scale the problem based on the speed of light: speed is measured in
  fraction of the speed of light (i.e. use $v=0.95$ if Enterprise is travelling
  at $0.95c$), time is measured in \emph{years}, and distance is measured in
  \emph{light-years}.
  
  \question A rocket ship leaves the Earth at a speed of $0.8c$. When a clock
  on the rocket says 1 hour has elapsed, the rocket ship sends a light signal
  back to Earth.
  \begin{parts}
    \part According to Earth clocks, when was the signal sent?
    \part According to Earth clocks, how long after the rocket left did the
    signal arrive back on Earth?
    \part According to the rocket clock, how long after the rocket left did the
    signal arrive back on Earth?
  \end{parts}

  \question The spaceship Viking goes to a planet in a star system 30 light
  years away from Earth with a speed of $0.99c$, spends 1 year on the planet,
  and then returns home. The entire trip takes 10 years for the crew.
  \begin{parts}
    \part How far is the planet according to crew?
    \part How long does it take the crew to get to the planet?
    \part How long does it take the crew to return to Earth?
    \part What is the speed of the crew on return? Warning! The distance for
    the crew is not the same as the distance on their way to the planet.
    \part How far is the Earth from the planet according to crew on their
    return?
    \part How long did the entire trip take for the Earth observers?
  \end{parts}
\end{questions}
\end{document}
