\documentclass{../../../oss-classkick}

\begin{document}
\genheader

\gentitle{1}{MECHANICAL WAVES AND SOUND WAVES}

\genmultidirections

\gengravity

\raggedcolumns
\begin{multicols}{2}

  \begin{enumerate}[leftmargin=18pt]
  \item Which of the following is an example of a longitudinal wave?
    \begin{enumerate}[nosep,leftmargin=18pt,label=(\Alph*)]
    \item Water wave
    \item Microwave
    \item Sound wave
    \item Radio wave
    \item X-ray
    \end{enumerate}
    \vspace{.7in}
    
  \item Which of the following distances describes the amplitude of a wave?
    \begin{enumerate}[nosep,leftmargin=18pt,label=(\Alph*)]
    \item Crest to trough
    \item Crest to crest
    \item Trough to trough
    \item Top of crest to bottom of trough
    \item Crest to equilibrium position
    \end{enumerate}
    \vspace{.7in}
    
  \item A wave has a frequency of 100 Hz. What is the period of the wave?
    \begin{enumerate}[nosep,leftmargin=18pt,label=(\Alph*)]
    \item 0.5 s
    \item 0.01 s
    \item 0.1 s
    \item 1 s
    \item 100 s
    \end{enumerate}
    
  \item A wave has a frequency of \SI{100}{\hertz} and a wavelength of
    \SI1{\metre}. What is the speed of the wave?
    \begin{enumerate}[nosep,leftmargin=18pt,label=(\Alph*)]
    \item\SI{.01}{\metre\per\second}
    \item\SI{1}{\metre\per\second}
    \item\SI{10}{\metre\per\second}
    \item\SI{100}{\metre\per\second}
    \item\SI{1000}{\metre\per\second}
    \end{enumerate}
    
%455. The highest sound that a human can hear has a wavelength of 17.2 cm.
%What is the frequency of this wave? (Assume the speed of sound is 343
%m/s.)
%(A)
%(B)
%(C)
%(D)
%(E)
%20 Hz
%200 Hz
%2,000 Hz
%20 kHz
    %20 MHz
  \item Which of the following quantities remains constant as a mechanical
    wave travels from one type of spring into another?
    \begin{enumerate}[nosep,leftmargin=18pt,label=(\Alph*)]
    \item Frequency
    \item Wavelength
    \item Speed
    \item Amplitude
    \item Spring constant
    \end{enumerate}
    \vspace{.7in}
    
  \item A jackhammer operator wears a set of protective headphones. Through the
    headphones, a sound wave is broadcast that is \ang{180} out of phase with
    the jackhammer sound wave. The result is that he does not hear the sound of
    the jackhammer. These two sound waves are an example of which of the
    following?
    \begin{enumerate}[nosep,leftmargin=18pt,label=(\Alph*)]
    \item Standing wave
    \item Transverse wave
    \item Destructive interference
    \item Constructive interference
    \item Doppler effect
    \end{enumerate}
    \vspace{.7in}
    
  \item Half of a sound wave forms in a \SI{.500}{\metre} open tube when the
    fundamental frequency is played. What is this fundamental frequency
    when the speed of sound is \SI{343}{\metre\per\second}?
    \begin{enumerate}[nosep,leftmargin=18pt,label=(\Alph*)]
    \item 34 Hz
    \item 86 Hz
    \item 172 Hz
    \item 343 Hz
    \item 686 Hz
    \end{enumerate}
    \columnbreak
    
  \item Two waves have the same frequency. What other characteristic must be
    the same for these waves?
    \begin{enumerate}[nosep,leftmargin=18pt,label=(\Alph*)]
    \item Speed
    \item Period
    \item Amplitude
    \item Intensity
    \item Wavelength
    \end{enumerate}

  \item A child dips her finger repeatedly into the water to make waves. If she
    dips her finger more frequently, the wavelength \underline{\hspace{.3in}}
    and the speed \underline{\hspace{.3in}}.
    \begin{enumerate}[nosep,leftmargin=18pt,label=(\Alph*)]
    \item Increases; decreases
    \item Decreases; increases
    \item Increases; stays the same
    \item Decreases; stays the same
    \item Stays the same; increases
    \end{enumerate}
    \vspace{.7in}
    
  \item A \SI{.50}{\metre} tube is placed in a bucket of water. The tube can be
    moved up and down to vary the length of the column of air inside, and the
    temperature of the air is \SI{20}{\celsius}, which corresponds to a sound
    speed of \SI{343}{\metre\per\second}. A \SI{440}{\hertz} tuning fork is
    struck and placed over the mouth of the tube. The tube is moved up and down
    until the first resonance can be heard. What is the length of the column of
    air inside the tube when one antinode and one node form in the standing
    wave?
    \begin{enumerate}[nosep,leftmargin=18pt,label=(\Alph*)]
    \item 0.09 m
    \item 0.19 m
    \item 0.27 m
    \item 0.38 m
    \item 0.5 m
    \end{enumerate}

  \item A \SI{440}{\hertz} and a \SI{444}{\hertz} tuning fork are struck
    simultaneously. What is the beat frequency that you hear?
    \begin{enumerate}[nosep,leftmargin=18pt,label=(\Alph*)]
    \item 1 Hz
    \item 2 Hz
    \item 4 Hz
    \item 221 Hz
    \item 442 Hz
    \end{enumerate}
  \end{enumerate}
  \columnbreak
  
  \textbf{Questions \ref{des1}--\ref{des2}} use the following figure:
  \begin{center}
    \pic{.4}{2waves}
  \end{center}
  \begin{enumerate}[leftmargin=18pt,resume]
  \item Two waves are traveling on a string. The directions and amplitude of
    each wave are shown in the figure. When the two waves meet, what
    will be the amplitude of the resulting wave?
    \label{des1}
    \begin{enumerate}[nosep,leftmargin=18pt,label=(\Alph*)]
    \item $-4A/3$
    \item $-2A/3$
    \item $0$
    \item $2A/3$
    \item $4A/3$
    \end{enumerate}
    
  \item The figure depicts which of the following phenomena?
    \begin{enumerate}[nosep,leftmargin=18pt,label=(\Alph*)]
    \item Standing wave
    \item Transverse wave
    \item Destructive interference
    \item Constructive interference
    \item Doppler effect
    \end{enumerate}
    \vspace{.7in}
    
  \item After the waves interact, what will happen?
    \label{des2}
    \begin{enumerate}[nosep,leftmargin=18pt,label=(\Alph*)]
    \item One wave ($2A/3$) will travel to the right.
    \item One wave ($-2A/3$) will travel to the left.
    \item There will be no more waves.
    \item One wave ($+A$) will travel to the right, while one wave ($-A/3$)
      will travel to the left.
    \item One wave ($-A$) will travel to the right, while one wave ($+A/3$)
      will travel to the left.
    \end{enumerate}
    
    
%462. A tsunami wave travels at 720 km/h and has a period of 10 min. What
%is the wavelength of the wave?
%(A)
%(B)
%(C)
%(D)
%(E)
%2.0 km
%7.5 km
%120 km
%720 km
%750 km
%463. A favorite radio station is located on the dial at 100 MHz. What is the
%wavelength of the radio waves emitted from the radio station if the
%speed of these waves is 3.00 × 10 8 m/s?
%(A)
%(B)
%(C)
%(D)
%(E)
%3 m
%30 m
%300 m
%3 km
    %300 km
%  \item What phenomenon describes how sound changes frequency as it passes
%    a receiver?
%    \begin{enumerate}[nosep,leftmargin=18pt,label=(\Alph*)]
%    \item Young's modulus
%    \item Maxwell effect
%    \item Michelson shift
%    \item Doppler effect
%    \item Einstein bridge
%    \end{enumerate}
    

%466. Two waves are traveling on a string. The directions and amplitude of
%each wave are shown in the figure. When the two waves meet, what
%will be the amplitude of the resulting wave?
%(A)
%(B)
%(C)
%(D)
%(E)
%A/2
%A
%3A/2
%2A
%4A

  \item Which of the following measurements is used to find the wavelength?
    \begin{enumerate}[nosep,leftmargin=18pt,label=(\Alph*)]
    \item Crest to zero displacement
    \item Crest to trough
    \item Trough to zero displacement
    \item Trough to crest
    \item Crest to crest
    \end{enumerate}
    \vspace{.7in}
    
  \item What is true about a loud sound with a low pitch?
    \begin{enumerate}[nosep,leftmargin=18pt,label=(\Alph*)]
    \item It travels faster than a soft sound.
    \item It travels slower than a high-pitch sound.
    \item It has large amplitude and low frequency.
    \item It has small amplitude and high frequency.
    \item It has small amplitude and low frequency.
    \end{enumerate}
    \vspace{.7in}
    
%469. A girl sitting on a beach counts six waves passing a buoy in 3.0 s. She
%measured the distance between the waves to be 1.5 m. What is the
%speed of the waves the girl observed?
%(A)
%(B)
%(C)
%(D)
%(E)
%3.0 m/s
%9.0 m/s
%0.75 m/s
%1.3 m/s
%0.30 m/s

  \item As a wave is formed, what is the relationship between the wavelength
    and frequency?
    \begin{enumerate}[nosep,leftmargin=18pt,label=(\Alph*)]
    \item Linearly related and directly proportional
    \item Linearly related but not directly proportional
    \item Inversely proportional
    \item Parabolic
    \item Exponential
    \end{enumerate}
    \vspace{.7in}
    
  \item The sound from a loudspeaker vibrating at 256 Hz interferes with a
    trumpet vibrating at 252 Hz. What sound results?
    \begin{enumerate}[nosep,leftmargin=18pt,label=(\Alph*)]
    \item A 254-Hz pitch with a 4-Hz beat frequency
    \item A 4-Hz pitch with a 254-Hz beat frequency
    \item A melodic chord
    \item Two distinct pitches
    \item A resonance with a frequency of 308 Hz
    \end{enumerate}
    \columnbreak
    
%472. A girl sitting on a beach counts six waves passing a buoy in 3 s. What
%is the period of the buoy’s vibration?
%(A)
%(B)
%(C)
%(D)
%(E)
%0.5 s
%0.5 Hz
%2 s
%2 Hz
%6 Hz473. If a wave disturbance travels 16 m each second and the distance
%between each crest is 4 m, determine the frequency of the disturbance.
%(A)
%(B)
%(C)
%(D)
%(E)
%0.25 Hz
%2 Hz
%4 Hz
%32 Hz
%64 Hz
%474. What are the approximate amplitude and wavelength, respectively, of
%the wave shown in the diagram?
%(A)
%(B)
%(C)
%(D)
%(E)
%40 cm; 10 cm
%20 cm; 5 cm
%40 cm; 5 cm
%20 cm; 10 cm
%10 cm; 20 cm

  \item Which of the following is observed as a constantly moving source of
    sound passes a receiver that is at rest?
    \begin{enumerate}[nosep,leftmargin=18pt,label=(\Alph*)]
    \item The frequency and the speed increase.
    \item The frequency decreases, and the speed increases.
    \item The frequency decreases, and the speed stays the same.
    \item The frequency and the speed stay the same.
    \item The frequency increases, and the speed decreases.
    \end{enumerate}
    \vspace{.7in}
    
%  \item Which of the following will cause the phenomenon of sound beats to
%    occur?
%    \begin{enumerate}[nosep,leftmargin=18pt,label=(\Alph*)]
%(A) Two slightly different frequencies sounding together
%(B) When the frequency of one object matches the natural frequency of another
%(C) A sound wave reflecting off a boundary back onto itself
%(D) A source approaching the receiver at a speed faster than its sound
%wave
%(E) When a source varies its amplitude at regular intervals
  \item What is observed when sound beats occur?
    \begin{enumerate}[nosep,leftmargin=18pt,label=(\Alph*)]
    \item A rhythmic change in the pitch of the sound
    \item A regular increase and decrease in the speed of the sound
    \item An increase in frequency of the sound
    \item A dramatic growth in amplitude of the sound
    \item A sound that gets louder and softer at regular intervals
    \end{enumerate}
    \vspace{.7in}
    
  \item Which of the following best describes a wave?
    \begin{enumerate}[nosep,leftmargin=18pt,label=(\Alph*)]
    \item pattern resembling a sine wave
    \item An object that oscillates back and forth at a characteristic frequency
    \item A disturbance that carries energy and momentum from one place to
      another with the transfer of mass
    \item A disturbance that carries energy and momentum from one place to
      another without the transfer of mass
    \item An oscillating electric and magnetic field that cannot travel through
      a vacuum
    \end{enumerate}
    
%    479. Which of the following best describes the role of the medium with a
%transverse mechanical wave?
%(A) The medium vibrates back and forth parallel to the motion of the
%wave.
%(B) The medium vibrates back and forth perpendicular to the motion
%of the wave.
%(C) The medium oscillates between parallel and perpendicularvibrations.
%(D) There is a snakelike slither through the medium.
%(E) Oscillating compressions and rarefactions move through the
%medium.
%480. Which of the following quantities is defined as the amount of
%vibrations a medium experiences in a given amount of time?
%(A)
%(B)
%(C)
%(D)
%(E)
%Speed
%Period
%Wavelength
%Velocity
  %Frequency
    
%482. An observer notices a 2.00-s delay between seeing fireworks and
%hearing them. How far away are the fireworks if the speed of sound is
%344 m/s?
%(A)
%(B)
%(C)
%(D)
%(E)
%86 m
%172 m
%344 m
%688 m
%1,376 m
%483. A tuning fork vibrates 256 times each second. What is the distance
%between compressions if the speed of the sound waves is 345 m/s?
%(A) 0.70 m
%(B) 0.740 m(C) 1.40 m
%(D) 89.0 m
    %(E) 601 m
    
%485. Which of the following best describes the air particles as sound travels
%through air?
%(A) The air particles vibrate along lines perpendicular to the motion of
%the wave.
%(B) The air particles vibrate along lines parallel to the motion of the
%wave.
%(C) The air particles don’t move unless the transverse wave has a
%direct collision with them.
%(D) The air particles remain stationary as the wave travels through
%them.
%(E) The air particles move from the source of the wave to the receiver.

%487. A dolphin swimming at the surface of the sea emits an ultrasonic sound
%wave toward the ocean floor. It takes 0.15 s for the sound to go from
%the dolphin to the floor of the ocean and back. Determine the depth ofthe ocean if the speed of sound in the water is 1,400 m/s.
%(A)
%(B)
%(C)
%(D)
%(E)
%93 m
%105 m
%186 m
%210 m
    %420 m
    
%  \end{enumerate}
%    
%  Multi-select: For questions \ref{multi-1st}--\ref{multi-last}, two of the
%  suggested answers will be correct.
%  \begin{enumerate}[leftmargin=18pt,resume]
    
  \item Which of the following are true about sound waves?
    \label{multi-1st}
    \begin{enumerate}[nosep,leftmargin=18pt,label=(\Alph*)]
    \item Their speed increases slightly with temperature.
    \item They travel faster than light waves.
    \item Their speed gets greater as the pitch of the sound increases.
    \item They travel faster in water than in steel.
    \item They can travel in the vacuum of space.
    \end{enumerate}
    \vspace{.7in}
    
%  \item Which of the following are necessary for a standing wave to form?
%    \begin{enumerate}[nosep,leftmargin=18pt,label=(\Alph*)]
%    \item A vibration at a natural frequency
%    \item Reflection off a boundary
%    \item At least two different vibration frequencies
%    \item The source must move faster than the speed of the wave.
%    \item The wave must be transverse.
%    \end{enumerate}
%    \vspace{.7in}
    
  \item The speed of a sound wave depends on which of the following?
    \begin{enumerate}[nosep,leftmargin=18pt,label=(\Alph*)]
    \item The loudness and pitch of the sound
    \item The intensity of the vibration
    \item The frequency of the source of vibration
    \item The motion of the observer
    %\item The characteristics of the medium
    \item The type of medium
    \end{enumerate}

  \item Which of the following may NOT occur when a single wave hits a boundary
    between one medium and another?
    \begin{enumerate}[nosep,leftmargin=18pt,label=(\Alph*)]
    \item It reflects back into the original medium.
    \item It transmits into the new medium.
    \item Its frequency changes.
    \item Its wavelength decreases.
    %\item It gains energy.
    \item Some of its energy is absorbed as thermal energy.
    \end{enumerate}
    \vspace{.7in}
    
  \item Which of the following are examples of resonance?
    \label{multi-last}
    \begin{enumerate}[nosep,leftmargin=18pt,label=(\Alph*)]
    \item A low frequency beat is detected when two tuning forks are played
      together.
    \item A high frequency is detected when a moving object approaches an
      observer.
    \item A tuning fork starts vibrating when an identical tuning fork vibrates
      next to it.
%    \item A wineglass vibrates dramatically and shatters when a certain pitch
%      is played on a nearby speaker.
    \item Two waves combine to form a wave with a larger amplitude.
    \end{enumerate}
  \end{enumerate}
\end{multicols}
\newpage

\genfreetitle{1}{MECHANICAL WAVES AND SOUND WAVES}{4}

\genfreedirections

% THIS TAKEN FROM THE 2012 AP PHYSICS B EXAM FREE-RESPPONSE QUESTION #6
\cpic{.8}{set-up.png}
\begin{enumerate}[leftmargin=15pt]
\item You are given the apparatus represented in the figure above. A glass tube
  is fitted with a movable piston that allows the indicated length $L$ to be
  adjusted. A sine-wave generator with an adjustable frequency is connected to
  a speaker near the open end of the tube. The output of a microphone at the
  open end is connected to a waveform display. You are to use this apparatus to
  measure the speed of sound in air.
  \begin{enumerate}
  \item Describe a procedure using the apparatus that would allow you to
    determine the speed of sound in air. Clearly indicate what quantities you
    would measure and with what instrument each measurement would be made.
    Represent each measured quantity with a different symbol.
  \item Using the symbols defined in part (a), indicate how your measurements
    can be used to determine an experimental value of the speed of sound.
  \item A more accurate experimental value can be obtained by varying one of
    the measured quantities to obtain multiple sets of data. Indicate one
    quantity that can be varied, and describe how a graph of the resulting data
    could be used to determine the speed of sound. Clearly identify independent
    and dependent variables, and indicate how the slope of the graph relates to
    the speed of sound.
  \end{enumerate}
  \newpage

  % TAKEN FROM THE 2016 AP PHYSICS 1 EXAM FREE-RESPONSE QUESTION #5
  \begin{center}
    \pic{.3}{dangling.png}
  \end{center}
\item (Suggested time 13 minutes) The figure above on the left shows a
  uniformly thick rope hanging vertically from an oscillator that is turned
  off. When the oscillator is on and set at a certain frequency, the rope forms
  the standing wave shown above on the right. $P$ and $Q$ are two points on the
  rope.
  \begin{enumerate}
  \item The tension at point $P$ is greater than the tension at point $Q$.
    Briefly explain why.
  \item A student hypothesizes that increasing the tension in a rope increases
    the speed at which waves travel along the rope. In a clear, coherent
    paragraph-length response that may also contain figures and/or equations,
    explain why the standing wave shown above supports the student’s hypothesis.
  \end{enumerate}
  \newpage

  % TAKEN FROM THE 2018 AP PHYSICS 1 EXAM FREE-RESPONSE QUESTION #4
\item (Suggested time 13 minutes) A transverse wave travels to the right
  along a string.
  \begin{enumerate}[leftmargin=18pt]
  \item  Two dots have been painted on the string. In the diagrams below, those
    dots are labeled $P$ and $Q$.
    \begin{enumerate}[leftmargin=12pt]
    \item The figure below shows the string at an instant in time. At the
      instant shown, dot $P$ has maximum displacement and dot $Q$ has zero
      displacement from equilibrium. At each of $P$ and $Q$, draw an
      arrow indicating the direction of the instantaneous velocity of that dot.
      If either dot has zero velocity, write ``$v=0$'' next to the dot.
      \begin{center}
        \pic{.4}{wave}
      \end{center}
    \item The figure below shows the string at the same instant as shown in
      part (a)i. At each of $P$ and $Q$, draw an arrow indicating the
      direction of the instantaneous acceleration of that dot. If either dot
      has zero acceleration, write ``$a=0$'' next to the dot.
      \begin{center}
        \pic{.4}{wave}
      \end{center}
    \end{enumerate}
  \end{enumerate}
  The figure below represents the string at time $t=0$, the same instant as
  shown in part (a) when dot $P$ is at it maximum displacement from
  equilibrium. For simplicity, dot $Q$ is not shown.
  \begin{center}
    \pic{.4}{wave2}
  \end{center}
  \begin{enumerate}[leftmargin=18pt,resume]
  \item 
    \begin{enumerate}[leftmargin=18pt]
    \item On the grid below, draw the string at a later time $t=T/4$, where
      $T$ is the period of the wave.
      \begin{center}
    \pic{.45}{grid}
  \end{center}
    \item On your drawing above, draw a dot to indicate the position of dot
      $P$ on the string at time $t=T/4$ and clearly label the dot with the
      letter $P$.
    \end{enumerate}
  \item Now consider the wave at time $t=T$. Determine the distance traveled
    (not the displacement) by dot $P$ between times $t=0$ and $t=T$.
    \vspace{1in}
  \end{enumerate}
  \newpage
  
  % TAKEN FROM THE 2018 AP PHYSICS 1 EXAM FREE-RESPONSE QUESTION #4
  \begin{center}
    \pic{.65}{string-mass}
  \end{center}
\item (7 points, suggested time 13 minutes) The figure above shows a string
  with one end attached to an oscillator and the other end attached to a block.
  The string passes over a massless pulley that turns with negligible friction.
  Four such strings, $A$, $B$, $C$, and $D$, are set up side by side, as shown
  in the diagram below. Each oscillator is adjusted to vibrate the string at
  its fundamental frequency $f$. The distance between each oscillator and
  pulley $L$ is the same, and the mass $M$ of each block is the same. However,
  the fundamental frequency of each string is different.
  \begin{center}
    \pic{.65}{topview}
  \end{center}
  The equation for the velocity $v$ of a wave on a string is
  $\displaystyle v=\sqrt{\frac{F_T}{m/L}}$, where $F_T$ is the tension of the
  string and $m/L$ is the mass per unit length (linear mass density) of the
  string.
  \begin{enumerate}
  \item What is different about the four strings shown above that would result
    in their having different fundamental frequencies? Explain how you arrived
    at your answer.
  \item  A student graphs frequency as a function of the inverse of the linear
    mass density. Will the graph be linear? Explain how you arrived at your
    answer.
  \item The frequency of the oscillator connected to string $D$ is changed so
    that the string vibrates in its second harmonic. On the side view of string
    $D$ below, mark and label the points on the string that have the greatest
    average vertical speed.
    \begin{center}
      \pic{.65}{sideview}
    \end{center}
  \end{enumerate}
\end{enumerate}
\end{document}
