\documentclass{../../../oss-ap12ibhl}

\begin{document}
\genheader
\gentitle{8}{MECHANICAL WAVES AND SOUND WAVES}

\begin{questions}
  \question Which of the following is an example of a longitudinal wave?
  \begin{choices}
    \choice Water wave
    \choice Microwave
    \choice Sound wave
    \choice Radio wave
    \choice X-ray
  \end{choices}
    
  \question Which of the following distances describes the amplitude of a wave?
  \begin{choices}
    \choice Crest to trough
    \choice Crest to crest
    \choice Trough to trough
    \choice Top of crest to bottom of trough
    \choice Crest to equilibrium position
  \end{choices}
    
%  \question A wave has a frequency of 100 Hz. What is the period of the wave?
%  \begin{choices}
%    \choice 0.5 s
%    \choice 0.01 s
%    \choice 0.1 s
%    \choice 1 s
%    \choice 100 s
%  \end{choices}

  \question Which of the following measurements is used to find the wavelength?
  \begin{choices}
    \choice Crest to zero displacement
    \choice Crest to trough
    \choice Trough to zero displacement
    \choice Trough to crest
    \choice Crest to crest
  \end{choices}  

  \question A wave has a frequency of \SI{100}{\hertz} and a wavelength of
  \SI1{\metre}. What is the speed of the wave?
  \begin{choices}
    \choice\SI{.01}{\metre\per\second}
    \choice\SI{1}{\metre\per\second}
    \choice\SI{10}{\metre\per\second}
    \choice\SI{100}{\metre\per\second}
    \choice\SI{1000}{\metre\per\second}
  \end{choices}
    
  \question Which of the following quantities remains constant as a mechanical
  wave travels from one type of spring into another?
  \begin{choices}
    \choice Frequency
    \choice Wavelength
    \choice Speed
    \choice Amplitude
    \choice Spring constant
    \end{choices}
    
  \question A jackhammer operator wears a set of protective headphones. Through
  the headphones, a sound wave is broadcast that is \ang{180} out of phase with
  the jackhammer sound wave. The result is that he does not hear the sound of
  the jackhammer. These two sound waves are an example of which of the
  following?
  \begin{choices}
    \choice Standing wave
    \choice Transverse wave
    \choice Destructive interference
    \choice Constructive interference
    \choice Doppler effect
  \end{choices}
    
  \question Half of a sound wave forms in a \SI{.500}{\metre} open tube when the
  fundamental frequency is played. What is this fundamental frequency when the
  speed of sound is \SI{343}{\metre\per\second}?
  \begin{choices}
    \choice 34 Hz
    \choice 86 Hz
    \choice 172 Hz
    \choice 343 Hz
    \choice 686 Hz
  \end{choices}
    
  \question Two waves have the same frequency. What other characteristic must be
  the same for these waves?
  \begin{choices}
    \choice Speed
    \choice Period
    \choice Amplitude
    \choice Intensity
    \choice Wavelength
  \end{choices}

  \question A child dips her finger repeatedly into the water to make waves. If
  she dips her finger more frequently, the wavelength \underline{\hspace{.3in}}
  and the speed \underline{\hspace{.3in}}.
  \begin{choices}
    \choice Increases; decreases
    \choice Decreases; increases
    \choice Increases; stays the same
    \choice Decreases; stays the same
    \choice Stays the same; increases
  \end{choices}
    
  \question A \SI{.50}{\metre} tube is placed in a bucket of water. The tube
  can be moved up and down to vary the length of the column of air inside, and
  the temperature of the air is \SI{20}{\celsius}, which corresponds to a sound
  speed of \SI{343}{\metre\per\second}. A \SI{440}{\hertz} tuning fork is
  struck and placed over the mouth of the tube. The tube is moved up and down
  until the first resonance can be heard. What is the length of the column of
  air inside the tube when one antinode and one node form in the standing wave?
  \begin{choices}
    \choice 0.09 m
    \choice 0.19 m
    \choice 0.27 m
    \choice 0.38 m
    \choice 0.5 m
  \end{choices}
  \newpage

%  \item A \SI{440}{\hertz} and a \SI{444}{\hertz} tuning fork are struck
%    simultaneously. What is the beat frequency that you hear?
%    \begin{enumerate}[nosep,leftmargin=18pt,label=(\Alph*)]
%    \item 1 Hz
%    \item 2 Hz
%    \item 4 Hz
%    \item 221 Hz
%    \item 442 Hz
%    \end{enumerate}
  \uplevel{  
    \textbf{Questions \ref{des1}--\ref{des2}} use the following figure:
    \cpic{.5}{2waves}
  }

  \question Two waves are traveling on a string. The directions and amplitude of
  each wave are shown in the figure. When the two waves meet, what will be the
  amplitude of the resulting wave?
  \label{des1}
  \begin{choices}
    \choice $-4A/3$
    \choice $-2A/3$
    \choice $0$
    \choice $2A/3$
    \choice $4A/3$
  \end{choices}
    
  \question The figure depicts which of the following phenomena?
  \begin{choices}
    \choice Standing wave
    \choice Transverse wave
    \choice Destructive interference
    \choice Constructive interference
    \choice Doppler effect
  \end{choices}
    
  \question After the waves interact, what will happen?
  \label{des2}
  \begin{choices}
    \choice One wave ($2A/3$) will travel to the right.
    \choice One wave ($-2A/3$) will travel to the left.
    \choice There will be no more waves.
    \choice One wave ($+A$) will travel to the right, while one wave ($-A/3$)
    will travel to the left.
    \choice One wave ($-A$) will travel to the right, while one wave ($+A/3$)
    will travel to the left.
  \end{choices}
    
  \question What is true about a loud sound with a low pitch?
  \begin{choices}
    \choice It travels faster than a soft sound.
    \choice It travels slower than a high-pitch sound.
    \choice It has large amplitude and low frequency.
    \choice It has small amplitude and high frequency.
    \choice It has small amplitude and low frequency.
  \end{choices}
    
  \question As a wave is formed, what is the relationship between the wavelength
  and frequency?
  \begin{choices}
    \choice Linearly related and directly proportional
    \choice Linearly related but not directly proportional
    \choice Inversely proportional
    \choice Parabolic
    \choice Exponential
  \end{choices}
    
%  \question The sound from a loudspeaker vibrating at 256 Hz interferes with a
%  trumpet vibrating at 252 Hz. What sound results?
%  \begin{choices}
%    \choice A 254-Hz pitch with a 4-Hz beat frequency
%    \choice A 4-Hz pitch with a 254-Hz beat frequency
%    \choice A melodic chord
%    \choice Two distinct pitches
%    \choice A resonance with a frequency of 308 Hz
%  \end{choices}
    
  \question Which of the following is observed as a constantly moving source of
  sound passes a receiver that is at rest?
  \begin{choices}
    \choice The frequency and the speed increase.
    \choice The frequency decreases, and the speed increases.
    \choice The frequency decreases, and the speed stays the same.
    \choice The frequency and the speed stay the same.
    \choice The frequency increases, and the speed decreases.
  \end{choices}
    
  \question What is observed when sound beats occur?
  \begin{choices}
    \choice A rhythmic change in the pitch of the sound
    \choice A regular increase and decrease in the speed of the sound
    \choice An increase in frequency of the sound
    \choice A dramatic growth in amplitude of the sound
    \choice A sound that gets louder and softer at regular intervals
    \end{choices}
    
  \question Which of the following best describes a wave?
  \begin{choices}
    \choice pattern resembling a sine wave
    \choice An object that oscillates back and forth at a characteristic
    frequency
    \choice A disturbance that carries energy and momentum from one place to
    another with the transfer of mass
    \choice A disturbance that carries energy and momentum from one place to
    another without the transfer of mass
    \choice An oscillating electric and magnetic field that cannot travel
    through a vacuum
  \end{choices}
    
  \question Which of the following are true about sound waves?
  \label{multi-1st}
  \begin{choices}
    \choice Their speed increases slightly with temperature.
    \choice They travel faster than light waves.
    \choice Their speed gets greater as the pitch of the sound increases.
    \choice They travel faster in water than in steel.
    \choice They can travel in the vacuum of space.
  \end{choices}
    
  \question The speed of a sound wave depends on which of the following?
  \begin{choices}
    \choice The loudness and pitch of the sound
    \choice The intensity of the vibration
    \choice The frequency of the source of vibration
    \choice The motion of the observer
    \choice The characteristics of the medium
  \end{choices}
    
%  \question Which of the following may NOT occur when a single wave hits a
%  boundary between one medium and another?
%  \begin{choices}
%    \choice It reflects back into the original medium.
%    \choice It transmits into the new medium.
%    \choice Its frequency changes.
%    \choice Its wavelength decreases.
%    %\choice It gains energy.
%    \choice Some of its energy is absorbed as thermal energy.
%  \end{choices}
    
  \question Which of the following are examples of resonance?
  \label{multi-last}
  \begin{choices}
    \choice A low frequency beat is detected when two tuning forks are played
    together.
    \choice A high frequency is detected when a moving object approaches an
    observer.
    \choice A tuning fork starts vibrating when an identical tuning fork
    vibrates next to it.
    %\choice A wineglass vibrates dramatically and shatters when a certain pitch
    %is played on a nearby speaker.
    \choice Two waves combine to form a wave with a larger amplitude.
  \end{choices}
  \newpage


  % TAKEN FROM THE 2012 AP PHYSICS B EXAM FREE-RESPPONSE QUESTION #6
  \uplevel{
    \cpic{.8}{set-up}
  }
  \question You are given the apparatus represented in the figure above. A
  glass tube is fitted with a movable piston that allows the indicated length
  $L$ to be adjusted. A sine-wave generator with an adjustable frequency is
  connected to a speaker near the open end of the tube. The output of a
  microphone at the open end is connected to a waveform display. You are to use
  this apparatus to measure the speed of sound in air.
  \begin{parts}
    \part Describe a procedure using the apparatus that would allow you to
    determine the speed of sound in air. Clearly indicate what quantities you
    would measure and with what instrument each measurement would be made.
    Represent each measured quantity with a different symbol.
    
    \part Using the symbols defined in part (a), indicate how your measurements
    can be used to determine an experimental value of the speed of sound.
    
    \part A more accurate experimental value can be obtained by varying one of
    the measured quantities to obtain multiple sets of data. Indicate one
    quantity that can be varied, and describe how a graph of the resulting data
    could be used to determine the speed of sound. Clearly identify independent
    and dependent variables, and indicate how the slope of the graph relates to
    the speed of sound.
  \end{parts}
  \newpage

  % TAKEN FROM THE 2016 AP PHYSICS 1 EXAM FREE-RESPONSE QUESTION #5
  \uplevel{
    \cpic{.3}{dangling}
  }
  \question The figure above on the left shows a uniformly thick rope hanging
  vertically from an oscillator that is turned off. When the oscillator is on
  and set at a certain frequency, the rope forms the standing wave shown above
  on the right. $P$ and $Q$ are two points on the rope.
  \begin{parts}
    \part The tension at point $P$ is greater than the tension at point $Q$.
    Briefly explain why.
    
    \part A student hypothesizes that increasing the tension in a rope increases
    the speed at which waves travel along the rope. In a clear, coherent
    paragraph-length response that may also contain figures and/or equations,
    explain why the standing wave shown above supports the student’s hypothesis.
  \end{parts}
  \newpage

  % TAKEN FROM THE 2018 AP PHYSICS 1 EXAM FREE-RESPONSE QUESTION #4
  \question A transverse wave travels to the right along a string.
  \begin{parts}
    \part Two dots have been painted on the string. In the diagrams below, those
    dots are labeled $P$ and $Q$.
    
    \begin{subparts}
      \subpart The figure below shows the string at an instant in time. At the
      instant shown, dot $P$ has maximum displacement and dot $Q$ has zero
      displacement from equilibrium. At each of $P$ and $Q$, draw an
      arrow indicating the direction of the instantaneous velocity of that dot.
      If either dot has zero velocity, write ``$v=0$'' next to the dot.
      \cpic{.4}{wave}
      
      \subpart The figure below shows the string at the same instant as shown in
      part (a)i. At each of $P$ and $Q$, draw an arrow indicating the
      direction of the instantaneous acceleration of that dot. If either dot
      has zero acceleration, write ``$a=0$'' next to the dot.
      \cpic{.4}{wave}  
    \end{subparts}
    
    \uplevel{
      The figure below represents the string at time $t=0$, the same instant as
      shown in part (a) when dot $P$ is at it maximum displacement from
      equilibrium. For simplicity, dot $Q$ is not shown.
      \cpic{.4}{wave2}
    }
    \part 
    \begin{subparts}
      \subpart On the grid below, draw the string at a later time $t=T/4$, where
      $T$ is the period of the wave.
      \cpic{.45}{grid}
      
      \subpart On your drawing above, draw a dot to indicate the position of dot
      $P$ on the string at time $t=T/4$ and clearly label the dot with the
      letter $P$.
    \end{subparts}
    
    \part Now consider the wave at time $t=T$. Determine the distance traveled
    (not the displacement) by dot $P$ between times $t=0$ and $t=T$.
  \end{parts}
  \newpage
  
  % TAKEN FROM THE 2018 AP PHYSICS 1 EXAM FREE-RESPONSE QUESTION #4
  \uplevel{
    \cpic{.65}{string-mass}
  }
  \question The figure above shows a string
  with one end attached to an oscillator and the other end attached to a block.
  The string passes over a massless pulley that turns with negligible friction.
  Four such strings, $A$, $B$, $C$, and $D$, are set up side by side, as shown
  in the diagram below. Each oscillator is adjusted to vibrate the string at
  its fundamental frequency $f$. The distance between each oscillator and
  pulley $L$ is the same, and the mass $M$ of each block is the same. However,
  the fundamental frequency of each string is different.
  \cpic{.65}{topview}

  The equation for the velocity $v$ of a wave on a string is
  $v=\sqrt{\dfrac{F_T}{m/L}}$, where $F_T$ is the tension of the string and
  $m/L$ is the mass per unit length (linear mass density) of the string.
  \begin{parts}
    \part What is different about the four strings shown above that would result
    in their having different fundamental frequencies? Explain how you arrived
    at your answer.
    
    \part A student graphs frequency as a function of the inverse of the linear
    mass density. Will the graph be linear? Explain how you arrived at your
    answer.
    
    \part The frequency of the oscillator connected to string $D$ is changed so
    that the string vibrates in its second harmonic. On the side view of string
    $D$ below, mark and label the points on the string that have the greatest
    average vertical speed.
    \cpic{.65}{sideview}
  \end{parts}
\end{questions}
\end{document}
