\documentclass[12pt,compress,aspectratio=169]{beamer}

\usetheme{metropolis}
\setbeamersize{text margin left=.5cm,text margin right=.5cm}
%\usefonttheme{professionalfonts}
\usepackage[lf]{carlito}
\usepackage{amsmath,bm}
\usepackage{siunitx}
\usepackage{tikz}
\usepackage{mathpazo}

%\setsansfont{Roboto Light}
\setmonofont{Ubuntu Mono}
\setlength{\parskip}{0pt}
\renewcommand{\baselinestretch}{1}

\sisetup{
  per-mode=symbol
}

\title{Topic 4: Momentum, Impulse and Collisions}
\subtitle{Advanced Placement Physics 1}
\author[TML]{Dr.\ Timothy Leung}
\institute{Olympiads School}
\date{Last Updated: \today}

\newcommand{\iii}{\ensuremath\hat{\bm{\imath}}}
\newcommand{\jjj}{\ensuremath\hat{\bm{\jmath}}}
\newcommand{\kkk}{\ensuremath\hat{\bm{k}}}
\newcommand{\pic}[2]{\includegraphics[width=#1\textwidth]{#2}}
\newcommand{\mb}[1]{\ensuremath\mathbf{#1}}
\newcommand{\eq}[2]{\vspace{#1}{\Large\begin{displaymath}#2\end{displaymath}}}

\begin{document}

\begin{frame}
  \maketitle
\end{frame}

\section{Momentum}


\begin{frame}{Momentum}
  \textbf{Linear momentum}\footnote{or \textbf{translational momentum}, or just
    \textbf{momentum}} is a quantity of motion defined as the product of mass
  $m$ and velocity $\mb{v}$. The unit of momentum is
  \textbf{kilogram meter per second} (\si{\kilo\gram\metre\per\second}):

  \eq{-.25in}{
    \boxed{\mb{p}=m\mb{v}}
  }
  \begin{itemize}
  \item Momentum is a \emph{vector} quantity; the direction of $\mb{p}$ is the
    same as $\mb{v}$
  \item For rotational motion of rigid bodies, there is also a similar concept
    called \textbf{angular momentum} which will be studied in a later topic
  \end{itemize}
\end{frame}



\begin{frame}{General Form of Second Law of Motion}
  From an inertial frame of reference, for a constant mass $m$, the average
  rate of change of momentum gives a familiar result:

  \eq{-.15in}{
    \frac{\Delta\mb{p}}{\Delta t}=\frac{\Delta(m\mb{v})}{\Delta t}
    =m\frac{\Delta\mb{v}}{\Delta t}=m\mb{a}=\overline{\mb{F}}_\text{net}
  }
  
  which is the familiar form of the second law of motion. In fact, the
  \emph{general form} of the first and second laws of motion is that
  \textbf{the average net external force on an object is the change of its
    momentum over a finite time interval}:

  \eq{-.2in}{
    \boxed{\overline{\mb{F}}_\text{net}=\frac{\Delta\mb{p}}{\Delta t}}
  }
\end{frame}



\begin{frame}{First Law of Motion \& Conservation of Momentum}
  \eq{0in}{
    \boxed{\mb{F}_\text{net}=\frac{\Delta\mb{p}}{\Delta t}}
  }

  \textbf{First law of motion:} The momentum state of an object is conserved
  unless a net unbalanced external force acts on it
  \begin{itemize}
  \item Without net external forces, the \textbf{conservation of momentum}
    states that the initial and final momentum must be constant:
    
    \eq{-.25in}{
      \sum\mb{p}_i=\sum\mb{p}_i'
    }
    
  \item This is applied to collision and explosion problems.
  \end{itemize} 
\end{frame}


\begin{frame}{Impulse}
  Rearranging the variables in the general form of the second law of motion, we
  obtain the \textbf{impulse-momentum theorem}:
  
  \eq{-.15in}{
    \overline{\mb{F}}_{\text{net}}=\frac{\Delta\mb{p}}{\Delta t}\;\rightarrow\;
    \boxed{\mb{J}_{\text{net}}=\overline{\mb{F}}_{\text{net}}\Delta t=\Delta\mb{p}}
  }

  where $\mb{J}_{\text{net}}$ is called the \textbf{net impulse}. Since
  $\mb{F}$, $\mb{p}$ and $\mb{J}_{\text{net}}$ are all vectors, impulse can be
  evaluated separately in each of the $\iii$, $\jjj$ and $\kkk$ directions.

  \eq{-.2in}{
    \mb{J}=J_x\iii + J_x\jjj + J_z\kkk
  }

  \vspace{-.2in}For the $\iii$ direction:

  \eq{-.2in}{ J_x=F_x\Delta t=\Delta p_x }
\end{frame}



\begin{frame}{Impulse}
  Any average force $\overline{\mb{F}}$ acting on an object over a time
  $\Delta t$ generates an impulse $\mb{J}$, regardless of whether or not there
  is a change in momentum:

  \eq{-.2in}{
    \boxed{\mb{J}=\overline{\mb{F}}\Delta t}
  }
  
  The change in momentum only depends on \emph{net} impulse $\mb{J}_{\text{net}}$.
  The force is a \emph{time-averaged} vector that generates the same impulse.
  It is often used in introductory physics courses to avoid calculus

\end{frame}



\begin{frame}{Impulse: An Example}
  \textbf{Example 1:} Jim pushes a box with mass \SI{1.}{\kilo\gram} with a
  \SI{5.}{\newton} force for \SI{10}{\second} while the box stays on the same
  place. Find the impulse of the pushing force, friction force, the
  gravitational force, and the net force.
\end{frame}



\begin{frame}{Rocket Propulsion Problem}
  \textbf{Example 2:} A rocket generates a thrust force by ejecting hot gases
  from an engine. If it takes \SI{1}{\milli\second} to combust
  \SI{1.}{\kilo\gram} of fuel, ejecting it at a speed of
  \SI{1000}{\metre\per\second}, what thrust is generated?
  
  \vspace{.15in}\begin{enumerate}[A.]
  \item \SI{1000}{\newton}
  \item \SI{10000}{\newton}
  \item \SI{100000}{\newton}
  \item \SI{1000000}{\newton}
  \end{enumerate}
\end{frame}



\begin{frame}{Another Space Example}
  \textbf{Example 3:} A rocket for mining the asteroid belt is designed like a
  large scoop. It is approaching asteroids at a velocity of
  \SI{e4}{\metre\per\second}. The asteroids are much smaller than the rocket.
  If the rocket scoops asteroids at a rate of \SI{100}{\kilo\gram\per\second},
  what thrust (force) must the rocket's engine provide in order for the rocket
  to maintain constant velocity? Ignore any variation in the rocket's mass due
  to the burning fuel.

  \vspace{.15in}\begin{enumerate}[A.]
  \item\SI{e3}{\newton}
  \item\SI{e6}{\newton}
  \item\SI{e9}{\newton}
  \item\SI{e12}{\newton}
  \end{enumerate}
\end{frame}


%\begin{frame}{Impulse: Another Example}
%  \textbf{Example 5:} Two balls of the same mass are dropped from the same
%  height onto the floor. The first ball bounces upwards from the floor
%  elastically. The second ball sticks to the floor. The first applies an
%  impulse to the floor of $I_1$ and the second applies an impulse $I_2$. The
%  two impulses obey:
%  \begin{enumerate}[(a)]
%  \item $I_2=2I_1$
%  \item $I_2=I_1/2$
%  \item $I_2=4I_1$
%  \item $I_2=I_1/4$
%  \end{enumerate}
%\end{frame}


\section{Collisions}

\begin{frame}{Conservation of Momentum}
  \begin{itemize}
  \item From the third law of motion, we know that the action and reaction
    forces are always equal in magnitude and in opposite direction. Thus, their
    total impulse would be zero.
  \item When there is no external force, the momentum of the total system will
    always be constant. We saw that a few slides ago:
    
    \eq{-.2in}{
      \sum_i\mb{p}=\sum_i\mb{p}'
    }
  \end{itemize}
\end{frame}



\begin{frame}{Classifications of Collisions}
  \begin{itemize}
  \item Elastic Collision:
    \begin{itemize}
    \item Total kinetic energy is conserved
    \item<alert@2> Momentum is conserved
    \end{itemize}
  \item Inelastic collision:
    \begin{itemize}
    \item Kinetic energy is \textbf{not} conserved
    \item<alert@2> Momentum is conserved
    \end{itemize}
  \item Completely inelastic collision:
    \begin{itemize}
    \item ``Perfectly inelastic collision''
    \item A special case of inelastic collision
    \item The objects move together after the collision
    \item Kinetic energy is \textbf{not} conserved
    \item<alert@2> Momentum is conserved
    \end{itemize}
  \end{itemize}
\end{frame}



\begin{frame}{How to Solve Conservation of Momentum Problem}
  \begin{enumerate}
  \item Check whether the condition for the conservation of momentum is
    satisfied (i.e.\ are there any external forces?)
  \item If so, write out expressions for initial momentum and final momentum,
    and equate the two. You will get 1 to 3 equations (one for each direction).
  \item Solve these equations, find the quantity you need to find.
  \end{enumerate}
  Remember that momentum is a vector. If there is no external force component
  in some direction, then the momentum component in this direction is still
  conserved.
\end{frame}



\begin{frame}{Before We Dive Into Some Exercises}
  The most typical applications of momentum conservation are collision and
  explosions
  \begin{itemize}
  \item\textbf{Collision: A hits B}
    \begin{itemize}
    \item Regardless of whether they move together or not afterwards, momentum
      is conserved
    \item Head-on collisions are usually 1D
    \item Glancing collisions are usually 2D or 3D
    \end{itemize}
  \item\textbf{Explosion: A explodes and becomes B and C (and D and E\ldots)}
    \begin{itemize}
    \item A perfectly inelastic collision in reverse
    \item Total momentum of B and C (and D and E\ldots) is the same as A in the
      beginning
    \item Usually a 2D or 3D problem
    \end{itemize}
  \end{itemize}
\end{frame}



\begin{frame}{Example}
  \textbf{Example 4:} Two blocks A and B, both have mass \SI{1.}{\kilo\gram}.
  Block A has velocity \SI{3.}{\metre\per\second} and block B is at rest. Their
  distance is \SI{1.}{\metre}. The surface is has kinetic friction coefficient
  0.02. After they collide, they move together, what would be the final
  velocity of these two blocks? How far can they go after the collision?
\end{frame}



\begin{frame}{Collision Problem}
  \textbf{Example 5:} Two objects with equal mass are heading toward each
  other with equal speeds, undergo a head-on collision. Which one of the
  following statement is correct?

  \vspace{.15in}\begin{enumerate}[A.]
  \item Their final velocities are zero
  \item Their final velocities may be zero
  \item Each must have a final velocity equal to the other's initial velocity
  \item Their velocities must be reduced in magnitude
  \end{enumerate}
\end{frame}



\begin{frame}{Glancing Collision}
  \vspace{.3in}\textbf{Example 7:} A billiard ball of mass \SI{.155}{\kilo\gram}
  (``cue ball'') moves with a velocity of \SI{1.25}{\metre\per\second} towards
  a stationary billiard ball (``eight ball'') of identical mass and strikes it
  with a glancing blow. The cue ball moves off at an angle of \ang{29.7}
  clockwise from its original direction, with a speed of
  \SI{.956}{\metre\per\second}. What is the final velocity of the eight ball?
\end{frame}



%\begin{frame}{Example}
%  \textbf{Example 12:} A ball is dropped from a height $h$. It hits the ground
%  and bounces up with a momentum loss of \SI{10}{\percent} due to the impact.
%  The maximum height it will reach is:
%  \begin{enumerate}[(a)]
%  \item\num{.90}$h$
%  \item\num{.81}$h$
%  \item\num{.949}$h$
%  \item\num{.3}$h$
%  \end{enumerate}
%\end{frame}

\section{Elastic Collisions}

\begin{frame}{Elastic Collision Problems}
  In elastic collisions, \emph{both} momentum and kinetic energy is conserved.
  In a 1D collision, both equations below have to be satisfied:

  \vspace{-.2in}{\Large
    \begin{align*}
      \sum m_iv_i&=\sum m_iv_i'\\
      \sum\frac12 m_iv_i^2&=\sum\frac12 m_iv_i'^2
    \end{align*}
  }

  \textbf{How kinetic energy is conserved:} In an elastic collision, energy is
  first converted into a potential energy (e.g.\ elastic potential energy in a
  spring), and then all the energy is released back as kinetic energy.
\end{frame}



\begin{frame}{Conservation of Momentum \& Energy in Elastic Collisions}
  For collision of two objects, the conservation of momentum equation can be
  expressed as:

  \vspace{-.15in}{\Large
    \begin{equation}
      \boxed{m_1(v_1-v_1')=m_2(v_2'-v_2)}
    \end{equation}
  }

  By moving $m_1$ terms to the left, and $m_2$ terms to the right. Likewise,
  the conservation of energy can also be arranged as:

  \vspace{-.15in}{\Large
    \begin{equation}
      \boxed{m_1(v_1^2-v_1'^2)=m_2(v_2'^2-v_2^2)}
    \end{equation}
  }

  By multiplying every term by \num{2}, and again, moving $m_1$ terms to the
  left, and $v_2$ terms to the right.
\end{frame}



\begin{frame}{Conservation of Momentum \& Energy in Elastic Collisions}
  Dividing the equations (2) by (1) from the last slide, we get:

  
  \eq{-.15in}{
    \frac{(2)}{(1)}\quad\rightarrow\quad
    \frac{m_1(v_1^2-v_1'^2)}{m_1(v_1-v_1')}=
    \frac{m_2(v_2'^2-v_2^2)}{m_2(v_2'-v_2)}
  }

  $m_1$ and $m_2$ terms cancel out, while the terms in the numerator can be
  expanded as the difference of two squares which is then simplified:

  \eq{-.15in}{
    \frac{(v_1+v_1')(v_1-v_1')}{(v_1-v_1')}=
    \frac{(v_2'+v_2)(v_2'-v_2)}{(v_2'-v_2)}
  }
  
  Leading to the final expression, which is substituted back into (1)
  
  \eq{-.25in}{
    v_1 +v_1'= v_2+v_2'
  }
\end{frame}



%\begin{frame}{Solving the Example Problem}
%  
%  {\Large
%    \begin{displaymath}
%      \boxed{v_A'=\frac{m_A-m_B}{m_A+m_B}v_A}\quad
%      \boxed{v_B'=\frac{2m_A}{m_A+m_B}v_A}
%    \end{displaymath}
%  }
%
%  These equations work for \emph{all} elastic impact where object B (in this
%  example, the truck) is stationary when impact occurs. Substituting values for
%  $m_A$, $m_B$ and $v_A$, we get:
%  \begin{displaymath}
%    v_A'=\frac{m_A-m_B}{m_A+m_B}v_A=\frac{(1000-3000)}{(1000+3000)}\times 20
%    = \boxed{\SI{-10}{\metre\per\second}}
%  \end{displaymath}
%  \begin{displaymath}
%    v_B'=\frac{2m_A}{m_A+m_B}v_A=\frac{(2\times 1000)}{(1000+3000)}\times 20
%    = \boxed{\SI{10}{\metre\per\second}}
%  \end{displaymath}
%\end{frame}




\begin{frame}{Final Velocities in an Elastic Collision}
  When two objects \num{1} and \num{2} of mass $m_1$ and $m_2$ and initial
  velocities $v_1$ and $v_2$ collide elastically, their final velocities will
  be:
  
  \vspace{-.2in}{\Large
    \begin{align*}
      v_1'&=\frac{v_1(m_1-m_2)+2m_2v_2}{m_1+m_2}\\
      v_2'&=\frac{v_2(m_2-m_1)+2m_1v_1}{m_1+m_2}
    \end{align*}
  }

  These equations are not provided in the AP exam equation sheet, which means
  that we are more interested in the behavior qualitatively rather than
  quantitatively.
\end{frame}




\begin{frame}{Special Cases}
  If both objects have equal mass ($m_1=m_2=m$) and the second object is
  initially at rest ($v_2=0$), then the equations simplifies to
  
  \vspace{-.2in}{\Large
    \begin{align*}
      v_1'&=\frac{v_1(m-m)+2mv_2}{m+m}=0\\
      v_2'&=\frac{v_2(m-m)+2mv_1}{m+m}=v_1
    \end{align*}
  }

  All the momentum and energy from $m_1$ is transferred to $m_2$. Object 1
  stops all together, while object 2 continues with the initial momentum and
  velocity of Object 1.
\end{frame}



\begin{frame}{Special Cases}
  Another special case is when $m_1\gg m_2$ and $v_2=0$ (i.e.\ a large object
  colliding with a small stationary object) then we can effectively ``ignore''
  $m_2$:
  
  \vspace{-.2in}{\Large
    \begin{align*}
      v_1'&=\frac{v_1(m_1-m_2)+2m_2v_2}{m_1+m_2}\approx
      \frac{m_1v_1}{m_1}=v_1\\
      v_2'&=\frac{v_2(m_2-m_1)+2m_1v_1}{m_1+m_2}\approx
      \frac{2m_1v_1}{m_1}=2v_1
    \end{align*}
  }

  Object 1 continues to move like nothing happened, but object 2 is pushed to
  move at \emph{twice} the initial speed of object 1.
\end{frame}



\begin{frame}{Special Cases}
  In the reverse case, if $m_1\ll m_2$, and $v_2=0$ (a small object colliding
  with a large stationary object), then we can ``ignore'' the $m_1$ term:
  
  \vspace{-.2in}{\Large
    \begin{align*}
      v_1'&=\frac{v_1(m_1-m_2)+2m_2v_2}{m_1+m_2}\approx
      \frac{-m_2v_1}{m_2}=-v_1\\
      v_2'&=\frac{v_2(m_2-m_1)+2m_1v_1}{m_1+m_2}\approx 0
    \end{align*}
  }

  Object 1 bounces off object 2, and travels in the opposite direction with the
  same velocity magnitude, while object 2 does not move.
\end{frame}



\begin{frame}{Elastic Collision Example}
  \textbf{Example 8:} Blocks A and B have the same mass; A hits B with a speed
  of \SI{5.}{\metre\per\second} while B is initially at rest. If the collision
  is elastic, what would be the final speed of these two objects?
\end{frame}



\begin{frame}{Elastic Collision Example}
  \textbf{Example 9:} Blocks A and B with the same mass; A has a velocity
  \SI{3.}{\metre\per\second} to the east while B has \SI{2}{\metre\per\second}
  to the west. If the collision is elastic, after the collision, what would the
  velocity of the two blocks be?
\end{frame}



\begin{frame}{Elastic Collision Example}
  \textbf{Example 10:} Throw a ball to a really big wall, when the ball reaches
  the wall, it has a velocity \SI{10}{\metre\per\second} toward the wall. If
  the collision is elastic, what would the final velocity of the ball be?
\end{frame}



\begin{frame}{Elastic Collision Example}
  \textbf{Example 11:} Throw a ball with a velocity \SI{4.}{\metre\per\second}
  toward a train with a velocity \SI{40}{\metre\per\second} toward the ball.
  If the collision is elastic, what would the final velocity of the ball be?
\end{frame}


\begin{frame}{Inelastic Collision: Calculating Energy Loss}
  \textbf{Example 12:} Two blocks A and B with mass \SI{2.}{\kilo\gram}, block
  A hits B with velocity \SI{4.}{\metre\per\second} while B is at rest.
  \begin{enumerate}[A.]
  \item Suppose the collision is completely inelastic, what would the final
    velocity of A and B be?
  \item What is the loss of energy?
  \end{enumerate}
\end{frame}

\end{document}
