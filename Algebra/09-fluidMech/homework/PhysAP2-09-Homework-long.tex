\documentclass{../../../oss-legalpaper} 


\begin{document}
\genheader

\gentitle{2}{FLUID MECHANICS}

\genmultidirections

\gengravity

\raggedcolumns
\begin{multicols}{2}

  \begin{enumerate}[leftmargin=18pt]

  \item Two blocks of different sizes and masses float in a tray of water. Each
    block is half submerged, as shown in the figure. Water has a density of
    \SI{1000}{\kilo\gram\per\metre^3}. What can be concluded about the
    densities of the two blocks?
    
    \pic{.39}{mc-q1.png}
    \begin{enumerate}[nosep,leftmargin=18pt,label=(\Alph*)]
    \item\vspace{-.1in} The two blocks have different densities, both of which
      are less than \SI{1000}{kg/m^3}.
    \item The two blocks have the same density of \SI{500}{kg/m^3}.
    \item The two blocks have the same density, but the density cannot be
      determined with the information given.
    \item The larger block has a greater density than the smaller block, but
      the densities of the blocks cannot be determined with the information
      given.
    \end{enumerate}
    \vspace{.7in}
    
  \item The figure shows four cylinders of various diameters filled to different
    heights with water. A hole in the side of each cylinder is plugged by a
    cork. All cylinders are open at the top. The corks are removed. Which
    of the following is the correct ranking of the velocity of the water ($v$)
    as it exits each cylinder?
    
    \pic{.4}{mc-q2}
    \begin{enumerate}[nosep,leftmargin=18pt,label=(\Alph*)]
    \item $v_A > v_D > v_C > v_B$
    \item $v_A = v_D > v_C > v_B$
    \item $v_B > v_C > v_A = v_D$
    \item $v_C > v_A = v_B = v_D$
    \end{enumerate}
    \vspace{.7in}
    
  \item A 1 cm diameter pipe leads to a showerhead with twenty 1 mm diameter
    exit holes. The velocity of the water in the pipe is $v$. What is the
    velocity of the water exiting the holes?
    \begin{enumerate}[nosep,leftmargin=18pt,label=(\Alph*)]
    \item $0.05v$
    \item $0.5v$
    \item $5v$
    \item $100v$
    \end{enumerate}
  \end{enumerate}
  \columnbreak
  
  \textbf{Questions \ref{cyl1} and \ref{cyl2}}

  Four differently shaped sealed containers are completely filled with alcohol,
  as shown below. Containers $A$ and $B$ are cylindrical. Containers $C$ and
  $D$ are truncated conical shapes. The top and bottom diameters of the
  containers are shown.
  \pic{.45}{mc-q3-4}
  \begin{enumerate}[leftmargin=18pt,resume]
    
  \item Which of the following is the correct ranking of the pressure ($P$) at
    the bottom of the containers?
    \label{cyl1}
    \begin{enumerate}[nosep,leftmargin=18pt,label=(\Alph*)]
    \item $P_A = P_B = P_C = P_D$
    \item $P_A = P_D > P_C = P_B$
    \item $P_A > P_D > P_C > P_B$
    \item $P_D > P_A > P_C > P_B$
    \end{enumerate}
    \vspace{.7in}
    
  \item The force on the bottom of container $A$ due to the fluid inside the
    container is $F$. What is the force on the bottom of container $B$ due to
    the fluid inside?
    \label{cyl2}
    \begin{enumerate}[nosep,leftmargin=18pt,label=(\Alph*)]
    \item $F$
    \item $F/4$
    \item $F/8$
    \item $F/16$
    \end{enumerate}

  \item A mass $m$ is suspended in a fluid of density $\rho$ by a string,
    as shown in the figure below. The tension in the string is $T$. Which of
    the following is an appropriate equation for the buoyancy force? Select
    two answers.
    \begin{center}
      \vspace{-.15in}\pic{.3}{mc-q6}
    \end{center}
    \begin{enumerate}[nosep,leftmargin=18pt,label=(\Alph*)]
    \item $F_b=mg$
    \item $F_b=mg-T$
    \item $F_c=a_2 \rho gh_1$
    \item $F_d=a\rho g(h_1-h_2)$
    \end{enumerate}
    \columnbreak
    
  \item Two cylinders filled with a fluid are connected by a pipe so that fluid
    can pass between the cylinders, as shown in the figure. The cylinder
    on the right has 4 times the diameter of the cylinder on the left. Both
    cylinders are fitted with a movable piston and a platform on top. A
    person stands on the left platform. Which of the following lists the
    correct number of people that need to stand on the right platform so
    neither platform moves. Assume that the platform and piston have
    negligible mass and that all the people have the same mass.

    \vspace{-.1in}\pic{.35}{mc-q5}
    \begin{enumerate}[nosep,leftmargin=18pt,label=(\Alph*)]
    \item \num{16} people
    \item \num{4} people
    \item \num{1} person
    \item It is impossible to balance the system because you need 1/16 of a
      person on the right side.
    \end{enumerate}
    \vspace{.7in}
    
  \item Three wooden blocks of different masses and sizes float in a container
    of water, as shown in the figure. Each of the masses has a weight on top.
    Which of the following correctly ranks the buoyancy force on the wooden
    blocks?
    \begin{center}
      \vspace{-.15in}\pic{.35}{mc-q7}
    \end{center}
    \begin{enumerate}[nosep,leftmargin=18pt,label=(\Alph*)]
    \item $A > B = C$
    \item $A = B > C$
    \item $B > A = C$
    \item $B > A > C$
    \end{enumerate}
    
  \item Two blocks of the same dimensions are floating in a container of water,
    as shown in the figure. Which of the following is a correct statement about
    the two blocks?
    \begin{center}
      \vspace{-.15in}\pic{.3}{mc-q8}
    \end{center}
    \begin{enumerate}[nosep,leftmargin=18pt,label=(\Alph*)]
    \item The net force on both blocks is the same.
    \item The buoyancy force exerted on both blocks is the same.
    \item The density of both blocks is the same.
    \item The pressure exerted on the bottom of each block is the same.
    \end{enumerate}
    \columnbreak
    
  \item The figure shows four cubes of the same volume at rest in a container
    of water. Cube $C$ is partially submerged. Cubes $A$, $B$, and $D$ are fully
    submerged, with $B$ resting on the bottom of the container. Which of the
    following correctly ranks the densities ($\rho$) of the cubes? Assume the
    water to be incompressible.
    \begin{center}
      \vspace{-.15in}
      \pic{.3}{mc-q9}
    \end{center}
    \begin{enumerate}[nosep,leftmargin=18pt,label=(\Alph*)]
    \item $\rho_C >\rho_D >\rho_A >\rho_B$
    \item $\rho_B >\rho_A >\rho_D >\rho_C$
    \item $\rho_B >\rho_A =\rho_D >\rho_C$
    \item $\rho_B >\rho_A =\rho_D =\rho_C$
    \end{enumerate}
    \vspace{.7in}
    
  \item A beaker of water sits on a balance. A metal block with a mass of
    70 g is held suspended in the water by a spring scale in position 1, as
    shown. In this position, the reading on the balance is
    \SI{1260}{\gram}, and the spring scale reads 120 g. When the block is
    lifted from the water to position 2, what are the readings on the balance
    and spring scale?
    \begin{center}
      \vspace{-.15in}
      \pic{.2}{mc-q10}

      \begin{tabular}{c c c}
        & \textbf{Balance reading} & \textbf{Spring scale reading}\\
        (A) & \SI{1190}{\gram} & \SI{120}{\gram}\\
        (B) & \SI{1190}{\gram} & \SI{190}{\gram}\\
        (C) & \SI{1260}{\gram} & \SI{120}{\gram}\\
        (D) & \SI{1330}{\gram} & \SI{120}{\gram}
      \end{tabular}
    \end{center}
    \columnbreak
    
  \item Blood cells pass through an artery that has a buildup of plaque along
    both walls, as shown in the figure. Which of the following correctly
    describes the behavior of the blood cells as they move from the right
    side of the figure through the area of plaque? Assume the blood cells
    can change volume.
    \begin{center}
      \vspace{-.15in}
      \pic{.25}{mc-q11}
     \end{center}
    \begin{enumerate}[nosep,leftmargin=18pt,label=(\Alph*)]
    \item\vspace{-.1in} The blood cells increase in speed and expand in volume.
    \item The blood cells increase in speed and decrease in volume.
    \item The blood cells decrease in speed and expand in volume.
    \item The blood cells decrease in speed and decrease in volume.
    \end{enumerate}
    \columnbreak
    
  \item Firefighters use a hose with a 2 cm exit nozzle connected to a hydrant
    with an 8 cm diameter opening to attack a fire on the second floor of a
    building 6 m above the hydrant, as shown in the figure. What pressure must
    be supplied at the hydrant to produce an exit velocity of 15 m/s? (Assume
    the density of water is \SI{1000}{kg/m^3}, and the exit pressure is
    \SI{1e5}{\pascal}.)
    \begin{center}
      \vspace{-.2in}
      \pic{.4}{mc-q12}
     \end{center}
    \begin{enumerate}[nosep,leftmargin=18pt,label=(\Alph*)]
    \item\SI{1.7e5}{\pascal}
    \item\SI{2.0e5}{\pascal}
    \item\SI{2.6e5}{\pascal}
    \item\SI{3.2e5}{\pascal}
    \end{enumerate}
  \end{enumerate}
\end{multicols}
\end{document}
