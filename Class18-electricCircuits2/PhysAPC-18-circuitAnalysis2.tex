\documentclass[12pt,aspectratio=169]{beamer}
\usetheme{metropolis}
\setbeamersize{text margin left=.5cm,text margin right=.5cm}
\usepackage[lf]{carlito}
\usepackage{siunitx}
\usepackage{tikz}
\usepackage{mathpazo}
\usepackage{bm}
\usepackage{mathtools}
\usepackage[ISO]{diffcoeff}
\diffdef{}{ op-symbol=\mathsf{d} }
\usepackage{xcolor,colortbl}

\setmonofont{Ubuntu Mono}
\setlength{\parskip}{0pt}
\renewcommand{\baselinestretch}{1}

\sisetup{
  inter-unit-product=\cdot,
  per-mode=symbol
}

\tikzset{
  >=latex
}

%\newcommand{\iii}{\hat{\bm\imath}}
%\newcommand{\jjj}{\hat{\bm\jmath}}
%\newcommand{\kkk}{\hat{\bm k}}


\title{Classs 18: Circuit Analysis}
\subtitle{AP Physics C}
\author[TML]{Dr.\ Timothy Leung}
\institute{Olympiads School}
\date{Updated: Summer 2022}

\newcommand{\pic}[2]{
  \includegraphics[width=#1\textwidth]{#2}
}
\newcommand{\eq}[2]{
  \vspace{#1}{\Large
    \begin{displaymath}
      #2
    \end{displaymath}
  }
}
%\newcommand{\iii}{\ensuremath\hat{\bm{\imath}}}
%\newcommand{\jjj}{\ensuremath\hat{\bm{\jmath}}}
%\newcommand{\kkk}{\ensuremath\hat{\bm{k}}}
\newcommand{\iii}{\ensuremath\hat\imath}
\newcommand{\jjj}{\ensuremath\hat\jmath}
\newcommand{\kkk}{\ensuremath\hat k}



\begin{document}

\begin{frame}
  \maketitle
\end{frame}

%\section{Electric Current}
%
%\begin{frame}{Current}
%  The \textbf{electric current} is defined as the rate at which \textbf{charge
%    carriers} pass through a point in a circuit:
%  
%  \eq{-.3in}{
%    \boxed{I(t)=\diff Qt}
%  }
%
%  Expanding the expression:
%
%  \eq{-.15in}{
%    I = \diff Qt = \frac QV\diff Vt = (ne)(Av_d)
%  }
%  \begin{itemize}
%  \item $Q/V$ is the amount of charge carriers \emph{per volume}, which
%    is just
%    the \textbf{charge carrier density} (number of charge carriers per volume)
%    $n$ times the \textbf{elementary charge} $e$
%  \item $\diff V/t$ is the rate the volume of charges moves through the
%    conductor, give by the wire's cross-section area $A$ times the
%    \textbf{drift velocity} $v_d$ of the charge carrier
%  \end{itemize}
%\end{frame}
%
%
%
%\begin{frame}{Current Through the Conductor}
%  Combining the terms:
%
%  \eq{-.2in}{
%    \boxed{I=\diff Qt=neAv_d}
%  }
%  \begin{center}
%    \begin{tabular}{l|c|c}
%      \rowcolor{pink}
%      \textbf{Quantity} & \textbf{Symbol} & \textbf{SI Unit} \\ \hline
%      Current                               & $I$ & \si{\ampere} \\
%      Charge carrier density                & $n$ & \si{\per\metre\cubed} \\
%      Elementary charge                     & $e$ & \si{\coulomb}\\
%      Cross-section area of the conductor   & $A$ & \si{\metre\squared}\\
%      Drift velocity of the charge carriers & $v_d$ & \si{\metre\per\second}
%    \end{tabular}
%  \end{center}
%  The calculation for the charge carrier density $n$ requires some additional
%  thoughts.
%\end{frame}
%
%
%
%\begin{frame}{Charge Carrier Density}
%  Calculating the charge carrier density in a \emph{metal} conductor involves
%  some physical information about the metal:
%  \begin{enumerate}
%  \item Divide the metal's density $\rho$ by its molar mass $M$ to find the
%    \emph{number of moles of atoms per unit volume}
%  \item Multiply by Avogadro's number $N_A=\SI{6.0221e23}{\per mol}$ to find
%    \emph{number of atoms per unit volume}
%  \item Multiply by the number of free electrons per atom $k$ for that
%    particular metal
%  \end{enumerate}
%\end{frame}
%
%
%
%\begin{frame}{Charge Carrier Density}
%  Collecting all the terms from the last slide, we have:
%  
%  \eq{-.15in}{
%    \boxed{n=\frac{\rho kN_A}M}
%  }
%  \begin{center}
%    \begin{tabular}{l|c|c}
%      \rowcolor{pink}
%      \textbf{Quantity} & \textbf{Symbol} & \textbf{SI Unit} \\ \hline
%      Charge carrier density   & $n$    & \si{\per\metre\cubed} \\
%      Density of material      & $\rho$ & \si{\kilo\gram\per\metre\cubed} \\
%      Free electrons per atom  & $k$    & \\
%      Avogadro's number        & $N_A$  & \si{\per\mol}\\
%      Molar mass               & $M$    & \si{\kilo\gram\per\mol}
%    \end{tabular}
%  \end{center}
%  For copper, $M=\SI{63.54e-3}{\kilo\gram\per\mol}$,
%  $\rho=\SI{8.96e3}{\kilo\gram\per\metre\cubed}$, $k=1$ and therefore
%  $n=\SI{8.5e28}{\per\metre\cubed}$. The drift velocity is in the order of
%  $\approx\SI1{\milli\metre\per\second}$.
%\end{frame}
%
%
%
%\begin{frame}{Current}
%  Another alternate description of the electric current is to express it in
%  terms of the \textbf{current density} $J$, with a unit of of \emph{amp\`{e}re
%    per meters squared} (\si{\ampere\per\meter\squared}).
%
%  \eq{-.2in}{
%    \boxed{I(t)=J(t)A}
%  }
%
%  It is obvious from the previous expression that the current density is the
%  product of the charge carrier density, elementary charge, and the drift
%  velocity:
%
%  \eq{-.2in}{
%    \boxed{J=nev_d}
%  }
%\end{frame}
%
%
%
%\begin{frame}{Current Through the Conductor}
%  Combining the terms:
%
%  \eq{-.1in}{
%    \boxed{I=\diff Qt=neAv_d}
%  }
%  \begin{center}
%    \begin{tabular}{l|c|c}
%      \rowcolor{pink}
%      \textbf{Quantity} & \textbf{Symbol} & \textbf{SI Unit} \\ \hline
%      Current                               & $I$ & \si{\ampere} \\
%      Charge carrier density                & $n$ & \si{\per\metre\cubed} \\
%      Elementary charge                     & $e$ & \si{\coulomb}\\
%      Cross-section area of the conductor   & $A$ & \si{\metre\squared}\\
%      Drift velocity of the charge carriers & $v_d$ & \si{\metre\per\second}
%    \end{tabular}
%  \end{center}
%  The calculation for the charge carrier density $n$ requires some additional
%  thoughts.
%\end{frame}
%
%
%
%\begin{frame}{Electric Current: Conventional vs.\ Electron Flow}
%  The flow of electric current assumes the flow of \emph{positive} charges. We
%  call this the \textbf{conventional current}:
%  \begin{center}
%    \begin{tikzpicture}[scale=.8]
%      \draw (0,0)--(10,0);
%      \draw (0,1)--(10,1) node[midway,above]{$I\longrightarrow$};
%      \foreach \x in {1,3,...,9}{
%        \draw[thick,fill=pink!40](\x,.5) circle(.25) node{$+$};
%        \draw[thick,->](\x+.25,.5)--(\x+.75,.5);
%      }
%    \end{tikzpicture}
%  \end{center}
%  In a conducting wire, however, negatively charged electrons flow in the
%  opposite direction. We call this the \textbf{electron current}:
%  \begin{center}
%    \begin{tikzpicture}[scale=.8]
%      \draw (0,0)--(10,0);
%      \draw (0,1)--(10,1) node[midway,above]{$I\longrightarrow$};
%      \foreach \x in {1,3,...,9}{
%        \draw[thick,fill=blue!40!gray!20](\x,.5) circle(.25) node{$-$};
%        \draw[thick,->](\x-.25,.5)--(\x-.75,.5);
%      }
%    \end{tikzpicture}
%  \end{center}
%  %Even though it is actually electron that are moving, we will continue to
%  %treat electric current as the flow of positive charges in the direction of
%  %the conventional current.
%\end{frame}
%
%
%
%\section{Resistors}
%
%\begin{frame}{Resistivity and Electric Field}
%  The resistivity of a material is proportional to the electric field and
%  current density:
%
%  \eq{-.1in}{
%    \boxed{\bm{E}=\rho\bm{J}}
%    \quad\text{or}\quad
%    \boxed{\rho=\left|\frac EJ\right|}
%  }
%  \begin{center}
%    \begin{tabular}{l|c|c}
%      \rowcolor{pink}
%      \textbf{Quantity} & \textbf{Symbol} & \textbf{SI Unit} \\ \hline
%      Electric field & $\bm{E}$ & \si{\newton\per\coulomb} \\
%      Current density & $\bm{J}$ & \si{\ampere\per\metre\squared} \\
%      Resistivity & $\rho$ & \si{\ohm.\metre}
%    \end{tabular}
%  \end{center}
%  \begin{itemize}
%  \item In a conductor, the electrons are free to move, and the electric
%    field tend to be weak, and the resistivity is low.
%  \item In an insulator, electrons cannot move easily, therefore the electric
%    field are generally strong, and the resistivity is high.
%  \end{itemize}
%\end{frame}
%
%
%
%\begin{frame}{Resistance of a Conductor}
%  The resistance of a conductor is proportional to the resistivity $\rho$ and
%  its length $L$, and inversely proportional to the cross-sectional area $A$:
%
%  \eq{-.2in}{
%    \boxed{R = \rho\frac LA}
%  }
%  \begin{center}
%    \begin{tabular}{l|c|c}
%      \rowcolor{pink}
%      \textbf{Quantity} & \textbf{Symbol} & \textbf{SI Unit} \\ \hline
%      Resistance           & $R$    & \si{\ohm} \\
%      Resistivity          & $\rho$ & \si{\ohm.\metre} \\
%      Length of conductor  & $L$    & \si{\metre} \\
%      Cross-sectional area & $A$    & \si{\metre\squared}
%    \end{tabular}
%  \end{center}
%\end{frame}
%
%
%\begin{frame}{Resistance of a Conductor}
%  \eq{-.01in}{
%    \boxed{R = \rho\frac LA}
%  }
%  
%  \begin{columns}[t]
%    \column{.5\textwidth}
%    \centering
%    \begin{tabular}{c|c|c}
%      \rowcolor{blue!50}
%      {\color{white}Gauge} & 
%      {\color{white}Diameter} & 
%      {\color{white}$R/L$} \\
%      \rowcolor{blue!50}
%      & {\color{white}(\si{mm})} & 
%      {\color{white}(\SI{e-3}{\ohm\per\metre})}\\ \hline
%      0  & \num{9.35} & \num{0.31} \\
%      10 & \num{2.59} & \num{2.20} \\
%      14 & \num{1.63} & \num{8.54} \\
%      18 & \num{1.02} & \num{21.90} \\
%      22 & \num{0.64} & \num{51.70}
%    \end{tabular}
%    
%    \column{.5\textwidth}
%    \centering
%    \begin{tabular}{c|c}
%      \rowcolor{blue!50}
%      {\color{white} Material} & 
%      {\color{white} Resistivity $\rho$ (\si{\ohm.\metre})}\\ \hline
%      silver    & \num{1.6e-8} \\
%      copper    & \num{1.7e-8} \\
%      aluminum  & \num{2.7e-8} \\
%      tungsten  & \num{5.6e-8} \\
%      Nichrome  & \num{100e-8} \\
%      carbon    & \num{3500e-8}\\
%      germanium & \num{.46} \\
%      glass     & \num{e10} to \num{e14}
%    \end{tabular}
%  \end{columns}
%\end{frame}
%
%
%
%\section{Ohm's Law}
%
%\begin{frame}{Ohm's Law}
%  The electric potential difference $V$ across a ``load'' (resistor) equals the
%  product of the current $I$ through the load and the resistance $R$ of the
%  load.
%
%  \eq{-.2in}{
%    \boxed{V=IR}
%  }
%  \begin{center}
%    \begin{tabular}{l|c|c}
%      \rowcolor{pink}
%      \textbf{Quantity} & \textbf{Symbol} & \textbf{SI Unit} \\ \hline
%      Potential difference & $V$    & \si{\volt} \\
%      Current              & $I$    & \si{\ampere}\\
%      Resistance           & $R$    & \si{\ohm}
%    \end{tabular}
%  \end{center}
%  A resistor is considered ``ohmic'' if it obeys Ohm's law
%\end{frame}
%
%
%\begin{frame}{Power Dissipated by a Resistor}
%  Power is the rate at which work $W$ is done, and from electrostatics, the
%  change in electric potential energy $\Delta E_q$ is proportional to the
%  amount of charge $q$ and the voltage $V$. This gives a very simple expression
%  for power through a resistor:
%  
%  \eq{-.15in}{
%    P=\diff Wt=\diff {E_q}t=\diff{(qV)}t=\left(\diff qt\right)V
%    \;\rightarrow\;\boxed{P=IV}
%  }
%%  \begin{center}
%%    \begin{tabular}{l|c|c}
%%      \rowcolor{pink}
%%      \textbf{Quantity} & \textbf{Symbol} & \textbf{SI Unit} \\ \hline
%%      Power through a resistor    & $P$ & \si{\watt} \\
%%      Current through a resistor  & $I$ & \si{\ampere} \\
%%      Voltage across the resistor & $V$ & \si{\volt}
%%    \end{tabular}
%%  \end{center}
%%\end{frame}
%%
%%
%%
%%\begin{frame}{Other Equations for Power}
%
%  Combining Ohm's law with the above equation gives two additional expressions
%  for power through a resistor:
%
%  \eq{-.2in}{
%    \boxed{P=\frac{V^2}{R}}\quad\boxed{P=I^2R}
%  }
%%  \begin{center}
%%    \begin{tabular}{l|c|c}
%%      \rowcolor{pink}
%%      \textbf{Quantity} & \textbf{Symbol} & \textbf{SI Unit} \\ \hline
%%      Power      & $P$ & \si{\watt} \\
%%      Voltage    & $V$ & \si{\volt} \\
%%      Resistance & $R$ & \si{\ohm}  \\
%%      Current    & $I$ & \si{\ampere}
%%    \end{tabular}
%%  \end{center}
%\end{frame}
%
%
%\section{Kirchhoff's Laws}
%
%\begin{frame}{Kirchhoff's Current Law}
%  The electric current that flows into any junction in an electric circuit must
%  be equal to the current which flows out.
%
%  \vspace{.2in}
%  \begin{columns}
%    \column{.4\textwidth}
%    \begin{tikzpicture}[scale=1.6]
%      \begin{scope}[thick]
%        \draw(.2,1) to[short,o-*] (1.5,1)--(3.5,1)--(3.5,-.2);
%        \draw(1.5,1)--(1.5,-.2);
%        \draw(2.5,1)--(2.5,-.2);
%      \end{scope}
%      \begin{scope}[ultra thick,->,red]
%        \foreach\x in {1,...,3}{
%          \draw(\x+.5,.9)--(\x+.5,.1)   node[midway,left]{$I_{\x}$};
%        }
%        \draw(.4,1)--(1.3,1) node[midway,above]{$I$};
%      \end{scope}
%    \end{tikzpicture}
%
%    \column{.6\textwidth}
%    In the example on the left, with $I$ going into the junction, and $I_1$,
%    $I_2$ and $I_4$ coming out, the current law says that
%
%    \eq{-.2in}{
%      I=I_1+I_2+I_3
%    }
%
%    \vspace{-.1in}Basically, it means that there cannot be any accumulation of
%    charges anywhere in the circuit. The law is a consequence of conservation of
%    energy.
%  \end{columns}
%\end{frame}
%
%
%
%\begin{frame}{Kirchhoff's Voltage Law}
%  The voltage changes around any closed loop in the circuit must sum to zero,
%  no matter what path you take through an electric circuit.
%
%  \vspace{.1in}
%  \begin{columns}
%    \column{.3\textwidth}
%    \begin{center}
%      \begin{tikzpicture}[scale=1.8,american voltages]
%        \draw[thick](0,0) to[battery1,l=$\mathcal{E}$] (0,1.5)--(1.2,1.5)
%        to[R=$R$] (1.2,0)--(0,0);
%      \end{tikzpicture}
%    \end{center}
%    \column{.7\textwidth}
%    Assume that the current flows clockwise and we draw a clockwise loop, we
%    get
%
%    \eq{-.2in}{
%      \mathcal{E}-V_R=0\;\;\rightarrow\;\; \mathcal{E}-IR=0
%    }
%
%%    \vspace{-.25in}If I incorrectly guess that $I$ flows counterclockwise, I
%%    will still have a similar expression
%%
%%    \eq{-.45in}{-V_R-V=0\;\;\rightarrow\;\; -V-IR=0}
%  \end{columns}
%%  \vspace{-.1in}When solving for $I$, we get a negative number, indicating
%%  that my guess was in the wrong direction.
%\end{frame}
%
%
%
%\section{Resistors in Circuits}
%
%\begin{frame}{Resistors in Parallel}
%  \begin{columns}
%    \column{.32\textwidth}
%    \begin{tikzpicture}
%     \draw[thick] (0,2) to[short,o-] (1,2) to[R=$R_1$] (1,0) to[short,-o] (0,0);
%     \draw[thick](1,2)  to[short] (2.25,2)to[R=$R_2$] (2.25,0)to[short] (1,0);
%     \draw[thick](2.25,2)to[short](3.5,2) to[R=$R_3\cdots$] (3.5,0)
%     to[short] (2.25,0);
%    \end{tikzpicture}
%    
%    \column{.68\textwidth}
%    The total current is the current through all the resistors, which can be
%    rewritten in terms of voltage and resistance using Ohm's law:
%
%    \eq{-.25in}{
%      I=I_1+I_2+I_3\cdots=\frac{V_1}{R_1}+\frac{V_2}{R_2}+\frac{V_3}{R_3}\cdots
%    }
%    
%    Since $V_1=V_2=V_3=\cdots=V$ from the voltage law, we can re-write as
%
%    \eq{-.3in}{
%      I=\frac{V}{R_p}=V\left(\frac{1}{R_1}+\frac{1}{R_2}+\frac{1}{R_3}
%      \cdots\right)
%    }
%  \end{columns}
%\end{frame}
%
%
%
%\begin{frame}{Equivalent Resistance of Resistors in Parallel} 
%  %Through applying Ohm's Law and Kirchkoff's laws, we find the equivalent
%  %resistance of a parallel circuit, which we have known since Grade 9:
%  The reciprocal of the equivalent resistance for resistors connected in
%  parallel is the sum of the inverses of the individual resistances.
%
%  \eq{-.2in}{
%    \boxed{
%      \frac{1}{R_p}
%      =\frac{1}{R_1}+\frac{1}{R_2}+\cdots+\frac{1}{R_N}
%    }
%  }
%  \begin{center}
%    \begin{tabular}{l|c|c}
%      \rowcolor{pink}
%      \textbf{Quantity} & \textbf{Symbol} & \textbf{SI Unit} \\ \hline
%      Equivalent resistance in parallel & $R_p$ & \si{\ohm} \\
%      Resistance of individual loads    & $R_{1,2,3,\cdots,N}$ & \si{\ohm}
%    \end{tabular}
%  \end{center}
%\end{frame}
%
%
%
%\begin{frame}{Resistors in Series}
%  \begin{center}
%    \begin{tikzpicture}
%      \draw[thick](0,0) to[R=$R_1$,o-] (2,0) to[R=$R_2$] (4,0)
%      to[R=$R_3$,-o] (6,0);
%    \end{tikzpicture}
%  \end{center}
%
%  \vspace{.1in}The analysis for resistors in series is similar (but easier).
%  From the current law, the current through each resistor is the same:
%
%  \eq{-.2in}{I_1=I_2=I_3=\cdots=I}
%
%  \vspace{-.15in}And the total voltage drop across all resistor is therefore:
%
%  \eq{-.2in}{
%    V=V_1+V_2+V_3+\cdots=I(R_1+R_2+R_3+\cdots)
%  }
%\end{frame}
%
%
%
%\begin{frame}{Equivalent Resistance: Resistors in Series}
%  %Again, through applying Ohm's Law and Kirchkoff's laws, we find that when
%  %resistors are connected in series:
%  The equivalent resistance of loads is the sum of the resistances of the
%  individual loads.
%  
%  \eq{-.2in}{
%    \boxed{R_s=\sum_{i=1}^{N}R_i}
%  }
%  \begin{center}
%    \begin{tabular}{l|c|c}
%      \rowcolor{pink}
%      \textbf{Quantity} & \textbf{Symbol} & \textbf{SI Unit} \\ \hline
%      Equivalent resistance in series & $R_s$ & \si{\ohm} \\
%      Resistance of individual loads & $R_{1,2,3,\cdots,N}$ & \si{\ohm}
%    \end{tabular}
%  \end{center}
%\end{frame}
%
%
%%\begin{frame}{Example Problem (Simple)}
%%  A simple circuit analysis problem will involve one voltage source and
%%  resistors connected, some in parallel, and some in series. Below is a typical
%%  example:
%%
%%  \vspace{.2in}
%%  \begin{columns}
%%    \column{.4\textwidth}
%%    \begin{tikzpicture}[scale=.8,american voltages]
%%      \draw(0,0) to[R=2<\ohm>](0,2) to[battery=100<\volt>] (0,4) to[short]
%%      (1,4) to [R=10<\ohm>] (3,4)--(3,4.5) to[R=40<\ohm>] (5,4.5)
%%      to[short] (5,4) to[short] (5.5,4)--(5.5,0)--(0,0);
%%      \draw(3,4)--(3,3.5) to[R,l_=10<\ohm>] (5,3.5)--(5,4);
%%      \draw[dashed] (-1.8,0.25) rectangle(0.6,3.75);
%%    \end{tikzpicture}
%%    \column{.6\textwidth}
%%    Two \SI{10}{\ohm} resistors and a \SI{40}{\ohm} resistor are connected as
%%    shown to a \SI{100}{\volt} emf source with internal resistance
%%    \SI{2}{\ohm}. How much power is dissipated by the \SI{40}{\ohm} resistor?
%%    \begin{enumerate}[(A)]
%%    \item\SI{160}{\watt}
%%    \item\SI{40}{\watt}
%%    \item\SI{400}{\watt}
%%    \item\SI{5}{\watt}
%%    \item\SI{500}{\watt}
%%    \end{enumerate}
%%  \end{columns}
%%\end{frame}
%
%
%
%\begin{frame}{Tips for Solving ``Simple'' Circuit Problems}
%  \begin{enumerate}
%  \item Identify groups of resistors that are in parallel or in series, and
%    find their equivalent resistance.
%  \item Gradually reduce the entire circuit to one voltage source and one
%    resistor.
%  \item Using Ohm's law, find the current out of the battery.
%  \item Using Kirchhoff's laws, find the current through each of the resistors.
%  \end{enumerate}
%\end{frame}
%
%
%
%\begin{frame}{Circuits Aren't Always Simple}
%  Some of these problems require you to solve a system of linear equations.
%  The following is a simple example with two voltage sources:
%  \begin{center}
%    \begin{tikzpicture}[american voltages,scale=1.3]
%      \draw[thick](0,0) to[battery,l=$\mathcal{E}_1$](0,2) to[R,l=$R_1$](2,2)
%      to[R,l=$R_3$](2,0)--(0,0);
%      \draw[thick](2,0)--(4,0) to[battery,l_=$\mathcal{E}_2$](4,2)
%      to[R,l_=$R_2$](2,2);
%      \uncover<2->{
%        \draw[very thick,red,->](2,0.4)--(2,0)
%        node[midway,right]{\small $I_1+I_2$};
%        \draw[very thick,red,->] (0.6,0.4)..controls (0,2) and (2,2)..(1.6,.4)
%        node[midway,below]{\small $I_1$};
%        \draw[very thick,red,->] (3.4,0.4)..controls (4,2) and (2,2)..(2.4,.4)
%        node[midway,below]{\small $I_2$};
%      }
%    \end{tikzpicture}
%  \end{center}
%  \uncover<2->{
%    In this case, we have to draw two loops of current.
%  }
%\end{frame}
%
%
%
%%\begin{frame}{A More Difficult Example}
%%  \begin{columns}
%%    \column{.4\textwidth}
%%    
%%    \vspace{-.3in}
%%    \begin{tikzpicture}[american voltages]
%%      \draw(0,0) to[battery=$V_1$](0,2) to[R=$R_1$](2,2)to[R=$R_3$](2,0)--(0,0);
%%      \draw(2,0)--(4,0) to[battery,l_=$V_2$](4,2) to[R,l_=$R_2$](2,2);
%%      \draw[very thick,red,->](2,0.4)--(2,0)
%%      node[midway,right]{\tiny $I_1+I_2$};
%%      \draw[very thick,red,->] (0.6,0.4)..controls (0,2) and (2,2)..(1.6,.4)
%%      node[midway,below]{\tiny $I_1$};
%%      \draw[very thick,red,->] (3.4,0.4)..controls (4,2) and (2,2)..(2.4,.4)
%%      node[midway,below]{\tiny $I_2$};
%%    \end{tikzpicture}
%%    \column{.6\textwidth}
%%    We split the circuit into two loops, and apply Kirchkoff's voltage in both:
%%
%%    \vspace{-.4in}{\Large
%%      \begin{align*}
%%        V_1-I_1R_1-(I_1+I_2)R_3&=0\\
%%        V_2-I_2R_2-(I_1+I_2)R_3&=0
%%      \end{align*}
%%    }
%%  \end{columns}
%%
%%  \vspace{-.1in}Two equations, two unknowns ($I_1$ and $I_2$). We can subtract
%%  (2) from (1), then solve for $I_1$ and $I_2$:
%%
%%  \eq{-.4in}{
%%    I_1=\frac{V_1-I_2R_3}{R_1+R_3}
%%    \quad\quad
%%    I_2=
%%    \frac{\left[V_2-\frac{(V_1-V_2)R_3}{R_1}\right]}
%%         {\left[R_2+\frac{(R_1+R_2)R_3}{R_1}\right]}
%%  }
%%
%%  (Try this at home as an exercise.)
%%\end{frame}
%
%
%\begin{frame}{As Difficult As It Gets}
%  \begin{columns}
%    \column{.45\textwidth}
%    \begin{tikzpicture}[american voltages,scale=1.2]
%      \draw[thick](0,0)--(0,2) to[R=3<\ohm>] (0,4)
%      to[battery1,l=42<\volt>] (2,4) to[R=3<\ohm>] (4,4)--(4,2)
%      to[R=4<\ohm>](4,0)--(2,0) to[battery1,l=6<\volt>] (2,2)
%      to[R,l_=4<\ohm>] (0,2);
%      \draw[thick](0,0) to[R,l_=6<\ohm>](2,0);
%      \draw[thick](2,2) to[R,l_=6<\ohm>](4,2);
%      \uncover<2->{
%        \draw[ultra thick,red,->]
%        (.5,3.5)..controls(4.75,4)and(4.75,2)..(.5,2.5)
%        node[midway,left]{$I_1$};
%        \draw[ultra thick,red,->] (0.6,0.4)..controls (0,2) and (2,2)..(1.4,.4)
%        node[midway,below]{$I_3$};
%        \draw[ultra thick,red,->] (2.6,0.4)..controls (2,2) and (4,2)..(3.4,.4)
%        node[midway,below]{$I_2$};
%      }
%    \end{tikzpicture}
%
%    
%    \column{.55\textwidth}
%    \begin{itemize}
%    \item To solve this problem, we define a few ``loops'' around the circuit:
%      one on top, one on bottom left, and one on bottom right.
%    \item<2-> Apply the voltage law in the loops. For example, in the
%      lower left:
%
%      \eq{-.5in}{
%        4(I_1-I_3)-6-6I_3=0
%      }      
%    \item<2->\vspace{-.1in} Solve the linear system to find the current. If the
%      current that you worked out is negative, it means that you have the
%      direction wrong.
%    \end{itemize}
%  \end{columns}
%\end{frame}
%


\section{Capacitors in Circuit}

\begin{frame}{Capacitors in Parallel}
  \begin{center}
    \begin{tikzpicture}[scale=1.2]
      \draw[thick](0,1)to[short,o-](1,1) to[C=$C_1$] (1,0)to[short,-o](0,0);
      \draw[thick](1,1)--(3,1) to[C=$C_2$]       (3,0)--(1,0);
      \draw[thick](3,1)--(5,1) to[C=$C_3\ldots$] (5,0)--(3,0);
    \end{tikzpicture}
  \end{center}
  From the voltage law, we know that the voltage across all the capacitors are
  the same, i.e.\ $V_1=V_2=V_3-\cdots=V$. We can express the total charge
  $Q_\text{tot}$ stored across all the capacitors in terms of capacitance and
  this common voltage $V$: 

  \eq{-.3in}{
    Q_\text{tot}=Q_1+Q_2+Q_3+\cdots=C_1V+C_2V+C_3V+\cdots
  }
  
  \vspace{-.15in}Factoring out $V$ from each term gives us the equivalent
  capacitance:

  \eq{-.15in}{
    \boxed{C_p=\sum_i C_i}
  }
\end{frame}


\begin{frame}{Capacitors in Series}
  Likewise, we can do a similar analysis to capacitors connected in series.
  \begin{center}
    \begin{tikzpicture}[scale=1.2]
      \draw[thick](0,0) to[C=$C_1$,o-] (1.25,0) to[C=$C_2$] (2.5,0)
      to[C=$C_3$,-o] (3.75,0);
    \end{tikzpicture}
  \end{center}
  The total voltage across these capacitors are the sum of the voltages across
  each of them, i.e.\ $V_\text{tot}=V_1+V_2+V_3+\cdots$
  
  \vspace{.1in}The charge stored on all the capacitors must be the same! The
  total voltage in terms of capacitance and charge is:

  \eq{-.25in}{
    V_\text{tot}=\frac Q{C_1}+\frac Q{C_2}+\frac Q{C_3}+\cdots
  }
\end{frame}




\begin{frame}{Equivalent Capacitance in Series}
  The inverse of the equivalent capacitance for $N$ capacitors connected in
  series is the sum of the inverses of the individual capacitance.

  \eq{-.2in}{
    \boxed{ \frac1{C_s}=\sum_i\frac1{C_i} }
  }
  
  Make sure we don't confuse ourselves with resistors.
\end{frame}



\begin{frame}{How Do We Know That Charges Are The Same?}
  It's simple to show that the charges across all the capacitors are the same
  \begin{center}
    \begin{tikzpicture}[scale=1.5]
      \draw[thick](0,0) to[C=$C_1$,o-] (1.25,0) to[C=$C_2$,-o] (2.5,0);
      \draw[dashed](0.625,-.75) rectangle (1.875,1.2);
    \end{tikzpicture}
  \end{center}
  The capacitor plates and the wire connecting them are really one piece of
  conductor. There is nowhere for the charges to leave the conductor, therefore
  when charges are accumulating on $C_1$, $C_2$ must also have the same charge
  because of conservation of charges.
\end{frame}



\section{R-C Circuits}

\begin{frame}{Circuits with Resistors and Capacitors}
  An \textbf{RC circuit} is one that has both resistors and capacitors. The
  simplest form is a resistor and capacitor connected in series, and then
  connect to a voltage source.
  \begin{center}
    \begin{tikzpicture}[american voltages]
      \draw[thick](0,0)
      to[battery,l=$V$] (0,2)
      to[R=$R$] (2,2)
      to[C=$C$] (2,0)--(0,0);
    \end{tikzpicture}
  \end{center}
  Because of the nature of capacitors, the current through the circuit will not
  be steady as were the case with only resistors.
\end{frame}



\begin{frame}{Discharging a Capacitor}
  \begin{columns}
    \column{.3\textwidth}
    \centering
    \begin{tikzpicture}[scale=1.4]
      \draw[thick](0,0) to[R=$R$] (0,2)--(2,2) to[C=$C$] (2,0)--(0,0);
    \end{tikzpicture}
    
    \column{.7\textwidth}
    The analysis starts with something simpler. There is no voltage source,
    and the capacitor is already charged to $V_c=Q_\text{tot}/C$. What happens
    when the current begin to flow?
    
    \vspace{.15in}As current starts to flow, the charge on the capacitor
    decreases. Over time the current decreases, until the capacitor is fully
    discharged, and current stops flowing.
  \end{columns}

  \vspace{.15in}Now we apply the voltage law for the circuit. In this case,
  as the current flow in the circuit \emph{decreases} the total charge in the
  capacitor, $I=-\diff Q/t$, while the voltage across a capacitor is
  $V_c=Q/C$:

  \eq{-.25in}{
    V_c-IR=0\quad\rightarrow\quad
    \frac QC+R\diff Qt=0}
\end{frame}


\begin{frame}{Discharging a Capacitor}
  Separating the variable gives the first-order linear differential equation:

  \eq{-.2in}{
    \frac{\dl Q}Q = \frac{-\dl t}{RC}
  }
  
  which we can now integrate and ``exponentiate'':

  \eq{-.15in}{
    \int\frac{\dl Q}Q = \int\frac{-\dl t}{RC}
    \;\rightarrow\;
    \ln Q=\frac{-t}{RC} + K
    \;\rightarrow\;
    Q=e^Ke^{-t/RC}
  }

  The constant of integration $K$ is the initial charge on the capacitor
  $Q_\text{tot}$:

  \eq{-.2in}{
    e^K=Q_\text{tot}
  }
\end{frame}



\begin{frame}{Discharging a Capacitor}
  The expression of charge across the capacitor is time-dependent:

  \eq{-.15in}{\boxed{Q(t)=Q_0e^{-t/\tau}}}

  \vspace{-.1in}where $Q_0=Q_\text{tot}$ is the initial charge on the
  capacitor, and $\tau=RC$ is called the \textbf{time constant}. Taking the
  time derivative of $Q(t)$ gives us the current through the circuit:

  \eq{-.15in}{
    \boxed{I(t)=\diff Qt=I_0e^{-t/\tau}}
  }

  where the initially current at $t=0$ is given by
  $I_0=Q_\text{tot}/\tau$.
\end{frame}



\begin{frame}{Charging a Capacitor}
  \begin{columns}
    \column{.3\textwidth}
    \centering
    \begin{tikzpicture}[american voltages,scale=1.25]
      \draw[thick](0,0)
      to[battery,l=$V$] (0,2)
      to[R=$R$] (2,2)
      to[C=$C$] (2,0)--(0,0);
    \end{tikzpicture}
    
    \column{.7\textwidth}
    In charging up the capacitor, we go back to our original circuit, and apply
    the voltage law, then substitute the expression for current and voltage
    across the capacitor:

    \eq{-.2in}{
      V-R\diff Qt-\frac QC=0
    }
  \end{columns}
  \vspace{.2in}Again, separating variables, and integrating, we get:

  \eq{-.2in}{
    \int\frac{\dl Q}{CV-Q}=\int\frac{\dl t}{RC}
    \quad\rightarrow\quad-\ln(CV-Q)=\frac t{RC}+K
  }
 \end{frame}



\begin{frame}{Charging a Capacitor}
  ``Exponentiating'' both sides, we have
  
  \eq{-.2in}{
    CV-Q=e^Ke^{-t/RC}
  }

  To find the constant of integration $K$, we note that at $t=0$, the charge
  across the capacitor is $0$, and we get

  \eq{-.2in}{e^K=CV=Q_\text{tot}}

  which is the charge stored in the capacitor at the end. Substitute this back
  into the equation:

  \eq{-.2in}{
    \boxed{Q(t)=Q_\text{tot}(1-e^{-t/RC})}
  }
\end{frame}




\begin{frame}{Capacitors}

  \eq{-.15in}{\boxed{Q(t)=Q_\text{tot}(1-e^{-t/\tau})}}

  \vspace{-.1in}Charging a capacitor has the same time constant $\tau=RC$ as
  during discharge. We can also differentiate to find the current through the
  circuit; it is identical to the equation for discharge:

  \eq{-.2in}{
    \boxed{I(t)=\diff Qt=I_0e^{-t/\tau}}
  }

  where the initial current is given by $I_o=Q_\text{tot}/\tau=V/R$. This
  makes sense because $V_C(t=0)=0$, and all of the energy must be dissipated
  through the resistor. Similarly at $I(t=\infty)=0$.
\end{frame}



\begin{frame}{Two Small Notes}
  \begin{enumerate}
  \item When a capacitor is uncharged, there is no voltage across the plate,
    it acts like a short circuit.
  \item When a capacitor is charged, there is a voltage across it, but no
    current flows \emph{through} it. Effectively it acts like an open circuit.
  \end{enumerate}
\end{frame}



\begin{frame}{A Slightly More Difficult Problem}
  \begin{center}
    \begin{tikzpicture}[scale=1.2,american voltages]
      \draw[thick] (0,0) to[battery1,l=12<\volt>] (0,2)
      to[R=4<\ohm>] (2,2)
      to[short,-*](2.46,2.3);
      \draw[thick](2.5,2) to[short,*-] (3,2) to[short] (4,2)
      to[C=\SI{6}{\micro\farad}] (4,0)--(0,0);
      \draw[thick] (3,0) to[R=\SI{8}{\ohm}] (3,2);
    \end{tikzpicture}
  \end{center}
  \textbf{Example:} The capacitor in the circuit is initially uncharged. Find
  the current through the battery
  \begin{enumerate}
  \item Immediately after the switch is closed
  \item A long time after the switch is closed
  \end{enumerate}
\end{frame}


\section{LR Circuits}

\begin{frame}{Circuits with Inductors}
  \begin{itemize}
  \item Coils and solenoids in circuits are known as ``inductors'' and have
    large self inductance $L$
  \item Self inductance prevents currents rising and falling instantaneously
  \item A basic circuit containing a resistor and an inductor is called an
    \textbf{\emph{LR} circuit}:
    \begin{center}
      \begin{tikzpicture}[american voltages,scale=1.2]
        \draw[thick](0,0) to[battery,l=$\mathcal E_0$] (0,2)
        to[short,-*](0.5,2);
        \draw[thick](0.51,2.3) to[short,*-*](1,2)--(1.5,2) to [R=$R$] (3,2)
        to [L=$L$] (3,0)--(0,0);
      \end{tikzpicture}
    \end{center}
  \end{itemize}
\end{frame}



\begin{frame}{Analyzing LR Circuits}
  \begin{columns}
    \column{.35\textwidth}
    \begin{tikzpicture}[american voltages,scale=1.2]
      \draw[thick](0,0) to[battery,l=$\mathcal E_0$] (0,2)
      to[short,-*](0.5,2);
      \draw[thick](0.51,2.3) to[short,*-*](1,2)--(1.5,2) to [R=$R$] (3,2)
      to [L=$L$] (3,0)--(0,0);
    \end{tikzpicture}

    \column{.65\textwidth}
    Applying Kirchhoff's voltage law:

    \eq{-.2in}{\mathcal{E}_0-IR-L\diff It=0}

    Following the same procedure as charging a capacitor, the time-dependent
    current is found to be:

    \eq{-.2in}{I(t)=\frac{\mathcal E_0}R\left(1-e^{-t/\tau}\right)}

    Where the time constant $\tau$ is defined as:
    
    \eq{-.2in}{\tau=\frac LR}
  \end{columns}
\end{frame}



\section{LC Circuit}

\begin{frame}{LC Circuit}
  The final type of circuit in AP Physics is the LC circuit. In its simplest
  form, the circuit has an inductor and capacitor connected in series:
  \begin{center}
    \begin{tikzpicture}[american voltages,scale=1.2]
      \draw[thick](0,0) to[L=$L$] (0,2)--(2,2) to[C=$C$] (2,0)--(0,0);
    \end{tikzpicture}
  \end{center}
  We apply the Kirchhoff's voltage law:
  
  \eq{-.2in}{
    -V_L-V_C=0
    \quad\rightarrow\quad
    L\diff It+\frac QC=0
  }
\end{frame}

\begin{frame}{LC Circuits}
  Since both terms are continuously differentiable, we can differentiate both
  sides of the equation, which gives:

  \eq{-.25in}{
    L\diff[2]It+\frac1C\diff Qt=0
  }

  In fact, the above equation a second-order ordinary differential equation
  with constant coefficients.

  \eq{-.25in}{
    \diff[2]It+\frac1{LC}I=0
  }

  The solution to such an equation is the simple harmonic motion.
%
%
%\end{frame}
%
%
%\begin{frame}{LC Circuit}
%  The current inside of an LC circuit is given by:

  \eq{-.23in}{
    I(t)=I_0\sin(\omega t + \varphi)\quad\text{where}\quad
    \omega=\frac1{\sqrt{LC}}
  }
\end{frame}
\end{document}
