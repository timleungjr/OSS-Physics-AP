\documentclass[12pt]{article}

\usepackage[margin=.75in,letterpaper]{geometry}
\usepackage{enumitem}
\usepackage{graphicx}
\usepackage{tikz}
\usepackage{mathpazo}
\usepackage[scaled]{helvet}
\usepackage{siunitx}

\usetikzlibrary{decorations.pathmorphing,patterns}

\sisetup{number-math-rm=\mathnormal}

\renewcommand{\familydefault}{\sfdefault}

\newcommand{\pic}[2]{\includegraphics[width=#1\textwidth]{#2}}
\newcommand{\magdir}[2]{$#1\;[\mathrm{#2}]$}
\newcommand{\mb}[1]{\mathbf{#1}}

\setlength{\parindent}{0pt}

\begin{document}

\begin{center}
  Student \#: \underline{\hspace{1in}}\hspace{1.9in}
  Student Name: \underline{\hspace{2in}}\\
  \vspace{0.3in}{\LARGE
    AP Physics Class 10 \& 11: Electrostatics \& Gauss's Law}
\end{center}

The questions in this homework assignment cover AP 1 and C exams. Some of the
questions are not typically questions.

\begin{enumerate}[leftmargin=50pt,label=\underline{\hspace{0.4in}} \arabic*.]


\item Two electric objects experience a repulsive force. What happens to that
  force if the distance between the objects is doubled?
  \begin{enumerate}[noitemsep,topsep=0pt,leftmargin=18pt]
  \item It decreases to one-fourth its value.
  \item It decreases to one-half its value.
  \item It stays the same.
  \item It doubles.
  \item It quadruples.
  \end{enumerate}

\item A pith ball is a tiny piece of Styrofoam that is covered with a
  conductive paint. One pith ball initially has a charge of \SI{6.4e-8}{C},
  and it touches an identical, neutral pith ball. After the pith balls are
  separated, what is the charge on the pith ball that had the initial charge?
  \begin{enumerate}[noitemsep,topsep=0pt,leftmargin=18pt]
  \item\SI{6.4e-8}{C}
  \item\SI{3.2e-8}{C}
  \item\SI{0}{C}
  \item\SI{-3.2e-8}{C}
  \item\SI{-6.4e-8}{C}
  \end{enumerate}

\item Glass becomes positively charged when it is rubbed with silk. Which
  of the following is the best description of what’s happening?
  \begin{enumerate}[noitemsep,topsep=0pt,leftmargin=18pt]
  \item Electrons are rubbed off the glass onto the silk.
  \item Electrons are rubbed off the silk onto the glass.
  \item Protons are rubbed off the glass onto the silk.
  \item Protons are rubbed off the silk onto the glass.
  \item Neutrons in the glass have an affinity for positive charge.
  \end{enumerate}
  
\item Consider an isolated, neutral system consisting of wool fabric and a
  rubber rod. If the rubber rod is rubbed with wool to become negatively
  charged, what can be said about the wool fabric?
  \begin{enumerate}[noitemsep,topsep=0pt,leftmargin=18pt]
  \item It becomes equally negatively charged.
  \item It becomes equally positively charged.
  \item It becomes negatively charged but not equally.
  \item It becomes positively charged but not equally.
  \item In a neutral system, neither object can become charged.
  \end{enumerate}

\item An electron and a proton are separated by \SI{1.50e-10}{m}. If they are
  released, which one will accelerate at a greater rate, and what is the
  magnitude of that acceleration?
  \begin{enumerate}[noitemsep,topsep=0pt,leftmargin=18pt]
  \item The electron; \SI{1.12e22}{m/s^2}
  \item The proton; \SI{1.12e22}{m/s^2}
  \item The electron; \SI{6.13e18}{m/s^2}
  \item The proton; \SI{6.13e18}{m/s^2}
  \item They both accelerate at the same rate; \SI{1.02e-8}{m/s^2}
  \end{enumerate}

%\item A proposed ``space elevator'' can lift a \SI{1,000}{kg} payload to an
%  orbit of \SI{150}{km} above the Earth's surface. The radius of the Earth is
%  \SI{6.4e6}{m}, and the Earth's mass is \SI{6e24}{kg}. What is the
%  gravitational potential energy of the payload when it reaches orbit?
%  \begin{enumerate}[noitemsep,topsep=0pt,leftmargin=18pt]
%  \item\SI{1.0e3}{J}
%  \item\SI{2.7e6}{J}
%  \item\SI{6.1e10}{J}
%  \item\SI{2.7e12}{J}
%  \item\SI{1.0e15}{J}
%  \end{enumerate}
%
%\item The Earth is at an average distance of \SI{1}{AU} from the Sun and has an
%  orbital period of \SI{1}{year}. Jupiter orbits the Sun at approximately
%  \SI{5}{AU}. About how long is the orbital period of Jupiter?
%  \begin{enumerate}[noitemsep,topsep=0pt,leftmargin=18pt]
%  \item\SI{1}{year}
%  \item\SI{2}{years}
%  \item\SI{5}{years}
%  \item\SI{11}{years}
%  \item\SI{125}{years}
%  \end{enumerate}
%  
%\item A satellite orbits the Earth at a distance of \SI{200}{km}. If the mass
%  of the Earth is \SI{6.0e24}{kg} and the Earth's radius is \SI{6.4e6}{m}, what
%  is the satellite's speed?
%  \begin{enumerate}[noitemsep,topsep=0pt,leftmargin=18pt]
%  \item\SI{1e3}{m/s}
%  \item\SI{3.5e3}{m/s}
%  \item\SI{7.8e3}{m/s}
%  \item\SI{5e6}{m/s}
%  \item\SI{6.1e7}{m/s}
%  \end{enumerate}
%  
%\item Mars orbits the Sun at a distance of \SI{2.3e11}{m}. The mass of the Sun
%  is \SI{2e30}{kg}, and the mass of Mars is \SI{6.4e23}{kg}. Approximately
%  what is the gravitational force that the Sun exerts on Mars?
%  \begin{enumerate}[noitemsep,topsep=0pt,leftmargin=18pt]
%  \item\SI{1.6e20}{N}
%  \item\SI{1.6e21}{N}
%  \item\SI{3.7e21}{N}
%  \item\SI{3.7e32}{N}
%  \item\SI{3.7e42}{N}
%  \end{enumerate}
%
%\item When climbing from sea level to the top of Mount Everest, a hiker
%  changes elevation by \SI{8848}{m}. By what percentage will the
%  gravitational field of the Earth change during the climb? (The Earth's
%  mass is \SI{6.0e24}{kg}, and its radius is \SI{6.4e6}{m}.)
%  \begin{enumerate}[noitemsep,topsep=0pt,leftmargin=18pt]
%  \item It will increase by approximately $0.3\%$.
%  \item It will decrease by approximately $0.3\%$.
%  \item It will increase by approximately $12\%$.
%  \item It will decrease by approximately $12\%$.
%  \item The gravitational field strength will not change.
%  \end{enumerate}
%  \newpage
%  
%\item Four planets, A through D, orbit the same star. The relative masses and
%  distances from the star for each planet are shown in the table. For
%  example, Planet A has twice the mass of Planet B, and Planet D has
%  three times the orbital radius of Planet A. Which planet has the highest
%  gravitational attraction to the star?
%  \begin{center}
%    \begin{tabular}{lll}
%      \hline
%      \textbf{Planet} & \textbf{Relative mass} & \textbf{Relative distance}\\
%      \hline
%      A\hspace{0.4in}& $2m$   & $r$    \\ \hline
%      B & $m$    & $0.1r$\hspace{0.25in} \\ \hline
%      C & $0.5m$\hspace{0.25in} & $2r$   \\ \hline
%      D & $4m$   & $3r$   \\ \hline
%    \end{tabular}
%  \end{center}
%  \begin{enumerate}[noitemsep,topsep=0pt,leftmargin=18pt]
%  \item Planet A
%  \item Planet B
%  \item Planet C
%  \item Planet D
%  \item All have the same gravitational attraction to the star.
%  \end{enumerate}
%
%\item A satellite orbits the Earth at a distance that is four times the radius
%  of the Earth. If the acceleration due to gravity near the surface of the Earth
%  is $g$, the acceleration of the satellite is most nearly
%  \begin{enumerate}[noitemsep,topsep=0pt,leftmargin=18pt]
%  \item zero
%  \item $g/2$
%  \item $g/4$
%  \item $g/8$
%  \item $g/16$
%  \end{enumerate}
%
%\item The mass of a planet is $1⁄4$ that of Earth and its radius is half of
%  Earth's radius. The acceleration due to gravity on this planet is most nearly
%  \begin{enumerate}[noitemsep,topsep=0pt,leftmargin=18pt]
%  \item\SI{2 }{m/s^2}
%  \item\SI{4 }{m/s^2}
%  \item\SI{5 }{m/s^2}
%  \item\SI{10}{m/s^2}
%  \item\SI{20}{m/s^2}
%  \end{enumerate}
%  \begin{center}
%    \vspace{-.25in}
%    \begin{tikzpicture}[scale=1.4]
%      \tikzstyle{balloon}=[ball color=gray];
%      \draw(0,0) ellipse (2 and 1);
%      \draw[fill=black](2,0) circle(0.05) node[right]{A};
%      \draw[fill=black](-2,0) circle(0.05) node[left]{B};
%      \shade[balloon] (0.5,0) circle (0.2);
%    \end{tikzpicture}
%  \end{center}
%  
%\item A satellite orbits the Earth in an elliptical orbit, with point A being
%  close to the Earth and point B farther away. As the satellite moves from
%  point A to point B, which of the following is true of the angular momentum and
%  kinetic energy of the satellite?
%  
%  \begin{tabular}{lll}
%    & \underline{Angular momentum} & \underline{Kinetic energy}\\
%    (a) & Increases & Remains constant \\
%    (b) & Remains constant & Increases \\
%    (c) & Decreases & Remains constant \\
%    (d) & Remains constant & Decreases \\
%    (e) & Remains constant & Remains constant
%  \end{tabular}
%
%\item Two planets of mass $M$ and $9M$ are in the same solar system. The
%  radius of the planet of mass $M$ is $R$. In order for the acceleration due to
%  gravity to be the same for each planet, the radius of the planet of mass
%  $9M$ would have to be
%  \begin{enumerate}[noitemsep,topsep=0pt,leftmargin=18pt]
%  \item $1/2\;R$
%  \item $R$
%  \item $2R$
%  \item $3R$
%  \item $9R$
%  \end{enumerate}
%  \vspace{-0.5in}
%  \begin{center}
%    \begin{tikzpicture}[scale=0.7]
%      \draw[fill=gray!60](0,0) circle(0.25);
%      \draw[dashed](0,0) circle(1);
%      \draw[dashed](0,0) circle(3);
%      \draw(0,0)--(1,0) node[pos=1,right]{$x$} node[midway,above]{$R$};
%      \draw[fill=black](1,0) circle(0.05);
%      \begin{scope}[rotate=230]
%        \draw(0,0)--(3,0) node[pos=1,right]{$y$} node[pos=0.7,right]{$3R$};
%        \draw[fill=black](3,0) circle(0.05);
%      \end{scope}
%    \end{tikzpicture}
%  \end{center}
%\item Two planets, X and Y, orbit a star. Planet X orbits at a radius $R$, and
%  Planet Y orbits at a radius $3R$. Which of the following best represents
%  the relationship between the acceleration $a_X$ of Planet X and the
%  acceleration $a_Y$ of Planet Y?
%  \begin{enumerate}[noitemsep,topsep=0pt,leftmargin=18pt]  
%  \item $a_X = 9a_Y$
%  \item $9a_X = a_Y$
%  \item $a_X = 3a_Y$
%  \item $3a_X = a_Y$
%  \item $a_X = a_Y$
%  \end{enumerate}
%\item A satellite is in a stable circular orbit around the Earth at a radius $R$
%  and speed $v$. At what radius would the satellite travel in a stable orbit
%  with a speed $2v$?
%  \begin{enumerate}[noitemsep,topsep=0pt,leftmargin=18pt]  
%  \item $1⁄4\;R$
%  \item $1⁄2\;R$
%  \item $R$
%  \item $2R$
%  \item $4R$
%  \end{enumerate}
%
%\item The Earth and the moon apply a gravitational force to each other.
%  Which of the following statements is true?
%  \begin{enumerate}[noitemsep,topsep=0pt,leftmargin=18pt]  
%  \item The Earth applies a greater force on the moon than the moon exerts on
%    the Earth.
%  \item The Earth applies a smaller force on the moon than the moon exerts on
%    the Earth.
%  \item The Earth applies a force on the moon, but the moon does not exert a
%    force on the Earth.
%  \item The Earth does not apply a force on the moon, but the moon exerts a
%    force on the Earth.
%  \item The force the Earth applies to the moon is equal and opposite to the
%    force the moon applies to the Earth.
%  \end{enumerate}
%  \newpage
%  
%\item Two masses exert a gravitational force $F$ on each other. If one of the
%  masses is doubled, and the distance between the masses is tripled, the
%  new force between them is
%  \begin{enumerate}[noitemsep,topsep=0pt,leftmargin=18pt]  
%  \item $6F$
%  \item $2/3\;F$
%  \item $2/9\;F$
%  \item $3/2\;F$
%  \item $4/9\;F$
%  \end{enumerate}
%
%\item A planet orbits at a radius $R$ around a star of mass $M$. The period of
%  orbit of the planet is
%  \begin{enumerate}[noitemsep,topsep=0pt,leftmargin=18pt]  
%  \item $\displaystyle\sqrt{\frac{4\pi^2R^2}{GM}}$
%  \item $\displaystyle\frac{4\pi^2R^3}{GM}$
%  \item $\displaystyle\sqrt{\frac{4\pi^2R^3}{GM}}$
%  \item $\displaystyle\sqrt{\frac{4\pi^2R}{GM}}$
%  \item $\displaystyle\frac{GM}{4\pi^2R}$
%  \end{enumerate}
%  
%\item A moon orbits a large planet in an elliptical orbit, with its closest
%  approach at a distance $a$, and its farthest distance $b$. The speed of the
%  moon at point b is $v$. The speed at point $a$ is
%  \begin{enumerate}[noitemsep,topsep=0pt,leftmargin=18pt]  
%  \item $\displaystyle\frac{av}{b}$
%  \item $\displaystyle\frac{bv}{a}$
%  \item $\displaystyle\frac{(a+b)v}{b}$
%  \item $\displaystyle\frac{(b-a)v}{b}$
%  \item $\displaystyle\frac{2bv}{a}$
%  \end{enumerate}
%
%
%\item A satellite orbits the Earth in an elliptical orbit. Which of the
%  following statements is true?
%  \begin{enumerate}[noitemsep,topsep=0pt,leftmargin=18pt]  
%  \item The angular velocity of the satellite increases as it travels
%    farther from the Earth.
%  \item The acceleration of the satellite increases as it travels closer
%    to the Earth.
%  \item The angular momentum of the satellite increases as it travels
%    closer to the Earth.
%  \item The potential energy of the satellite is equal to its kinetic
%    energy at all points in the orbit.
%  \item The speed of the satellite must remain constant for it to remain
%    in orbit around the Earth.
%  \end{enumerate}
%
%  \begin{center}
%    \begin{tikzpicture}[scale=0.8]
%      \tikzstyle{balloon}=[ball color=gray];
%      \draw(0,0) circle(0.25);
%      \draw[dashed](0,0) circle(3) node[below]{$M$};
%      \begin{scope}[rotate=50]
%        \draw[->](0.25,0)--(2.8,0) node[midway,right]{$R$};
%        \shade[balloon] (3,0) circle (0.2) node[right]{$m$};
%        \draw[thick,->](3,0.25)--(3,1.25) node[pos=1,above]{$v$};
%      \end{scope}
%      \begin{scope}[rotate=130]
%        \shade[balloon] (3,0) circle (0.2) node[left]{$2m$};
%        \draw[thick,->](3,-0.25)--(3,-1.25) node[pos=1,above]{$v$};
%      \end{scope}
%    \end{tikzpicture}
%  \end{center}
%  
%\item\vspace{-.15in}Two moons of mass $m$ and $2m$ orbit a planet of mass $M$
%  at the same radius $R$ and speed $v$ toward each other, as shown. The moons
%  collide and stick together without destroying either moon. The total momentum
%  of the moons after the collision is
%  \begin{enumerate}[noitemsep,topsep=0pt,leftmargin=18pt]  
%  \item $mv$
%  \item $2mv$
%  \item $3mv$
%  \item $6mv$
%  \item zero
%  \end{enumerate}
%
%\item The velocity of the two masses after the collision above is
%  \begin{enumerate}[noitemsep,topsep=0pt,leftmargin=18pt]  
%  \item $v$ counterclockwise
%  \item $v/2$ counterclockwise
%  \item $v/2$ clockwise
%  \item $v/3$ counterclockwise
%  \item $v/3$ clockwise
%  \end{enumerate}
%
%\item Consider a two-star system shown above, which consists of two stars of
%  mass $m$ rotating in a circle of radius $r$ about their center of mass. What
%  is the total energy of the two-star system?
%  \begin{enumerate}[noitemsep,topsep=0pt,leftmargin=18pt]  
%  \item $-Gm^2/2r$
%  \item $Gm^2/2r$
%  \item $Gm^2/4r$
%  \item $3Gm^2/4r$
%  \item $-Gm^2/4r$
%  \end{enumerate}
%
%\item If a planet has twice the radius of Earth and half of Earth's density,
%  what is the acceleration due to gravity on the surface of the planet (in
%  terms of the gravitational acceleration $g$ on the surface of Earth)?
%  \begin{enumerate}[noitemsep,topsep=0pt,leftmargin=18pt]  
%  \item $4g$
%  \item $2g$
%  \item $g$
%  \item $g/2$
%  \item $g/4$
%  \end{enumerate}
\end{enumerate}

\newpage
\noindent\textbf{Free-Response Questions:}

\begin{enumerate}[leftmargin=15pt]

\item A charged spherical shell
%  charge and touches an identical pith ball that is initially neutral. The
%  diagram below (not to scale) shows the final configuration of the two pith
%  balls as they hang from threads:
%  \begin{center}
%    \begin{tikzpicture}[scale=1.3]
%      \draw[thick](0,0)--(1,-2);
%      \draw[thick](0,0)--(-1,-2);
%      \draw[fill=gray!70](1,-2) circle(0.2);
%      \draw[fill=gray!70](-1,-2) circle(0.2);
%      \draw(-1,-2.5)--(1,-2.5) node[midway,below]{\SI{52}{cm}};
%      \draw(-1,-2.35)--(-1,-2.65);
%      \draw( 1,-2.35)--( 1,-2.65);
%    \end{tikzpicture}
%  \end{center}
%  \begin{enumerate}[noitemsep]
%  \item Explain the value of the charge on each pith ball after they are
%    separated.
%  \item Construct a free-body diagram with symbols representing all the forces
%    on the pith ball on the right.
%  \item Calculate the value of all the forces on the pith ball in your
%    free-body diagram.
%  \item Determine the angle between the threads in the diagram above.
%  \end{enumerate}
%
%
%  
\end{enumerate}
\end{document}
