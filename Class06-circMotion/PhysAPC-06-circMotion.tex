\documentclass[12pt,compress,aspectratio=169]{beamer}
\usetheme{metropolis}
\setbeamersize{text margin left=.5cm,text margin right=.5cm}
\usepackage[lf]{carlito}
\usepackage{siunitx}
\usepackage{tikz}
\usepackage{mathpazo}
\usepackage{bm}
\usepackage{mathtools}
\usepackage[ISO]{diffcoeff}
\diffdef{}{ op-symbol=\mathsf{d} }
\usepackage{xcolor,colortbl}

\setmonofont{Ubuntu Mono}
\setlength{\parskip}{0pt}
\renewcommand{\baselinestretch}{1}

\sisetup{
  inter-unit-product=\cdot,
  per-mode=symbol
}

\tikzset{
  >=latex
}

%\newcommand{\iii}{\hat{\bm\imath}}
%\newcommand{\jjj}{\hat{\bm\jmath}}
%\newcommand{\kkk}{\hat{\bm k}}


\usetikzlibrary{patterns}

\title{Class 6: Circular Motion}
\subtitle{Advanced Placement Physics C}
\author[TML]{Dr.\ Timothy Leung}
\institute{Olympiads School}
\date{Updated: Summer 2022}

\newcommand{\pic}[2]{
  \includegraphics[width=#1\textwidth]{#2}
}
\newcommand{\eq}[2]{
  \vspace{#1}{\Large
    \begin{displaymath}
      #2
    \end{displaymath}
  }
}
%\newcommand{\iii}{\ensuremath\hat{\bm{\imath}}}
%\newcommand{\jjj}{\ensuremath\hat{\bm{\jmath}}}
%\newcommand{\kkk}{\ensuremath\hat{\bm{k}}}
\newcommand{\iii}{\ensuremath\hat\imath}
\newcommand{\jjj}{\ensuremath\hat\jmath}
\newcommand{\kkk}{\ensuremath\hat k}



\begin{document}

\begin{frame}
  \maketitle
\end{frame}



\begin{frame}{Review of Circular Motion}
  In a \textbf{circular motion}, an object of mass $m$ moves in a circular path
  about a fixed center. In Grade 12 Physics, you should have studied
  \emph{uniform} circular motion, where:
  \begin{itemize}
  \item the object's speed (magnitude of velocity) is constant
  \item the object's \textbf{centripetal acceleration} is toward the center
  \item the object's acceleration is caused by a \textbf{centripetal force}
  \end{itemize}
\end{frame}



\section{Polar Coordinates}

\begin{frame}{Polar Coordinate System in 2D}
  \begin{columns}
    \column{.32\textwidth}
    \begin{tikzpicture}[scale=.75]
      \draw[axes] (-3,0)--(3,0) node[right]{$x$};
      \draw[axes] (0,-3)--(0,3) node[above]{$y$};
      \draw[vectors] (0,0)--(1,0) node[below]{$\iii$};
      \draw[vectors] (0,0)--(0,1) node[left] {$\jjj$};
      \draw circle (2.5);
      \begin{scope}[rotate=38]
        \draw[vectors] (0,0)--(2.45,0) node[midway,above]{$r$};
        \draw[vectors] (2.5,0)--(3.5,0) node[right]{$\hat r$};
        \draw[vectors] (2.5,0)--(2.5,1) node[above]{$\hat\theta$};
        \draw[mass] (2.5,0) circle (.1);
      \end{scope}
      \draw[axes] (1.5,0) arc (0:38:1.5) node[pos=.55,right]{$\theta$};
    \end{tikzpicture}

    \column{.68\textwidth}
    In the Cartesian coordinate system, an object's position is described by
    its $x$ and $y$ coordinates:

    \eq{-.13in}{
      \vec r(t)=x(t)\iii + y(t)\jjj
      }

    \vspace{-.15in}For circular motion or general rotational motion, the
    \textbf{polar coordinate system} is preferred. The position of an object
    is described by:

    \eq{-.17in}{
      \vec r(t)=r(t)\hat r + \theta(t)\hat\theta
    }
    \begin{itemize}
    \item\vspace{-.15in}$r$ is distance from the origin
    \item $\theta$ is the standard angle, measured counter clockwise from the
      $x$ axis in \emph{radians}
    \end{itemize}
  \end{columns}
\end{frame}



\begin{frame}{Polar Coordinate System in 2D}
  \begin{columns}
    \column{.3\textwidth}
    \begin{tikzpicture}[scale=.75]
      \draw[axes] (-3,0)--(3,0) node[right]{$x$};
      \draw[axes] (0,-3)--(0,3) node[above]{$y$};
      \draw[vectors] (0,0)--(1,0) node[below]{$\iii$};
      \draw[vectors] (0,0)--(0,1) node[left] {$\jjj$};
      \draw circle (2.5);
      \begin{scope}[rotate=38]
        \draw[vectors] (0,0)--(2.45,0) node[midway,above]{$r$};
        \draw[vectors] (2.5,0)--(3.5,0) node[right]{$\hat r$};
        \draw[vectors] (2.5,0)--(2.5,1) node[above]{$\hat\theta$};
        \draw[mass] (2.5,0) circle (.1);
      \end{scope}
      \draw[axes] (1.5,0) arc (0:38:1.5) node[pos=.55,right]{$\theta$};
    \end{tikzpicture}

    \column{.7\textwidth}
    \begin{itemize}
    \item Like the Cartesian system, the polar coordinate system is also
      right-handed
    \item Both basis vectors $\hat r$ (radial direction) and $\hat\theta$
      (angular direction) rotate as the object moves
    \item Simpler way is to think of position as just two
      parameters, which is \emph{exactly} how position vectors are expressed in
      Grade 11/12 Physics: magnitude ($r$) and direction ($\theta$)!
    \item Cartesian and polar coordinates are related by:
      
      \vspace{-.35in}{\large
        \begin{align*}
          x&=r\cos\theta\\
          y&=r\sin\theta
        \end{align*}
      }
    \end{itemize}
  \end{columns}
\end{frame}



\begin{frame}{Cylindrical Coordinates in 3D}
  \begin{columns}
    \column{.45\textwidth}
    \begin{tikzpicture}[scale=.75]
      \draw[axes] (0,0)--(-2.5,-2.5) node[below]{$x$};
      \draw[axes] (0,0)--(5,0) node[right]{$y$};
      \draw[axes] (0,0)--(0,5) node[above]{$z$};
      \draw[axes] (-1,-1) arc (-110:-45:2) node[midway,below]{$\theta$};
      \draw[dashed,fill=green!40,opacity=.4](0,0)--(3,-1.5)
      node[pos=.6,below left,opacity=1]{$r$}--(3,2.5)
      node[midway,right,black,opacity=1]{$z$}--(0,4);
      \fill (3,2.5) circle(.1) node[right]{$\vec r(r,\theta,z)$};
    \end{tikzpicture}

    \column{.55\textwidth}
    One way to extend the coordinates coordinate system into 3D is the
    \textbf{cylindrical coordinate system}. Note that the discussions for this
    topic focuses on $xy$ plane. Since the $z$-axis is linearly independent of
    the $xy$ plane, motion along that direction is independent.
  \end{columns}
\end{frame}



\section{General Circular Motion}

\begin{frame}{Angular Position and Angular Velocity}
  \vspace{.2in}
  \begin{columns}[T]
    \column{.35\textwidth}
    \begin{tikzpicture}[scale=.75]
      \draw[axes] (-3,0)--(3,0) node[right]{$x$};
      \draw[axes] (0,-3)--(0,3) node[above]{$y$};
      \draw[axes] (1,0) arc (0:38:1) node[midway,right]{$\theta$};
      \draw circle(2.5);
      \begin{scope}[rotate=38]
        \draw[vectors] (0,0)--(2.44,0) node[midway,above]{$\vec r$};
        \draw[mass] (2.5,0) circle (.1);
      \end{scope}
    \end{tikzpicture}
    
    \column{.65\textwidth}
    For constant $r$, the \textbf{angular position} $\theta$ determines an
    object's position as a function of time. It is usually measured in
    \textbf{radian} (\si{\radian}):

    \eq{-.1in}{
      \boxed{\theta=\theta(t)}
    }
    
    \vspace{-.1in}\textbf{Angular velocity} $\omega$ (or \textbf{angular
      frequency}) is its time derivative, measured in \textbf{radian per second}
    (\si{\radian\per\second}):
      
    \eq{-.1in}{
      \boxed{\omega(t)=\diff{\theta}t}
    }
  \end{columns}
\end{frame}



\begin{frame}{Velocity and Angular Velocity}
  \begin{columns}
    \column{.35\textwidth}
    \begin{tikzpicture}[scale=.75]
      \draw[axes] (-3,0)--(3,0) node[right]{$x$};
      \draw[axes] (0,-3)--(0,3) node[above]{$y$};
      \draw[axes] (1,0) arc (0:38:1) node[midway,right]{$\theta$};
      \draw circle(2.5);
      \begin{scope}[rotate=38]
        \draw[vectors] (0,0)--(2.44,0) node[midway,above]{$\vec r$};
        \draw[vectors] (2.5,.08)--(2.5,1.5) node[above]{$\vec v$};
        \draw[mass] (2.5,0) circle (.1);
      \end{scope}
    \end{tikzpicture}

    \column{.65\textwidth}
    The velocity of the object in circular motion is related to the angular
    velocity (or angular frequency) by:

    \eq{-.1in}{
      v=r\omega
    }
    \begin{itemize}
    \item The direction of $\vec v$ is tangent to circle, along
      $\hat\theta$, and therefore $\perp$ to $\hat r$
    \item If $\omega>0$, the motion is counter-clockwise
    \item If $\omega<0$, the motion is clockwise
    \end{itemize}
  \end{columns}
\end{frame}



\begin{frame}{Velocity and Angular Velocity}
  \begin{columns}
    \column{.3\textwidth}
    \begin{tikzpicture}[scale=.75]
      \draw[axes] (-3,0)--(3,0) node[right]{$x$};
      \draw[axes] (0,-3)--(0,3) node[above]{$y$};
      \draw[axes] (1,0) arc (0:38:1) node[midway,right]{$\theta$};
      \draw circle (2.5);
      \begin{scope}[rotate=38]
        \draw[vectors] (0,0)--(2.44,0) node[midway,above]{$\vec r$};
        \draw[vectors] (2.5,.08)--(2.5,1.5) node[above]{$\vec v$};
        \draw[mass] (2.5,0) circle (.1);
      \end{scope}
    \end{tikzpicture}

    \column{.7\textwidth}
    The velocity of the object in circular motion is more properly related to
    the angular velocity using this vector cross product:

    \eq{-.1in}{
      \vec v=\vec\omega\times\vec r
    }

    \begin{itemize}
    \item\vspace{-.2in}$\vec\omega$: out of the page if motion is
      counter-clockwise
    \item $\vec\omega$: into the page if motion is clockwise
    \end{itemize}
    Visualizing $\vec\omega$ takes practice, but this vector notation is
    mathematically rigorous and consistent
  \end{columns}
\end{frame}



%\begin{frame}{Relativity Velocity}
%  If two points $A$ and $B$ are rotating with the same angular velocity with the
%  same center, their relative position is given by:
%
%  \eq{-.1in}{
%    \boxed{
%      \vec V_B=\vec V_A+ \vec\omega\times\vec r_{BA}
%    }
%  }
%
%  Where $\vec r_{BA}$ is the position of $B$ relative to $A$.
%  
%\end{frame}




\begin{frame}{Period \& Frequency}
  \begin{columns}
    \column{.3\textwidth}
    \begin{tikzpicture}[scale=.75]
      \draw[axes] (-3,0)--(3,0) node[right]{$x$};
      \draw[axes] (0,-3)--(0,3) node[above]{$y$};
      \draw[axes] (1,0) arc (0:38:1) node[midway,right]{$\theta$};
      \draw circle (2.5);
      \begin{scope}[rotate=38]
        \draw[vectors] (0,0)--(2.44,0) node[midway,above]{$\vec r$};
        \draw[vectors] (2.5,.08)--(2.5,1.5) node[above]{$\vec v$};
        \draw[mass] (2.5,0) circle (.1);
      \end{scope}
    \end{tikzpicture}

    \column{.7\textwidth}
    For constant angular velocity $\omega$ (uniform circular motion), the
    motion is periodic. Its \textbf{frequency} and \textbf{period} are given by:

    \eq{-.1in}{
      f=\frac\omega{2\pi}\quad
      T=\frac{2\pi}\omega\quad
      f=\frac1T
    }
    
    $T$ is in \textbf{seconds} (\si\second) and $f$ is in \textbf{hertz}
    (\si\hertz)
  \end{columns}
\end{frame}



%\begin{frame}{Rotating Object Without Slipping}
%  A tire with radius $r$ rolls along the road with an angular velocity $\omega$
%  \emph{without slipping}. (This is a very common case for analysis.)  What
%  is its velocity $v$
%  \begin{enumerate}[a.]
%  \item at the contact between the ground and the tire?
%  \item at the center?
%  \item at the top of the tire?
%  \end{enumerate}
%
%  \vspace{-.4in}
%  \begin{center}
%    \hspace{1in}
%    \begin{tikzpicture}[scale=.7]
%      \draw[fill=black] circle(3);
%      \draw[fill=white] circle(2.2);
%      \draw[ultra thick](-6,-3)--(6,-3);
%      \draw[ultra thick,blue!70,->](-.707,-.707) arc(225:135:1)
%      node[midway,left]{$\omega$};
%      \draw[fill=blue!70,blue!70](0,-3) circle(.1) node[below right]{$v_a$};
%      \draw[fill=blue!70,blue!70] circle(.1);
%      \draw[ultra thick,blue!70,->](0,0)--(1,0) node[right]{$v_b$};
%      \draw[fill=blue!70,blue!70](0,3) circle(.1);
%      \draw[ultra thick,blue!70,->](0,3)--(2,3) node[right]{$v_c$};
%    \end{tikzpicture}
%  \end{center}
%\end{frame}



\begin{frame}{Angular Acceleration}
  The time derivative of $\omega$ is \textbf{angular acceleration}, which
  has a unit of \si{\radian\per\second\squared}:

  \eq{-.1in}{
    \alpha=\diff{\omega}t=\diff[2]{\theta}t
  }

  Similar to the relationship between velocity and angular velocity,
  \textbf{tangential acceleration} $a_t$ is related to angular acceleration
  $\alpha$ by the radius $r$:
    
  \eq{-.1in}{
    a_t(t)=\diff vt=r\diff{\omega}t=r\alpha
  }
    
  For \emph{uniform} circular motion, $\omega$ is constant, and therefore
  $a_t=0$
\end{frame}



\begin{frame}{With Calculus}
  Relationship between angular position and angular velocity:

  \eq{-.1in}{
    \omega(t)=\diff{\theta}t\quad\quad
    \theta(t)=\int\omega(t)\dl t +\theta_0
  }

  Relationship between angular velocity and angular acceleration:

  \eq{-.1in}{
    \alpha(t)=\diff{\omega}t=\diff[2]{\theta}t
    \quad\quad\omega(t)=\int\alpha(t)\dl t+\omega_0
  }

  The relationships are the same as in rectilinear motion.
\end{frame}




\begin{frame}{Kinematics in the Angular Direction}
  For constant $\alpha$, the kinematic equations are just like in rectilinear
  motion:

  \vspace{-.3in}{\large
    \begin{align*}
      \theta&=\theta_0 + \omega_0 t + \frac12\alpha t^2\\
      \theta&=\theta_0+ \frac{\omega_0+\omega}2 t\\
      \omega^2& = \omega_0^2+ 2\alpha(\theta-\theta_0)
    \end{align*}
  }
  
  For non-constant $\alpha$, calculus will be required.
\end{frame}



\begin{frame}{A Simple Example}
  \textbf{Example:} An object moves in a circle with angular acceleration
  \SI{3.0}{\radian\per\second\squared}. The radius is \SI{2.0}{\metre} and it
  starts from rest. How long does it take for this object to finish a circle?
\end{frame}



\begin{frame}{Centripetal Acceleration \& Centripetal Force}
  There is also a component of acceleration toward the center of the motion,
  called the \textbf{centripetal acceleration} $a_c$:

  \eq{-.1in}{
    \boxed{\vec a_c=-\frac{v^2}r\hat r=-(\omega^2r)\hat r}
  }

  (The negative sign indicates that $\vec a_c$ is radially inward toward the
  center of motion.) The force that causes the centripetal acceleration is
  called the \textbf{centripetal force}:

  \eq{-.1in}{
    \boxed{
      \vec F_c=m\vec a_c=-\frac{mv^2}r\hat r=-(m\omega^2r)\hat r
    }
  }
\end{frame}



\begin{frame}{Centripetal Acceleration for Uniform Circular Motion}
  In uniform circular motion ($\alpha=0$) problems where the period or
  frequency are known, the speed of the object is:

  \eq{-.1in}{
    v=\omega r = 2\pi rf = \frac{2\pi r}T
  }

  Centripetal acceleration can therefore be expressed based on $T$ or $f$:

  \eq{-.1in}{
    \vec a_c=-(\omega^2r)\hat r\quad\rightarrow\quad
    \boxed{
      \vec a_c=-\frac{4\pi^2r}{T^2}\hat r=-4\pi^2rf^2\hat r
    }
  }
\end{frame}



\begin{frame}{Acceleration: The General Case}
  \begin{columns}
    \column{.25\textwidth}
    \begin{tikzpicture}[scale=4.2]
      \draw[dashed] (.866,-.5) arc (-30:30:1);
      \draw[vectors,magenta] (1,0)--(.5,0) node[midway,below]{$a_c=\omega^2r$};
      \draw[vectors,cyan] (1,0)--(1,.3) node[right]{$a_t=r\alpha$};
      \fill (1,0) circle (.02);
    \end{tikzpicture}
    
    \column{.75\textwidth}
    In general circular motion, there are two components of acceleration:
    \begin{itemize}
    \item\textcolor{magenta}{\textbf{Centripetal acceleration} $a_c$} depends on
      radius of curvature $r$ and instantaneous speed $v$. The direction of
      the acceleration is toward the center of the circle.
    \item \textcolor{cyan}{\textbf{Tangential acceleration} $a_t$}
      depends on radius $r$  and angular acceleration $\alpha$. The direction
      of the acceleration is tangent to the circle
    \end{itemize}
  \end{columns}
\end{frame}



\begin{frame}{How to Solve Circular Motion Problems}
  The condition for circular motion is the second law of motion:

  \eq{-.1in}{
    \vec F_c=\sum\vec F=m\vec a_c
  }
  
  The forces that generate the centripetal force comes from the free-body
  diagram. It may include:
  \begin{itemize}
  \item Gravity
  \item Friction
  \item Normal force
  \item Tension
  \item Etc.
  \end{itemize}
\end{frame}



\begin{frame}{Example: Horizontal Motion}
  \begin{columns}
    \column{.4\textwidth}
    \pic1{puck-on-table}
    
    \column{.6\textwidth}
    \textbf{Example:} In the figure on the left, a mass $m_1$ is rolling around
    a frictionless table with radius $R$ with a speed $v$. What is the mass of
    $m_2$?
  \end{columns}
\end{frame}


\begin{frame}{Banked Curves on Highways and Racetracks}
  \begin{columns}
    \column{.35\textwidth}
    \centering
    \pic{.8}{banked-turn-acceleration}
    
    \begin{tikzpicture}[vectors]
      \fill circle (.08);
      \draw[rotate=-30] (0,0)--(0,1) node[above]{$\vec N$};
      \draw (0,0)--(0,-1) node[below]{$\vec F_g$};
      \draw[rotate=60] (0,0)--(0,-1) node[right]{$\vec f$};
    \end{tikzpicture}
    \begin{tikzpicture}[axes]
      \draw (0,0)--(1,0) node[right]{$x$};
      \draw (0,0)--(0,1) node[above]{$y$};
    \end{tikzpicture}

    \column{.65\textwidth}
    No motion in the $y$ direction, i.e.\ no net force:

    \eq{-.1in}{
      \sum F_y=N\cos\theta-f\sin\theta-F_g=0
    }

    Net force in the $x$ direction is the centripetal force:

    \eq{-.1in}{
      \sum F_x=N\sin\theta +f\cos\theta = \frac{mv^2}r
    }

    Friction force $\vec f$ may be static or kinetic.
  \end{columns}
\end{frame}



\begin{frame}{Banked Curves on Highways and Racetracks}
  \begin{columns}
    \column{.35\textwidth}
    \centering
    \pic{.8}{banked-turn-acceleration}
    
    \begin{tikzpicture}[vectors]
      \fill circle (.08);
      \draw[rotate=-30] (0,0)--(0,1)node[above]{$\vec N$};
      \draw (0,0)--(0,-1)node[below]{$\vec F_g$};
      \draw[rotate=60] (0,0)--(0,-1)node[right]{$\vec f$};
    \end{tikzpicture}
    \begin{tikzpicture}[axes]
      \draw (0,0)--(1,0) node[right]{$x$};
      \draw (0,0)--(0,1) node[above]{$y$};
    \end{tikzpicture}

    \column{.65\textwidth}
    For analysis, use the simplified equation for friction $f=\mu N$ (i.e.\
    assume either kinetic friction or maximum static friction), and weight
    $F_g=mg$, the equations on the previous slides can be arranged as:

    \vspace{-.3in}{\large
      \begin{align*}
        N\left(\cos\theta-\mu\sin\theta\right) &=mg\\
        N\left(\sin\theta+\mu\cos\theta\right) &=\frac{mv^2}r
      \end{align*}
    }
  \end{columns}
\end{frame}


\begin{frame}{Banked Curves on Highways and Racetracks}
  \begin{columns}
    \column{.35\textwidth}
    \centering
    \pic{.8}{banked-turn-acceleration}\\
    \begin{tikzpicture}[vectors]
      \fill circle(.08);
      \draw[rotate=-30] (0,0)--(0,1) node[above]{$\vec N$};
      \draw (0,0)--(0,-1) node[below]{$\vec F_g$};
      \draw[rotate=60] (0,0)--(0,-1) node[right]{$\vec f$};
    \end{tikzpicture}
    \begin{tikzpicture}[axes]
      \draw (0,0)--(1,0) node[right]{$x$};
      \draw (0,0)--(0,1) node[above]{$y$};
    \end{tikzpicture}

    \column{.65\textwidth}
    Dividing the two equations removes both the normal force and mass terms:

    \eq{-.1in}{
      \frac{\sin\theta+\mu\cos\theta}{\cos\theta-\mu\sin\theta}
      =\frac{v^2}{rg}
    }

    The \emph{maximum} velocity $v_\text{max}$ can be expressed as:

    \eq{-.1in}{
      \boxed{v_\text{max}=
        \sqrt{rg\frac{\sin\theta+\mu\cos\theta}{\cos\theta-\mu\sin\theta}}
      }
    }

    Note that $v_\text{max}$ does not depend on mass.
  \end{columns}
\end{frame}



\begin{frame}{Banked Curves on Highways and Racetracks}
  In the limit of $\mu=0$ (frictionless case), the equation reduces to:

  \eq{-.1in}{
    \boxed{ v_\text{max}=\sqrt{rg\tan\theta} }
  }

  And in the limit of a flat roadway with no banking ($\theta=0$,
  $\sin\theta=0$ and $\cos\theta=1$), the equation reduces to:

  \eq{-.1in}{
    \boxed{
      v_{\text{max}}=\sqrt{\mu rg}
    }
  }
\end{frame}




%
%
%\begin{frame}{Another Example: Exit Ramp}
%  \textbf{Example:} A car exits a highway on a ramp that is banked at
%  \ang{15} to the horizontal. The exit ramp has a radius of curvature of
%  \SI{65}{\metre}. If the conditions are extremely icy and the driver cannot
%  depend on any friction to help make the turn, at what speed should the driver
%  travel so that the car will not skid off the ramp? What if there is friction?
%\end{frame}


\section{Vertical Circles}

\begin{frame}{Vertical Circles}
  Circular motion with a horizontal path is straightforward. However, for
  vertical motion:
  \begin{itemize}
  \item Generally difficult to solve by dynamics and kinematics
  \item Instead, use conservation of energy to solve for $\vec v$
  \item Then use the equation for centripetal force to find other forces
  \end{itemize}

  \textbf{Remember:} If it is impossible to get the required centripetal
  force, then it could not continue the circular motion
\end{frame}



\begin{frame}{What About a Pendulum?}
  A simple pendulum is also like a vertical circular motion problem.

  \vspace{.1in}
  \begin{columns}
    \column{.35\textwidth}
    \begin{tikzpicture}[scale=.8]
      \fill[pattern=north east lines] (-1,0) rectangle (1,0.2);
      \draw[very thick] (-1,0)--(1,0);
      \begin{scope}[rotate=20]
        \draw[thick] (0,0)--(0,-5);
        \shade[ball color=red] (0,-5) circle (0.2) node[below right]{$m$};
        \begin{scope}[vectors,red]
          \draw (0,-5)--(0,-3.3) node[left]{$\vec F_T$};
          \draw[rotate around={-20:(0,-5)}] (0,-5)--(0,-6.5)
          node[below]{$\vec F_g$};
        \end{scope}
      \end{scope}
      \draw[dashed] (0,0)--(0,-5);
      \draw[dashed] (3.54,-3.54) arc (315:225:5);
    \end{tikzpicture}

    \column{.65\textwidth}
    \begin{itemize}
    \item There are two forces act on the pendulum: weight $\vec F_g$, and
    tension $\vec F_T$
    \item Speed of the pendulum at any height is found using conservation
      of energy
      \begin{itemize}
      \item $\vec F_T$ is always $\perp$ to motion, therefore it doesn't
        do any work
      \item Work is done by gravity (a conservative force) alone
      \end{itemize}
    \item Tangential and centripetal accelerations are based on the net force
      along the angular and radial directions
    \end{itemize}
  \end{columns}
\end{frame}



\begin{frame}{Simple Pendulum: Radial Forces at the Highest Point}
  \begin{columns}
    \column{.5\textwidth}
    \begin{tikzpicture}[scale=.75]
      \fill[pattern=north east lines] (-1,0) rectangle (1,0.2);
      \draw[very thick] (-1,0)--(1,0);
      \begin{scope}[rotate=45]
        \draw[thick] (0,0)--(0,-5);
        \shade[ball color=red] (0,-5) circle (.2) node[right=2.5]{$m$};
        \begin{scope}[vectors,red]
          \draw[dotted] (0,-5)--(-1.1,-5)node[left]{$F_g\sin\theta$};
          \draw[dotted] (0,-5)--(0,-6.1)
          node[right,fill=yellow!20]{$F_g\cos\theta$};
          \draw (0,-5)--(0,-3.9) node[left,fill=yellow!20]{$\vec F_T$};
          \draw[rotate around={-45:(0,-5)}] (0,-5)--(0,-6.5)
          node[below]{$\vec F_g$};
        \end{scope}
      \end{scope}
      \draw[dashed] (0,0)--(0,-5);
      \draw[dashed] (3.54,-3.54) arc (315:225:5);
      \draw[axes] (0,-2) arc (270:315:2) node[midway,below]{$\theta$};
    \end{tikzpicture}

    \column{.5\textwidth}
    At the top of the swing, velocity $v$ is zero, therefore centripetal
    acceleration is also zero:

    \eq{-.15in}{
      a_c=\frac{v^2}r=0
    }

    \vspace{-.1in}the net force along the radial direction $\hat r$ is zero.
    The tension force $F_T$ can be calculated:

    \eq{-.15in}{
      F_T=mg\cos\theta
    }
    
    \vspace{-.15in}At the highest point when $\theta$ is largest, tension is the
    lowest.
  \end{columns}
\end{frame}



\begin{frame}{Simple Pendulum: Tangential Forces at the Highest Point}
  \vspace{.2in}
  \begin{columns}[T]
    \column{.5\textwidth}
    \begin{tikzpicture}[scale=.75]
      \fill[pattern=north east lines] (-1,0) rectangle (1,0.2);
      \draw[very thick] (-1,0)--(1,0);
      \begin{scope}[rotate=45]
        \draw[thick] (0,0)--(0,-5);
        \shade[ball color=red] (0,-5) circle(.2) node[right=2.5]{$m$};
        \begin{scope}[vectors,red]
          \draw[dotted] (0,-5)--(-1.1,-5)
          node[left,fill=cyan!10]{$F_g\sin\theta$};
          \draw[dotted] (0,-5)--(0,-6.1) node[right]{$F_g\cos\theta$};
          \draw (0,-5)--(0,-3.9) node[left]{$\vec F_T$};
          \draw[rotate around={-45:(0,-5)}] (0,-5)--(0,-6.5)
          node[below]{$\vec F_g$};
        \end{scope}
      \end{scope}
      \draw[dashed] (0,0)--(0,-5);
      \draw[dashed] (3.54,-3.54) arc (315:225:5);
      \draw[axes] (0,-2) arc (270:315:2) node[midway,below]{$\theta$};
    \end{tikzpicture}

    \column{.5\textwidth}
    In the tangential direction $\hat\theta$, there is still a net force of
    $mg\sin\theta$, therefore, there is a tangential acceleration with a
    magnitude:

    \eq{-.15in}{
      a_t=g\sin\theta
    }

    %This is the same acceleration as an object sliding down a
    %frictionless ramp at an angle of $\theta$.
  \end{columns}
\end{frame}



\begin{frame}{Simple Pendulum: Bottom of Swing}
  At the bottom of the swing, the velocity is at its maximum value,
  therefore
  \vspace{.1in}\begin{columns}
    \column{.35\textwidth}
    \begin{tikzpicture}[scale=.8]
      \fill[pattern=north east lines] (-1,0) rectangle (1,.2);
      \draw[very thick] (-1,0)--(1,0);
      \draw[thick] (0,0)--(0,-5);
      \shade[ball color=red] (0,-5) circle (0.2) node[below right]{$m$};
      \draw[vectors,red] (0,-5)--(0,-2.5) node[right]{$\vec F_T$};
      \draw[vectors,red] (0,-5)--(0,-6.5) node[below]{$\vec F_g$};
      \draw[dashed] (3.54,-3.54) arc (315:225:5);
    \end{tikzpicture}

    \column{.65\textwidth}
    \begin{itemize}
    \item Centripetal acceleration is at maximum value:

      \eq{-.13in}{
        a_c=\frac{v^2}r
      }
    \item There is no tangential acceleration because there are no tangential
      forces:

      \eq{-.15in}{
        a_t=0
      }
    \item At the lowest point, tension is the highest:

      \eq{-.1in}{
        F_T=F_g+F_c=m\left(g+\frac{v^2}r\right)
      }
    \end{itemize}
  \end{columns}
\end{frame}



\begin{frame}{Example Problem: Vertical Motion}
  \textbf{Example:} You are playing with a yo-yo with a mass $M$. The full
  length of the string is $R$. You decide to see how slowly you can swing it in
  a vertical circle while keeping the string fully extended, even when the
  yo-yo is at the top of its swing.
  \begin{enumerate}[a.]
  \item Calculate the minimum speed at which you can swing the yo-yo while
    keeping it on a circular path.
  \item If the yo-yo is at its minumum speed at the top of its swing, find the
    tension in the string when the yo-yo is at the side and at the bottom.
  \end{enumerate}
\end{frame}



\begin{frame}{Example Problem: Vertical Motion}
  %This is a very typical problem for vertical motion.
  %To solve this problem, we
  First, we draw free-body diagrams for each of the positions in the circle.
  There are two forces acting on the yo-yo: gravity ($\vec F_g$) and tension
  ($\vec F_T$).
  %\footnote{We are, of
  %course, ignoring drag and friction, but a this speed, this will not affect
  %our answers}
  \begin{center}
    \begin{tikzpicture}[scale=.75]
      \draw[thick] circle (2);
      \begin{scope}[red]
        \fill (0,2) circle (.1);
        \draw[vectors] (-.06,2)--+(0,-1) node[left]{$\vec F_T$};
        \draw[vectors] (.06,2)--+(0,-1.5) node[right]{$\vec F_g$};
      \end{scope}
      \begin{scope}[violet]
        \fill (-2,0) circle (.1);
        \draw[vectors] (-2,0)--+(1.5,0) node[below left]{$\vec F_T$};
        \draw[vectors] (-2,0)--+(0,-1.5) node[left]{$\vec F_g$};
      \end{scope}
      \begin{scope}[orange]
        \fill (0,-2) circle (.1);
        \draw[vectors] (0,-2)--+(0,1.7) node[right]{$\vec F_T$};
        \draw[vectors] (0,-2)--+(0,-1.5) node[left]{$\vec F_g$}; 
      \end{scope}
    \end{tikzpicture}
  \end{center}
  \vspace{-.2in}Since the circular motion is not uniform (i.e.\ the speed of
  the yo-yo is not constant), we have to also use conservation of energy to
  solve it.
\end{frame}




\begin{frame}{Example Problem: Vertical Motion}
  \centering
  \begin{tikzpicture}
    \draw[dashed] circle (2);
    \begin{scope}[red]
      \fill (0,2) circle (.1);
      \draw[vectors] (-.06,2)--+(0,-1) node[left]{$\vec F_T$};
      \draw[vectors] (.06,2)--+(0,-1.5) node[right]{$\vec F_g$};
      \draw[vectors,black] (-.2,2)--+(-1,0) node[left]{$\vec v$};
    \end{scope}

    \node[text width=7.8cm,fill=red!10] (fc) at (6.6,2.5){
      At the top of the circle, centripetal force is provided by both gravity
      and string tension, i.e.:
      
      \vspace{-.1in}\begin{displaymath}
        F_c = Ma_c\quad\rightarrow\quad
        F_T+F_g = \frac{Mv^2}R
      \end{displaymath}\par
    };
    \draw[axes,red] (fc) to[out=180,in=80] (0,2.2);
    \uncover<2->{
      \node[text width=7.8cm,fill=yellow!10] (min) at (6.6,0){
        Since $M$, $g$ and $R$ are constant, minimum velocity $v_\text{min}$ on
        the right side means $F_T=0$ on the left side. We are left with:
        
        \vspace{-.1in}\begin{displaymath}
          Mg = \frac{Mv_\text{min}^2}R
        \end{displaymath}\par
      };
    }
    \uncover<3->{
      \node[text width=7.8cm,fill=green!12] at (6.6,-2.4){
        Cancelling $M$ and solving for $v_\text{min}$, we have:
      
        \vspace{-.12in}\begin{displaymath}
          v^2=gR\quad\rightarrow\quad\boxed{v_\text{min} = \sqrt{gR}}
        \end{displaymath}\par
      };
    }
  \end{tikzpicture}
\end{frame}




\begin{frame}{Example Problem: Vertical Motion}
  \centering
  \begin{tikzpicture}
    \fill circle (.05);
    \draw[dashed] circle (2);
    \begin{scope}[violet]
      \fill (-2,0) circle (.1);
      \draw[vectors] (-2,0)--+(1.5,0) node[below left]{$\vec F_T$};
      \draw[vectors] (-2,0)--+(0,-1.5) node[right]{$\vec F_g$};
    \end{scope}

    \node[text width=5.5cm,fill=red!10] (fc) at (-5.2,2.2){
      At the side of the circle, centripetal force is provided only by
      tension:

      \vspace{-.1in}\begin{displaymath}
        F_c = Ma_c\quad\rightarrow\quad
        F_T = \frac{Mv^2}R
      \end{displaymath}
      But we do not know the speed $v$ of the yo-yo at this location yet.
    };
    
    \uncover<2->{
      \node[text width=4.5cm,fill=yellow!15] (min) at (4.5,2.6){
        Using conservation of energy:
        
        \vspace{-.2in}\begin{align*}
          K_\text{top} + U_\text{top} &= K_\text{side}\\
          \frac12Mv_\text{top}^2 + MgR &=\frac 12Mv_\text{side}^2
        \end{align*}
      };
    }
    
    \uncover<3->{
      \node[text width=4.5cm,fill=green!10] at (4.5,.4){
        Cancelling $M$ term and solving for $v_\text{side}^2$, we have:

        \vspace{-.1in}\begin{displaymath}
          v_\text{side}^2 = v_\text{top}^2+2gR
        \end{displaymath}\par
      };
    }


    \uncover<4->{
      \node[text width=4.5cm,fill=blue!15] at (4.5,-1.6){
        Since $v_\text{top}^2=v_\text{min}^2=gR$ that we have just calculated,

        \vspace{-.2in}\begin{displaymath}
          v_\text{side}^2 = (gR)^2+2gR=3gR
        \end{displaymath}\par
      };
    }

    \uncover<5->{
      \node[text width=5.5cm,fill=violet!15] at (-5.2,-1.1){
        Now the final expression for tension:
        
        \vspace{-.2in}\begin{displaymath}
          F_T = \frac{Mv^2}R = \frac{M(3rR)}R=\boxed{3Mg}
        \end{displaymath}
        Tension is 3 times the weight of the yo-yo!
      };
    }
  \end{tikzpicture}
\end{frame}



\begin{frame}{Example Problem: Vertical Motion}
  \centering
  \begin{tikzpicture}
    \fill circle (.05);
    \draw[dashed] circle (2);
    \begin{scope}[orange]
      \fill (0,-2) circle (.1);
      \draw[vectors] (0,-2)--+(0,1.7) node[right]{$\vec F_T$};
      \draw[vectors] (0,-2)--+(0,-1.5) node[left]{$\vec F_g$}; 
    \end{scope}

    \node[text width=5.6cm,fill=red!10] (fc) at (-5,1.2){
      At the bottom of the circle, tension contributes to centripetal force,
      while gravity contributes \emph{against} it:

      \vspace{-.22in}\begin{displaymath}
        F_c = Ma_c\quad\rightarrow\quad
        F_T-F_g = \frac{Mv^2}R
      \end{displaymath}
      Again, we need to find the speed of the yo-yo at this location.
    };
    
    \uncover<2->{
      \node[text width=4.6cm,fill=yellow!10] (min) at (4.5,1.7){
        Using conservation of energy again:
        
        \vspace{-.2in}\begin{align*}
          K_\text{top} + U_\text{top} &= K_\text{bottom}\\
          \frac12Mv_\text{top}^2 + MgR &=\frac 12Mv_\text{bottom}^2
        \end{align*}
      };
    }
    
    \uncover<3->{
      \node[text width=4.6cm,fill=green!10] at (4.5,-.6){
        Cancelling $M$ term and solving for $v_\text{bottom}^2$, we have:

        \vspace{-.1in}\begin{displaymath}
          v_\text{bottom}^2 = v_\text{top}^2+4gR
        \end{displaymath}\par
      };
    }


    \uncover<4->{
      \node[text width=4.6cm,fill=blue!15] at (4.5,-2.5){
        Recognizing that $v_\text{top}^2=gR$ like we did before:

        \vspace{-.2in}\begin{displaymath}
          v_\text{bottom}^2 = (gR)^2+4gR=5gR
        \end{displaymath}\par
      };
    }

    \uncover<5->{
      \node[text width=5.6cm,fill=violet!15] at (-5,-2.2){
        Now the final expression for tension:
        
        \vspace{-.2in}\begin{align*}
          F_T &= Mg+\frac{Mv^2}R =Mg + \frac{M(5rR)}R\\
          &=\boxed{6Mg}
        \end{align*}
        $F_T$ is 6 times the weight of the yo-yo
      };
    }
  \end{tikzpicture}
\end{frame}



\begin{frame}{Example Problem}
  \textbf{Example:} A cord is tied to a pail of water, and the pail is swung
  in a vertical circle of \SI{1.}\metre. What must be the minimum velocity of
  the pail be at its highest point so that no water spills out?
  \begin{enumerate}[(A)]
  \item\SI{3.1}{\metre\per\second}
  \item\SI{5.6}{\metre\per\second}
  \item\SI{20.7}{\metre\per\second}
  \item\SI{100.5}{\metre\per\second}
  \end{enumerate}
\end{frame}



\begin{frame}{Example: Roller Coaster}
  \textbf{Example:} A roller coaster car is on a track that forms a circular
  loop, of radius $R$, in the vertical plane. If the car is to maintain contact
  with the track at the top of the loop (generally considered to be a good
  thing), what is the minimum speed that the car must have at the bottom of the
  loop. Ignore air resistance and rolling friction.
  \begin{enumerate}[(A)]
  \item $\sqrt{2gR}$
  \item $\sqrt{3gR}$
  \item $\sqrt{4gR}$
  \item $\sqrt{5gR}$
  \end{enumerate}
\end{frame}



\begin{frame}{Example}
  \textbf{Example:} A stone of mass $m$ is attached to a light strong string
  and whirled in a \emph{vertical} circle of radius $r$. At the exact bottom of
  the path, the tension of the string is three times the weight of the stone.
  The stone's speed at that point is given by:
  \begin{enumerate}[(A)]
  \item $2\sqrt{gR}$
  \item $\sqrt{2gR}$
  \item $\sqrt{3gR}$
  \item $4gR$
  \end{enumerate}
\end{frame}
\end{document}
