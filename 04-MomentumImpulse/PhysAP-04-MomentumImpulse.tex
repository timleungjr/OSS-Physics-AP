\documentclass[12pt,compress,aspectratio=169]{beamer}

\mode<presentation>
{
  \usetheme{Singapore}
  \setbeamersize{text margin left=.5cm,text margin right=.5cm}
%  \setbeamertemplate{navigation symbols}{} % suppress nav bar
%  \setbeamercovered{transparent}
}
\usefonttheme{professionalfonts}
\usepackage{amsmath,bm}
\usepackage{siunitx}
\usepackage{tikz}
\usepackage{mathpazo}
\usepackage[scaled]{helvet}
\usepackage{xcolor,colortbl}

\sisetup{
  number-math-rm=\mathnormal,
  per-mode=symbol
}

\usetikzlibrary{decorations.pathmorphing,patterns}


\title{Topic 4: Momentum, Impulse and Collisions}
\subtitle{Advanced Placement Physics}
\author[TML]{Dr.\ Timothy Leung}
\institute{Olympiads School}
\date{Novemver 23, 2019}

\newcommand{\pic}[2]{\includegraphics[width=#1\textwidth]{#2}}
\newcommand{\mb}[1]{\ensuremath\mathbf{#1}}
\newcommand{\eq}[2]{\vspace{#1}{\Large\begin{displaymath}#2\end{displaymath}}}

\begin{document}

\begin{frame}
  \maketitle
\end{frame}

\section{Momentum}


\begin{frame}{Linear Momentum}
  \textbf{Linear momentum} is proportional to the object's \textbf{mass} and
  its \textbf{velocity}.

  \eq{-.2in}{
    \boxed{\mb{p}=m\mb{v}}
  }
  \begin{center}
    \begin{tabular}{l|c|c}
      \rowcolor{pink}
      \textbf{Quantity} & \textbf{Symbol} & \textbf{SI Unit} \\ \hline
      Momentum & $\mb{p}$ & \si{\kilo\gram.\metre\per\second} \\
      Mass      & $m$    & \si{\kilo\gram} \\
      Velocity  & $\mb{v}$ & \si{\metre\per\second}
    \end{tabular}
  \end{center}
  \begin{itemize}
  \item Momentum $\mb{p}$ is a vector in the same direction as velocity
  \item Like all vectors, $\mb{p}$ obeys the principle of superposition
  \end{itemize}
\end{frame}



\begin{frame}{Newton's Second Law}
  Start with the familiar form of Newton's second law of motion with constant
  $m$, we can find out how $\Delta\mb{p}$ relates to $\mb{F}$:
 
  \eq{-.2in}{
    \sum\mb{F}=m\mb{a}=m\frac{d\mb{v}}{dt}=\frac{d(m\mb{v})}{dt}
    =\frac{d\mb{p}}{dt}
  }

  In fact, this is the general form of Newtons second law of motion:
  \textbf{net force on an object is the rate of change of its momentum}

  \eq{-.2in}{
    \boxed{\mb{F}_{\textrm{net}}=\frac{d\mb{p}}{dt}}
  }

  $\mb{F}_\mathrm{net}=m\mb{a}$ is just a special case
%  \item Momentum is conserved (i.e.\ $\sum\mb{p}$ constant) when the net force
%    on an object or a system of objects is zero.
%  \item Internal forces do not contribute to net force, in that case:
%
%    \eq{-.2in}{
%      \sum_i\mb{p}_i(t_1)=\sum_i\mb{p}_i(t_2)
%    }
%  \end{itemize}
\end{frame}



\begin{frame}{Newton's Second Law}

  \eq{0in}{
    \boxed{\mb{F}_{\textrm{net}}=\frac{d\mb{p}}{dt}}
  }
  \begin{itemize}
  \item Momentum is conserved (i.e.\ $\sum\mb{p}$ constant) when the net force
    on an object or a system of objects is zero.
  \item Internal forces do not contribute to net force, in that case:

    \eq{-.2in}{
      \sum_i\mb{p}_i(t_1)=\sum_i\mb{p}_i(t_2)
    }
  \end{itemize} 
\end{frame}


\begin{frame}{Impulse}
  Let's get this by looking at Newton's second law again. If we rearrange the
  variables:
  
  \eq{-.2in}{
    \mb{F}_\mathrm{net}=\frac{d\mb{p}}{dt}\;\rightarrow\;
    \mb{F}_\mathrm{net}dt=d\mb{p}
  }

  We can integrate both sides to get the \textbf{impulse-momentum theorem}.
 
  \eq{-.2in}{
    \boxed{
      \mb{J}_{\textrm{net}}
      =\int_{t_1}^{t_2}\mb{F}_\mathrm{net}dt
      =\int d\mb{p}=\Delta\mb{p}}
  }
  
  The quantity $\mb{J}_{\textrm{net}}$ is called the \textbf{net impulse}.
\end{frame}



\begin{frame}{Impulse}
  $\mb{F}$, $\mb{p}$ and $\mb{J}$ are all vectors, so the integral can be
  evaluated in each of the $x$, $y$ and $z$ axis, i.e., for the $x$ direction:

  \eq{-.2in}{
    J_x=\int_{t_1}^{t_2}F_xdt=\int dp_x=\Delta p_x
  }

  \textbf{Average force} is a ``force'' that gets the same impulse, i.e.

  \eq{-.2in}{
    \overline{\mb{F}}=\frac{\int_{t_1}^{t_2}\mb{F}dt}{t_2-t_1}
    =\frac{\mb{J}}{\Delta t}
  }

  Note that impulse from each individual force does not depend on whether the
  object moves. The change in momentum only depends on \emph{net} impulse
\end{frame}

\begin{frame}
  \frametitle{Impulse: An Example}
  \textbf{Example 4:} Jim pushes a box with mass \SI{1.}{\kilo\gram} with a
  \SI{5.}{\newton} force for \SI{10}{\second} while the box stays on the same
  place. Find the impulse of the pushing force, friction force, the
  gravitational force, and the net force.
\end{frame}

\begin{frame}
  \frametitle{Impulse: Another Example}
  \textbf{Example 5:} Two balls of the same mass are dropped from the same
  height onto the floor. The first ball bounces upwards from the floor
  elastically. The second ball sticks to the floor. The first applies an
  impulse to the floor of $I_1$ and the second applies an impulse $I_2$. The
  two impulses obey:
  \begin{enumerate}[(a)]
  \item $I_2=2I_1$
  \item $I_2=I_1/2$
  \item $I_2=4I_1$
  \item $I_2=I_1/4$
  \end{enumerate}
\end{frame}

\begin{frame}
  \frametitle{Conservation of Momentum}
  \begin{itemize}
  \item From Newton's third law, we know that the action and reaction forces are
    always equal in magnitude and in opposite direction. Thus, their total
    impulse would be zero. 
    
  \item When there is no external force, the momentum of the total system will
    always be constant. We saw that a few slides ago:

    \eq{-.2in}{
      \sum\mb{p}(t_1)=\sum\mb{p}(t_2)
    }
  \end{itemize}
\end{frame}

\begin{frame}
  \frametitle{How to Solve Conservation of Momentum Problem}
  \begin{enumerate}
  \item Check whether the condition for the conservation of momentum is
    satisfied.
  \item If so, write out expressions for initial momentum and final momentum,
    and equate the two. You will get $1$ to $3$ equations (one for each
    direction).
  \item Solve these equations, find the quantity you need to find.
  \end{enumerate}
%\end{frame}
%
%\begin{frame}
%  \frametitle{Two Remarks}
%  \begin{itemize}
%  \item Sometimes, the external force \emph{does} exist, but are too small, or
%    the time interval of the external force is very short. In these cases, we
%    can still regard the total momentum as conserved.
  Remember that momentum is a vector. If there is no external force component
  in some direction, then the momentum component in this direction is still
  conserved.
%  \end{itemize}
\end{frame}

\begin{frame}
  \frametitle{Example}
  \textbf{Example 6:} Two blocks A and B, both have mass \SI{1.}{\kilo\gram}.
  Block A has velocity \SI{3.}{\metre\per\second} and block B is at rest. Their
  distance is \SI{1.}{\metre}. The surface is has dynamic friction coefficient
  $0.02$. After they collide, they move together, what would be the final
  velocity of these two blocks? How far can they go after the collision?
\end{frame}

%\begin{frame}
%  \frametitle{More Example}
%  Max throws a ball into the air with an initial speed $\SI{10}{\metre\per\second} at an
%  angle of $60$ degree with the horizontal direction. By accident, the ball
%  splits into two parts (horizontally) in the air. Suppose both parts land at
%  the same time, neglecting the air resistance,
%  \begin{enumerate}
%  \item If one part is \SI{5}{\m} away from its original position (same
%    direction as the initial speed), where is the second part?
%  \item How about one parties \SI{5}{\metre} away from the original position in the
%    direction that has an angle of $30$ degree with its initial speed?
%  \end{enumerate}
%\end{frame}
%
\begin{frame}
  \frametitle{Before We Dive Into Some Exercises}
  \begin{itemize}
  \item The most typical applications of momentum conservation are collision
    and explosions
  \item\textbf{Collision: object A hits object B}. Regardless of whether they
    move together or not afterwards, momentum is conserved.
    \begin{itemize}
    \item Head-on collisions are usually 1D
    \item Glancing collisions are usually 2D or 3D.
    \end{itemize}
  \item\textbf{Explosion: A explodes and becomes B and C (and D and E\ldots)}.
    Total momentum of B and C (and D and E\ldots) is the same as A in the
    beginning. 
  \end{itemize}
\end{frame}

\begin{frame}
  \frametitle{Collision Problem}
  \textbf{Example 7:} Two objects with equal mass are heading toward each
  other with equal speeds, undergo a head-on collision. Which one of the
  following statement is correct?
  \begin{enumerate}[(a)]
  \item Their final velocities are zero
  \item Their final velocities may be zero
  \item Each must have a final velocity equal to the other's initial velocity
  \item Their velocities must be reduced in magnitude
  \end{enumerate}
\end{frame}

\begin{frame}
  \frametitle{Conservation of Momentum Example}
  \textbf{Example 8:} Two astronauts, each of mass \SI{75}{\kilo\gram}, are
  floating next to each other in space, outside the space shuttle. One of them
  pushes the other through a distance of \SI{1.}{\metre} (about an arm's
  length) with a force of \SI{300}{\newton}. What is the final relative
  velocity of the two?
  \begin{enumerate}[(a)]
  \item \SI{2.}{\metre\per\second}
  \item \SI{2.83}{\metre\per\second}
  \item \SI{4.}{\metre\per\second}
  \item \SI{16.}{\metre\per\second}
  \end{enumerate}
\end{frame}

\begin{frame}
  \frametitle{Continuous Problems in the Application of Momentum}

%  \textbf{Example 9:} A water fountain sprays water with a flow of
%  \SI{30}{L/min}. Suppose the water has no initial velocity, find the impulse
%  of the pushing force in $1$ hour and estimate the pushing force, assuming the
%  force is constant.
%\end{frame}
%
%\begin{frame}
%  \frametitle{Example: Rocket Thrust}
  \textbf{Example 10:} A rocket generates a thrust force by ejecting hot gases
  from an engine. If it takes \SI{1}{\milli\second} to combust
  \SI{1.}{\kilo\gram} of fuel, ejecting it at a speed of
  \SI{1000}{\metre\per\second}, what thrust is generated?
  \begin{enumerate}[(a)]
  \item \SI{1000}{\newton}
  \item \SI{10000}{\newton}
  \item \SI{100000}{\newton}
  \item \SI{1000000}{\newton}
  \end{enumerate}
\end{frame}

\begin{frame}
  \frametitle{Another Space Example}
  \textbf{Example 11:} A rocket for mining the asteroid belt is designed like a
  large scoop. It is approaching asteroids at a velocity of
  \SI{e4}{\metre\per\second}. The asteroids are much smaller than the rocket.
  If the rocket scoops asteroids at a rate of \SI{100}{\kilo\gram\per\second},
  what thrust (force) must the rocket's engine provide in order for the rocket
  to maintain constant velocity? Ignore any variation in the rocket's mass due
  to the burning fuel.
  \begin{enumerate}[(a)]
  \item \SI{e3}{\newton}
  \item \SI{e6}{\newton}
  \item \SI{e9}{\newton}
  \item \SI{e12}{\newton}
  \end{enumerate}
\end{frame}


\begin{frame}
  \frametitle{Example}
  \textbf{Example 12:} A ball is dropped from a height $h$. It hits the ground
  and bounces up with a momentum loss of $10\%$ due to the impact. The maximum
  height it will reach is:
  \begin{enumerate}[(a)]
  \item $0.90h$
  \item $0.81h$
  \item $0.949h$
  \item $0.3h$
  \end{enumerate}
\end{frame}

\begin{frame}
  \frametitle{Conservation of Energy Example}
  \textbf{Example 13:} A simple pendulum has a bob of mass \SI{2}{\kilo\gram}
  hanging on a cord of length \SI{1}{\metre}. Suppose the pendulum is raised
  until it is horizontal (and angular displacement of \ang{90}) and then
  released. What is the speed of the bob at the bottom of its swing?
  \begin{enumerate}[(a)]
  \item\SI{9.91}{\metre\per\second}
  \item\SI{19.6}{\metre\per\second}
  \item\SI{3.13}{\metre\per\second}
  \item\SI{4.43}{\metre\per\second}
  \end{enumerate}
\end{frame}
 
\begin{frame}
  \frametitle{Conservation of Energy Example}
  \textbf{Example 14:} A toy firing a ball vertically consists of a vertical
  spring which is compressed by \SI{.10}{\metre}. A force of \SI{10.}{\newton}
  is needed to hold the spring at that compression. If a ball of mass
  \SI{.050}{\kilo\gram} is placed on the compressed spring and the spring is
  released, the ball will reach a height (above its initial position) of:
  \begin{enumerate}[(a)]
  \item \SI{1.}{\metre}
  \item \SI{1.2}{\metre}
  \item \SI{1.4}{\metre}
  \item \SI{1.6}{\metre}
  \end{enumerate}
\end{frame}


\section{Elastic Collisions}

\begin{frame}
  \frametitle{Classifications of Collisions}
  \begin{itemize}
  \item Elastic Collision:
    \begin{itemize}
    \item Total kinetic energy is conserved
    \item<alert@2> Momentum is conserved
    \end{itemize}
  \item Inelastic collision:
    \begin{itemize}
    \item Kinetic energy is \textbf{not} conserved
    \item<alert@2> Momentum is conserved
    \end{itemize}
  \item Completely inelastic collision:
    \begin{itemize}
    \item ``Perfectly inelastic collision''
    \item The objects move together after the collision
    \item Kinetic energy is \textbf{not} conserved
    \item<alert@2> Momentum is conserved
    \end{itemize}
  \end{itemize}
\end{frame}

\begin{frame}
  \frametitle{Elastic Collision}
  If two objects 1 and 2 of mass $m_1$ and $m_2$ and initial velocities
  $v_{1,i}$ and $v_{2,i}$ collide elastically, their final velocities will be:
  
  {\Large
    \begin{displaymath}
      v_{1,f}=\frac{v_{1,i}(m_1-m_2)+2m_2v_{2,i}}{m_1+m_2}
    \end{displaymath}
    
    \begin{displaymath}
      v_{2,f}=\frac{v_{2,i}(m_2-m_1)+2m_1v_{1,i}}{m_1+m_2}
    \end{displaymath}
  }
\end{frame}

\begin{frame}
  \frametitle{Elastic Collision Example}

  \textbf{Example 15:} Blocks A and B have the same mass; A hits B with a speed
  of \SI{5.}{\metre\per\second} while B is initially at rest. If the collision
  is elastic, what would be the final speed of these two objects?
\end{frame}


\begin{frame}
  \frametitle{Elastic Collision Example}
  \textbf{Example 16:} Blocks A and B with the same mass; A has a velocity
  \SI{3.}{\metre\per\second} to the east while B has \SI{2}{\metre\per\second}
  to the west. If the collision is elastic, after the collision, what would the
  velocity of the two blocks be?
\end{frame}


\begin{frame}
  \frametitle{Elastic Collision Example}
  
  \textbf{Example 17:} Throw a ball to a really big wall, when the ball reaches
  the wall, it has a velocity \SI{10}{\metre\per\second} toward the wall. If
  the collision is elastic, what would the final velocity of the ball be?
\end{frame}


\begin{frame}
  \frametitle{Elastic Collision Example}
  \textbf{Example 18:} Throw a ball with a velocity \SI{4.}{\metre\per\second}
  toward a train with a velocity \SI{40}{\metre\per\second} toward the ball.
  If the collision is elastic, what would the final velocity of the ball be?
\end{frame}


\begin{frame}
  \frametitle{Inelastic Collision: Calculating Energy Loss}
  \textbf{Example 19:} Two blocks A and B with mass \SI{2.}{\kilo\gram}, block
  A hits B with velocity \SI{4.}{\metre\per\second} while B is at rest.
  \begin{enumerate}[(a)]
  \item Suppose the collision is completely inelastic, what would the final
    velocity of A and B be?
  \item What is the loss of energy?
  \end{enumerate}
\end{frame}

\end{document}
